%!TEX root = soutenance_lei_2018.tex
\subsection{Detection of insulation flaws and thermal bridges in insulated truck box panels}
\frame{\tableofcontents[currentsection,currentsubsection]}

\begin{frame}{Outline}
    ATP standard tests: \textbf{Global} results\\
    % \textbf{Local} defects ==> \alert{Infrared Thermography}

    Time consuming!\\
    In summer, not economical.
    \\
    ATP standard method uses heating inside and cooling outside.\\
    Can we just cool the outside to see the difference?\\
    \pause
    ==> Using {\color{cyan}{compressed air}} instead of \alert{radiator!}
\end{frame}


\begin{frame}{Methods--Specimen}
    \begin{figure}[ht]
    % \centering
    \hspace*{-20pt}
    \subfloat[Panel detail]
    {
        \includegraphics[scale=0.35]{img/ch4/Panel_detail.png}
    }
    \pause
    \subfloat[Panel done]
    {
        \includegraphics[scale=0.265]{img/ch4/Panel_done.jpg}
    }
    \caption{Truck panel specimen}
    \end{figure}
 
\end{frame}


\begin{frame}{Experimental Set-up}
    \begin{figure}[ht]
        \centering
        \includegraphics[scale=0.4]{img/ch4/Exp.png}
    \end{figure}
\end{frame}

\begin{frame}{Simulation models}
 The governing pure conduction equation:
    \begin{equation*}
        \rho C_p \frac{\partial T}{\partial t}-\nabla \cdot (k\nabla T) = 0
    \end{equation*}
The boundary condition (convection and radiation):
    \begin{equation*}
        n(k\nabla T) = q_0 + h_{cv}(T_{amb}-T)+\sigma \epsilon(T_{amb}^4-T^4)
    \end{equation*}
\pause
    \begin{description}
        \item[Lamp {\color{red}{Heating}}] \textit{Heat transfer in Solid with surface-to-surface Radiation} module  
        \item[{\color{cyan}{Air Cooling}}] Heat transfer and CFD modules    
    \end{description}
\end{frame}

\begin{frame}{Mesh}
    \begin{figure}[ht]
    % \centering
    \hspace*{-22pt}
    \subfloat[Lamp {\color{red}{Heating}} meshes]
    {
        \includegraphics[scale=0.38]{img/ch4/Mesh_flash.png}
    }
    % \pause
    \subfloat[Air {\color{cyan}{Cooling}} meshes]
    {
        \includegraphics[scale=0.38]{img/ch4/Mesh_laminar.png}
    }
    \caption{Simulation meshes}
    \end{figure}

\end{frame}


\begin{frame}{Results}
Experimental results:

    \begin{figure}
        \hspace*{-20pt}
        \subfloat[Lamp {\color{red}{Heating}}]
        {
            \includegraphics[scale=0.42]{img/ch4/Heat_FT_AMP.png}
        }
        \subfloat[Air {\color{cyan}{Cooling}}]
        {
            \includegraphics[scale=0.42]{img/ch4/Cool_PCT3_2.png}
        }
        \caption{Experimental results (PCT 3rd image)}

    \end{figure}
\end{frame}

\begin{frame}{Results}
Computational results:

    \begin{figure}
        \hspace*{-20pt}
        \subfloat[Lamp {\color{red}{Heating}}]
        {
            \includegraphics[scale=0.50]{img/ch4/Truck_panel_flash_03.png}
        }
        \hspace*{8pt}
        \subfloat[Air {\color{cyan}{Cooling}}]
        {
            \includegraphics[scale=0.50]{img/ch4/Truck_panel_laminar_final_7_3.png}
        }
        \caption{Computational results}

    \end{figure}
\end{frame}


\begin{frame}{Quantitative comparison}

    \begin{figure}
        \hspace*{-20pt}
        \subfloat[{\color{red}{Exp.}} Lamp Heating]
        {
            \includegraphics[scale=0.42]{img/ch4/heating_evolution5.png}
        }
        % \hspace*{8pt}
        \subfloat[{\color{cyan}{Sim.}} Lamp Heating]
        {
            \includegraphics[scale=0.37]{img/ch4/Truck_panel_Flash_TGraph_4.png}
        }
        \caption{Quantitative results}

    \end{figure}
\end{frame}


\begin{frame}{Quantitative comparison}

    \begin{figure}
        \hspace*{-20pt}
        \subfloat[{\color{red}{Exp.}} Air Cooling]
        {
            \includegraphics[scale=0.42]{img/ch4/cooling_evolution4.png}
        }
        % \hspace*{8pt}
        \subfloat[{\color{cyan}{Sim.}} Air Cooling]
        {
            \includegraphics[scale=0.37]{img/ch4/Truck_panel_laminar_TGraph_4.png}
        }
        \caption{Quantitative results}

    \end{figure}
\end{frame}


\begin{frame}{Discussion}
    \begin{itemize}[<+->]
    \pause
    \large
        \item Detection of steel dusts, water defects and thermal bridge 
        \item Heating approach provides clearer results
        \item Compressed air spray is faster  
        \item Practically convenient with air spray than the heating method in providing successful detection
    \end{itemize}
\end{frame}
