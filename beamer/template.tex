\documentclass{beamer}
\usefonttheme[onlymath]{serif}

%\usepackage[utf8]{inputenc}
%\usepackage[T1]{fontenc}
\usepackage{fontspec}
\usepackage{lmodern}
%\usepackage[english]{babel}
%\frenchbsetup{CompactItemize=false}


% \usepackage[babel]{csquotes}
% \usepackage[url=false, doi=false, style=science, backend=bibtex, bibencoding=ascii]{biblatex}
% \bibliography{IEEEabrv,bib/OAM}

  
\graphicspath{{img/}}


\mode<presentation> {
	% \useoutertheme{infolines} % Pour les thèmes qui n'ont pas de pied-de-page
	\usetheme{ulaval}
	%\usecolortheme{ulaval}
	% \setbeamercovered{transparent}
	\setbeamercovered{invisible}
	%\setbeamertemplate{navigation symbols}{} % Enlever les icônes de navigation
}


%\usepackage{bm} 
% For typesetting bold math (not \mathbold)
%\includeonlyframes{}

%\usepackage{pgfpages}
%\setbeameroption{show notes on second screen}
%\setbeameroption{show notes}
%\setbeamertemplate{note page}[plain]

\logo{\includegraphics[height=0.6cm]{COPL}\hspace{.5cm}%
	\includegraphics[height=0.5cm]{UL_P}\hspace{.2cm}\vspace{.85\paperheight}}


\title[Short Title]{Long Title}
%\subtitle[]{}

\author[Short Author Name]{Long Author Name}
\institute[Université Laval]
{
	Electrical and Computer Engineering Department \\
	Laval University, Quebec City, Canada \\
	\medskip
	{\emph{email@ulaval.ca}}
}
\date{\today} % \today will show current date. 
% Alternatively, you can specify a date.


\AtBeginSection[]{
  \begin{frame}
	\Huge \centerline{\insertsection}
%  \small \tableofcontents[currentsection, hideothersubsections]
  \end{frame} 
}

\begin{document}



\begin{frame}[label=titre, plain]
	\titlepage
	\begin{center}\includegraphics[height=1.2cm]{COPL}%
		\hspace{2cm}\includegraphics[height=1cm]{UL_P}\end{center}
\end{frame}


\section*{Contents}

\begin{frame}[label=toc]{Outline}
	\setlength{\leftskip}{5cm}%
	\tableofcontents[subsectionstyle=show]
\end{frame}


\section{Introduction}


\begin{frame}[label=intro]{Introduction slide}
	This document shows how to use the \emph{ulaval} Beamer template.
	It gives examples to see how the resulting document looks like.
\end{frame}

\section{Main Section}

\subsection{List styles}

\subsubsection{Itemize}

\begin{frame}[label=itemize]\frametitle{Itemize sample} 
	\begin{itemize}
		\item Item 1
		\item Item 2
		\begin{itemize}
			\item Sub item 1
			\item Sub item 2
			\begin{itemize}
				\item Sub sub sub item 1
				\item Sub sub sub item 2
			\end{itemize}
			\item Sub item 3
		\end{itemize}
		\item Item 3
	\end{itemize}
\end{frame}

\subsubsection{Enumerate}

\begin{frame}[label=enumerate]\frametitle{Enumerate sample} 
	\begin{enumerate}
		\item Item 1
		\item Item 2
		\begin{enumerate}
			\item Sub item 1
			\item Sub item 2
			\begin{enumerate}
				\item Sub sub sub item 1
				\item Sub sub sub item 2
			\end{enumerate}
			\item Sub item 3
		\end{enumerate}
		\item Item 3
	\end{enumerate}
\end{frame}

\subsubsection{Description}

\begin{frame}[label=description]\frametitle{Description sample}
    
\begin{description}
	\item[Term 1:] Definition 1
	\item[Term 2:] Definition 2
	\item[Term 3:] Definition 3
\end{description}

\end{frame}

\subsection{Boxes styles}

\begin{frame}[label=boxes]\frametitle{Boxes Styles}
    
\begin{block}{Block Title}
	Block content
\end{block}

\begin{alertblock}{Alert Block Title}
	Alert block content
\end{alertblock}

\begin{exampleblock}{Example Block Title}
	Example block content
\end{exampleblock}

\end{frame}

\subsection{Block environments}

\begin{frame}[label=environments]\frametitle{Environments Samples}
    
	\begin{definition}
	Definition content
	\end{definition}
  
	\begin{example}
	Example content
	\end{example}

	\begin{proof}
	Proof content
	\end{proof}  
    
	\begin{theorem}
	Theorem content
	\end{theorem}

\end{frame}


\subsection{Math}

\begin{frame}[label=math]\frametitle{Math}
    
\begin{equation}
	V_0 = k_0 \rho \sqrt{n_1^2 - n_2^2}
\end{equation}

\end{frame}


\subsection{Text environments}

\begin{frame}[label=text]\frametitle{Text Environments}
    
\begin{quotation}
  Quotation environment line 1\\
  Quotation environment line 2
\end{quotation}
\begin{quote}
  Quote environment line 1\\
  Quote environment line 2
\end{quote}
\begin{semiverbatim}
  Semiverbatim environment
\end{semiverbatim}
\begin{verse}
  Verse environment line 1\\
  Verse environment line 2
\end{verse}  

\end{frame}

\section{Conclusion}

\begin{frame}[label=conclu]{Conclusion slide}
	That's all folks!
\end{frame}



% End of slides
\end{document}
