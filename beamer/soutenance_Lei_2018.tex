\documentclass{beamer}
\usefonttheme[onlymath]{serif}

%\usepackage[utf8]{inputenc}
%\usepackage[T1]{fontenc}
\usepackage{fontspec}
\defaultfontfeatures{Extension = .otf} % ...and this line

\usepackage{lmodern}
%\usepackage[english]{babel}
%\frenchbsetup{CompactItemize=false}


% \usepackage[babel]{csquotes}
% \usepackage[url=false, doi=false, style=science, backend=bibtex, bibencoding=ascii]{biblatex}
% \bibliography{IEEEabrv,bib/OAM}

  
\graphicspath{{img/}}
\DeclareGraphicsExtensions{.jpg,.png,.pdf,.eps}

\mode<presentation> {
	% \useoutertheme{infolines} % Pour les thèmes qui n'ont pas de pied-de-page
	\usetheme{ulaval}
	\usecolortheme{ulaval}
	% \setbeamercovered{transparent}
	\setbeamercovered{invisible}
	%\setbeamertemplate{navigation symbols}{} % Enlever les icônes de navigation
}


%\usepackage{bm} 
% For typesetting bold math (not \mathbold)
%\includeonlyframes{}

%\usepackage{pgfpages}
%\setbeameroption{show notes on second screen}
%\setbeameroption{show notes}
%\setbeamertemplate{note page}[plain]

\logo{
	\includegraphics[height=0.5cm]{UL_P}
	\hspace{.5cm}%
	\includegraphics[height=0.5cm]{lvsn}
	\hspace{.5cm}
	\vspace{.85\paperheight}
	\includegraphics[height=0.5cm]{logoCNR-ITC}}


\title[Cold food chain-IRT application]{Cold food chain: Infrared thermography applied to the evaluation of insulation anomalies in refrigerated vehicles for the transport of food \& Exploration of cold approach in infrared thermography for Non-Destructive Testing}
%\subtitle[]{}

\author[Lei Lei]{Lei Lei}
\institute[Université Laval]
{
	Electrical and Computer Engineering Department \\
	Laval University, Quebec City, Canada \\
	\medskip
	% {\emph{email@ulaval.ca}}
}
\date{\today} % \today will show current date. 
% Alternatively, you can specify a date.


\AtBeginSection[]{
  \begin{frame}
	\Huge \centerline{\insertsection}
 % \small \tableofcontents[currentsection, hideothersubsections]
  \end{frame} 
}

\begin{document}



\begin{frame}[label=titre, plain]
	\titlepage
	\begin{center}
		\includegraphics[height=0.9cm]{UL_P}%
		\hspace{1cm}
		\includegraphics[height=0.9cm]{lvsn}
		\hspace{1cm}
		\includegraphics[height=0.9cm]{logoCNR-ITC}
	\end{center}
\end{frame}


\section*{Contents}

\begin{frame}[label=toc]{Outline}
	\setlength{\leftskip}{5cm}%
	\tableofcontents[subsubsectionstyle=hide]
\end{frame}


\section{Introduction}


\begin{frame}[label=intro]{Introduction slide}
	This document shows how to use the \emph{ulaval} Beamer template.
	It gives examples to see how the resulting document looks like.
\end{frame}

\section{IRT application for ATP}
%!TEX root = soutenance_lei_2018.tex
\subsection{Mapping of the heat flux of an insulated small container by infrared thermography}
\frame{\tableofcontents[currentsection,currentsubsection]}

\begin{frame}{ATP test}
    the overall coefficient of heat transfer ($K$) is defined as:
    \begin{equation*}
        K = \frac{W}{S\cdot \Delta \theta}
    \end{equation*}
    $W$: Power \\
    $\Delta \theta$: a  steady  temperature  difference\\
    $S$ is the mean surface of the equipment:
    \begin{equation*}
        S = \sqrt{S_i \cdot S_e}
    \end{equation*}
\end{frame}


\begin{frame}{Theory \& Methods}
    Symbolically Ohm’s law:
    \begin{equation*}
        I = \frac{\Delta V}{R_e}
    \end{equation*}
    \pause
    Analogy to an electrical circuit, Fourier’s law can be written similarly as
    \begin{equation*}
        q = \frac{\Delta T}{R_t}
    \end{equation*}

    \pause
    \begin{figure}
        \centering
        \includegraphics[scale=0.55]{img/ch2/Therm_Res.png}
        % \caption{Overall thermal resistance of the roll-container}
        % \label{Therm_Res}
    \end{figure}

    \centering
    \begin{minipage}[c]{0.8\textwidth}
        $\theta_i$: inside temperature,  $\theta_{wi}$: internal wall temperature\\
        $\theta_e$: outside temperature, $\theta_{we}$: external wall temperature
    \end{minipage}

\end{frame}


\begin{frame}{Theory \& Methods}
    Heat flux computation:\\ 
    by a thermal flux meter in a reference zone:
    \begin{equation*}
        q_{ref} = \frac{\theta_{we}(x_r,y_r)-\theta_e}{1/h_e}
    \end{equation*}
    which gives that:
    \begin{equation*}
        h_e = \frac{q_{ref}}{\theta_{we}(x_r,y_r)-\theta_e}
    \end{equation*}
    then:
    \pause
    \begin{equation*}
        q(x,y) = \frac{q_{ref}}{\theta_{we}(x_r,y_r)-\theta_e}(\theta_{we}(x,y)-\theta_e)
    \end{equation*}
    where $\theta_{we}(x,y)$ is the temperature at each point of the external surface.    
\end{frame}


\begin{frame}{Experimental Setup}
Roll-container:
    \begin{figure}[!htbp]
        \centering
        \includegraphics[scale=0.08]{img/ch2/DSC_0191.jpg}
        \includegraphics[scale=0.08]{img/ch2/DSC_0189.jpg}
        \caption{The roll container used for the test [outside and inside]}
        % \label{box}
    \end{figure}
\end{frame}


\begin{frame}{Results}
Values from heat flux meter:
    \begin{figure}
        % \centering
        \hspace*{-18pt}
        \includegraphics[scale=0.275]{img/ch2/It_project_2014_QProfile.png}
        \includegraphics[scale=0.275]{img/ch2/It_project_2014_TProfile.png}
        % \caption{Data from thermal flux meter (heat flux and temperature profiles)}
        % \label{flux_meter}
    \end{figure}
    \centering
    \small{Mean values (steady condition) are $q_r=9.73\; W/(m^2 K)$ and $T_r = 7.83\; ^\circ$C.}
\end{frame}


\begin{frame}{Results--\small{Thermal raw images}}
    \begin{figure}
        \vspace*{-5pt}
        \centering
        \includegraphics[scale=0.25]{img/ch2/IR_front_m.jpg}
        \hspace{5pt}
        \includegraphics[scale=0.25]{img/ch2/IR_back_m.jpg}
        \includegraphics[scale=0.25]{img/ch2/IR_left_m.jpg}
        \hspace{5pt}
        \includegraphics[scale=0.25]{img/ch2/IR_right_m.jpg}
        \includegraphics[scale=0.25]{img/ch2/IR_top.jpg}
        \vspace*{-5pt}
        \caption{\footnotesize{The temperature map of the roll container (front, rear, left, right and top surfaces)}}
        % \label{Q_box}
    \end{figure}
\end{frame}

\begin{frame}{Results--\small{Image correction}}
    Several image geometrical corrections exit, while an easier way to realize is to apply the \alert{homography} technique from computer vision.\\
    \pause
    Briefly, the planar homography relates the transformation between two planes (up to a scale factor):
    \begin{equation*}
        s\begin{bmatrix}
        x'\\y'\\1
        \end{bmatrix}
        =H \begin{bmatrix}
        x\\y\\1
        \end{bmatrix}
        = \begin{bmatrix}
        h_{11} & h_{12} & h_{13}\\h_{21} & h_{22} & h_{23}\\h_{31} & h_{32} & h_{33}
        \end{bmatrix}
        \begin{bmatrix}
        x\\y\\1
        \end{bmatrix}
    \end{equation*}
It is generally normalized with $h_{33}=1$
\end{frame}

\begin{frame}{Results--\small{Homography}}
    \begin{figure}[ht]
        \centering
        \includegraphics[scale=0.45]{img/ch2/homography_perspective_correction.jpg}
    \end{figure}
    \pause
    In our case, the ratio between the length and the width of the roll container is known. Once the projective transformation matrix in images has been obtained, a bilinear interpolation with the projective transformation matrix into the raw images will be performed.

    Recall: 
        \begin{equation*}
            q(x,y) = \frac{q_r}{\theta_{we}(x_r,y_r)-\theta_e}(\theta_{we}(x,y)-\theta_e)
        \end{equation*}
\end{frame}


\begin{frame}{Results--\small{Heat flux map}}
    \begin{figure}
        \vspace*{-5pt}
        \centering
        \includegraphics[scale=0.25]{img/ch2/Q_front_m.jpg}
        \hspace{5pt}
        \includegraphics[scale=0.25]{img/ch2/Q_back_m.jpg}
        \hspace*{5pt}
        \includegraphics[scale=0.25]{img/ch2/Q_left_m.jpg}
        \hspace{5pt}
        \includegraphics[scale=0.25]{img/ch2/Q_right_m.jpg}
        \includegraphics[scale=0.25]{img/ch2/Q_top.jpg}
        \vspace*{-5pt}
        \caption{\footnotesize{The heat flux map of the roll container (front, rear, left, right and top surfaces)}}
        % \label{Q_box}
    \end{figure}
\end{frame}


\begin{frame}{Results--\small{Vehicle results}}
    \begin{figure}
        \vspace*{-5pt}
        \centering
        \includegraphics[scale=0.25]{img/ch2/IR_truck_lt_m}
        \hspace{3pt}
        \includegraphics[scale=0.25]{img/ch2/IR_truck_rt_m}
        % \vspace{3pt}
        \includegraphics[scale=0.25]{img/ch2/IR_truck_bk11_m}
        \hspace{6pt}
        \includegraphics[scale=0.25]{img/ch2/IR_truck_bk12_m}
        \includegraphics[scale=0.25]{img/ch2/IR_truck_tp_m}
        \vspace*{-5pt}
        \caption{\footnotesize{The temperature map of the refrigerated vehicle (left, right, rear1, rear2 and top surfaces)}}
    \end{figure}
\end{frame}


\begin{frame}{Discussion}
    \begin{itemize}[<+->]
    \pause
    \large
        \item Thermal resistance model works well
        \item Homography application offers good results in roll-container
        \item The convective heat transfer coefficient not constant around the roll-container surfaces
        \item Unable to apply homography in large size of vehicle panel 
        \item Thermal reflection in vehicle results
    \end{itemize}
\end{frame}

%!TEX root = soutenance_lei_2018.tex
\subsection{Panoramic view of the heat flux inside the vehicle}
\frame{\tableofcontents[currentsection,currentsubsection]}

\begin{frame}{Outline}
 As a truck is normally huge, the homography transformation would then have a large deviation in the final thermal images.\\
 \pause
 ==> Take photo inside the truck ?\\
 \pause
 ==> \alert{Panoramic view!}
\end{frame}


\begin{frame}{Experimental Set-up}
    \begin{figure}[ht]
        \centering
        \includegraphics[scale=0.3]{img/ch3/Camera_setup.jpg}
        \caption{Pan-tilt device for experiment test}
    \end{figure}
\end{frame}


\begin{frame}{Image processing--\small{spherical projection}}
    \begin{figure}
        % \centering
        \hspace*{-15pt}
        \includegraphics[scale=0.238]{img/ch3/Sph_1.jpg}
        \includegraphics[scale=0.238]{img/ch3/Sph_2.jpg}
        \caption{Spherical projection.}
        \label{Sph_pro}
    \end{figure}
\end{frame}

\begin{frame}{Image processing--\small{spherical projection}}
Therefore, given the focal length $ f $ and the image coordinates $ (x, y) $, the corresponding spherical coordinates $ (x', y') $ are:

    \begin{align*}
        x'={} f \cdot tan(\dfrac{x-x_c}{f})+x_c \notag \\
        y'=f \cdot \frac{tan(\dfrac{y-y_c}{f})}{cos(\dfrac{x-x_c}{f})} +y_c
    \end{align*}
where $ (x_c,y_c) $ are the center coordinates of the spherical image.

\end{frame}


\begin{frame}{Image processing--\small{spherical projection}}
The original image and its spherical projection are then:
    \begin{figure}
        % \centering
        \hspace*{-15pt}
        \includegraphics[scale=0.336]{img/ch3/img7.jpg}
        \includegraphics[scale=0.336]{img/ch3/sph_img7.jpg}
        \caption{Original IR image (left) and its spherical projection (right).}
        \label{Orig_sph}
    \end{figure}
\end{frame}


\begin{frame}{Image projection--\small{Harris corner}}
    \begin{figure}
        %   \centering
        \hspace*{-15pt}
        \includegraphics[scale=0.50]{img/ch3/Harris1.png}
        \includegraphics[scale=0.50]{img/ch3/Harris2.png}
        \caption{Harris corners detected (green crosses) in the figure above and its precedent image of the whole series. }
        % \label{Harris}
    \end{figure}
    % In these two continuous images, the IR camera scanned from top to bottom inside the vehicle and there is a ``clear" (from our vision) overlapping part (left bottom part in the first and left top part in the second) in these two images. However, from the Harris corners detection results, all the features obtained could not be matched correctly.
\end{frame}


\begin{frame}{Image translation and stitching}
    \begin{figure}
        \centering
        \vspace*{-10pt}
        \includegraphics[scale=0.3]{img/ch3/img_trs.jpg}\\
        \pause
        \includegraphics[scale=0.3]{img/ch3/img_sti.jpg}
    \end{figure}
\end{frame}


\begin{frame}{Temperature panorama}
    \begin{figure}
        \hspace*{-15pt}
        \includegraphics[scale=0.222]{img/ch3/Pano_T_Final.jpg}
    \end{figure}
    \centering
    Mean value: $ \overline{T} = 305.94$ K $= 32.79$ °C
\end{frame}

\begin{frame}{Heat flux meter}
    \begin{figure}
    \centering
    \vspace*{-18pt}
    \includegraphics[scale=0.31]{img/ch3/QIRT2015Aisa_Fig8.png}
    \end{figure}
    % \pause
    Mean value after 10 hours (steady condition): $q_{ref}=11.460\; W/m^2$
\end{frame}

\begin{frame}{Heat flux panorama}
    \begin{figure}
        \hspace*{-15pt}
        \includegraphics[scale=0.222]{img/ch3/Pano_Q_Final.jpg}
    \end{figure}
    \centering
    Mean value: $\bar{q}=11.724\; W/m^2$
\end{frame}


\begin{frame}{Results \& discussion}
 Therefore, the final K-value from IR thermography is:

    \begin{equation*}
    K_{th}=\frac{\bar{q}}{\Delta ̅\theta} =\frac{11.724\; W/m^2}{(32.79-7.5)\;K}=0.464\; W/K\cdot m^2 
    \end{equation*}

\pause
Comparing with ATP standard:
    \begin{table}[ht]
        \centering
        \caption{ATP test results.}
        \begin{tabular}{l|r}
            \hline
            Fans Power [W] (Mean over 6 hours) & 144 \\
            \hline 
            Heaters Power [W] (Mean over 6 hours) &   988\\
            \hline
            Internal temperature [°C] (Mean over 6 hours) &   32.5\\
            \hline
            External temperature [°C] (Mean over 6 hours) &   7.5\\
            \hline
            K-Value [ W/(K m$^2 $)] & 0.46 \\
            \hline
        \end{tabular}
        % \label{ATP_res}
    \end{table}
\pause
The error between these two results is then:

    \begin{equation*}
        e=  \frac{|K_{th}-K|}{K}=\frac{0.464-0.46}{0.46}=0.0087=0.87\%
    \end{equation*}
\end{frame}

\begin{frame}{Results \& discussion}
    \begin{itemize}[<+->]
    \pause
    \large
        \item Thermal resistance model works well
        \item Favorable result in panoramic view compared to ATP
        \item Uniform and repeatable texture inside the truck toughened the automatic feature detection and image stitching
        \item Amelioration in need for image processing with advanced feature detection and description
    \end{itemize}
\end{frame}



\section{Exploration of cold approach}
%!TEX root = soutenance_lei_2018.tex
\subsection{Detection of insulation flaws and thermal bridges in insulated truck box panels}
\frame{\tableofcontents[currentsection,currentsubsection]}

\begin{frame}{Outline}
    ATP standard tests: \textbf{Global} results\\
    \textbf{Local} defects ==> \alert{Infrared Thermography}

    Time consuming!\\
    In summer, not practical, nor convenient.

    ATP standard method uses heating inside and cooling outside.\\
    Can we just cool the outside to see the difference?\\
    ==> Using {\color{cyan}{compressed air}} instead of \alert{radiator!}
\end{frame}


\begin{frame}{Methods--Specimen}
    \begin{figure}[ht]
    % \centering
    \hspace*{-20pt}
    \subfloat[Panel detail]
    {
        \includegraphics[scale=0.35]{img/ch4/Panel_detail.png}
    }
    \pause
    \subfloat[Panel done]
    {
        \includegraphics[scale=0.265]{img/ch4/Panel_done.jpg}
    }
    \caption{Truck panel specimen}
    \end{figure}
 
\end{frame}


\begin{frame}{Experimental Set-up}
    \begin{figure}[ht]
        \centering
        \includegraphics[scale=0.4]{img/ch4/Exp.png}
    \end{figure}
\end{frame}

\begin{frame}{Simulation models}
 The governing pure conduction equation:
    \begin{equation*}
        \rho C_p \frac{\partial T}{\partial t}-\nabla \cdot (k\nabla T) = 0
    \end{equation*}
The boundary condition (convection and radiation):
    \begin{equation*}
        n(k\nabla T) = q_0 + h_{cv}(T_{amb}-T)+\sigma \epsilon(T_{amb}^4-T^4)
    \end{equation*}
\pause
    \begin{description}
        \item[Lamp {\color{red}{Heating}}] \textit{Heat transfer in Solid with surface-to-surface Radiation} module  
        \item[{\color{cyan}{Air Cooling}}] Heat transfer and CFD modules    
    \end{description}
\end{frame}

\begin{frame}{Mesh}
    \begin{figure}[ht]
    % \centering
    \hspace*{-22pt}
    \subfloat[Lamp {\color{red}{Heating}} meshes]
    {
        \includegraphics[scale=0.38]{img/ch4/Mesh_flash.png}
    }
    % \pause
    \subfloat[Air {\color{cyan}{Cooling}} meshes]
    {
        \includegraphics[scale=0.38]{img/ch4/Mesh_laminar.png}
    }
    \caption{Simulation meshes}
    \end{figure}

\end{frame}


\begin{frame}{Results}
Experimental results:

    \begin{figure}
        \hspace*{-20pt}
        \subfloat[Lamp {\color{red}{Heating}}]
        {
            \includegraphics[scale=0.42]{img/ch4/Heat_FT_AMP.png}
        }
        \subfloat[Air {\color{cyan}{Cooling}}]
        {
            \includegraphics[scale=0.42]{img/ch4/Cool_PCT3_2.png}
        }
        \caption{Experimental results (PCT 3rd image)}

    \end{figure}
\end{frame}

\begin{frame}{Results}
Computational results:

    \begin{figure}
        \hspace*{-20pt}
        \subfloat[Lamp {\color{red}{Heating}}]
        {
            \includegraphics[scale=0.50]{img/ch4/Truck_panel_flash_03.png}
        }
        \hspace*{8pt}
        \subfloat[Air {\color{cyan}{Cooling}}]
        {
            \includegraphics[scale=0.50]{img/ch4/Truck_panel_laminar_final_7_3.png}
        }
        \caption{Computational results}

    \end{figure}
\end{frame}


\begin{frame}{Quantitative comparison}

    \begin{figure}
        \hspace*{-20pt}
        \subfloat[{\color{red}{Exp.}} Lamp Heating]
        {
            \includegraphics[scale=0.42]{img/ch4/heating_evolution5.png}
        }
        % \hspace*{8pt}
        \subfloat[{\color{cyan}{Sim.}} Lamp Heating]
        {
            \includegraphics[scale=0.37]{img/ch4/Truck_panel_Flash_TGraph_4.png}
        }
        \caption{Quantitative results}

    \end{figure}
\end{frame}


\begin{frame}{Quantitative comparison}

    \begin{figure}
        \hspace*{-20pt}
        \subfloat[{\color{red}{Exp.}} Air Cooling]
        {
            \includegraphics[scale=0.42]{img/ch4/cooling_evolution4.png}
        }
        % \hspace*{8pt}
        \subfloat[{\color{cyan}{Sim.}} Air Cooling]
        {
            \includegraphics[scale=0.37]{img/ch4/Truck_panel_laminar_TGraph_4.png}
        }
        \caption{Quantitative results}

    \end{figure}
\end{frame}


\begin{frame}{Discussion}
    \begin{itemize}[<+->]
    \pause
    \large
        \item Detection of steel dusts, water defects and thermal bridge 
        \item Heating approach provides more clear results
        \item Compressed air spray is more rapid  
        \item Practically convenient with air spray than the heating method in providing successful detection
    \end{itemize}
\end{frame}

%!TEX root = soutenance_lei_2018.tex
\subsection{Liquid Nitrogen Cooling in IR Thermography applied to steel specimen}
\frame{\tableofcontents[currentsection,currentsubsection]}





% \subsection{List styles}

% \subsubsection{Itemize}

% \begin{frame}[label=itemize]\frametitle{Itemize sample} 
% 	\begin{itemize}
% 		\item Item 1
% 		\item Item 2
% 		\begin{itemize}
% 			\item Sub item 1
% 			\item Sub item 2
% 			\begin{itemize}
% 				\item Sub sub sub item 1
% 				\item Sub sub sub item 2
% 			\end{itemize}
% 			\item Sub item 3
% 		\end{itemize}
% 		\item Item 3
% 	\end{itemize}
% \end{frame}

% \subsubsection{Enumerate}

% \begin{frame}[label=enumerate]\frametitle{Enumerate sample} 
% 	\begin{enumerate}
% 		\item Item 1
% 		\item Item 2
% 		\begin{enumerate}
% 			\item Sub item 1
% 			\item Sub item 2
% 			\begin{enumerate}
% 				\item Sub sub sub item 1
% 				\item Sub sub sub item 2
% 			\end{enumerate}
% 			\item Sub item 3
% 		\end{enumerate}
% 		\item Item 3
% 	\end{enumerate}
% \end{frame}

% \subsubsection{Description}

% \begin{frame}[label=description]\frametitle{Description sample}
    
% \begin{description}
% 	\item[Term 1:] Definition 1
% 	\pause
% 	\item[Term 2:] Definition 2
% 	\item[Term 3:] Definition 3
% \end{description}

% \end{frame}

% \subsection{Boxes styles}

% \begin{frame}[label=boxes]\frametitle{Boxes Styles}
    
% \begin{block}{Block Title}
% 	Block content
% \end{block}

% \begin{alertblock}{Alert Block Title}
% 	Alert block content
% \end{alertblock}

% \begin{exampleblock}{Example Block Title}
% 	Example block content
% \end{exampleblock}

% \end{frame}

% \subsection{Block environments}

% \begin{frame}[label=environments]\frametitle{Environments Samples}
    
% 	\begin{definition}
% 	Definition content
% 	\end{definition}
  
% 	\begin{example}
% 	Example content
% 	\end{example}

% 	\begin{proof}
% 	Proof content
% 	\end{proof}  
    
% 	\begin{theorem}
% 	Theorem content
% 	\end{theorem}

% \end{frame}


% \subsection{Math}

% \begin{frame}[label=math]\frametitle{Math}
    
% \begin{equation}
% 	V_0 = k_0 \rho \sqrt{n_1^2 - n_2^2}
% \end{equation}

% \end{frame}


% \subsection{Text environments}

% \begin{frame}[label=text]\frametitle{Text Environments}
    
% \begin{quotation}
%   Quotation environment line 1\\
%   Quotation environment line 2
% \end{quotation}
% \begin{quote}
%   Quote environment line 1\\
%   Quote environment line 2
% \end{quote}
% \begin{semiverbatim}
%   Semiverbatim environment
% \end{semiverbatim}
% \begin{verse}
%   Verse environment line 1\\
%   Verse environment line 2
% \end{verse}  

% \end{frame}

\section{Conclusion}

\begin{frame}[label=conclu]{Conclusion slide}
	That's all folks!
\end{frame}



% End of slides
\end{document}
