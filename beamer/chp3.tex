%!TEX root = soutenance_lei_2018.tex
\subsection{Panoramic view of the heat flux inside the vehicle}
\frame{\tableofcontents[currentsection,currentsubsection]}

\begin{frame}{Outline}
 As a truck is normally huge, the homography transformation would then have a large deviation in the final thermal images.\\
 \pause
 ==> Take photo inside of the truck ?\\
 \pause
 ==> \alert{Panoramic view!}
\end{frame}


\begin{frame}{Experimental Set-up}
    \begin{figure}[ht]
        \centering
        \includegraphics[scale=0.3]{img/ch3/Camera_setup.jpg}
        \caption{Pan-tilt device for experiment test}
    \end{figure}
\end{frame}


\begin{frame}{Image processing--\small{spherical projection}}
    \begin{figure}
        % \centering
        \hspace*{-15pt}
        \includegraphics[scale=0.238]{img/ch3/Sph_1.jpg}
        \includegraphics[scale=0.238]{img/ch3/Sph_2.jpg}
        \caption{Spherical projection.}
        \label{Sph_pro}
    \end{figure}
\end{frame}

\begin{frame}{Image processing--\small{spherical projection}}
Therefore, given the focal length $ f $ and the image coordinates $ (x, y) $, the corresponding spherical coordinates $ (x', y') $ are:

    \begin{align*}
        x'={} f \cdot tan(\dfrac{x-x_c}{f})+x_c \notag \\
        y'=f \cdot \frac{tan(\dfrac{y-y_c}{f})}{cos(\dfrac{x-x_c}{f})} +y_c
    \end{align*}
where $ (x_c,y_c) $ are the center coordinates of the spherical image.

\end{frame}


\begin{frame}{Image processing--\small{spherical projection}}
The original image and its spherical projection are then:
    \begin{figure}
        % \centering
        \hspace*{-15pt}
        \includegraphics[scale=0.336]{img/ch3/img7.jpg}
        \includegraphics[scale=0.336]{img/ch3/sph_img7.jpg}
        \caption{Original IR image (left) and its spherical projection (right).}
        \label{Orig_sph}
    \end{figure}
\end{frame}


\begin{frame}{Image projection--\small{Harris corner}}
    \begin{figure}
        %   \centering
        \hspace*{-15pt}
        \includegraphics[scale=0.50]{img/ch3/Harris1.png}
        \includegraphics[scale=0.50]{img/ch3/Harris2.png}
        \caption{Harris corners detected (green crosses) in the figure above and its precedent image of the whole series. }
        % \label{Harris}
    \end{figure}
    % In these two continuous images, the IR camera scanned from top to bottom inside the vehicle and there is a ``clear" (from our vision) overlapping part (left bottom part in the first and left top part in the second) in these two images. However, from the Harris corners detection results, all the features obtained could not be matched correctly.
\end{frame}


\begin{frame}{Image translation and stitching}
    \begin{figure}
        \centering
        \vspace*{-10pt}
        \includegraphics[scale=0.3]{img/ch3/img_trs.jpg}\\
        \pause
        \includegraphics[scale=0.3]{img/ch3/img_sti.jpg}
    \end{figure}
\end{frame}


\begin{frame}{Temperature panorama}
    \begin{figure}
        \hspace*{-15pt}
        \includegraphics[scale=0.222]{img/ch3/Pano_T_Final.jpg}
    \end{figure}
    \centering
    Mean value: $ \overline{T} = 305.94$ K $= 32.79$ °C
\end{frame}

\begin{frame}{Heat flux meter}
    \begin{figure}
    \centering
    \vspace*{-18pt}
    \includegraphics[scale=0.31]{img/ch3/QIRT2015Aisa_Fig8.png}
    \end{figure}
    % \pause
    Mean value after 10 hours (steady condition): $q_{ref}=11.460\; W/m^2$
\end{frame}

\begin{frame}{Heat flux panorama}
    \begin{figure}
        \hspace*{-15pt}
        \includegraphics[scale=0.222]{img/ch3/Pano_Q_Final.jpg}
    \end{figure}
    \centering
    Mean value: $\bar{q}=11.724\; W/m^2$
\end{frame}


\begin{frame}{Results \& discussion}
 Therefore, the final K-value from IR thermography is:

    \begin{equation*}
    K_{th}=\frac{\bar{q}}{\Delta ̅\theta} =\frac{11.724\; W/m^2}{(32.79-7.5)\;K}=0.464\; W/K\cdot m^2 
    \end{equation*}

\pause
Comparing with ATP standard:
    \begin{table}[ht]
        \centering
        \caption{ATP test results.}
        \begin{tabular}{l|r}
            \hline
            Fans Power [W] (Mean over 6 hours) & 144 \\
            \hline 
            Heaters Power [W] (Mean over 6 hours) &   988\\
            \hline
            Internal temperature [°C] (Mean over 6 hours) &   32.5\\
            \hline
            External temperature [°C] (Mean over 6 hours) &   7.5\\
            \hline
            K-Value [ W/(K m$^2 $)] & 0.46 \\
            \hline
        \end{tabular}
        % \label{ATP_res}
    \end{table}
\pause
The error between these two results is then:

    \begin{equation*}
        e=  \frac{|K_{th}-K|}{K}=\frac{0.464-0.46}{0.46}=0.0087=0.87\%
    \end{equation*}
\end{frame}

\begin{frame}{Results \& discussion}
    \begin{itemize}[<+->]
    \pause
    \large
        \item Thermal resistance model works well
        \item Favorable result in panoramic view compared to ATP
        \item Uniform and repeatable texture inside the truck toughened the automatic feature detection and image stitching
        \item Amelioration in need for image processing with advanced feature detection and description
    \end{itemize}
\end{frame}
