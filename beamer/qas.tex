%!TEX root = soutenance_lei_2018.tex
\begin{frame}{Q \& A--Chpt 1 \small{Bottom surface}}
    \begin{figure}
    \includegraphics[scale=0.06]{img/qas/DC_51949.jpg}
    \includegraphics[scale=0.06]{img/qas/DC_51951.jpg}\\
    \includegraphics[scale=0.2]{img/qas/IR_51948.jpg}
    \includegraphics[scale=0.2]{img/qas/IR_51950.jpg}
    \end{figure}

\end{frame}

\begin{frame}{Q \& A--Chpt 1}
    \begin{figure}
    \includegraphics[scale=0.095]{img/qas/DC_51957.jpg}
    \includegraphics[scale=0.3]{img/qas/IR_51956.jpg}

    \end{figure}

\end{frame}


\begin{frame}{Q \& A--Chpt 1 \small{Results comparison}}
Global K-value for roll-container:  $0.53\; W/(m^2 K)$\\
Temperature  difference: $\Delta \theta = 25.3 \;$ °C\\
\bigskip
Average value of heat flux from thermography: $q_{ir} = 10.7\; W/m^2$\\
which gives:
$K_{ir} = q_{ir}/\Delta \theta = 0.42\; W/(m^2 K)$

\bigskip
The influence of the convective heat transfer around the roll-container surfaces is more important than imaging.
\end{frame}

\begin{frame}{Q \& A--Chpt 4 \small{Specimen}}
    \begin{table}
    \centering

    \caption{Specimen specification details (dimensions in mm).}
    \hspace*{-20pt}
    \small
    \begin{tabular}{c|c|c|c}
        \hline
         Aluminum,plaques & Foam       & Air Gap     & Thermal Bridge \\
         250*150*1        & 250*150*25 & 15*150 *25  & 1*150*25       \\ 
         \hline
         Wet Napkin       & Sawdust    & Steel dusts & Water          \\ 
         40*20*10         & 20*20*15   & 20*20*15    & 40*20*10       \\ 
         \hline
    \end{tabular}
    \end{table}
\end{frame}



\begin{frame}{Q \& A--Chpt 4 \small{Materials properties}}
    \begin{figure}[ht]
      \centering
      \includegraphics[scale=0.45]{img/qas/table_mat.png}
    \end{figure}

\end{frame}



\begin{frame}{Q \& A--Chpt 4 }
    \begin{figure}
      \centering
      \includegraphics[scale=0.5]{img/qas/Laminar_model_full.png}
    \end{figure}
\end{frame}

\begin{frame}{Q \& A--Chpt 5 \small{Exp Set-up}}
    Equipments' detail:
    \begin{itemize}
        \item Infrared Camera FLIR SC3000 (spatial resolution equal to 320$\times$240 pixels, frame rate up to 50Hz, GaAs sensor, spectral range 8-9 $\mu m$)
        \item Two pairs of flash lamps for a total of 10 kJ (electric) released in 5 $ms $ 
        \item One pair of modulated halogen lamps with 1kW each served as Lock-in    stimulation   
        \item An isolated bottle (500 ml) filled with  Liquid Nitrogen.
\end{itemize}
\end{frame}


\begin{frame}{Q \& A--Chpt 5 \small{raw images:}}
    % \small{Thermal raw images:}
    \begin{figure}
     % \centering
     \vspace{-18pt}
       \subfloat[Flash raw frame 23]
       {
          \includegraphics[scale=0.35]{img/ch5/flash_raw23_2.png}
       }
       % \hspace{10pt}
       \subfloat[Flash raw frame 60]
       {
          \includegraphics[scale=0.35]{img/ch5/flash_raw60_2.png}
       }
       % \hspace{10pt}
       \\
       \vspace*{-10pt}
       \subfloat[LN$_2$ raw frame 41]
       {
          \includegraphics[scale=0.35]{img/ch5/cool_raw_2.png}
       }
       \caption{Thermal raw images of PT and LN$_2$ stimulation techniques}      
    \end{figure}
\end{frame}


\begin{frame}{Q \& A--Chpt 5 \small{ROC computation}}
The main algorithm in the calculation of TPR and FPR is clarified as:
\begin{enumerate}
   \item Choose the post-processed thermogram in which most defects shown as the test image;
%    \replaced [id=LL] {Choose the post-processed thermogram in which most defects shown as the test image;}{Identify the best gray-scale result from the post-processing thermal images as the test image;}
   \item Resize the defect map to the same size as that of the test image;
   \item Choose a thresholding step number $N$($N=1000$ in this study) and establish the step value of thresholding [from $0$, $\frac{1}{N}$, $\frac{2}{N}$ till $1 (=\frac{N}{N})$];
   \item For each thresholding step, binarize the test image with the thresholding and then compare it to the defect map, in order to obtain the corresponding TPR and FPR values;
   \item Iterate the binarization and comparison so as to plot a whole curve.
\end{enumerate}
\end{frame}


\begin{frame}{Q \& A}
CFRP (25 defects inside):\\
heating one side with lamp and cooling the other side with liquid nitrogen 
  \begin{figure}
    \centering
    \includegraphics[scale=0.15]{img/qas/ref_noisy.jpg}
  \end{figure}

\end{frame}
