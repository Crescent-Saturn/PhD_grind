%!TEX root = soutenance_lei_2018.tex
\subsection{Mapping of the heat flux of an insulated small container by infrared thermography}
\frame{\tableofcontents[currentsection,currentsubsection]}

\begin{frame}{ATP test}
    the overall coefficient of heat transfer ($K$) is defined as:
    \begin{equation*}
        K = \frac{W}{S\cdot \Delta \theta}
    \end{equation*}
    $W$: Power \\
    $\Delta \theta$: a  steady  temperature  difference\\
    $S$ is the mean surface of the equipment:
    \begin{equation*}
        S = \sqrt{S_i \cdot S_e}
    \end{equation*}
\end{frame}


\begin{frame}{Theory \& Methods}
    Symbolically Ohm’s law:
    \begin{equation*}
        I = \frac{\Delta V}{R_e}
    \end{equation*}
    \pause
    Analogy to an electrical circuit, Fourier’s law can be written similarly as
    \begin{equation*}
        q = \frac{\Delta T}{R_t}
    \end{equation*}

    \pause
    \begin{figure}
        \centering
        \includegraphics[scale=0.55]{img/ch2/Therm_Res.png}
        % \caption{Overall thermal resistance of the roll-container}
        % \label{Therm_Res}
    \end{figure}

    \centering
    \begin{minipage}[c]{0.8\textwidth}
        $\theta_i$: inside temperature,  $\theta_{wi}$: internal wall temperature\\
        $\theta_e$: outside temperature, $\theta_{we}$: external wall temperature
    \end{minipage}

\end{frame}


\begin{frame}{Theory \& Methods}
    Heat flux computation:\\ 
    by a thermal flux meter in a reference zone:
    \begin{equation*}
        q_{ref} = \frac{\theta_{we}(x_r,y_r)-\theta_e}{1/h_e}
    \end{equation*}
    which gives that:
    \begin{equation*}
        h_e = \frac{q_{ref}}{\theta_{we}(x_r,y_r)-\theta_e}
    \end{equation*}
    then:
    \begin{equation*}
        q(x,y) = \frac{q_{ref}}{\theta_{we}(x_r,y_r)-\theta_e}(\theta_{we}(x,y)-\theta_e)
    \end{equation*}
    Where $\theta_{we}(x,y)$ is the temperature at each point of the external surface.    
\end{frame}


\begin{frame}{Experimental Setup}
Roll-container:
    \begin{figure}[!htbp]
        \centering
        \includegraphics[scale=0.08]{img/ch2/DSC_0191.jpg}
        \includegraphics[scale=0.08]{img/ch2/DSC_0189.jpg}
        \caption{The roll container used for the test [outside and inside]}
        % \label{box}
    \end{figure}
\end{frame}


\begin{frame}{Results}
Values from heat flux meter:
    \begin{figure}
        % \centering
        \hspace*{-18pt}
        \includegraphics[scale=0.275]{img/ch2/It_project_2014_QProfile.png}
        \includegraphics[scale=0.275]{img/ch2/It_project_2014_TProfile.png}
        % \caption{Data from thermal flux meter (heat flux and temperature profiles)}
        % \label{flux_meter}
    \end{figure}
    \centering
    \small{Mean values (steady condition) are $q_r=9.73\; W/(m^2 K)$ and $T_r = 7.83\; ^\circ$C.}
\end{frame}


\begin{frame}{Results--\small{Thermal raw images}}
    \begin{figure}
        \vspace*{-5pt}
        \centering
        \includegraphics[scale=0.25]{img/ch2/IR_front_m.jpg}
        \hspace{5pt}
        \includegraphics[scale=0.25]{img/ch2/IR_back_m.jpg}
        \includegraphics[scale=0.25]{img/ch2/IR_left_m.jpg}
        \hspace{5pt}
        \includegraphics[scale=0.25]{img/ch2/IR_right_m.jpg}
        \includegraphics[scale=0.25]{img/ch2/IR_top.jpg}
        \vspace*{-5pt}
        \caption{\footnotesize{The temperature map of the roll container (front, rear, left, right and top surfaces)}}
        % \label{Q_box}
    \end{figure}
\end{frame}

\begin{frame}{Results--\small{Image correction}}
    Several image geometrical corrections exit, while an easier way to realize is to apply the \alert{homography} technique from computer vision.\\
    \pause
    Briefly, the planar homography relates the transformation between two planes (up to a scale factor):
    \begin{equation*}
        s\begin{bmatrix}
        x'\\y'\\1
        \end{bmatrix}
        =H \begin{bmatrix}
        x\\y\\1
        \end{bmatrix}
        = \begin{bmatrix}
        h_{11} & h_{12} & h_{13}\\h_{21} & h_{22} & h_{23}\\h_{31} & h_{32} & h_{33}
        \end{bmatrix}
        \begin{bmatrix}
        x\\y\\1
        \end{bmatrix}
    \end{equation*}
It is generally normalized with $h_{33}=1$
\end{frame}

\begin{frame}{Results--\small{Homography}}
    \begin{figure}[ht]
        \centering
        \includegraphics[scale=0.45]{img/ch2/homography_perspective_correction.jpg}
    \end{figure}
    \pause
    In our case, the ratio between the length and the width of the roll container is known. Once the projective transformation matrix in images has been obtained, a bilinear interpolation with the projective transformation matrix into the raw images will be performed.

    Recall: 
        \begin{equation*}
            q(x,y) = \frac{q_r}{\theta_{we}(x_r,y_r)-\theta_e}(\theta_{we}(x,y)-\theta_e)
        \end{equation*}
\end{frame}


\begin{frame}{Results--\small{Heat flux map}}
    \begin{figure}
        \vspace*{-5pt}
        \centering
        \includegraphics[scale=0.25]{img/ch2/Q_front_m.jpg}
        \hspace{5pt}
        \includegraphics[scale=0.25]{img/ch2/Q_back_m.jpg}
        \hspace*{5pt}
        \includegraphics[scale=0.25]{img/ch2/Q_left_m.jpg}
        \hspace{5pt}
        \includegraphics[scale=0.25]{img/ch2/Q_right_m.jpg}
        \includegraphics[scale=0.25]{img/ch2/Q_top.jpg}
        \vspace*{-5pt}
        \caption{\footnotesize{The heat flux map of the roll container (front, rear, left, right and top surfaces)}}
        % \label{Q_box}
    \end{figure}
\end{frame}


\begin{frame}{Results--\small{Vehicle results}}
    \begin{figure}
        \vspace*{-5pt}
        \centering
        \includegraphics[scale=0.25]{img/ch2/IR_truck_lt_m}
        \hspace{3pt}
        \includegraphics[scale=0.25]{img/ch2/IR_truck_rt_m}
        % \vspace{3pt}
        \includegraphics[scale=0.25]{img/ch2/IR_truck_bk11_m}
        \hspace{6pt}
        \includegraphics[scale=0.25]{img/ch2/IR_truck_bk12_m}
        \includegraphics[scale=0.25]{img/ch2/IR_truck_tp_m}
        \vspace*{-5pt}
        \caption{\footnotesize{The temperature map of the refrigerated vehicle (left, right, rear1, rear2 and top surfaces)}}
    \end{figure}
\end{frame}


\begin{frame}{Discussion}
    \begin{itemize}[<+->]
    \pause
    \large
        \item Thermal resistance model works well
        \item Homography application offers good results in roll-container
        \item Large size of vehicle panel 
        \item Thermal reflection in vehicle results
    \end{itemize}
\end{frame}
