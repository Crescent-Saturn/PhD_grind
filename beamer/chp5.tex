%!TEX root = soutenance_lei_2018.tex
\subsection{Liquid Nitrogen Cooling in IR Thermography applied to steel specimen}
\frame{\tableofcontents[currentsection,currentsubsection]}

\begin{frame}{Outline}
    \begin{columns}
      \column{0.5\textwidth}
           \begin{itemize}
                \item Stimulation Techniques:
                \begin{itemize}
                    \item Pulse Thermography
                    \item Lock-in Thermography
                    \item Liquid Nitrogen cooling
                \end{itemize}
            \end{itemize}
      \column{0.5\textwidth}
      \pause
      \centering
      \includegraphics[scale=0.26]{img/ch5/Outline_tech.jpg}
    \end{columns}
\end{frame}

\begin{frame}{Outline}
    \begin{columns}
        \column{0.5\textwidth}
            \begin{itemize}
                \item Processing Methods:
                \begin{itemize}
                    \item Principal Component Thermography (PCT)
                    \item FFT Amplitude \& Phase
                    \item Receiver Operating Characteristic (ROC) curve analysis
                \end{itemize}  
            \end{itemize}
        \column{0.5\textwidth}
        \pause
      \centering
      \includegraphics[scale=0.28]{img/ch5/Outline_analy.jpg}
    \end{columns}
\end{frame}


\begin{frame}{Specimen dimensions}
    Steel with Flat Bottom holes:
    \begin{figure}
        \centering
        \includegraphics[scale=0.3]{img/ch5/specimen_schema.pdf}
    \end{figure}


\end{frame} 


\begin{frame}{Experimental Set-up}
    Hot Stimulation:
    \begin{figure}
        \centering
        \subfloat[Flash stimulation]{
            \includegraphics[scale=0.33]{img/ch5/ch5_flash_setup.png}
        }
        \subfloat[Lock-in Stimulation]{
            \includegraphics[scale=0.33]{img/ch5/ch5_lockin_setup.png}
        }
       \caption{Experimental set-up in the \textit{reflection} mode}
    \end{figure}

\end{frame}


\begin{frame}{Experimental Set-up}
    LN$_2$ Cooling:
    \begin{figure}
        \centering
        \includegraphics[scale=0.5]{img/ch5/ch5_ln2_setup.png}
    \end{figure}

\end{frame}


% \begin{frame}{Experimental Set-up}
%     Equipments' detail:
%     \begin{itemize}
%         \item Infrared Camera FLIR SC3000 (spatial resolution equal to 320$\times$240 pixels, frame rate up to 50Hz, GaAs sensor, spectral range 8-9 $\mu m$)
%         \item Two pairs of flash lamps for a total of 10 kJ (electric) released in 5 $ms $ 
%         \item One pair of modulated halogen lamps with 1kW each served as Lock-in    stimulation   
%         \item An isolated bottle (500 ml) filled with  Liquid Nitrogen.
% \end{itemize}
% \end{frame}


% \begin{frame}{Results--\small{Thermal raw images:}}
%     % \small{Thermal raw images:}
%     \begin{figure}
%      % \centering
%      \vspace{-18pt}
%        \subfloat[Flash raw frame 23]
%        {
%           \includegraphics[scale=0.35]{img/ch5/flash_raw23_2.png}
%        }
%        % \hspace{10pt}
%        \subfloat[Flash raw frame 60]
%        {
%           \includegraphics[scale=0.35]{img/ch5/flash_raw60_2.png}
%        }
%        % \hspace{10pt}
%        \\
%        \vspace*{-10pt}
%        \subfloat[LN$_2$ raw frame 41]
%        {
%           \includegraphics[scale=0.35]{img/ch5/cool_raw_2.png}
%        }
%        \caption{Thermal raw images of PT and LN$_2$ stimulation techniques}      
%     \end{figure}
% \end{frame}


\begin{frame}{Results--\small{Lock-in FFT images}}
    \begin{figure}
        \vspace*{-18pt}
        \hspace*{-15pt}
        \subfloat[LIT4 FFT in amplitude]
        {
          \includegraphics[scale=0.31]{img/ch5/LIT4_AMP.png}
        }
        \subfloat[LIT8 FFT in amplitude]
        {
          \includegraphics[scale=0.31]{img/ch5/LIT8_AMP.png}
        }
        \subfloat[LIT16 FFT in amplitude]
        {
          \includegraphics[scale=0.31]{img/ch5/LIT16_AMP.png}
        }        
        \\
        \hspace*{-15pt}
        \subfloat[LIT4 FFT in phase]
        {
          \includegraphics[scale=0.31]{img/ch5/LIT4_PHA.png}
        }
        \subfloat[LIT8 FFT in phase]
        {
          \includegraphics[scale=0.31]{img/ch5/LIT8_PHA.png}
        }
        \subfloat[LIT16 FFT in phase]
        {
          \includegraphics[scale=0.31]{img/ch5/LIT16_PHA.png}
        }

        \caption{FFT in amplitude and phase results for LIT}      
    \end{figure}

\end{frame}

\begin{frame}{Results--\small{PCT results}}
    \begin{figure}
        % \hspace*{-10pt}
        \subfloat[Flash PCT 2nd image]
        {
          \includegraphics[scale=0.315]{img/ch5/Flash_PCT_2.png}
        }
        \subfloat[LN$_2$ PCT 2nd Image]
        {
          \includegraphics[scale=0.315]{img/ch5/Cool_PCT_2.png}
        }   
        \\
        \vspace*{-6pt}
        \hspace*{-15pt}
        \subfloat[LIT4 PCT 3rd Image]
        {
          \includegraphics[scale=0.315]{img/ch5/LIT4_PCT_3.png}
        }
        \subfloat[LIT8 PCT 3rd Image]
        {
          \includegraphics[scale=0.315]{img/ch5/LIT8_PCT_3.png}
        }
        \subfloat[LIT16 PCT 3rd Image]
        {
          \includegraphics[scale=0.315]{img/ch5/LIT16_PCT_3.png}
        }
        \caption{PCT results of corresponding technique}      
    \end{figure}

\end{frame}

\begin{frame}{Results--\small{ROC curves}}
    \begin{figure}[ht]
       \centering
       \subfloat[Binary map of defects]
       {
          \includegraphics[scale=0.195]{img/ch5/Schema_done.png}
       }
       \subfloat[Cool map]
       {
          \includegraphics[scale=0.8]{img/ch5/Cool_ROC.png}
       }   
       \caption{One example of ROC analysis (LN$_2$ results) and binary map of defect locations}
    \end{figure}
\end{frame}


\begin{frame}{Results--\small{ROC curves}}
    \begin{figure}
        \centering
            \includegraphics[scale=0.35]{img/ch5/ROC_PCT_2017.pdf}
        \\
            \includegraphics[scale=0.35]{img/ch5/ROC_LIT_AMP_2017.pdf}
            \includegraphics[scale=0.35]{img/ch5/ROC_LIT_PHA_2017.pdf}
        \caption{ROC curves comparison}
    \end{figure}

\end{frame}


\begin{frame}{Results--\small{AUC values}}
Area under curve analysis and comparison:
    \begin{figure}[ht]
        \vspace*{-5pt}
        \centering
        \includegraphics[scale=0.32]{img/ch5/AUC_Value.png}
        \caption{AUC value comparison}
    \end{figure}
\end{frame}


\begin{frame}{Discussion}
    \begin{itemize}[<+->]
    \pause
        \item All techniques highlight part of the flaws in the sample
        \item PCT post-processing method displays a better results for all procedures
        \item More defects are exhibited in Flash stimulation with PCT processing
        \item ROC curve analysis and its AUC analysis have elucidated a  straightforward classification comparison
        \item The best values are obtained with the Flash technique with PCT processing, trailed narrowly by the Liquid Nitrogen method.
    \end{itemize}
\end{frame}
