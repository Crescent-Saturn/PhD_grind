%!TEX root = soutenance_lei_2018.tex

\begin{frame}{Cold chain concept}
    \begin{figure}[ht]
        \centering
        \includegraphics[scale=0.5]{img/ch1/AbyssFig.pdf}    
    \end{figure}
    \begin{itemize}
    \item A cold chain is a temperature-controlled and monitored supply chain
    \item The goal is to keep a sample or material within a certain temperature range during all stages of delivery, processing and storage.
    \end{itemize}
\end{frame}


\begin{frame}{Cold chain concept}
An unexpected winner at Russian World Cup 2018? \\
\pause  
\centering
\alert{Chinese crayfish!}
\vspace*{-6pt}
    % \pause
    \begin{figure}[ht]
        \centering
        \includegraphics[width=150pt,height=100pt]{img/ch1/crayfish_eat.jpg}
        \pause
        \includegraphics[width=150pt,height=100pt]{img/ch1/crayfish_ln2.jpg}
        \pause
        \includegraphics[width=150pt,height=100pt]{img/ch1/crayfish_pkg.jpg}
        \pause
        \includegraphics[width=150pt,height=100pt]{img/ch1/crayfish_train.jpg}
    \end{figure}

\end{frame}


\begin{frame}{ATP}
    ATP: Agreement on the Transport of Perishable foodstuffs
    \begin{figure}
        \centering
        \includegraphics[scale=0.16]{img/ch1/Technology_FreezerTruck.png}
    \end{figure}
    \small{the correct transport of perishable foodstuffs in refrigerated vehicles is a critical link in the food chain not only in terms of maintaining the temperature integrity of the transported products but also its impact on energy consumption and CO$_2$ emissions. }
\end{frame}


\begin{frame}{ATP}
\small{Transport temperature requirements of food products (\textit{Tassou et al. 2009}):}
    \begin{figure}
        \centering
        \includegraphics[scale=0.40]{img/ch1/temp_reqs.png}
        % \caption{Transport temperature requirements of food products (\textit{Tassou et al. 2009})}
    \end{figure}
    \scriptsize{The ATP classifies insulated vehicles and bodies as either Normally Insulated Equipment (\textbf{IN}, isotherme normal: K coefficient equal or less than $ 0.7\; W/(m^2 K) $) or Heavily Insulated Equipment (\textbf{IR}, isotherme renforcé: K coefficient equal or less than $ 0.4\; W/(m^2 K )$. (\textit{Tassou et al. 2009})}
\end{frame}


\begin{frame}{Infrared Thermography}
    \footnotesize{Infrared Thermography: the science of measuring and mapping surface temperatures}
    % \pause
        \begin{figure}
            \includegraphics[scale=0.36]{img/ch1/Infrared_Th.png}
        \end{figure}
        \scriptsize{(image from: \hyperlink{Engineering World}{http://engineering-world.blogspot.ca/2011/07/tcr-arabia-wins-more-thermography.html})}
\end{frame}



\begin{frame}{Infrared thermography in ATP}
\small
ATP: a robust approach ==> \alert{global} result of K-value\\
\pause
What about local thermal brides, lack of insulation, anomalous aging zones, air leakage, etc.?\\
\pause
Infrared thermography ==> \alert{local} approach for identification
\pause
    \begin{figure}
        % \centering
        \includegraphics[scale=0.48]{img/ch1/Estrada-Flores1996use_1.jpg}
        \includegraphics[scale=0.48]{img/ch1/Estrada-Flores1996use_2.jpg}
        \caption{Side view and rear doors of an insulated van (\textit{Estrada-Flores et al. 1996})}
    \end{figure}
\end{frame}


\begin{frame}{Objectives}
    \begin{itemize}
        \item \textbf{Objective 1$ ^\circ $ } - Development of quantitative methods for identifying thermal insulation anomalies in refrigerated vehicles for the transport of food.
        \item \textbf{Objective 2$ ^\circ $ } - Development of an aging model of panels constituting the refrigerated ``box" of the vehicle in order to predict its objective remaining useful lives. 
        \item \textbf{Objective 3$ ^\circ $ } - Development of a model using a cold stimulation device to identify thermal insulation anomalies in insulated materials.
    \end{itemize}
\end{frame}
