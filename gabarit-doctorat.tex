%% GABARIT POUR THÈSE STANDARD
%%
%% Consulter la documentation de la classe ulthese pour une
%% description détaillée de la classe, de ce gabarit et des options
%% disponibles.
%%
%% [Ne pas hésiter à supprimer les commentaires après les avoir lus.]
%%
%% Déclaration de la classe avec le type de grade
%%   [l'un de LLD, DMus, DPsy, DThP, PhD]
%% et les langues les plus courantes. Le français sera la langue par
%% défaut du document.
\documentclass[PhD, nonatbib, french,english, 12pt]{ulthese}
  %% Encodage utilisé pour les caractères accentués dans les fichiers
  %% source du document. Les gabarits sont encodés en UTF-8. Inutile
  %% avec XeLaTeX, qui gère Unicode nativement.
  \ifxetex\else \usepackage[utf8]{inputenc} \fi

  %% Charger ici les autres paquetages nécessaires pour le document.
  %% Quelques exemples; décommenter au besoin.
  \usepackage{amsmath}       % recommandé pour les mathématiques
  \usepackage{ncccomma}      % gestion de la virgule dans les nombres
  \usepackage{pdfpages}
  \usepackage[caption=false]{subfig}
  \usepackage{nomencl}
  \makenomenclature
  
  \usepackage{chapterbib}
  \usepackage[sectionbib, authoryear,square]{natbib}
  
  \renewcommand{\nomname}{List of Acronyms}
  \renewcommand{\bibname}{Reference}  

%  \usepackage{paralist}  % Paralist items
  %% Utilisation d'une autre police de caractères pour le document.
  %% - Sous LaTeX
  %\usepackage{mathpazo}      % texte et mathématiques en Palatino
  %\usepackage{mathptmx}      % texte et mathématiques en Times
  %% - Sous XeLaTeX
  %\setmainfont{TeX Gyre Pagella}      % texte en Pagella (Palatino)
  %\setmathfont{TeX Gyre Pagella Math} % mathématiques en Pagella (Palatino)
  %\setmainfont{TeX Gyre Termes}       % texte en Termes (Times)
  %\setmathfont{TeX Gyre Termes Math}  % mathématiques en Termes (Times)

\graphicspath{{./graph/}}               % Images folders
\DeclareGraphicsExtensions{.jpg,.png,.pdf,.eps}

  %% Gestion des hyperliens dans le document. S'assurer que hyperref
  %% est le dernier paquetage chargé.
  \usepackage{hyperref}
  \hypersetup{colorlinks,allcolors=ULlinkcolor}

  %% Options de mise en forme du mode français de babel. Consulter la
  %% documentation du paquetage babel pour les options disponibles.
  %% Désactiver (effacer ou mettre en commentaire) si l'option
  %% 'nobabel' est spécifiée au chargement de la classe.
  \frenchbsetup{%
    StandardItemizeEnv=true,       % format standard des listes
    ThinSpaceInFrenchNumbers=true, % espace fine dans les nombres
    og=«, fg=»                     % caractères « et » sont les guillemets
  }

%% Suppression du numéro de section de la bibliographie. Utilisation
%% de \extrasfrench parce que c'est la dernière langue déclarée dans
%% \documentclass, ci-dessus.
\addto\extrasenglish{%
 \renewcommand{\bibsection}{\section*{\bibname}\prebibhook}}

  %% Style de la bibliographie.
%  \setcitestyle{numbers, square}

%  \bibliographystyle{plainnat}

  %% Déclarations des pages de titre. Remplacer les éléments entre < >.
  %% Supprimer les caractères < >. Couper un long titre ou un long
  %% sous-titre manuellement avec \\.
%  \titre{Application of infrared thermography in maintaining the ``cold chain" \&\& Exploration of cold approach in infrared thermography for Non-Destructive Testing}
  \titre{Cold food chain: Infrared thermography applied to the evaluation of insulation anomalies in refrigerated vehicles for the transport of food \& Exploration of cold approach in infrared thermography for Non-Destructive Testing}
  % \titre{Ceci est un exemple de long titre \\
  %   avec saut de ligne manuel}
  % \soustitre{Sous-titre le cas échéant}
  % \soustitre{Ceci est un exemple de long sous-titre \\
  %   avec saut de ligne manuel}
  \auteur{Lei Lei}
  \annee{2018}
  \programme{Doctorat en génie électrique}
  \direction{Xavier P.V. Maldague, directeur de recherche}
  % \codirection{<Prénom Nom>, <codirecteur ou codirectrice> de recherche}
  % \codirection{<Prénom Nom>, <codirecteur ou codirectrice> de recherche \\
  %              <Prénom Nom>, <codirecteur ou codirectrice> de recherche}

\begin{document}

\frontmatter                    % pages liminaires

\pagestitre                     % production des pages de titre

%!TEX root = gabarit-doctorat.tex
\chapter*{Résumé}                      % ne pas numéroter
\phantomsection\addcontentsline{toc}{chapter}{Résumé} % inclure dans TdM

\begin{otherlanguage*}{french}
  Le coût croissant de l'énergie a fait de l'économie d'énergie une nécessité vitale dans le monde actuel. Un des exemples consiste à ``maintenir la chaîne du froid", c'est-à-dire le transport correct des aliments périssables dans les véhicules réfrigérés, en particulier pour les produits laitiers, la viande et les aliments congelés. Tout en conservant une conservation appropriée des denrées alimentaires, l'ATP (Agreement on Transport of Perishable Foodstuffs) est l'un des accords concernant les essais d'isolation thermique qui déterminent l'adéquation du transport.
  
  Le test standard ATP est une procédure pour mesurer l'état isolant des équipements avec une approche globale. Néanmoins, certains défauts locaux dans la structure de l'équipement ne peuvent pas être visualisés dans cette procédure. Dans ce contexte, la technique de thermographie pourrait être particulièrement utile à ces problèmes. Deux exemples de cette application sont présentés dans cette thèse, l'un d'eux se concentre sur la cartographie du flux de chaleur sur la surface externe d'un rouleau-conteneur isolé par la technique de thermographie infrarouge. La seconde tente d'établir une vue panoramique du flux de chaleur sur la surface interne d'un véhicule isolé.
  
  Encouragé par les résultats favorables précédents, une exploration de l'approche à froid dans la thermographie infrarouge pour les Tests Non-Destructifs et l'Évaluation est introduite et réalisée dans ce qui suit. Une approche se concentre sur la détection des défauts isolés et des ponts thermiques dans les panneaux de caisses de camions isolés par chauffage à lampe et refroidissement par air, deux moyens d'excitation opposés. L'autre examine un refroidissement à l'azote liquide appliqué à un échantillon d'acier avec des trous à fond plat de différentes profondeurs et tailles.
  
  Différentes méthodes de traitement des données et de modélisation et de simulation sont effectuées dans des chapitres connexes.
\end{otherlanguage*}
                % résumé français
\chapter*{Abstract}                      % ne pas numéroter
\phantomsection\addcontentsline{toc}{chapter}{Abstract} % inclure dans TdM

\begin{otherlanguage*}{english}
   The increasing cost of energy has made energy saving a vital necessity in the current world. One of the examples involves, ``Maintaining the cold chain", which is the correct transport of perishable foodstuffs in refrigerated vehicles, especially for dairy products, meat and frozen foods.  In this respect a suitable thermal insulation implemented in refrigerated vehicles is essential for saving energy while maintaining an appropriate conservation of the foodstuffs. ATP is one of the agreements concerning thermal insulation tests ensuing the suitability of the transport.
   
   The ATP standard test is a procedure to measure the insulating status of equipment with a global approach. Nonetheless, some local defects in the structure of equipment cannot be visualized in this procedure. The thermography technique could be particularly helpful for these issues. Two examples of this application are presented in this thesis, one focuses on mapping the heat flux on the external surface of an insulated roll-container by infrared thermography technique. The second one attempts to establish a panoramic view of the heat flux on the internal surface of an insulated vehicle. 
   
   Encouraged by previous favorable results, an exploration of the cold approach in infrared thermography for Non-Destructive Testing \& Evaluation is introduced and performed herein. One approach focuses on the detection of insulated flaws and thermal bridges in insulated truck box panels by lamp heating and air cooling, two opposite means of excitation. The other approach investigates the application of  liquid nitrogen cooling to a steel specimen with flat-bottom holes of different depths and sizes.
   
   Different data processing methods and modeling and simulation are also carried out.% in related chapters. 
\end{otherlanguage*}
              % résumé anglais
\cleardoublepage

\tableofcontents                % production de la TdM
\cleardoublepage

\listoftables                   % production de la liste des tableaux
\cleardoublepage

\listoffigures                  % production de la liste des figures
\cleardoublepage

\nomenclature{CFT}{Continuous Fourier Transform}
\nomenclature{PCs}{Principal Components}
\nomenclature{PCT}{Principal Component Thermography}
\nomenclature{PPT}{Pulsed Phase Thermography}
\nomenclature{PT}{Pulsed Thermography}
\nomenclature{PCT}{Principal Component Thermography}
\nomenclature{PCT}{Principal Component Thermography}
\nomenclature{PCT}{Principal Component Thermography}

\printnomenclature
\phantomsection\addcontentsline{toc}{chapter}{List of Acronyms} 
\cleardoublepage

\dedicace{For Steven \& Alice}
\cleardoublepage

\epigraphe{A minute's success pays the failure of years.}{Robert Browning}
\cleardoublepage

\chapter*{Acknowledgements}         % ne pas numéroter
\phantomsection\addcontentsline{toc}{chapter}{Acknowledgements} % inclure dans TdM

I would like to first thank 
         % remerciements
\chapter*{Preface}         % ne pas numéroter
\phantomsection\addcontentsline{toc}{chapter}{Preface} % inclure dans TdM

This thesis is submitted to the ``Faculté des études supérieures de l'Université Laval" to obtain the degree of Philosophiae Doctor of Science (Ph.D.). The current thesis is composed of six chapters. The first chapter is a literature review of maintaining the ``cold chain" in industry field, as well as cold approaches that are used nowadays in infrared thermography for Non-Destructive Testing \& Evaluation. This chapter concludes with the issue, hypothesis and objectives. In addition, several common data analyzing and processing methods, including modeling and simulation applied in whole research studies, are also presented and highlighted. Chapters two and three are presented in form of papers and describe the application of infrared thermogaphy in the ``cold chain" and corresponding discussion. Chapters four and five are composed of manuscripts and describe the exploration of cold approaches in infrared thermography including several experimental results and discussions. Finally, in order to provide a closure on the results obtained in the current thesis, chapter six presents a general conclusions and proposed perspectives of the performed studies.

The first manuscript is entitled ``Mapping of the heat flux of an insulated small container by infrared thermography" and was presented in the 24th IIR International Congress of Refrigeration. Authors: Paolo Bison, Alessandro Bortolin, Gianluca Cadelano, Giovanni Ferrarini, Lei Lei, Xavier Maldague, Stefano Rossi.

The second paper is entitled ``Panoramic View of the Heat Flux Inside an Insulated Vehicle by Infrared Thermography" and was first presented at an oral session of the 1st QIRT Asia Conference 2015 Mamallapuram, India. Then it was selected for publication in the Quantitative InfraRed Thermography Journal. Authors: Lei Lei, Alessandro Bortolin, Gianluca Cadelano, Giovanni Ferrarini, Stefano Rossi, Paolo Bison, Xavier Maldague. 

Then, the third article is entitled ``Detection of insulation flaws and thermal bridges in insulated truck box panels" and was first presented at an oral session of the 13th Quantitative InfraRed Thermography Conference 2016 at Gdańsk University of Technology in Poland. Then this paper was invited for publication in the Quantitative InfraRed Thermography Journal. Authors: Lei Lei, Alessandro Bortolin, Paolo Bison, Xavier Maldague.

The last manuscript is entitled ``Liquid nitrogen cooling in IR thermography applied to steel specimen" and it was accepted for publication in the Proceedings Volume 10214 of Thermosense: Thermal Infrared Applications XXXIX; 102140T (2017). Authors: Lei Lei, Giovanni Ferrarini, Alessandro Bortolin, Gianluca Cadelano, Paolo Bison, Xavier Maldague.

The main objectives of the present thesis are first to deploy the infrared thermography technique in the procedure of maintaining the ``cold food chain'', especially in insulated vehicles of ATP standards. The benefits of time-saving and the accuracy of the results in the determination of K-value could be applied for assessment at a commercial level. Then based on the previous favorable results, another objective is to explore cold approaches (such as compressed air, liquid nitrogen, etc.) in infrared thermography for Non-Destructive Testing \& Evaluation. The first part of this research was a cooperative project supported by the governments of Italy and Quebec (Ministère des Relations internationales et de la Francophonie) through the Joint Subcommittee Québec-Italy. Our collaborator, the Construction Technologies Institute of the Italian National Research Council (ITC-CNR), has made a significant contribution to this research. Three main experiments were performed at ITC-CNR, and Dr. Paolo Bison oversaw all steps of the work there.

For the exploration of cold approaches, the research supervisor, prof. Xavier Maldague collaborated in the set up and work planning of experiments, equipment and participated in the interpretation of the results and the revision of the manuscripts.

\textbf{TO BE COMPLETED}


%Finally, the candidate has been invited to present the results of this thesis in a good management practices guide for pigs from farm to slaughter. This guide will be developed for use by the pork meat industry and the Canadian hog producers.
           % avant-propos

\mainmatter                     % corps du document

\chapter*{Introduction}         % ne pas numéroter
\phantomsection\addcontentsline{toc}{chapter}{Introduction} % inclure dans TdM

Une thèse ou un mémoire devrait normalement débuter par une
introduction. Celle-ci est traitée comme un chapitre normal, sauf
qu'elle n'est pas numérotée.
          % introduction
\chapter{State of the art}     % numéroté
\section*{Maintain the ``cold chain"}
\phantomsection\addcontentsline{toc}{section}{Maintain the ``cold chain"}


\section*{Infrared thermography for NDT \& E}
\phantomsection\addcontentsline{toc}{section}{Infrared thermography for NDT \& E} % inclure dans TdM

In the past few decades, a Nondestructive Testing \& Evaluation (NDT\&E) technique: Infrared Thermography (IT), also commonly referred as \textit{Thermal imaging}, or \textit{thermography}, has received growing attention and applications for diagnostics. This is due to its characteristics that ``it allows the mapping of thermal patterns, i.e. thermograms, on the surface of objects, bodies, or systems through the use of an infrared imaging instrument, such as an infrared camera" \citep{maldague3introduction}. Besides, its impressive quickness and convenience, the inspection attracted a wide variety of applications including biological, civil engineering, aerospace, cultural heritage, etc \citep{2007-Ibarra-Castanedo,2000-Li,cielo1987thermographie,shoja2011inspection,pradere2009microscale,avdelidis2004applications,maierhofer2005quantitative}. Other utilization such evaluation or measurements of heat transfer coefficients (or absorption) can be found in \citep{dragano2009experimental,grinzato2010r,grinzatoquality,grinzato1comparison,rossi2009k}.

Infrared thermography is often divided into two approaches: passive thermography, in which materials and structures are naturally at different temperature than the environment; and active thermography, in which an external simulation is added to induce a thermal response \citep{Maldague2001theory}. The two approaches are described in next two sections.
\section{Passive thermography}
This approach is due to the principle of energy conservation and states that an important quantity of heat is released by any process consuming energy because of the law of entropy, which is known as \textit{the First law of thermodynamics} \citep{thdy1}. 

Therefore, the passive thermography is often qualitative, such as the diagnosis of the presence of a given abnormality or hot spot with respect to the immediate surroundings.  A delta-T of a few degrees (> 5°C) is generally
found suspicious while greater values indicate strong evidences of abnormal behaviors. Some applications in civil engineering can be found in \citep{2000-Li,stanley1994non,lo2004building}, where the thermography approach was proved to be reliable in detecting the debonded ceramic tiles on a building finish, and a temperature contours over the surface of a target object to provide an appropriate measure of the damaged building or structure has been mapped. The inspection of buildings, for different purposes such as energy monitoring, moisture mapping and many others, has also been widely deployed \citep{laranjeirapassive,bison1993automatic,bison2012geometrical}. 

\section{Active thermography}
Contrary to ``\textit{Passive}", the word ``\textit{Active}" seems more progressive. Thus might be the reason why this technique finds a large number of applications in NDT. The experimental configuration for active infrared thermography can be seen in Fig \ref{exp_active} \citep{sfarra2010comparative}. 
\begin{figure}[!htbp]
	\centering
	\includegraphics[scale=0.4]{art/exp_active}
	\caption{Experimental setup for the active thermography}
	\label{exp_active}
\end{figure}
where \textcircled{1} are two different stimulation sources. They can be located in the same side of the camera, which is known as \textit{reflection} mode; or in the opposite side of the camera, which is known as \textit{transmission} mode.  \textcircled{2} is the specimen under test, \textcircled{3} is the IR camera for recording, and \textcircled{4} is the PC for processing. The whole system is connected by a synchronization control.

Practically, \textit{any} form of energy can be used to produce a measurable thermal contrast. In addition, since the external stimulation can be accurately controlled, quantitative procedure is possible. 

\subsection*{Excitation methods}
Generally, energy used as the external resources can be delivered by the following mechanisms:
\begin{itemize}
	\item Conduction: heating blanket, hot bag or cool bags (as snow or ice);
	\item Convection: Hot (or cold) water (or gas(air));
	\item Radiation: Lamps, flashes, infrared heaters;
	\item Mechanical stimulation: ultrasonic vibration;
\end{itemize}
Even though ``any" form of energy can be used, the heating sources are often preferred. They can be commonly divided as {ibarra2013infrared}:
\begin{itemize}
	\item optical: Photographic flashes or lamps are utilized, known as \textit{Pulsed Thermography} (sometimes even using laser as heating resource \citep{suzuki2002application,burrows2007combined}). When using periodic heating at a given frequency to measure the amplitude and (or) phase delay  of the thermal response, that is known as \textit{Lock-in thermography} \citep{wu1998lock,duan2013quantitative,2007-Ibarra-Castanedo};
	\item mechanical: Sound or ultrasound waves are injected to the specimen to produce heat by friction, known as \textit{Vibrothermography} \citep{2007-ClementeIbarra-Castanedo,2007-Ibarra-Castanedo};
	\item induction: Eddy currents are generated by a coil inside the specimen \citep{riegert2004lockin,zenzinger2007thermographic}. This type of heating source is limited to conductive materials;
	\item microwave: Heat is introduced into the specimen by a time-gated microwave source \citep{myers1979microwave,land1987clinical};
\end{itemize}
Since the main technique in this work is using the excitation method, \textit{Pulsed Thermography} will be presented in details in the following sections \citep{Maldague2001theory,ibarra2013infrared}.
\subsection{Pulsed Thermography}
%In this technique, the specimen surface is submitted to a heat pulse generated by a high power and fast heat source such as flashes. Then the thermal front propagates under the surface by diffusion. Then as time passes, the surface temperature decreases uniformly for a piece without any defects. While on the contrary,  the diffusion rate can be changed and abnormal temperature patterns will be produced at the subsurface, by any kind of surface discontinuities (such as air leakage, delaminations, thermal bridges, porosity, inclusions, etc.). The detection of these discontinuities  with an IR camera depends on their sizes.

Pulsed thermography (PT) is one of the most popular thermal stimulation method in IR thermography. One reason for this popularity is the quickness of the inspection relying on a thermal stimulation pulse, with duration going from a few ms for high thermal conductivity material inspection (such as metal parts) to a few seconds for low thermal conductivity specimens (such as plastics, graphite epoxy components \citep{Maldague1993Nondestructive,Maldague1994bInfra}). Such quick thermal stimulation allows direct deployment on the plant floor with convenient heating sources. Moreover, the brief heating prevents damage to the component (heating is generally limited to a few degrees above the initial component temperature).

Basically, PT consists of briefly heating the specimen and then recording its temperature decay curve. Qualitatively, the phenomenon is as follows. The temperature of the material first rises during the pulse. After the pulse, it then decays because the energy - the thermal front - propagates by diffusion under the surface. Later, the presence of a subsurface defect (example: a disbonding) reduces the diffusion rate so that when observing the surface temperature, such a subsurface defect appears as an area of higher temperature with respect to the surrounding sound area. In fact in such a case the reduced diffusion rate caused by the subsurface defect presence translates into “heat accumulation” and hence higher surface temperature just over the defect. Moreover, such phenomenon occurs in time so that, deeper defects are observed later and with a reduced “diluted” or “spread” thermal contrast.

Theoretically, the 1D solution of the \textit{Fourier} equation for the propagation of a \textit{Dirac} heat pulse through a semi-infinite homogeneous material is given by \citep{carslaw1986heat}:
\begin{equation}
T(z,t) = T_0 + \frac{Q}{\sqrt{k\rho C_p \pi t}}exp(-\frac{z^2}{4\alpha t})
\end{equation}
where $Q$ is the energy absorbed, $T_0$  is the initial temperature, $k$ the conductivity of the material, $C_p$ the heat capacity at constant pressure and $\alpha$ thermal diffusivity.

Thus, at the surface ($z=0$), one has:
\begin{equation}
T(0,t) = T_0 + \frac{Q}{\sqrt{k\rho C_p \pi t}}=T_0 + \frac{Q}{e\sqrt{\pi t}}
\label{PT_eq}
\end{equation}
where $e=\sqrt{k\rho C_p}$ is defined as the thermal effusivity, which measures the material ability to exchange heat with its surroundings. Therefore, surface temperature will decay as a function of $t^{1/2}$.

\noindent The energy source can be applied in various ways:
\begin{description}
	\item \textbf{Point inspection}: heating with a laser or a focused light beam; advantages: repeatable heating, uniformity; drawback: the necessity to move the inspection head to fully inspect a surface slows down the inspection process.
	\item \textbf{Line inspection}: heating using line lamps, heated wire, scanning laser, line of air jets (cool or hot); advantages: fast inspection rate (up to 1 $m^2/s$) and good uniformity thanks to the lateral motion; drawback: only part of the temperature history curve is available due to the lateral motion of the specimen and the fixed distance between thermal stimulation and temperature signal pick-up. Projection of a series of line heating strips is also used to detect surface cracks.
	\item \textbf{Surface inspection}: heating using lamps, flash lamps, scanning laser; advantages: the complete analysis of the phenomenon is possible since the whole temperature history curve is recorded; drawback: concerns about non-uniformity of the heating (lamps, flashes, heat gun, laser, microwave).
\end{description}

If the temperature of the part to inspect is already higher than ambient temperature, it can be of interest to make use of a cold thermal source such as a line of air jets (or water jets; sudden contact with ice, snow, etc.). In fact, a thermal front propagates the same way whether being hot or cold: what is important is the temperature differential between the thermal source and the specimen. An advantage of a cold thermal source is that it does not induce spurious thermal reflections into the IR camera as in the case of a hot thermal source. The main limitations of cold stimulation sources are related to practical considerations as for instance it is generally easier and more efficient, to heat rather then to cool a part. For this reason, our work on cold stimulation needs to be investigated in detail and better understood.

\subsection{Step Heating}
When the specimen is  continuously heated, the increase of surface temperature is monitored during the application of a stepped heating pulse, thus the principle of \textit{Step Heating}. Same as in pulsed thermography, variations of surface temperature with time are related to specimen features. So it can be called \textit{Long pulse}. Besides, it is also referred as  time-resolved  infrared radiometry (TRIR) \citep{spicer1992time}.  More details can be found in \citep{ibarra2013infrared,osiander1998thermal}

\subsection{Lock-in Thermography}
Lock-in thermography (LT) is based on thermal waves generated inside the inspected specimen and detected remotely. Wave generation is for instance performed by periodically depositing heat on the specimen surface (e.g. through sine-modulated lamp heating) while the resulting oscillating temperature field in the stationary regime is remotely recorded through its thermal infrared emission \citep{wu1998lock}.

The lock-in terminology refers to the necessity to monitor the exact time dependence between the output signal and the reference input signal (i.e. the modulated heating). This is done with a locking amplifier in a point by point laser heating or by computer in full-field (lamp) deployment so that both phase and magnitude images becomes available. Phase images are related to the propagation time and since they are relatively insensitive to local optical surface features (such as non uniform heating), they are interesting for NDE purposes. The depth range of images is inversely proportional to the modulation frequency so that higher modulation frequencies restrict the analysis in a near surface region \citep{Maldague2001theory}.

\subsection{Vibrothermography}
Known as \textit{Ultrasound thermography} \citep{dillenz2001progress}, vibrothemography (VT) is an active Infrared Thermography technique based on that principle: under the effect of mechanical vibrations (0 to 25 kHz) induced externally to the structure, thanks to direct conversion from mechanical to thermal energy, heat is released by friction precisely at locations where defects such as cracks and delaminations are located.

Ultrasonic waves are ideal for NDT in the sense that, defect detection is independent from of its orientation inside the specimen, and both internal and open surface defects can be detected. Thus, VT is very useful for the detection of cracks and delaminations. The range for ultrasonic waves is often between 20 kHz and 1 MHz. Unlike electromagnetic waves, mechanical elastic waves such as sonic and ultrasonic waves can not propagate in a vacuum. They require a medium for traveling. They travel faster in solids and liquids than the air. This indicates an important aspect of VT: the common approach in VT is to use a coupling media between a transducer and the specimen to reduce losses. Therefore, one of the Infrared thermography characteristic ``contactless" is meaningless in this technique. However, image acquisition can still be carried out from a distance using an IR camera.

\section{Advantages and limitations of infrared thermography}
As every coin has two sides, all technique has its strengths and weaknesses. For Infrared thermography, its advantages are as follows \citep{maldague3introduction, Maldague2001theory}:
\begin{itemize}
	\item Fast, surface inspection
	\item Ease of deployment
	\item Contactles, no coupling needed as in the case of conventional ultrasounds.
	\item Security, there's no damaging radiation.
	\item Easy access to results thanks to the imaging capabilities.
	\item Numerous applications.
	\item Unique inspection tool in some tasks.
\end{itemize}
While, on the other hand, some limitations specific are evident:% \citep{maldague3introduction, Maldague2001theory}:
\begin{itemize}
	\item Non-uniform heating, when over a large surface.
	\item Emissivity differs from materials.
	\item Defects detected are generally shallow.
	\item Thermal losses, absorption by the environment.
	\item Inspected thickness of material under the surface has a limitation.
	\item Cost of the apparatus.
\end{itemize}

\section{Recent research on cold approach}
The first mention of this technique was found in \citep{Maldague1993Nondestructive,Maldague1994bInfra}, where a filament-wound graphite-epoxy tank was tested by TNDT using a liquid nitrogen spray to identify defects and bonded Al structures were inspected  in a thermographical \textit{reflection} procedure by propagation of a cold front (cool air jet) Shown in Fig \ref{Al_stru_cold}. 
\begin{figure}[!htbp]
	\centering
	\includegraphics[scale=0.81]{art/Al_stru_cold}
	\caption{Thermographic inspection by propagation of a cold front for the inspection of bonded Al structures.}
	\label{Al_stru_cold}
\end{figure}
In the latter's thermal image, the structure is initially at a temperature of some 10°C above room temperature and a line
of air jets is used to quickly cool the inspected area. The image shows the hotter central area which corresponds to the bonded area (oriented vertically on the thermogram). The cooler region at the center of the bonding line reveals a bonding defect (lack of adhesive).
Another example of stimulation by propagation of a cool front in the same mode on an Al-foam panel was also presented in \citep{Maldague1993Nondestructive}(Fig \ref{Al_foam_hc}).
\begin{figure}[!htbp]
	\centering
	\includegraphics[scale=0.90]{art/Al_foam_hc}
	\caption{Thermographic inspection using a line of air jets with pre-heating then cooling.}
	\label{Al_foam_hc}
\end{figure}
In this case, it is necessary to detect an non-bonded area in a Al-foam laminate. The panel is first uniformly heated at a temperature of about 10$°C$ above room temperature. The surface is then cooled by a line of cool air jets during the lateral moving of the panel (at a constant speed of 2.4 $cm/s$). In the figure, two thermograms are shown, one for a sound area and one for an area where a circular-shaped disbonding (4 cm diameter) is present. The defect is clearly visible at the image
center. 
Besides, after comparing  the radiative-heat injection and convection-heat removal approach in \citep{Maldague1993Nondestructive}. They found that, a reduced thermal contrast is obtained with the cool air stimulation due to both the reduced heat capacity of air and the smaller temperature differential between the thermal perturbing source and the inspected surface (case of air versus high temperature radiative source). 

Other applications have then been made during the past decades in several domains. \citep{endohdynamical2012} proposed constructing an active thermographic imaging system, in which a cooling material contacts the surface and scans over a welded zone of two stainless plates (Fig \ref{endoh_fig}).
\begin{figure}[!htbp]
	\centering
	\includegraphics[scale=1]{art/endoh_axe}\\
	\includegraphics[scale=1]{art/endoh_lat}
	\caption{ Experimental setup (a) Axial movement, (b) lateral movement of a coolant material.}
	\label{endoh_fig}
\end{figure}
They recorded both a time-varying thermographic image and a real-time response of each pixel. The results are exhibited in Fig \ref{endoh_res_1}, \ref{endoh_res_2}.
\begin{figure}[!htbp]
	\centering
	\includegraphics[scale=0.9]{art/endoh_res_1}
	\caption{(a)Thermal image, (b) temperature profile along A-A’ line, (c) temperature profile along B-B’ line (v=10$mm/s$).}
	\label{endoh_res_1}
\end{figure}
\begin{figure}[!htbp]
	\centering
	\includegraphics[scale=1.0]{art/endoh_res_2}
	\caption{(a)Thermal image, (b) temperature profile along A-A’ line (v=50$ mm/s$).}
	\label{endoh_res_2}
\end{figure}

In both two cases of scanning (v= 10 and v=50 $mm/s$), it is observed that the locally high-temperature appears in low-temperature region of the simulated welding region. For axial movement of coolant material, the heat inside and the surface of the specimen transfers toward the interface surface of the specimen to the coolant, due to conservation of the axial symmetry of heat transfer. However, for lateral movement of coolant, it was squeezed for the spatial distribution of the temperature along the direction of coolant movement. Therefore, an elliptical shape of temperature distribution was obtained.

In \citep{2012-LewisHom} a custom, microfluidic  heat  sink  with  an  IR-transparent working  fluid  (0.75  LPM)  was  manufactured  to  cool  an instrumented test chip while permitting optical access for IR thermal imaging.  Then a detailed system calibration was conducted to account for the temperature-dependent optical properties of the chip and heat sink. Their experiments  confirm  that  the  dynamic  range  of  the infrared  microscope  can  be  greatly  enhanced  by decreasing the  thickness  of  the  liquid  layer. \citep{rodriguez2014cooling} deployed an thermographic test to the detection of subsurface cracks in welding. The procedure started with the thermal excitation of the material, following with the monitoring of the cooling process with infraRed thermography. They used the natural convection between the material and the environment, but two method of heating were deployed in to same specimen. One is heating from the front surface (Method 1), the other is heating from back surface (Method 2). 

\begin{figure}
	\hspace{-45pt}
%	\centering
	\includegraphics[scale=0.65]{art/cooling_res}
	\caption{Thermal images of each defect for two methods Natural convection cooling}
	\label{cooling_res}
\end{figure}

The experimental results (Fig \ref{cooling_res}) have shown that the technique has a potential to detect hidden flaws due to the influence of the physical differences between the defect and the non-defect area in the cooling parameters. But the results depended on the depth of defect: ``Greater depths require larger heating and greater ranges of precision in thermographic data acquisitions". %Still, the limitation of that proposed method in this literature is only by natural convection. %Our work will apply the forced convection by spraying the nitrogen gas.

\section{Issue, hypothesis and objectives}
\subsection{Issue}

\subsection{Hypothesis}

\subsection{Objectives}




\section{Methodology}
Thought infrared thermography has the advantages such as fast inspection, ease of deployment, contactless, security and easy access to results with the imaging capabilities, raw infrared thermography results is difficult to handle and analyze in case of reflections and non-uniform external stimulation. To improve the inspection results, there are various post-processing techniques which have been developed. The recently popular data analyzing and processing methods are presented in detail in the following subsections.
\subsection{Fourier Transform (FT)}
Among all data processing methods, Fourier Transform (FT) is especial because it helps retrieve phase and amplitude data from raw results, since our principal results are images, which can be seen as signals with two dimensions.

It is well-known that any wave form, periodic or not, can be approximated by the sum of purely harmonic waves oscillating at different frequencies. The Continuous Fourier Transform (CFT) is then given by:
\begin{equation}
F(\omega) = \int_{-\infty}^{\infty}f(t)e^{-j\omega t}dt = A(\omega)e^{i\phi(\omega)}
\end{equation}
where $\omega = 2 \pi f$. the FT technique serves to transform the perception of signal from a time-based domain to a frequency-based domain.

In case of discrete situation, the Discrete Fourier Transform (DFT) is possible to analyze the data in the frequency domain:
\begin{equation}
F_n = \Delta t \sum_{k=0}^{N-1}T(k\Delta t)e^{-\tfrac{i2\pi nk}{N}} = \Re(F_n) + i\Im(F_n)
\label{DFT}
\end{equation}
where $n$ designates the frequency increment ($n=0, 1, ..., N$), $\Delta t$ is the sampling interval, $N$ is the total number of infrared images, and $\Re$ and $\Im$ are the real and the imaginary parts of the transform, respectively.
DFT is often applied in Pulsed Phase Thermography (PPT), which analyzes phase data obtained from PT results. In addition, the Fast Fourier Transform (FFT) algorithm is often applied to reduce computation time.

\subsection{PPT}
From Eq.\ref{DFT}, amplitude $A_n$ and phase delay $\Phi_n$ are given by:
\begin{equation}
A_n = \sqrt{\Re(F_n)^2 + \Im(F_n)^2} \qquad \Phi_n = \tan^{-1}\frac{\Re(F_n)}{\Im(F_n)}
\end{equation}
It should be noted here that The processed sequence is less affected than the original data by undesired noise
sources such as environmental reflections, emissivity variations, non-uniform heating.

By selecting two pixels, the first in correspondence of a reference zone, the second in correspondence of a possible defect zone, and following both of them in time, after the pulse, the two profiles shown in Fig. \ref{T_profile_PPT} are obtained. By taking the FFT of the two signals, the two profiles of amplitude (Fig. \ref{FT_AM_profile_PPT}) and phase (Fig. \ref{FT_PH_profile_PPT})
as a function of frequency are obtained.
\begin{figure}[!h]
	\centering
	\includegraphics[scale=0.35]{art/T_profile_PPT}
	\caption{Temperature profiles of a reference zone (blue) and a defect zone (red)}
	\label{T_profile_PPT}
\end{figure}

\begin{figure}[!h]
	\centering
	\includegraphics[scale=0.35]{art/FT_AM_T_profile_PPT}
	\caption{Amplitude as a function of frequency: blue reference, red defect}
	\label{FT_AM_profile_PPT}
\end{figure}

\begin{figure}[!h]
	\centering
	\includegraphics[scale=0.35]{art/FT_PH_T_profile_PPT}
	\caption{Phase as a function of frequency: blue reference, red defect}
	\label{FT_PH_profile_PPT}
\end{figure}

\subsection{Differential Absolute Contrast (DAC)}
Traditionally, once the temperature of a sound area (reference zone)$T_s(t) $ and that of a defect zone $T_{def}(t) $ are known, contrast methods can be applied by simply:
\begin{equation}
C_{ac}(t) = T_{def}(t) - T_s(t)
\label{AC_eq}
\end{equation}
which is known as the absolute contrast. However, this method becomes inconvenient when the sound area cannot be pratical defined. In addition, the common case of non-uniform heating has a strong effect on the results.

To improve this, the Differential Absolute Contrast (DAC) has proven its amelioration for non-uniform heating situations\citep{Benitez2008, pilla2002new}.

DAC method starts from Eq. \ref{PT_eq}, the surface temperature increase based on one-dimensional model of the Fourier equation after an instantaneous Dirac heating pulse is applied:
\begin{equation}
\Delta T = T(0,t) - T_0  = \frac{Q}{e\sqrt{\pi t}}
\label{PT_eq_2}
\end{equation}
The temperature of the sound area at the surface ($z=0$) $T_s$ at time $t_1$ is given by:
\begin{equation}
\Delta T_s(t_1) = \frac{Q}{e\sqrt{\pi t_1}}
\end{equation}
Then at time $t$, the temperature can be written as:
\begin{equation}
\Delta T_s(t) = \frac{Q}{e\sqrt{\pi t_1}} = \sqrt{\frac{t_1}{t}}\cdot \Delta T(t_1)
\end{equation}
where $t_1$ is a time between the thermal pulse lasting and the time at which the first temperature spot of the subsurface defects appear.\\
The absolute temperature contrast (Eq. \ref{AC_eq}) can be rewritten as:
\begin{align}
C_{ac}(t) = & [T_{def}(t) -T_0] - [T_s(t) - T_0] \\ 
		  = & \Delta T_{def}(t) - \sqrt{\frac{t_1}{t}}\cdot \Delta T(t_1)
\end{align}
DAC Applications on different type of material samples can be found in \citep{pilla2002new}, which has proven this method has an effective improvement on the signal to noise ratio.

\subsection{Temperature Signal Reconstruction (TSR)}
The thermographic signal reconstruction (TSR) data processing technique is one of the most recent improvements which raise thermography to the level of the most established NDE techniques \citep{Balageas2015}.

Same as DAC method, Eq. \ref{PT_eq_2} can be rewritten  in logarithmic way as:
\begin{equation}
\log (\Delta T) = \log (\frac{Q}{e}) - \frac{1}{2}\log (\pi t)
\end{equation}
In a general way, as: %with degree $n$:
\begin{equation}
\log (\Delta T) = a_0 + a_1\log (t) + a_2[\log (t)]^2 +...+ [a_n\log(t)]^n
\end{equation}
This technique usually consists in the fitting of the experimental log-log plot by a polynomial of degree $n$, and also the computation of the 1$^{st}$ and 2$^{nd}$ derivatives of the thermograms.



\subsection{Principal Component Thermography (PCT)}

\subsection{Receiver Operating 	Characteristic (ROC)}
The Receiver operating characteristic (ROC) curve is a technique in statistics which helps visualize, organize and select classifiers based on their performance. The curve graph is created by plotting the the true positive rate (TPR) against the false positive rate (FPR) at various threshold settings. More details about concepts and definitions can be found in \citep{Fawcett2006}. While being frequently chosen a standard method in several scientific fields, ROC are rarely applied in the thermographic field.\citep{Bison2014a} 

In the ROC curves definitions, the \textit{sensitivity} is the True positive rate (\textit{tp rate}), and \textit{1-specificity} is the False positive rate (\textit{fp rate}).


\section{Modeling and simulation}
In this research, all modelling and simulation work are undertaken in the platform COMSOL Multiphysics$^{\textregistered}$, which might be helpful when comparing with the experimental results.
%\section{Introduction to COMSOL Multiphysics}

COMSOL Multiphysics is a general-purpose software platform, based on advanced numerical methods, for modeling and simulating physics-based problems. It is a finite element analysis, solver and Simulation software / FEA Software package for various physics and engineering applications, especially coupled phenomena, or multiphysics.

The advantages of using COMSOL Multiphysics for modeling:
\begin{itemize}
	\item COMSOL Multiphysics has an integrated modeling environment.
	\item COMSOL Multiphysics takes a semi-analytic approach: once questions specified, COMSOL symbolically assembles finite-element method matrices and organizes the bookkeeping.
	\item COMSOL Multiphysics is fully compatible with MATLAB, so user could define programming for the modeling,organizing the computation, or the post-processing has full functionality. COMSOL Script is a MATLAB-like integrated programming environment that can also provide these facilities.
	\item COMSOL Multiphysics provides pre-built templates as Application Modes and in the Model Library for common modeling applications.
	\item COMSOL Multiphysics provides multiphysics modeling--linking well known ``application modes" transparently.
	\item COMSOL Multiphysics innovated extended multiphysics--coupling between logically distinct domains and models that permits simultaneous solution.
\end{itemize}

All the simulation parameters and conditions can be found in corresponding chapters.             % chapitre 1

\chapter*{Maintain the ``cold chain"}     % numéroté
\phantomsection\addcontentsline{toc}{chapter}{Maintain the ``cold chain"} % inclure dans TdM
The following two chapters will present two published paper concerning the application of infrared thermography for Non-Destructive Testing \& Evaluation applied in ``Maintain the cold chain" procedure.

The first study mainly focused on mapping the heat flux on the external surface of an insulated roll-container by Infrared thermography technique. The ATP standard measurement was performed to obtain the experimental results, meanwhile IR images of roll-container have been taken when the steady condition arrived, in order to analyze and compute the corresponding heat flux on entire surfaces. A simple thermal resistance model has been applied to realize the computation. Final temperature figures showed a good uniform distribution, and several defects in the structure like thermal bridges or air leakages have been identified. A reference zone of the external wall is measured by a thermal flux meter, then with that reference the whole surface heat flux map have been figured out. Besides, for a better view of the heat flux map, the homography technique has been performed into the raw images by applying a bilinear interpolation with the projective transformation matrix. The final corrected heat flux map has been demonstrated for each surface, in which the right one showed a smaller value than the others. 

In the second research, a panoramic view of the heat flux on the internal surface of an insulated vehicle by infrared thermography technique has been established in this study. The ATP measurement was performed to obtain the experimental results.  An IR camera was mounted on a pan-tilt head and automatically driven by a suitable software to map the temperature of the inner walls at the steady condition, in order to analyze and compute the corresponding heat flux on the entire surface.  The final result of the K-value obtained by IR thermography is accurate enough and compares well with that of the ATP test.

\chapter{Mapping of the heat flux of an insulated small container by infrared thermography}
\section{Introduction}
Nowadays, the public is increasingly aware to the need of applying rigorous standards all along the ``food chain'', considering a better life with a good food supply.  Thus the transport of food in the refrigerated vehicles (such as trucks, trailers, containers, etc.), especially for dairy products, meat and frozen foods, is of great interest. Moreover, the ever increasing cost of energy incites limiting the minimum refrigeration. So it is essential to ensure a perfect thermal insulation at the vehicles inspection. 

There exits some agreements of the thermal insulation tests which ensures the suitability for the transport of food in refrigerated conditions---ATP.
``The Agreement on the International Carriage of Perishable Foodstuffs and on the Special Equipment to be Used for such Carriage (ATP) done at Geneva on 1 September 1970 entered into force on 21 November 1976'' \citep{Geneva1970}, which establishes standards for the international transport of perishable food between the states that ratify the treaty. It has been updated through amendment a number of times and as of 2013 has 48 state parties, most of which are in Europe or Central Asia. It is open to ratification by states that are members of the United Nations Economic Commission for Europe (UNECE) and states that otherwise participate in UNECE activities\citep{ATP_wiki}.
The details contents of the agreements can be found in \citep{ATP_wiki, rossi2009k}, therefore no more description in this report will be presented.

The institute (CNR-ITC) has extensive experience in the measurement of heat transfer coefficient $K$ applied to refrigerated vehicles \citep{rossi2009k,bison1993automatic,bison2012geometrical,dragano2009experimental,grinzato2010r}. It is also responsible for the certification of these vehicles throughout Italy.

In summary, the major methods and procedures for measuring and checking the insulation of the equipment during the ATP test, is following \citep{rossi2009k}: 
\begin{itemize}
	\item \textbf{insulated equipment} built with insulated envelope such that the heat exchanged between inside and outside is limited in such a way that the overall coefficient of heat transfer ($K$-value) is assignable into 2 classes: a)equal to or less than 0.7 $W/(m^2 K)$ for normally insulated equipment; b) equal to or less than 0.4 $W/(m^2 K)$ for heavily insulated equipment; 
	
	\item\textbf{refrigerated equipment} that are insulated equipment which utilize some source of cold like ice, eutectic plates, dry ice etc. This equipment, with an outside temperature of $30^{\circ}$C must be able to lower the inside temperature to: + $7^{\circ}$C (class A); $-10^{\circ}$C (class B); $-20^{\circ}$C (class C); $0^{\circ}$C (class D); 
	
	\item \textbf{mechanically refrigerated equipment} that  are  insulated  equipment  furnished  of  its  own  refrigerating appliance.  The  appliance,  with  an  outside  temperature  of  $30^{\circ}$C,  must  be  capable  of  lowering  the  inside temperature to: from $+12^{\circ}$C to $0^{\circ}$C (class A); from $+ 12^{\circ}$C to $-10^{\circ}$C (class B); from $+ 12^{\circ}$C to $- 20^{\circ}$C (class C); 
	
	\item \textbf{heated  equipment} that  can  heat  the  inside  (to  avoid  the  freezing  of  foodstuffs)  are  used  in  very  cold countries.
\end{itemize}
Especially, the overall coefficient of heat transfer ($K$) is defined as:
\begin{equation}
K = \frac{W}{S\cdot \Delta \theta}
\end{equation}
where W is  the  power  necessary  to  maintain  a  steady  temperature  difference $\Delta \theta$ between  the  mean internal and external air temperature of the equipment. $S$ is the mean surface of the equipment, given by the geometric mean of the inside and outside surface areas:
\begin{equation}
S = \sqrt{S_i \cdot S_e}
\end{equation}

The ATP standard test is a procedure to measure the insulating status of equipments with a global approach. Its robustness has been well demonstrated. While, on the other hand, some local defects in the structure of equipment, such as thermal bridges, air leakages or zones of anomalous aging, cannot be visualized in this procedure. Then the thermography technique could be particularly helpful to these issues. In fact, all the defects mentioned above lead to a variation of the heat flux and temperature on the surface of the equipment \citep{grinzatoquality,grinzato1comparison}. Therefore the local heat flux map of the equipment by Infrared thermography could give a straightforward visualization of the structure, and also a local evaluation of the K-value.

The work in this internship consists the modeling and simulation for an insulated roll container by COMSOL in theory part. In practice party, the experimentation will be performed to map the heat flux on its external surface by Infrared thermography. Besides, another test for a refrigerated truck will also be done. These primary results would be obtained and discussed. Some conclusions and perspectives come in the end.


\section{Theory \& Methods}
\subsection{The heat transfer model}
Analogy to an electrical circuit, the heat transfer can be modelled as the heat flux is represented by current, temperatures are represented by voltages \citep{Therm_Re}. Therefore, the resistors in the heat ``circuit" is then the thermal resistance. Symbolically Ohm’s law can be expressed as
\begin{equation}
I = \frac{\Delta V}{R_e}
\end{equation}
where $I$ is the current flowing through an element, $\Delta V$ is the voltage across the element, and $R_e$ is the electrical resistance across the element. With the observed analogy, Fourier’s law can be written similarly as
\begin{equation}
q = \frac{\Delta T}{R_t}
\end{equation}
where $q$ is the flux of heat conduction, $\Delta T$ is the temperature difference between the surfaces of a slab, and $R_t$ is the thermal resistance.

In this work, the thermal resistance model is applied to a roll-container, from inside to outside (More details about the container can be found in Chapter \ref{box_detail}). For the standard ATP requirement, a radiation heater is working inside the container to maintain a fixed a higher internal air temperature. After the steady conditions are reached, a heat power $W$ is delivered. Heat flow is transferred by convection from the hot inside air to the internal wall of the box, and then by conduction through internal wall to external wall, and again by convection from the external wall to the outside air, which is cooled by the ATP system. The total scheme is presented in Fig \ref{Therm_Res}
\begin{figure}[!htpb]
	\centering
	\includegraphics{mapping/Therm_Res}
	\caption{Overall thermal resistance of the roll-container}
	\label{Therm_Res}
\end{figure}

\noindent where $\theta_i$ and $\theta_e$ are the internal and external temperature of the container, respectively. $\theta_{wi}$ and $\theta_{we}$ are the internal and external wall temperature of the container. In essentially 1D hypothesis, $h_i$ and $h_e$ are respectively the convective heat exchange internal and external coefficients. $\lambda$ is the thermal conductivity of the container wall and $l$ its thickness, and $S$ is the mean surface of the box.


\subsection{Simulation by Comsol}

A simple simulation work for this model has been performed by COMSOL MultiPhysics, which helps to better understand the distribution of the temperature of final result. The measurements of box dimension can be found in Tab \ref{tab_box_dim}.

\begin{table}[h]
	\centering
	%\begin{tabular}{p{85pt}p{85pt}p{85pt}p{85pt}}
	\begin{tabular}{c|c|c|c||c}
		\hline
		Inside:  & $L_i=0.864$ m  & $D_i=0.613$ m  & $H_i=1.55$ m & Thickness\\
		\hline
		Outside:  & $L_o=0.988$ m & $D_o=0.74$ m & $H_o=1.67$ m  &  $50$ mm \\
		\hline
	\end{tabular}
	\caption{Roll container dimensions}
	\label{tab_box_dim}
\end{table}

Same as the ATP standard test, the box inside air temperature is set as $32.5^{\circ}$C, and the outside air temperature is kept as $7.2^{\circ}$C. 

The Heat Transfer in Solid was used during the simulation in this modelling. A heat flux is added inside the box between the air and the internal surface, with a free convection, then the coefficient is set as $h_i=15 W/(m^2 K)$. Another heat flux was added outside between the air and the external surface with a forced convection, by $h_e=25 W/(m^2 K)$ [referring \citep{airhe,htwiki}]. The material of the box is made by high density polyurethane foam, with a conductivity about 0.0026 $W/(mK)$ \citep{jarfelt2006thermal}. Two high conductive thin layer are added into the material, same as the experimental one. The study condition is set to the steady status, as in real test the whole period is more than 12 hours.

The simulation result is shown in Fig. \ref{3D_T}. 
\begin{figure}[!htbp]
	\centering
	\includegraphics[scale=0.65]{mapping/3D_T}
	\caption{Temperature distribution of the roll container in COMSOL}
	\label{3D_T}
\end{figure}

The distribution of temperature of the entire container presented above indicates an uniform result, since the model and simulation is in ideal condition. Neither air leakage no thermal bridge is found here.

The following two figures (\ref{cut_plane}) introduce two cut plane inside the box, which illustrate the temperature details in the layers. As one can see that, the heat transfers from inside to outside, with temperature decreasing from internal surface to external surface. Moreover, the high thickness in the corners lead to a less transferred heat, showing a lower corresponding temperature.
\begin{figure} [htbp]
	%	\centering
	\hspace{-20pt}
	\includegraphics[scale=0.40]{mapping/xy_plane}
	\includegraphics[scale=0.40]{mapping/2D_xy_plane}
	\vspace{5pt}
	\hspace{-20pt}
	\includegraphics[scale=0.40]{mapping/yz_plane}
	\includegraphics[scale=0.40]{mapping/2D_yz_plane}
	\caption{Temperature distribution of the cute planes}
	\label{cut_plane}
\end{figure}

\section{Experimental setup}
\subsection{Roll-container}
In this part the experimental installation will be presented. The aim of this work is to map the heat flux on the external surface of a roll-container, and Fig \ref{box} shows the installed probes on the surface of the roll container.

\begin{figure}[!htbp]
	\centering
	\includegraphics[scale=0.13]{mapping/DSC_0191}
	\includegraphics[scale=0.13]{mapping/DSC_0189}\\
	\vspace{4pt}
	\includegraphics[scale=0.13]{mapping/DC_51939}
	\includegraphics[scale=0.13]{mapping/DC_51941}
	\caption{The roll container used for the test[outside and inside(up-right)]}
	\label{box}
\end{figure}

This box is actually made by sandwich panels with three layers: two internal-external skins made by polyester-fiberglass and a core made by high density polyurethane foam, and total thickness is 50 $mm$. \label{box_detail}

12 points of measurements (thermal couple) are positioned at the 8 corners (inside and outside) and also at the center of 4 surfaces (front, left,back and right)[Fig \ref{therm_couple}]. All the thermal couples are at a distance of 10 cm from the wall.
\begin{figure}[!htbp]
	\centering
	\includegraphics[scale=0.40]{mapping/therm_couple}
	\caption{The position of the thermal couples}
	\label{therm_couple}
\end{figure}
Besides, on the upper-center of the front surface, a heat flux meter is set to measure the corresponding heat flux (Fig \ref{box} left). The acquisition system used in this work is an infrared camera--FLIR SC-660. It records the thermography image series during the whole test for the front and left surfaces of container (schema shown in Fig \ref{box} left). What's more, when the steady conditions were reached (a state period of time not less than 12 hours, according to the ATP standard \citep{rossi2009k}), several infrared images were captured for the external walls such as front, left, back, right and top one (impossible to capture the bottom surface). 

With standard ATP measurement during the test, a radiation heater is heating the inside the container to maintain a temperature about $32.5^{\circ}$C. For outside, with the air circulating a velocity between 1 and 2 $m s^{-1}$ in the tunnel, the temperature is maintained constant at about $7.2^{\circ}$C. Thus makes a temperature difference between inside air and outside of $\Delta \theta = 25.3^{\circ}$C.

Two hypothesis are proposed in this test: 1) the  heat exchange coefficients, inside ($h_i$) and outside ($h_e$) of the container are constant;\label{hyp1} 2) the heat diffusion is mainly 1D. These hypothesis serve the computation of heat flux for each external surface in the following chapter.

\subsection{Refrigerated Vehicle}

Even though the main work is testing on the roll container, one has also applied the same experimental setup to a refrigerated truck (Fig \ref{truck}), which refers to the main objective of the ATP agreements.
\begin{figure}[!htbp]
	\hspace{-10mm}
	\includegraphics[scale=0.12]{mapping/DSC_0210}
	\includegraphics[scale=0.12]{mapping/DSC_0211}\\
	
	\includegraphics[scale=0.12]{mapping/DSC_0209}
	\includegraphics[scale=0.12]{mapping/DSC_0213}
	\caption{The refrigerated vehicle used for the test}
	\label{truck}
\end{figure}

For this test, only thermography images of several surfaces (top, left, back and right) have been captured when the steady condition was reached. Besides, the heat flux meter was first set on the left surface (Fig \ref{truck} upper-left) and then was moved to test the back surface (Fig \ref{truck} lower-right). The former result contains the influence of air streaming whose velocity is between 1 and 2 $ms^{-1}$, while in the later one, the ventilation system had been switched off.

\section{Results \& Discussion}
\subsection{Heat flux map}
The idea is to measure the heat flux by a thermal flux meter in a reference zone outside of the insulated envelope and to build successively the whole map of the heat flux as a linear relation of the temperature difference between the outside wall and the air.

The specific heat flux at a reference zone of the external wall is measured by a thermal flux meter, then with the wall temperature in the proximity of the flux meter and by measuring the air temperature, one can draw the local approach like (\citep{rossi2009k}):
\begin{equation}
q_r = \frac{\theta_{we}(x_r,y_r)-\theta_e}{1/h_e}
\end{equation}
which gives that:
\begin{equation}
h_e = \frac{q_r}{\theta_{we}(x_r,y_r)-\theta_e}
\end{equation}
where $q_r$ is the heat flux measured at the reference point ($x_r,y_r$) by the thermal flux meter. $\theta_{we}$ is the temperature measured by thermography at the reference point. With the hypothesis 1) mentioned in Section \ref{hyp1}, the heat flux map of the whole surface could be determined with the temperature map according to:
\begin{equation}
q(x,y) = \frac{q_r}{\theta_{we}(x_r,y_r)-\theta_e}(\theta_{we}(x,y)-\theta_e)
\label{eq_q}
\end{equation}
Where $\theta_{we}(x,y)$ is the temperature at each point of the external surface. This equation indicates that the  heat flux has a linear relation of the temperature difference between the outside wall and the air.

Finally the temperature of all the surfaces are distributed as Fig \ref{IR_box}:
\begin{figure}[!htbp]
	%	\hspace{-10mm}
	\centering
	\includegraphics[scale=0.50]{mapping/IR_front_m}
	\hspace{6pt}
	\includegraphics[scale=0.50]{mapping/IR_back_m}
	\vspace{3pt}
	\includegraphics[scale=0.50]{mapping/IR_left_m}
	\hspace{6pt}
	\includegraphics[scale=0.50]{mapping/IR_right_m}
	\includegraphics[scale=0.50]{mapping/IR_top}
	\caption{The temperature map of the roll container (front, back, left, right and top surfaces)}
	\label{IR_box}
\end{figure}
All these IR images were captured at steady condition. 
In the figure, a good uniform distribution of temperature was well shown, even though there are several abnormal lines which may be the thermal bridges or air leakages (at front, left, right and top surfaces). An attention might be paid is that there is an area of the light reflection on the container surface (on the second image left part). It could be no influence here as the back surface is the main part of the second figure.

Comparing all the surface temperature, one can see that on the right surface of container, the temperature value is a little lower than other surfaces. This result may conclude the Hypothesis 1) \ref{hyp1} is not fully correct, then the convective heat transfer coefficient could be different around the roll container during the test.

The temperature at the reference point around the thermal flux meter was presented in the figure (the first one), with that one could figure out the map of heat flux of the entire surface by Eq \ref{eq_q}. And the data measured by the thermal flux meter are presented in Fig \ref{flux_meter}.
\begin{figure}[!htbp]
%	\hspace{-35pt}
	\centering
	\includegraphics[scale=0.65]{mapping/It_project_2014_QProfile}
%	\vspace{20pt}
%	\hspace{-35pt}
	\includegraphics[scale=0.65]{mapping/It_project_2014_TProfile}
	\caption{Data from thermal flux meter (heat flux and temperature profiles)}
	\label{flux_meter}
\end{figure}
These data recorded the whole test. Thought the fluctuation is very high, ones took the steady period after 5 hours from the beginning. Then mean values are $q_r=9.73 W/(m^2 K)$ and $T_r = 7.83 ^\circ$C, which served the computation of heat flux.

Moreover, for a better view, some techniques are often applied for images corrections \citep{bison2012geometrical}. In this work, an easier way to realize the image correction is to apply the homography technique.

\subsection{Homography application}
In practical applications, such as image rectification, image registration, or computation of camera motion—rotation and translation—between two images, a homography is often performed in the field of computer vision \citep{homo_wiki}.

In this work, as one knows the radio between the length and the width of the roll container (or between the width and the height in same way), a projective transformation matrix in images has been obtained. Then applying a bilinear interpolation with the projective transformation matrix, into the raw images, one finally got the corrected heat flux mapping of each surfaces, demonstrated in Fig \ref{Q_box}.
\begin{figure}[!htbp]
	%	\hspace{-10mm}
	\centering
	\includegraphics[scale=0.55]{mapping/Q_front_m}
	\hspace{5pt}
	\includegraphics[scale=0.55]{mapping/Q_back_m}
	\includegraphics[scale=0.55]{mapping/Q_left_m}
	\hspace{5pt}
	\includegraphics[scale=0.55]{mapping/Q_right_m}
	\includegraphics[scale=0.55]{mapping/Q_top_m}
	\caption{The corrected heat flux map of the roll container (front,  back, left, right and top surfaces)}
	\label{Q_box}
\end{figure}

The Heat flux map exhibits that in right surface of the roll container, one gets smaller heat flux value rather than those on other surfaces, such as front, top, back and left.
On the other hand, the heat flux map of back, top and left surface show a little higher mean value than that of the front one, which is due to the air stream.

A contradiction is found that, normally the right surface's heat flux should be a bit greater than the front surface where the air stagnation is probable, as air stream flows the lateral surface, and that leads to a lower temperature in the right surface. While in the thermography images, one got the contrary result. The reason for this is due to the linear relationship between heat flux and the difference of temperature between external surface and the air. Since in our model Eq.\ref{eq_q}, lower $\theta_{we}(x,y)$ would lead to a lower $q(x,y)$. This might also indicate that the influence of the convective heat transfer in lateral surface is more important than imagining.
Comparing to the standard ATP measurement, the global K-value obtained is about $0.53 W/(m^2 K)$, then multiplied by the temperature difference $\Delta \theta =25.3^\circ$C, thus it gives us a global heat flux value of the roll container $13.41 W/m^2$. With thermography one obtained the local heat flux in the map about $10.7 W/m^2$, which is more or less a good result.

``A local variation of thermal conductivity leads to a local increase of the heat flux with a consequent variation of the local internal and external wall temperature"\citep{rossi2009k}, this phenomena can be well seen in all figures above.

\subsection{Vehicle results}
The temperature map of several surfaces of the refrigerated truck are distributed in Fig \ref{IR_truck}.
\begin{figure}[!htbp]
	%	\hspace{-10mm}
	\centering
	\includegraphics[scale=0.50]{mapping/IR_truck_lt_m}
	\hspace{6pt}
	\includegraphics[scale=0.50]{mapping/IR_truck_rt_m}
	\vspace{3pt}
	\includegraphics[scale=0.50]{mapping/IR_truck_bk11_m}
	\hspace{6pt}
	\includegraphics[scale=0.50]{mapping/IR_truck_bk12_m}
	\includegraphics[scale=0.50]{mapping/IR_truck_tp_m}
	\caption{The temperature map of the refrigerated vehicle (left, right, back1, back2 and top surfaces)}
	\label{IR_truck}
\end{figure}
From all the figures, one can see a good uniform of the temperature distribution on all the external surfaces of the vehicle. However, the reflection effects are very heavy on the left, right and top surfaces (which are not the abnormal lines in figures). Besides, in this test, lower temperature are found on the lateral surfaces than the back one, thanks to the air streaming.

%Two tests of the thermal flux have been performed: one on the left surface with air streaming; another on the back surface without ventilation system. The corresponding two final data profiles can be found in Fig \ref{truck_meter}.
%\begin{figure}[!htbp]
%	\centering
%	\includegraphics[scale=0.5]{mapping/Q_truck_left_profil}
%	\includegraphics[scale=0.5]{mapping/T_truck_left_profil}
%	\includegraphics[scale=0.6]{mapping/Q_truck_back_profil}
%	\includegraphics[scale=0.6]{mapping/T_truck_back_profil}
%	\caption{Data from thermal flux meter for the refrigerated vehicle (up:left surface; down:back surface)}
%	\label{truck_meter}
%\end{figure}
%The mean values of all these data in steady status are in following table (Tab \ref{tab_truck}).
%\begin{table}[h]
%	\centering
%	%\begin{tabular}{p{85pt}p{85pt}p{85pt}p{85pt}}
%	\begin{tabular}{l|r}
%		\hline
%		left surface  & back surface  \\
%		\hline
%		$\overline{q}=10.34 W/(m^2 K)$ & $\overline{q}=5.62 W/(m^2 K)$  \\
%		\hline
%		$\overline{T}=7.84^\circ$C & $\overline{T}=8.97^\circ$C  \\
%		\hline
%	\end{tabular}
%	\caption{Thermal flux meter data for truck}
%	\label{tab_truck}
%\end{table}
%
%The huge difference (45\% error) between the heat flux of the back and left surface makes no sense, since two test were almost at the same condition. Therefore, that might because when switching the thermal flux meter from left surface to back surface, something wrong has been done to influence the accuracy of the equipment, like not attaching well to the surface. Another consideration is that, for the back surface test, the measuring time was not long, as it has already arrived in the steady condition. While in the profiles, a little increasing tendency in temperature could be observed. This may indicate again that the vehicle result of back surface was not so good.

As the main work of the roll container has been done, more tests on the truck will be undertaken in future measurements.


\section{Conclusion \& Perspectives}

%\addcontentsline{toc}{chapter}{Conclusion \& Perspectives}
This preliminary work mainly focused on mapping the heat flux on the external surface of an insulated roll-container by Infrared thermography technique. The ATP standard measurement was performed to obtain the experimental results, meanwhile IR images of roll-container have been taken when the steady condition arrived, in order to analyze and compute the corresponding heat flux on entire surfaces. A simple thermal resistance model has been applied to realize the computation. Final temperature figures showed a good uniform distribution, and several defects in the structure like thermal bridges or air leakages have been identified. A reference zone of the external wall is measured by a thermal flux meter, then with that reference the whole surface heat flux map have been figured out. Besides, for a better view of the heat flux map, the homography technique has been performed into the raw images by applying a bilinear interpolation with the projective transformation matrix. The final corrected heat flux map has been demonstrated for each surface, in which the right one showed a smaller value than the others. Due to the air streaming, one got a bit smaller temperatures in the lateral surfaces than other surfaces, thus making the smaller heat flux values. That also indicated the convective heat transfer coefficient was not constant around the roll-container surfaces, contrary to our Hypothesis 1 in theory.

For the refrigerated vehicle test, two surfaces (left and back) have been taken into measurement with the thermal flux meter. The former one was tested with the air streaming, while the ventilation system was switched off for the later one. The final IR images presented a big reflection influence on the lateral surfaces. And a huge difference between the heat flux value on two test surfaces was found, which might be because the equipment was not attached well to the surface when moved. 

For future work, more tests will be taken place on refrigerated vehicle with two thermal flux meters measuring on the same time, to avoiding the problem encountered in this work. And the suppression of thermal reflections in thermal imaging \citep{vollmer2004identification} may be taken into consideration during the IR image processing. 

This project topic can be also extended to buildings where the monitoring of the effective transmittance is crucial for the energy saving \citep{grinzato2010r}.             % chapitre 2, etc.

%\includepdf[pages=-]{Prop_these_lei}

\chapter{Panoramic view of the heat flux inside the vehicle}     % numéroté
(Accepted for publication in the Quantitative InfraRed Thermography Journal, in February 2018).


The results of this study were firstly presented at an oral session of the 1st QIRT Asia Conference 2015 Mamallapuram, India. % Then it was selected for the publication in Quantitative InfraRed Thermography Journal.

\textbf{\texttt{Contributing authors:}}

\textbf{\textsf{Lei Lei}} (Ph.D candidate): developing protocol, experiment preparation, data analysis,  personnel coordination and manuscript preparation.

\textbf{Alessandro Bortolin} (Ph.D student of CNR-ITC): discussion and experiment preparation.

\textbf{Gianluca Cadelano} (Ph.D student of CNR-ITC): data collection, data analysis, discussion and experiment preparation.

\textbf{Giovanni Ferrarini} (Researcher of CNR-ITC): discussion in developing protocol, data collection, experiment preparation.

\textbf{Stefano Rossi} (Ph.D, researcher of CNR-ITC): experiment planning and preparation.

\textbf{Paolo Bison} (Research supervisor of CNR-ITC): student supervision, revision and correction of the manuscript. 

\textbf{Xavier Maldague} (Research director of LVSN in University Laval): student supervision, revision and correction of the manuscript.


\phantomsection\addcontentsline{lot}{table}{3.1\quad Insulated container dimensions}
\phantomsection\addcontentsline{lot}{table}{3.2\quad ATP test results}

\phantomsection\addcontentsline{lof}{table}{3.1\quad Overall thermal behavior of the container being tested represented as an analogy of an electrical circuit.}
\phantomsection\addcontentsline{lof}{table}{3.2\quad The inside of the insulated container used for the test.}
\phantomsection\addcontentsline{lof}{table}{3.3\quad Spherical projection.}
\phantomsection\addcontentsline{lof}{table}{3.4\quad Original IR image (left) and its spherical projection (right).}
\phantomsection\addcontentsline{lof}{table}{3.5\quad Harris corners detected}
\phantomsection\addcontentsline{lof}{table}{3.6\quad Estimation of translation between images.}
\phantomsection\addcontentsline{lof}{table}{3.7\quad Stitching of images.}
\phantomsection\addcontentsline{lof}{table}{3.8\quad Temperature panorama of the inside of the vehicle.}
\phantomsection\addcontentsline{lof}{table}{3.9\quad Heat Flux meter measurement.}
\phantomsection\addcontentsline{lof}{table}{3.10\enspace Heat Flux panorama of the inner part of the vehicle.}
\phantomsection\addcontentsline{lof}{table}{3.11\enspace Schema of calculating real size of each image (Left: Side view; Right: Plan view).}
\phantomsection\addcontentsline{lof}{table}{3.12\enspace One element surface map (up) and the area panorama of the inner part of the vehicle.}

\includepdf[pages=-]{Revised_Version_2}


\chapter*{Exploration of cold approach}     % numéroté
\phantomsection\addcontentsline{toc}{chapter}{Exploration of cold approach} % inclure dans TdM
The following two chapters will present two published paper concerning the exploration of cold approach in infrared thermography applied in Non-Destructive Testing \& Evaluation.

The first study concentrates on the detection of defects and thermal bridges in insulated truck box panels, by active infrared thermography. Comparison between heating and cooling approaches for experiments and models has been established. In addition, passive thermography detection in computational models has been presented. Results demonstrate that the compressed air spray is more rapid than the traditional heating method in providing successful detection. Even if the traditional heating approach provides clearer results, in reality it is not easy and practical to heat a whole truck box to conduct inspection: the compressed air spray approach is much more convenient.

The second research investigates an external stimulation–cooling instead of heating in IR Thermography for NDT \& E. A steel specimen is used to test three different stimulations for thermal images and also ROC analysis comparison. Results shows that all techniques highlight part of the flaws in the sample, whereas the LN$_2$ technique represents the defects only at the beginning; this maybe due to the high conductivity of steel. In thermal results, the PCT post-processing method displays a better results for all procedures. More defects are exhibited in Flash stimulation with PCT processing.  The results of this study were firstly presented at an oral session of SPIE Thermosense: Thermal Infrared Applications XXXIX 2017, in the United States.

\chapter{Detection of insulation flaws and thermal bridges in insulated truck box panels}
(Accepted for publication in the Quantitative InfraRed Thermography Journal, in May 2017).

%
The results of this study were firstly presented at an oral session of the 13th Quantitative InfraRed Thermography Conference 2016 at Gdańsk University of Technology in Poland . Then it was selected for the publication in Quantitative InfraRed Thermography Journal.

\section*{Résumé}
Cet article se concentre sur la détection des défauts et des ponts thermiques dans les panneaux de caisses de camions isolés, en utilisant la thermographie infrarouge. Contrairement à la méthode traditionnelle de thermographie passive, cette recherche utilise des méthodes de chauffage et de refroidissement dans des configurations de thermographie active. Le chauffage de la lampe est utilisé comme stimulation externe chaude, tandis qu'un jet d'air comprimé est appliqué comme stimulation externe froide. Une caméra thermique capture tout le processus. En outre, des simulations numériques sous la plate-forme COMSOL$^®$ sont également menées. Les résultats expérimentaux et de simulation pour deux situations sont comparés et discutés.

\section*{Abstract}
This paper focuses on the detection of defects and thermal bridges in insulated truck box panels, utilizing infrared thermography. Unlike the traditional way in which passive thermography is applied, this research uses both heating and cooling methods in active thermography configurations. Lamp heating is used as the hot external stimulation, while a compressed air jet is applied as the cold external stimulation. A thermal camera captures the whole process. In addition, numerical simulations under COMSOL$^®$ platform are also conducted. Experimental and simulation results for two situations are compared and discussed.

\textbf{\texttt{Contributing authors:}}

\textbf{\textsf{Lei Lei}} (Ph.D candidate): developing protocol, experiment preparation and planning, data collection, personnel coordination and manuscript preparation.

\textbf{Alessandro Bortolin} (Ph.D student of CNR-ITC): data analysis, discussion and manuscript preparation.

\textbf{Paolo Bison} (Research supervisor of CNR-ITC): student supervision, revision and correction of the manuscript. 

\textbf{Xavier Maldague} (Research director of LVSN in University Laval): student supervision, revision and correction of the manuscript. 

\phantomsection\addcontentsline{lot}{table}{4.1\quad Specimen specification details}
\phantomsection\addcontentsline{lot}{table}{4.2\quad Air-cooling parameters}
\phantomsection\addcontentsline{lot}{table}{4.3\quad Materials properties}
\phantomsection\addcontentsline{lot}{table}{4.4\quad Thermal contrast peak talbe}

\phantomsection\addcontentsline{lof}{table}{4.1\quad Details of defects inside the specimen(left) and final specimen to test(right).}
\phantomsection\addcontentsline{lof}{table}{4.2\quad Experimental set-up.}
\phantomsection\addcontentsline{lof}{table}{4.3\quad Simulation 3D models transparency view.}
\phantomsection\addcontentsline{lof}{table}{4.4\quad Experimental results.}
\phantomsection\addcontentsline{lof}{table}{4.5\quad Simulation results.}
\phantomsection\addcontentsline{lof}{table}{4.6\quad Experimental quantitative results of panel surface.}
\phantomsection\addcontentsline{lof}{table}{4.7\quad Computational quantitative results of panel surface.}
\phantomsection\addcontentsline{lof}{table}{4.8\quad Temperature contrast profiles of simulation models.}
\phantomsection\addcontentsline{lof}{table}{4.9\quad Temperature contrast profiles of Lamp heating models.}


\includepdf[pages={2-11}]{Lei2017Detection}

\chapter{Liquid Nitrogen Cooling in IR Thermography applied to steel specimen}
The results of this study were firstly presented at an oral session of SPIE Commercial+ Scientific Sensing and Imaging. International Society for Optics and Photonics, 2017.
Then it was accepted for publication in Proceedings Volume 10214, Thermosense: Thermal Infrared Applications XXXIX; 102140T (2017).


%
\section*{Résumé}
Thermographie pulsée (PT) est l'une des méthodes les plus courantes dans les procédures de thermographie active de la thermographie pour NDT \& E ​​(essais non destructifs  \& évaluation), en raison de la rapidité et la commodité de cette technique d'inspection. Des éclairs ou des lampes sont souvent utilisés pour chauffer les échantillons dans le PT traditionnel. Cet article explore principalement exactement la stimulation externe opposée en IR Thermographie: refroidissement au lieu de chauffage. Un échantillon d'acier avec des trous à fond plat de différentes profondeurs et tailles a été testé. De l'azote liquide (LN $_2$) est répandu sur la surface de l'échantillon et l'ensemble du processus est capturé par une caméra thermique. Pour obtenir une bonne comparaison, deux autres techniques classiques de CND, la thermographie pulsée et la thermographie verrouillée, sont également utilisées. En particulier, la méthode Lock-in est implémentée avec trois fréquences différentes. Dans la procédure de traitement d'image, la méthode de thermographie en composantes principales (PCT) a été effectuée sur toutes les images thermiques. Pour les résultats Lock-in, les images de phase et d'amplitude sont générées par la transformée de Fourier rapide (FFT). Les résultats montrent que toutes les techniques présentaient une partie des défauts tandis que la technique LN $_2$ affiche les défauts seulement au début du test. De plus, un poste-traitement de seuil binaire est appliqué aux images thermiques, et en comparant ces images à une carte binaire de l'emplacement des défauts, les courbes caractéristiques de fonctionnement du récepteur (ROC) correspondantes sont établies et discutées. Une comparaison des résultats indique que la meilleure courbe ROC est obtenue en utilisant la technique flash avec la méthode de traitement PCT.

%Cette recherche étudie une stimulation externe - refroidissement au lieu de chauffer en thermographie infrarouge pour NDT \& E. Un spécimen en acier est utilisé afin de  tester trois stimulations différentes sur les images thermiques et également une comparaison d'analyse ROC. Les résultats montrent que toutes les techniques mettent en évidence une partie des défauts de l'échantillon, alors que la technique LN $ _2 $ ne représente les défauts qu'au début; ceci peut être dû à la conductivité élevée de l'acier. Dans les résultats thermiques, la méthode de post-traitement PCT affiche de meilleurs résultats pour toutes les procédures. Plus de défauts sont exposés dans la stimulation Flash avec le traitement PCT.
% Les résultats de cette étude ont d'abord été présentés lors d'une session orale de SPIE Thermosense: Thermal Infrared Applications XXXIX 2017, aux Etats-Unis.

\section*{Abstract}
Pulsed Thermography (PT) is one of the most common methods in Active Thermography procedures of the Thermography for NDT \& E (Nondestructive Testing \& Evaluation), due to the rapidity and convenience of this inspection technique. Flashes or lamps are often used to heat the samples in the traditional PT. This paper mainly explores exactly the opposite external stimulation in IR Thermography: cooling instead of heating. A steel sample with flat-bottom holes of different depths and sizes has been tested. Liquid nitrogen (LN$_2$) is sprinkled on the surface of the specimen and the whole process is captured by a thermal camera. To obtain a good comparison, two other classic NDT techniques, Pulsed Thermography and Lock-In Thermography, are also employed. In particular, the  Lock-in  method  is  implemented  with  three  different  frequencies.  In  the  image  processing  procedure,  the Principal Component Thermography (PCT) method has been performed on all thermal images. For Lock-In results, both Phase and Amplitude images are generated by Fast Fourier Transform (FFT). Results show that all techniques presented part of the defects while the LN$_2$ technique displays the flaws only at the beginning of the test. Moreover, a binary threshold post-processing is applied to the thermal images, and by comparing these images to a binary map of the location of the defects, the corresponding Receiver Operating Characteristic (ROC) curves are established and discussed. A comparison of the results indicates that the better ROC curve is obtained using the Flash technique with PCT processing method.  

\newpage
\textbf{\texttt{Contributing authors:}}

\textbf{\textsf{Lei Lei}} (Ph.D candidate): developing protocol, experiment preparation and planning, data analysis,  personnel coordination and manuscript preparation.

\textbf{Giovanni Ferrarini} (Researcher of CNR-ITC): discussion in developing protocol, experiment preparation.

\textbf{Alessandro Bortolin} (Ph.D student of CNR-ITC): discussion and experiment preparation.

\textbf{Gianluca Cadelano} (Ph.D student of CNR-ITC): data collection, discussion and experiment preparation.

\textbf{Paolo Bison} (Research supervisor of CNR-ITC): student supervision, revision and correction of the manuscript. 

\textbf{Xavier Maldague} (Research director of LVSN in University Laval): student supervision, revision and correction of the manuscript.

\phantomsection\addcontentsline{lof}{table}{5.1\quad Experimental set-up in the reflection mode}
\phantomsection\addcontentsline{lof}{table}{5.2\quad  Steel sample dimension details with Flat-Bottom Holes of different depths and sizes.}
\phantomsection\addcontentsline{lof}{table}{5.3\quad Experimental set-up for LN$_2$ cooling}
\phantomsection\addcontentsline{lof}{table}{5.4\quad One example of ROC analysis (LN$_2$ results) and binary map of defect locations}
\phantomsection\addcontentsline{lof}{table}{5.5\quad Thermal Raw Images of PT and LN$_2$ stimulation techniques}
\phantomsection\addcontentsline{lof}{table}{5.6\quad FFT in amplitude and Phase results for LIT}
\phantomsection\addcontentsline{lof}{table}{5.7\quad PCT results of corresponding technique}
\phantomsection\addcontentsline{lof}{table}{5.8\quad ROC curves obtained from above results}

\includepdf[pages=-]{Thermosense2017_Lei}

\chapter{Liquid Nitrogen Cooling in IR Thermography applied to steel specimen}
The results of this study were firstly presented at an oral session of SPIE Commercial+ Scientific Sensing and Imaging. International Society for Optics and Photonics, 2017.
Then it was accepted for publication in Proceedings Volume 10214, Thermosense: Thermal Infrared Applications XXXIX; 102140T (2017).


%
\section*{Résumé}
La thermographie pulsée (PT) est l'une des méthodes les plus courantes dans les procédures de thermographie active de la thermographie pour NDT \& E ​​(essais non destructifs  \& évaluation), en raison de la rapidité et la commodité de cette technique d'inspection. Des éclairs ou des lampes sont souvent utilisés pour chauffer les échantillons dans la PT traditionnelle. Cet article explore principalement exactement la stimulation externe opposée en IR Thermographie: refroidissement au lieu de chauffage. Un échantillon d'acier avec des trous à fond plat de différentes profondeurs et tailles a été testé. De l'azote liquide (LN $_2$) est répandu sur la surface de l'échantillon et l'ensemble du processus est capturé par une caméra thermique. Pour obtenir une bonne comparaison, deux autres techniques classiques de CND, la thermographie pulsée et la thermographie verrouillée, sont également utilisées. En particulier, la méthode Lock-in est implémentée avec trois fréquences différentes. Dans la procédure de traitement d'image, la méthode de thermographie en composantes principales (PCT) a été effectuée sur toutes les images thermiques. Pour les résultats Lock-in, les images de phase et d'amplitude sont générées par la transformée de Fourier rapide (FFT). Les résultats montrent que toutes les techniques présentaient en partie les défauts tandis que la technique LN $_2$ affichait les défauts seulement au début du test. De plus, un poste-traitement de seuil binaire est appliqué aux images thermiques, et en comparant ces images à une carte binaire de l'emplacement des défauts, les courbes caractéristiques de fonctionnement du récepteur (ROC) correspondantes sont établies et discutées. Une comparaison des résultats indique que la meilleure courbe ROC est obtenue en utilisant la technique flash avec la méthode de traitement PCT.

%Cette recherche étudie une stimulation externe - refroidissement au lieu de chauffer en thermographie infrarouge pour NDT \& E. Un spécimen en acier est utilisé afin de  tester trois stimulations différentes sur les images thermiques et également une comparaison d'analyse ROC. Les résultats montrent que toutes les techniques mettent en évidence une partie des défauts de l'échantillon, alors que la technique LN $ _2 $ ne représente les défauts qu'au début; ceci peut être dû à la conductivité élevée de l'acier. Dans les résultats thermiques, la méthode de post-traitement PCT affiche de meilleurs résultats pour toutes les procédures. Plus de défauts sont exposés dans la stimulation Flash avec le traitement PCT.
% Les résultats de cette étude ont d'abord été présentés lors d'une session orale de SPIE Thermosense: Thermal Infrared Applications XXXIX 2017, aux Etats-Unis.

\section*{Abstract}
Pulsed Thermography (PT) is one of the most common methods in Active Thermography procedures of the Thermography for NDT \& E (Nondestructive Testing \& Evaluation), due to the rapidity and convenience of this inspection technique. Flashes or lamps are often used to heat the samples in the traditional PT. This paper mainly explores exactly the opposite external stimulation in IR Thermography: cooling instead of heating. A steel sample with flat-bottom holes of different depths and sizes has been tested. Liquid nitrogen (LN$_2$) is sprinkled on the surface of the specimen and the whole process is captured by a thermal camera. To obtain a good comparison, two other classic NDT techniques, Pulsed Thermography and Lock-In Thermography, are also employed. In particular, the  Lock-in  method  is  implemented  with  three  different  frequencies.  In  the  image  processing  procedure,  the Principal Component Thermography (PCT) method has been performed on all thermal images. For Lock-In results, both Phase and Amplitude images are generated by Fast Fourier Transform (FFT). Results show that all techniques presented part of the defects while the LN$_2$ technique displays the flaws only at the beginning of the test. Moreover, a binary threshold post-processing is applied to the thermal images, and by comparing these images to a binary map of the location of the defects, the corresponding Receiver Operating Characteristic (ROC) curves are established and discussed. A comparison of the results indicates that the better ROC curve is obtained using the Flash technique with PCT processing method.  

\newpage
\textbf{\texttt{Contributing authors:}}

\textbf{\textsf{Lei Lei}} (Ph.D candidate): developing protocol, experiment preparation and planning, data analysis,  personnel coordination and manuscript preparation.

\textbf{Giovanni Ferrarini} (Researcher of CNR-ITC): discussion in developing protocol, experiment preparation.

\textbf{Alessandro Bortolin} (Ph.D student of CNR-ITC): discussion and experiment preparation.

\textbf{Gianluca Cadelano} (Ph.D student of CNR-ITC): data collection, discussion and experiment preparation.

\textbf{Paolo Bison} (Research supervisor of CNR-ITC): student supervision, revision and correction of the manuscript. 

\textbf{Xavier Maldague} (Research director of LVSN in University Laval): student supervision, revision and correction of the manuscript.

% \phantomsection\addcontentsline{lof}{table}{5.1\quad Experimental set-up in the reflection mode}
% \phantomsection\addcontentsline{lof}{table}{5.2\quad  Steel sample dimension details with Flat-Bottom Holes of different depths and sizes.}
% \phantomsection\addcontentsline{lof}{table}{5.3\quad Experimental set-up for LN$_2$ cooling}
% \phantomsection\addcontentsline{lof}{table}{5.4\quad One example of ROC analysis (LN$_2$ results) and binary map of defect locations}
% \phantomsection\addcontentsline{lof}{table}{5.5\quad Thermal Raw Images of PT and LN$_2$ stimulation techniques}
% \phantomsection\addcontentsline{lof}{table}{5.6\quad FFT in amplitude and Phase results for LIT}
% \phantomsection\addcontentsline{lof}{table}{5.7\quad PCT results of corresponding technique}
% \phantomsection\addcontentsline{lof}{table}{5.8\quad ROC curves obtained from above results}

% \includepdf[pages=-]{Thermosense2017_Lei}
\newpage
\section{Introduction}
\label{sect:intro}  % \label{} allows reference to this section
In the Nondestructive Testing \& Evaluation (NDT \& E) field, active InfraRed (IR) thermography \citet{Maldague2001theory} is a technique widely used in assessing the conditions of parts of material components, with an extremely broad range of applications \citet{Vavilov2017Thermal,cadelano2016}. Traditionally, Pulsed Thermography (PT) deploys a thermal stimulation pulse (flash or lamp heating) to produce a thermal contrast between the feature of interest and the background, then monitors the time evolution of the surface temperature by a thermal camera. With this rapidity and convenience, numerous studies have been devoted to this technique, that is now a standard procedure for thermal testing.\citet{Maldague1993Nondestructive,Maldague1994bInfra,2007-Ibarra-Castanedo,2011-ClementeIbarra-Castanedo,duan2013quantitative,vavilov_2015}. 
%\added [id=GF] {that is now a standard procedure for thermal testing.}

However, if the temperature of the material to inspect is already higher than the ambient temperature, it can be of interest to make use of a cold thermal source such as a line of air jets (or water jets; sudden contact with ice, snow, etc.). In fact, a thermal front propagates the same way whether being hot or cold: what is important is the temperature differential between the thermal source and the specimen. An advantage of a cold thermal source is that it does not induce spurious thermal reflections into the IR camera as in the case of a hot thermal source. The main limitations of cold stimulation sources are related to practical considerations, as it is generally easier and more efficient to heat rather then to cool a part.
% \replaced [id=GF] {, as it is generally easier and more efficient to heat rather then to cool a part.} {as for instance it is generally easier and more efficient, to heat rather then to cool a part.} 
Thus, the advantage and convenience of using the cold stimulation in active infrared thermography  still remains to be investigated in detail and better understood. Using a cold source could extend the range of application of infrared thermography to cases where the specimen under analysis could not be heated, due to safety issues or physical reasons\citet{Livingston2018High}. A cold source could be required while handling biological or organic materials   that could not withstand a temperature increase, such as food. Another field of investigation is the survey of concrete and composite materials in order to find cracks and delaminations due to the presence of water or ice. Also in this case, a cold source would significantly decrease the risk of altering the physical characteristics of the specimen. 
% \added [id=GF] {Using a cold source could extend the range of application of infrared thermography to cases where the specimen under analysis could not be heated, due to safety issues or physical reasons. A cold source could be required while handling biological or organic materials that could not withstand a temperature increase, such as food. Another field of investigation is the survey of concrete and composite materials in order to find cracks and delaminations due to the presence of water or ice. Also in this case, a cold source would significantly decrease the risk of altering the physical characteristics of the specimen} 
Nonetheless,in the past in the scientific community, only a limited number of studies investigating a cold approach (cooling as the external stimulation in active infrared thermography) have been performed on industrial product inspection\citet{endohdynamical2012,2012-LewisHom}. A study by Burleigh\citet{1989Burleigh} showed that the cooling method with refrigerating liquids is feasible but requires caution to ensure the safety.
% \added [id=GF] {A study by Burleigh\citet{1989Burleigh} showed that the cooling method with refrigerating liquids is feasible but requires caution to ensure the safety.} 
The work of (Lei et al.)\citet{lei2017detection} chose instead to use air cooling to survey refrigerated vehicles. All the available works do not give a quantitative information about the reliability of the cooling technique, especially in comparison with the traditional heating procedure.
% \replaced [id=GF] {chose instead to use air cooling to survey refrigerated vehicles.} {demonstrates an example that air cooling approach applied in detection of the truck panel.} 
% \added [id=GF] {All the available works do not give a quantitative information about the reliability of the cooling technique, especially in comparison with the traditional heating procedure.}

Therefore, the aim of this paper is improving the current knowledge on cooling thermal stimulation
% \replaced [id=GF] {improving the current knowledge on cooling thermal stimulation} {then the continuous investigation of cold approach} 
in IR thermography. In order to perform a reliable comparison, three methods (two traditional techniques, Pulsed Thermography and Lock-in Thermography act as the reference) will be applied on a steel slab with different sizes of flat-bottom-holes. The thermographic images of the experiments will be  treated to  eventually produce a binary map of the location of the defects. This map will be statistically evaluated in terms of sensitivity and specificity~\citet{Fawcett2006} by comparison with the `true' map of the defects, furnishing a rank of the three stimulation methods. 

\section{Experimental setup} % (fold)
\label{sec:experimental_setup}
One side stimulation approach is often used in the Infrared Thermography NDT \& E field, which is also known as the reflection scheme: both the stimulation device and the camera stay on the same side of the sample being tested. This approach applied in reality is shown in Figure~\ref{Exp_setup}.

\begin{figure}[ht]
   % \centering
   % \begin{center}
   \hspace{-0.95cm}
   \begin{tabular}{c}
   \includegraphics[scale=0.95]{chp3/Exp_setup.png}
   \\
   \footnotesize{(a) Pulsed Thermography set-up} \hspace{4cm} \footnotesize{(b) Lock-in Thermography set-up}   
   \end{tabular}  
   % \end{center}
   \caption{Experimental set-up in the \textit{reflection} mode}
   \label{Exp_setup}
\end{figure}

% \begin{figure}[ht]
%    \centering
%    \subfloat[Pulsed Thermography set-up]
%    {
%       \includegraphics[scale=0.3]{chp3/Flash_Setup.png}
%    }
%    %\hspace{5pt}
%    \subfloat[Lock-in Thermography set-up]
%    {
%       \includegraphics[scale=0.3]{chp3/LIT_setup.png}
%    }
%    \caption{Experimental set-up in the \textit{reflection} mode}
%    \label{Exp_setup}
% \end{figure}

The following equipment was set up for this study:
\begin{itemize}
   \item Infrared Camera FLIR SC3000 (spatial resolution equal to 320$\times$240 pixels, frame rate up to 50Hz, GaAs sensor, spectral range 8-9 $\mu m$)
%    \replaced [id=GF] {spatial resolution equal to 320$\times$240 pixels, framerate up to 50Hz, GaAs sensor, spectral range 8-9 $\mu m$)}{320$\times$240 pixels, framerate 50Hz, GaAs , 8-9 $\mu m$)}
   \item Two pairs of flash 
%    \replaced [id=GF] {}{halogen} 
    lamps for a total of 10 kJ (electric) released in 5 $ms$ 
   \item One pair of modulated halogen lamps with 1kW each served as Lock-in stimulation
   \item An isolated bottle (500 ml)
%   ??? \added [id=LL] {need to be rechecked, thanks Giovanni~}
filled with
%    \replaced [id=GF] {(500 ml???) filled with}{full of}
   Liquid Nitrogen.
\end{itemize}

\subsection{Specimen} % (fold)
\label{sub:specimen}
In this study, a steel specimen comprising flat-bottom holes of different depths and sizes will be examined. Their dimensions are depicted in Figure~\ref{specimen}.
   \begin{figure}[ht]
   \centering   
   % \begin{tabular}{c} %% tabular useful for creating an array of images 
   \includegraphics[scale=0.4]{chp3/specimen_schema.pdf}
   % \end{tabular}
   \caption{Steel sample dimension details with Flat-Bottom Holes of different depths and sizes (mirror image)} 
%    \added [id=LL] {Mirror image}
%>>>> use \label inside caption to get Fig. number with \ref{}
    \label{specimen} 
   \end{figure}  

The steel plate has seventeen holes whose diameters vary from 0.4 $cm$  to 3 $cm$, and whose depths (from bottom) vary from 0.3 $cm$ to 0.9 $cm$. The entire thickness is 1 $cm$. This specimen is painted before the test, in order to increase its emissivity and to obtain a homogeneous external stimulation.

% subsection specimen (end)
\subsection{Stimulation Techniques} % (fold)
\label{sub:stimulation_techniques}
Three external stimulations were deployed on the sample, for the sake of  obtaining a good comparison in results: 
\begin{itemize}
   \item Pulse Thermography (PT) 
   \item Lock-in Thermography (LIT)
   \item Liquid Nitrogen cooling (LN$_2$)
\end{itemize}
Known as the traditional and fast technique in NDT \& E, Pulse Thermography (PT) acts as the reference during this test. One reason for this popularity is the quickness of the inspection relying on a thermal  stimulation pulse, with duration going from a few ms for high thermal conductivity material inspection (such as metal specimen in this study) to a few seconds for low thermal conductivity specimens.  Such quick thermal stimulation allows direct deployment on the plant floor with convenient heating sources.

% When neglecting heat exchange with the environment, the pulse of energy $Q$, delivered on a layer of thickness $L$, characterized by a density $\rho$, a specific heat $C_p$ and a thermal conductivity $\lambda$ (or a thermal diffusivity $\alpha$) produces a temperature increment behavior on the heated surface given by:
% \begin{equation}
%    T(t) = \frac{Q}{\rho C_p L}[1+2\sum_{n=1}^{\infty} e^{-\frac{n^2 \pi ^2\alpha t}{L^2}}]
%    \label{eq_pt}
% \end{equation}

Lock-in thermography (LIT) is based on thermal waves generated inside the inspected specimen and detected remotely. Wave generation is for instance performed by periodically depositing heat on the specimen surface (e.g. through sine-modulated lamp heating) while the resulting oscillating temperature field in the stationary regime is remotely recorded through its thermal infrared emission.

In this study, 3 different frequencies of 0.0625~$Hz$ (LIT16--one period is equivalent to 16 seconds), 0.125~$Hz$ (LIT8--one period is equivalent to 8 seconds) and 0.25~$Hz$ (LIT4--one period is equivalent to 4 seconds) are performed in Lock-in Thermography. 

% By the convolution integral, Eq~(\ref{eq_pt}) becomes:
% \begin{equation}
%    T(t) = \frac{W}{\lambda}\frac{\alpha}{L}\int_0^t d\tau \Big(1+\sin(\omega \tau - \frac{\pi}{2})\Big)\Big\{1+2\sum_{n=1}^{\infty} e^{-\frac{n^2 \pi ^2\alpha(t-\tau)}{L^2}}\Big\}
% \end{equation}
% where $W$ is the absorbed heating power.

The Liquid Nitrogen is applied in the test by means of pouring it out directly onto the surface in order to cool the sample. LN$_2$ was sprinkled onto the specimen center and allowed to spread out forwards the edges. The whole capture duration is 500 frames with 50~$Hz$ of image frequency, ie. 10 seconds of recording. The pouring time (cooling time) is about 2 seconds, due to the bottle with a volume of 500 ml, with 100 frames in the results sequence. This way of cooling has a limit of that the central part will be first and mainly cooled, then the cold front propagates from center to the edges. This can lead to a drawback that in the final thermograms defects closer to the edges will be difficulty to detected. The experimental set-up is shown in Figure~\ref{Exp_LN2}.

\begin{figure}[ht]
   \centering
   \includegraphics[scale=0.4]{chp3/LN2_setup.png}
   \caption{Experimental set-up for LN$_2$ cooling}
   \label{Exp_LN2}
\end{figure}

% subsection stimulation_techniques (end)

% section experimental_setup (end)


\section{Processing Methods} % (fold)
\label{sec:processing_methods}
The following image-processing techniques and data-analysis methods were employed for this study:
\begin{itemize}
   \item Principal Component Thermography (PCT)
   \item Phase and Amplitude images by Fast Fourier Transform (FFT)
   \item Receiver Operating Characteristic curves (ROC Curves)
\end{itemize}

\subsection{Principal Component Thermography (PCT)}
The Principal Component Thermography technique\citet{Rajic2002} uses ``singular value decomposition (SVD) to reduce the matrix of observations to a highly compact statistical representation of the spatial and temporal variations relating to contrast information associated with underlying structural flaws".

\subsection{FFT in Phase and Amplitude for LIT}
In addition to the common technique for NDT \& E, Fast Fourier Transform in LIT\citet{wu1998lock} is also one of the most applied techniques in IR Thermography, which is based on the periodic heating of the object being tested. A thermal wave is likewise generated and propagates inside the material. In real experimental cases the thermal wave is composed by a principal frequency and several harmonics where the amplitude of the Fast Fourier Transform is a function of frequency. By selecting the component with the highest amplitude it is possible to produce a phase map at the corresponding frequency where the defect appears enhanced.


\subsection{ROC Curve analysis} % (fold)
\label{sub:roc_curve_analysis}
The Receiver operating characteristic (ROC) curve is a technique in statistics which helps visualize, organize and select classifiers based on their performance. The curve graph is created by plotting the the true positive rate (TPR) against the false positive rate (FPR) at various threshold settings. More details about concepts and definitions can be found in \citet{Fawcett2006}. While being frequently chosen a standard method in several scientific fields, ROC are rarely applied in the thermographic field.\citet{Bison2014a} 

Implemented in this study, a binary map of the defect locations is built and correlated to the post-processed images in gray scale, and the lay-out is provided in Figure~\ref{binary}.
% \deleted [id=LL]{, and the lay-out is provided in Figure~\ref{binary}}. 
The main algorithm in the calculation of TPR and FPR is clarified as:
\begin{enumerate}
   \item Choose the post-processed thermogram in which most defects shown as the test image;
%    \replaced [id=LL] {Choose the post-processed thermogram in which most defects shown as the test image;}{Identify the best gray-scale result from the post-processing thermal images as the test image;}
   \item Resize the defect map to the same size as that of the test image;
   \item Choose a thresholding step number $N$($N=1000$ in this study) and establish the step value of thresholding [from $0$, $\frac{1}{N}$, $\frac{2}{N}$ till $1 (=\frac{N}{N})$];
   \item For each thresholding step, binarize the test image with the thresholding and then compare it to the defect map, in order to obtain the corresponding TPR and FPR values;
   \item Iterate the binarization and comparison so as to plot a whole curve.
\end{enumerate}

% \begin{figure}[ht]
%    \begin{center}
%    \begin{tabular}{c}
%    \includegraphics[scale=1.0]{ROC_exm.png}
%    \\\footnotesize{(a) Binary map of defects} \hspace{4cm} \footnotesize{(b) Cool map}   
%    \end{tabular}  
%    \end{center}
%    \caption{One example of ROC analysis (LN$_2$ results) and binary map of defect locations}
%    \label{binary}
% \end{figure}

\begin{figure}[ht]
   \centering
   \subfloat[Binary map of defects]
   {
      \includegraphics[scale=0.195]{chp3/Schema_done.png}
      \label{bin_map}
   }
   \subfloat[Cool map]
   {
      \includegraphics[scale=0.8]{chp3/Cool_ROC.png}
      \label{cool_map}
   }   
   \caption{One example of ROC analysis (LN$_2$ results) and binary map of defect locations}
   \label{binary}
\end{figure}

%% subsection roc_curve (end)


% section methods (end)
\section{Results \& Discussion} % (fold)
\label{sec:results_&_discussion}
Figure~\ref{raw_results} illustrates the thermal raw images of PT and LN$_2$. The Lock-In FFT results can be found in Figure~\ref{LIT_results}.

% \begin{figure}[ht]
%    \begin{center}
%    % \begin{tabular}{c}
%    \includegraphics[scale=0.60]{chp3/Raw_results.png}
%    % \\\footnotesize{(a) Binary map of defects} \hspace{4cm} \footnotesize{(b) Cool map}   
%    % \end{tabular}  
%    \end{center}
%    \caption{Thermal raw images of PT and LN$_2$ stimulation techniques}
%    \label{raw_results}
% \end{figure}


\begin{figure}[htpb]
   \centering
   \subfloat[Flash raw frame 23]
   {
      \includegraphics[scale=0.55]{chp3/flash_raw23_2.png}
      \label{Flash_raw23}
   }
   \hspace{10pt}
   \subfloat[Flash raw frame 60]
   {
      \includegraphics[scale=0.55]{chp3/flash_raw60_2.png}
      \label{Flash_raw60}
   }
   \hspace{10pt}
   \subfloat[LN$_2$ raw frame 41]
   {
      \includegraphics[scale=0.55]{chp3/cool_raw_2.png}
      \label{LN2_raw}
   }
   \caption{Thermal raw images of PT and LN$_2$ stimulation techniques}      
   \label{raw_results}
\end{figure}

% \begin{figure}[ht]
%    \begin{center}
%    % \begin{tabular}{c}
%    \includegraphics[scale=0.6]{chp3/LIT_results.png}
%    % \\\footnotesize{(a) Binary map of defects} \hspace{4cm} \footnotesize{(b) Cool map}   
%    % \end{tabular}  
%    \end{center}
%    \caption{FFT in amplitude and phase results for LIT}
%    \label{LIT_results}
% \end{figure}

\begin{figure}[htpb]
   \centering
   \subfloat[LIT4 FFT in amplitude]
   {
      \includegraphics[scale=0.57]{chp3/LIT4_AMP.png}
   }
   \hspace{10pt}
   \subfloat[LIT4 FFT in phase]
   {
      \includegraphics[scale=0.57]{chp3/LIT4_PHA.png}
   }
   \hspace{10pt}
   \subfloat[LIT8 FFT in amplitude]
   {
      \includegraphics[scale=0.57]{chp3/LIT8_AMP.png}
   }
   \hspace{10pt}
   \subfloat[LIT8 FFT in phase]
   {
      \includegraphics[scale=0.57]{chp3/LIT8_PHA.png}
      \label{LIT8_ph}
   }
   \hspace{10pt}
   \subfloat[LIT16 FFT in amplitude]
   {
      \includegraphics[scale=0.57]{chp3/LIT16_AMP.png}
      \label{LIT16_ph}
   }
   \hspace{10pt}
   \subfloat[LIT16 FFT in phase]
   {
      \includegraphics[scale=0.57]{chp3/LIT16_PHA.png}
   }
   \caption{FFT in amplitude and phase results for LIT}      
   \label{LIT_results}
\end{figure}

\subsection{Thermal images comparison} 

From the results above, raw images in Figure~\ref{raw_results} indicate that the most detectable flaws are the ones with a high aspect ratio (ie. diameter-to-depth). In addition, for the PT stimulation, a small hole with diameter 0.4 $cm$, depth\footnote{It should be noted that the depth values mentioned here and after are from the bottom of the sample, therefore the real corresponding depths should be these values subtracted from the thickness.} 0.9 $cm$ (left-upper in Figure~\ref{Flash_raw23}) appeared in frame 23, and disappeared after. Other holes, two with diameter 2 $cm$ and depth 0.9 $cm$, and two with diameter 3 $cm$ and depth 0.5 $cm$, 
% \replaced [id=GF]{. Other holes, two with diameter 2 $cm$ and depth 0.9 $cm$, and two with diameter 3 $cm$ and depth 0.5 $cm$,} {, while another two holes with diameter 2 $cm$, depth 0.9 $cm$ and two with diameter 3 $cm$, depth 0.5 $cm$}
(center in Figure~\ref{Flash_raw60}) appeared in frame 60.  Nonetheless, in LN$_2$ raw results, same defects as the one in Flash frame 60 appeared in LN$_2$ frame 41 (however, it should be noted that the first 100 frames of cooling time has been removed from the these final raw images. The reason is because there was much noise from nitrogen gases in the thermal images).
% \added [id=LL] {The reason is because there was much noise from nitrogent gases in the thermal images})
After this, there was no more defect that emerged. Another remark is that since the LN$_2$ is sprinkled onto the center of the surface, there could be the situation that the center part of specimen was over cooled while the `cold front' (opposite to heat front) might have not be able to propagate to the edges. Due to the high conductivity of steel, defects only showed up at the beginning
% \replaced [id=LL] {end} {beginning}
of the cooling procedure. These may be the main issues of pouring-out method.

For LIT results, FFT in amplitude has a better flaw detection capability than FFT in phase, as there is some noise in all of the phase images. LIT8 and LIT16 have about four more detected flaws than LIT4. Whereas, for FFT in phase for LIT8 and LIT16, there some inverse gray-scale values occur. This might be due to the reverse image question.

Following PCT processing, figure~\ref{PCT_results} exhibits a clearer result. It can be observed that most of the flaws are visible, especially in the flash image (Figure~\ref{PCT_Flash}). Less flaws are visible in the LIT4 PCT third image.
%The corresponding PCT results are represented in Figure~\ref{PCT_results}.
% \begin{figure}[ht]
%    \begin{center}
%    % \begin{tabular}{c}
%    \includegraphics[scale=0.6]{chp3/PCT_results.png}
%    % \\\footnotesize{(a) Binary map of defects} \hspace{4cm} \footnotesize{(b) Cool map}   
%    % \end{tabular}  
%    \end{center}
%    \caption{PCT results of corresponding technique}
%    \label{PCT_results}
% \end{figure}

\begin{figure}[htpb]
    \centering
    \subfloat[Flash PCT 2nd Image]{
      \includegraphics[scale=0.57]{chp3/Flash_PCT_2.png}
      \label{PCT_Flash}
      }
    \hspace{10pt}
    \subfloat[LN$_2$ PCT 2nd Image]{
      \includegraphics[scale=0.57]{chp3/Cool_PCT_2.png}
      }
    \hspace{10pt}
    \subfloat[LIT4 PCT 3rd Image]{
      \includegraphics[scale=0.57]{chp3/LIT4_PCT_3.png}
      }    
    \\ %\hspace{10pt}
    \subfloat[LIT8 PCT 3rd Image]{
      \includegraphics[scale=0.57]{chp3/LIT8_PCT_3.png}
      }
    \hspace{10pt}
    \subfloat[LIT16 PCT 3rd Image]{
      \includegraphics[scale=0.57]{chp3/LIT16_PCT_3.png}
      }
    % \includegraphics[scale=0.4]{graph/LIT4_PCT_3.png}
    % \includegraphics[scale=0.4]{graph/LIT8_PCT_3.png}
    % \includegraphics[scale=0.4]{graph/LIT16_PCT_3.png}
    \caption{PCT results of corresponding technique}
    \label{PCT_results}
\end{figure}


Comparing the processed thermal images, the following observations can be made:  
\begin{itemize}
    \item All techniques present part of the flaws in the sample;
    \item The PCT post-processing method displays a better results for all images;
    \item More defects are exhibited in Flash stimulation with PCT processing;
    %\item For LIT8 and LIT16, there are some reverse image question.
\end{itemize}


\subsection{Corresponding ROC curves comparison}
The ROC curves obtained from comparing the binary map of defect locations to the above results are represented in Figure \ref{roc_pct} and ~\ref{ROC_curve}.
\begin{figure}[htbp]
   \centering
   \includegraphics[scale=0.7]{chp3/ROC_PCT_2017.pdf}
   \caption{ROC curves for PCT processing results}
   \label{roc_pct}
\end{figure}

% \begin{figure}[htbp]
%    % \centering
%    \begin{center}
%    \begin{tabular}{c}
%    \includegraphics[scale=0.75]{chp3/ROC_LIT_results.png}
%    \\\footnotesize{(a) ROC from LIT FFT in amplitude results} \hspace{2.1cm} \footnotesize{(b)  ROC from LIT FFT in phase results}   
%    \end{tabular}
%    \end{center}  
%    \caption{ROC curves for LIT processing results}
%    \label{roc_lit}
% \end{figure}

\begin{figure}[ht]
    \centering
    % \subfloat[ROC from PCT Results]
    % {
    % \includegraphics[scale=0.75]{chp3/ROC_PCT_2017.pdf}
    % }
    % %\hspace{10pt}
    % \\
    \subfloat[ROC from LIT FFT in amplitude results]
    {
    \includegraphics[scale=0.70]{chp3/ROC_LIT_AMP_2017.pdf}
    \label{roc_amp}
    }
    \hspace{10pt}
    \subfloat[ROC from LIT FFT in phase results]
    {
    \includegraphics[scale=0.70]{chp3/ROC_LIT_PHA_2017.pdf}
    \label{roc_pha}
    }    
    \caption{ROC curves for LIT processing results}
    \label{ROC_curve}
\end{figure}

%\added[id=GF] {Before you called the true positive rate TPR (and the false FPR). From here you call them tp rate and fp rate, maybe it is better to choose the same nomenclature }

In the ROC curves definitions, the \textit{sensitivity} is the True Positive Rate (\textit{TPR}), and \textit{1-specificity} is the False Positive Rate (\textit{FPR}).
From the curves of the PCT results (Figure~\ref{roc_pct}), one can easily notice that the five curves have almost the same performance of classification in the beginning. When the \textit{TPR} arrives at $0.3$, the PT curve becomes the nearest to the northwest (where the \textit{TPR} is higher, the \textit{FPR} is lower or both). The second one is the LN$_2$ curve that has a slightly higher \textit{TPR} than PT after the \textit{FPR} reaches $0.4$. The LIT8 and LIT16 curves have almost the same performance before the \textit{TPR} attains $0.7$. After that the LIT16 has a higher \textit{TPR} than LIT8. The worst one is the LIT4 curve as it is the closest one to the diagonal line $y=x$, which represents the strategy of randomly guessing a class.

The situation is the same for the curves of LIT amplitude (Figure~\ref{roc_amp}), in which LIT4 has an unfavorable performance in classification. LIT16 also has a slightly higher \textit{TPR} than LIT8. However, in the curves of LIT phase results (Figure~\ref{roc_pha}), LIT16 shows a poor performance while LIT4 displays a better classification
% \added[id=GF] {while LIT4 displays a better classification}
, contrary to the former results. 
% \replaced[id=GF] {} {While LIT4 displays a better classification.}
Because of the reverse image problem, the result LIT8 presents a region (top-right) which is under the equal diagonal line.


\subsection{Area under curve analysis and comparison}
In ROC analysis, a common method to compare different classifiers is to calculate the area under the ROC curve, known as AUC.\citet{Fawcett2006} In this way, a single scalar value will represent the expected performance of the classifiers. As in the ROC curve profile, the AUC is a part of the unit square are, therefore, its value will be between 0 and 1. Thus, random guessing classifier produces a diagonal line between (0,0) and (1,1), which has an area of 0.5. It can be then indicated that a classifier AUC value less than 0.5 is even worse than a random guessing. 

The corresponding AUC obtained from the above ROC curves is found in Tab. \ref{tab_auc}. Their comparison in form of the bar chart is illustrated in Fig. \ref{fig_auc}. A straight comparison shows that the Flash method with PCT second component has the highest AUC value of 0.9. The second one is the Liquid Nitrogen method, that is very close to the flash method with an AUC value of 0.87. At the bottom of the ranking, LIT4 with PCT third component and its amplitude one have the worst values as  0.67 and 0.65 respectively, which are just a bit better and a random guessing.
On the contrary, LIT8 and LIT16 technique have a better result both in PCT processing and their amplitude images. This might be caused by the fact that during one period, 4 seconds of heating (LIT4) were not enough to penetrate the steel specimen deeply as those of 8 seconds (LIT8) and 16 seconds (LIT16). On the other hand, all the Lock-in Thermography techniques have similar AUC value in their phase images, which indicates that for phase image results, heating time does not make an effective influence in Lock-in Thermography in this study. Considering all these results, though the best approach is Flash technique with PCT processing, the LN$_2$ technique can be considered as a valid alternative method

% Figure AUC to be added.
% Table to be added.
\begin{table}[htbp]
    \centering
    \begin{tabular}{lr}%ccccccccc}
    \toprule
    \textbf{Method} & \textbf{Value} \\
    \midrule
    Flash-2nd & 0.90 \\
    LN2-2nd & 0.87 \\
    \midrule
    LIT4-3rd & 0.67 \\
    LIT8-3rd & 0.80 \\
    LIT16-3rd & 0.82 \\
    \midrule
    LIT4-AMP & 0.65 \\
    LIT8-AMP & 0.79 \\
    LIT16-AMP & 0.82 \\
    \midrule
    LIT4-PHA & 0.71 \\
    LIT8-PHA & 0.69 \\
    LIT16-PHA & 0.67 \\

%     0.90 & 0.87 & 0.67 & 0.80 & 0.82 & 0.65 & 0.79 & 0.82 & 0.71 & 0.69 & 0.67 \\
    \bottomrule
    \end{tabular}
    \caption{AUC value comparison}
    \label{tab_auc}
\end{table}

% $x = ['Flash_2nd', 'LN2_2nd', 'LIT4_3rd', 'LIT8_3rd', 'LIT16_3rd',
%      'LIT4_AMP', 'LIT8_AMP', 'LIT16_AMP', 'LIT4_PHA', 'LIT8_PHA', 'LIT16_PHA']\\
% y = np.array([0.90, 0.87, 0.67, 0.80, 0.82,
%               0.65, 0.79, 0.82, 0.71, 0.69, 0.67])$
              
\begin{figure}[htpb]
   \centering
   \includegraphics[scale=0.48]{chp3/AUC.png}
   \caption{AUC value comparison}
   \label{fig_auc}
\end{figure}

% section results_&_discussion (end)



\section{Conclusion} % (fold)
\label{sec:conclusion}
This study investigates an external stimulation--cooling instead of heating in IR Thermography for NDT \& E. 
A steel specimen is used to test three different stimulations for thermal images and also ROC analysis comparison. 
Results shows that all techniques highlight part of the flaws in the sample, whereas the LN$_2$ technique represents the defects at the end of cooling process; this maybe due to the high conductivity of steel. 
In thermal results, the PCT post-processing method displays a better results for all procedures. More defects are exhibited in Flash stimulation with PCT processing.
ROC curve analysis and its AUC analysis have elucidated a straightforward classification comparison, in which the best values are obtained with the Flash technique with PCT processing, trailed narrowly by the Liquid Nitrogen method.


The LN$_2$ technique should therefore be considered as a valid option for the survey of objects that should not be heated, such as biological tissues, organic materials, dry or iced samples. With this purpose, future work will analyze different kinds of specimens.
% \replaced[id=GF] {The LN$_2$ technique should therefore be considered as a valid technique for the survey of objects that should not be heated, such as biological tissues, organic materials, dry or iced samples. With this purpose, future work will analyze different kinds of specimens.} {In future work, other common composite materials such as CFRP (Carbon fiber reinforced polymer), GFRP (Glass Fiber Reinforced Polymer) will be chosen as specimen.}
The method of Liquid Nitrogen pouring may be replaced by spraying onto the sample surface, which can reduce the inhomogeneous cooling problem. To enhance the penetration of heat inside the sample, a proposition involving heating on one side of the specimen and cooling the other side might be taken into consideration.
% \replaced[id=GF] {} {Even though LN$_2$ technique in this study has not shown enough advantages,}
% \replaced[id=GF] {The} {this}
The exploration of an opposite means of external stimulation in InfraRed Thermography might favor new ideas 
% \replaced[id=GF] {} {of approaches}
for NDT \& E.


\section*{Acknowledgments}       
 
This research was supported by the governments of Italy and Quebec (Minist\`{e}re des Relations internationales et de la Francophonie) through the Joint Subcommittee Qu\'{e}bec-Italy, project n$^{\circ}$08.0203. It was also supported by  the Natural Sciences and Engineering Research Council of Canada (NSERC). We are also thankful to our collaborative institute CNR-ITC Padova which provided expertise that greatly helped in this research.


\bibliographystyle{unsrtnat}              % style de la bibliographie
\bibliography{U:/Desktop/Bibliography/Biblio_th} 


%\includepdf[pages=-]{Revised_Version_2}
%\includepdf[pages={2-11}]{Lei2017Detection}
%\includepdf[pages=-]{Thermosense2017_Lei}

\chapter{Conclusion \& Perspectives}         % ne pas numéroter
%\phantomsection\addcontentsline{toc}{chapter}{Conclusion \& Perspectives} % dans TdM

\section{General conclusions}
The objectives of the present thesis were, first, to deploy the infrared thermography technique in the procedure of maintaining the ``cold food chain'', especially in insulated vehicles of ATP standards. The application of infrared thermography aims to identify thermal insulation anomalies, which the standard ATP test cannot localize. 

The preliminary work focused on mapping the heat flux on the external surface of an insulated roll-container using the Infrared thermography technique. The ATP standard measurement was performed to obtain the experimental results, meanwhile IR images of the roll-container were obtained when the steady condition was reached, in order to analyze and compute the corresponding heat flux on the entire surface. A simple thermal resistance model was applied to conduct the computation. Final temperature figures showed a good uniform distribution, and several defects in the structure such as thermal bridges or air leakages were identified. A reference zone of the external wall was measured by a thermal flux meter, then with that reference the entire surface heat flux map was determined. In addition, for a better view of the heat flux map, the homography, one of the computer vision techniques, was performed in the raw images by applying a bilinear interpolation with the projective transformation matrix. The final corrected heat flux map was demonstrated for each surface, in which the right surface showed a smaller value than the others.  Due to the air streaming, temperatures in the lateral surfaces were a little smaller than other surfaces, thus this leads to the smaller heat flux values.

When implemented into the internal surface of the insulated vehicle, a panoramic view was needed, since the field of view (FOV) of the infrared camera could not capture the entire surface of insulated vehicle. With the help of an infrared camera mounted on a pan-tilt head and automatically driven by a suitable software, a series of thermal images of the inner walls of the vehicle under steady condition have been captured. Proper computer vision techniques such as inverse spherical projection and stitching images by translation helped to generate the final panorama. The same thermal resistance model was utilized to compute the corresponding heat flux map. Results demonstrated a good performance of the algorithm, though the manual creation of the panoramic view required more time for completion. Compared with the standard ATP test, the final K-value obtained by infrared thermography showed a good accuracy (0.87\% of error).

Then based on the previous favorable results, the second aspect of this project was to explore cold approaches (such as compressed air, liquid nitrogen, etc.) in infrared thermography for Non-Destructive Testing \& Evaluation. The first interesting idea of detection of insulation flaws and thermal bridges in insulated truck box panels then emerged. As it is not convenient to heat the entire vehicle panels for the detection, cooling them by compressed air therefore can be a better solution. The study then focused on the cooling approach for the truck box panels inspection by infrared thermography. Both heating and cooling methods were applied by lamp and compressed air respectively. Numerical simulations under COMSOL Multiphysics{\textregistered} platform were conducted as well. For a comprehensive analysis, passive thermography detection in computational models has been presented at the same time. Results demonstrate that the compressed air spray is more rapid than the traditional heating method in providing successful detection.

A consideration of replacing compressed air by liquid nitrogen was then explored in more detail. Thus a study was performed, in which a steel specimen was used to test three different stimulations for thermal images and also Receiver Operating Characteristic (ROC) analysis comparison. Results showed that all techniques highlighted part of the flaws in the sample, whereas the liquid nitrogen technique represented the defects only at the beginning; this may be due to the high conductivity of steel. In thermal results, the PCT post-processing method displayed better results for all procedures. More defects were exhibited in Flash stimulation with PCT processing. ROC curve analysis has elucidated a straightforward classification comparison, in which the best curve was obtained using the Flash technique with PCT processing. 



\section{Future perspectives}
Generally, the infrared thermography technique has been applied with promising results in ATP standard insulated vehicle tests, for the goal of maintaining the ``cold food chain". The results have demonstrated that the benefits of time-saving and the accuracy in the determination of the K-value could be applied for assessment at a commercial level.  For the panoramic view of the insulated vehicle internal surface, due to the repeated structure on the internal surface of vehicle, algorithms of automatic creation which have better feature detection and comparison remain to be explored.  A suitable software package may be created to simplify the post-processing of thermal images. This will facilitate the computation of the K-value.

On the other hand, the exploration of cold approaches in infrared thermography for Non-Destructive Testing and Evaluation has also shown several advantages. Compressed air cooling can be a good replacement of heating in the detection of insulation flaws for truck box panels. The strategy of heating one side and cooling another side can be deployed in practice, since ideal cases in simulation show favorable results.

The use of ROC curves to compare the different methods in infrared thermography is an interesting approach. This technique would benefit from more complete application and discussion, which would favor a more harmonious implementation in traditional techniques of NDT.



            % conclusion


%\appendix                       % annexes le cas échéant

%\chapter{Titre de l'annexe}     % numérotée

Texte de l'annexe.
                % annexe A

%\bibliography{U:/Desktop/Bibliography/Biblio_th}                 % production de la bibliographie

\end{document}
