%% GABARIT POUR THÈSE STANDARD
%%
%% Consulter la documentation de la classe ulthese pour une
%% description détaillée de la classe, de ce gabarit et des options
%% disponibles.
%%
%% [Ne pas hésiter à supprimer les commentaires après les avoir lus.]
%%
%% Déclaration de la classe avec le type de grade
%%   [l'un de LLD, DMus, DPsy, DThP, PhD]
%% et les langues les plus courantes. Le français sera la langue par
%% défaut du document.
\documentclass[PhD,french,english, 12pt]{ulthese}
  %% Encodage utilisé pour les caractères accentués dans les fichiers
  %% source du document. Les gabarits sont encodés en UTF-8. Inutile
  %% avec XeLaTeX, qui gère Unicode nativement.
  \ifxetex\else \usepackage[utf8]{inputenc} \fi

  %% Charger ici les autres paquetages nécessaires pour le document.
  %% Quelques exemples; décommenter au besoin.
  \usepackage{amsmath}       % recommandé pour les mathématiques
  \usepackage{ncccomma}      % gestion de la virgule dans les nombres
  \usepackage{pdfpages}
  \usepackage[caption=false]{subfig}
  \usepackage{nomencl}
  \makenomenclature
  
  \renewcommand{\nomname}{List of Acronyms}
%  \usepackage{paralist}  % Paralist items
  %% Utilisation d'une autre police de caractères pour le document.
  %% - Sous LaTeX
  %\usepackage{mathpazo}      % texte et mathématiques en Palatino
  %\usepackage{mathptmx}      % texte et mathématiques en Times
  %% - Sous XeLaTeX
  %\setmainfont{TeX Gyre Pagella}      % texte en Pagella (Palatino)
  %\setmathfont{TeX Gyre Pagella Math} % mathématiques en Pagella (Palatino)
  %\setmainfont{TeX Gyre Termes}       % texte en Termes (Times)
  %\setmathfont{TeX Gyre Termes Math}  % mathématiques en Termes (Times)

\graphicspath{{./graph/}}               % Images folders
\DeclareGraphicsExtensions{.jpg,.png,.pdf,.eps}

  %% Gestion des hyperliens dans le document. S'assurer que hyperref
  %% est le dernier paquetage chargé.
  \usepackage{hyperref}
  \hypersetup{colorlinks,allcolors=ULlinkcolor}

  %% Options de mise en forme du mode français de babel. Consulter la
  %% documentation du paquetage babel pour les options disponibles.
  %% Désactiver (effacer ou mettre en commentaire) si l'option
  %% 'nobabel' est spécifiée au chargement de la classe.
  \frenchbsetup{%
    StandardItemizeEnv=true,       % format standard des listes
    ThinSpaceInFrenchNumbers=true, % espace fine dans les nombres
    og=«, fg=»                     % caractères « et » sont les guillemets
  }

  %% Style de la bibliographie.
  \setcitestyle{authoryear, square}
  \bibliographystyle{plainnat}

  %% Déclarations des pages de titre. Remplacer les éléments entre < >.
  %% Supprimer les caractères < >. Couper un long titre ou un long
  %% sous-titre manuellement avec \\.
%  \titre{Application of infrared thermography in maintaining the ``cold chain" \&\& Exploration of cold approach in infrared thermography for Non-Destructive Testing}
  \titre{Cold food chain: Infrared thermography applied to the evaluation of insulation anomalies in refrigerated vehicles for the transport of food \&\& Exploration of cold approach in infrared thermography for Non-Destructive Testing}
  % \titre{Ceci est un exemple de long titre \\
  %   avec saut de ligne manuel}
  % \soustitre{Sous-titre le cas échéant}
  % \soustitre{Ceci est un exemple de long sous-titre \\
  %   avec saut de ligne manuel}
  \auteur{Lei Lei}
  \annee{2018}
  \programme{Doctorat en génie électrique}
  \direction{Xavier P.V. Maldague, directeur de recherche}
  % \codirection{<Prénom Nom>, <codirecteur ou codirectrice> de recherche}
  % \codirection{<Prénom Nom>, <codirecteur ou codirectrice> de recherche \\
  %              <Prénom Nom>, <codirecteur ou codirectrice> de recherche}

\begin{document}

\frontmatter                    % pages liminaires

\pagestitre                     % production des pages de titre

\chapter*{Résumé}                      % ne pas numéroter
\phantomsection\addcontentsline{toc}{chapter}{Résumé} % inclure dans TdM

\begin{otherlanguage*}{french}
  Le coût croissant de l'énergie a fait de l'économie d'énergie une nécessité vitale dans le monde actuel. Un des exemples consiste à ``maintenir la chaîne du froid", c'est-à-dire le transport correct des aliments périssables dans les véhicules réfrigérés, en particulier pour les produits laitiers, la viande et les aliments congelés. Tout en conservant une conservation appropriée des denrées alimentaires, l'ATP (Agreement on Transport of Perishable Foodstuffs) est l'un des accords concernant les essais d'isolation thermique qui déterminent l'adéquation du transport.
  
  Le test standard ATP est une procédure pour mesurer l'état isolant des équipements avec une approche globale. Néanmoins, certains défauts locaux dans la structure de l'équipement ne peuvent pas être visualisés dans cette procédure. Dans ce contexte, la technique de thermographie pourrait être particulièrement utile à ces problèmes. Deux exemples de cette application sont présentés dans cette thèse, l'un d'eux se concentre sur la cartographie du flux de chaleur sur la surface externe d'un rouleau-conteneur isolé par la technique de thermographie infrarouge. La seconde tente d'établir une vue panoramique du flux de chaleur sur la surface interne d'un véhicule isolé.
  
  Encouragé par les résultats favorables précédents, une exploration de l'approche à froid dans la thermographie infrarouge pour les Tests Non-Destructifs et l'Évaluation est introduite et réalisée dans ce qui suit. Une approche se concentre sur la détection des défauts isolés et des ponts thermiques dans les panneaux de caisses de camions isolés par chauffage à lampe et refroidissement par air, deux moyens d'excitation opposés. L'autre examine un refroidissement à l'azote liquide appliqué à un échantillon d'acier avec des trous à fond plat de différentes profondeurs et tailles.
  
  Différentes méthodes de traitement des données et de modélisation et de simulation sont effectuées dans des chapitres connexes.
\end{otherlanguage*}
                % résumé français
%!TEX root = gabarit-doctorat.tex
\chapter*{Abstract}                      % ne pas numéroter
\phantomsection\addcontentsline{toc}{chapter}{Abstract} % inclure dans TdM

\begin{otherlanguage*}{english}
   The increasing cost of energy has made energy saving a vital necessity in the current world. One of the examples involves, ``Maintaining the cold chain", which is the correct transport of perishable foodstuffs in refrigerated vehicles, especially for dairy products, meat and frozen foods.  In this respect a suitable thermal insulation implemented in refrigerated vehicles is essential for saving energy while maintaining an appropriate conservation of the foodstuffs. ATP (Agreement on Transport of Perishable Foodstuffs) is one of the agreements concerning thermal insulation tests ensuing the suitability of the transport.
   
   The ATP standard test is a procedure to measure the insulating status of equipment with a global approach. Nonetheless, some local defects in the structure of equipment cannot be visualized in this procedure. The thermography technique could be particularly helpful for these issues. Two examples of this application are presented in this thesis, one focuses on mapping the heat flux on the external surface of an insulated roll-container by infrared thermography technique. The second one attempts to establish a panoramic view of the heat flux on the internal surface of an insulated vehicle. 
   
   Encouraged by previous favorable results, an exploration of the cold approach in infrared thermography for Non-Destructive Testing \& Evaluation is introduced and performed herein. One approach focuses on the detection of insulated flaws and thermal bridges in insulated truck box panels by lamp heating and air cooling, two opposite means of excitation. The other approach investigates the application of  liquid nitrogen cooling to a steel specimen with flat-bottom holes of different depths and sizes.
   
   Different data processing methods and modeling and simulation are also carried out.% in related chapters. 
\end{otherlanguage*}
              % résumé anglais
\cleardoublepage

\tableofcontents                % production de la TdM
\cleardoublepage

\listoftables                   % production de la liste des tableaux
\cleardoublepage

\listoffigures                  % production de la liste des figures
\cleardoublepage

\nomenclature{CFT}{Continuous Fourier Transform}
\nomenclature{PCs}{Principal Components}
\nomenclature{PCT}{Principal Component Thermography}
\nomenclature{PPT}{Pulsed Phase Thermography}
\nomenclature{PT}{Pulsed Thermography}
\nomenclature{PCT}{Principal Component Thermography}
\nomenclature{PCT}{Principal Component Thermography}
\nomenclature{PCT}{Principal Component Thermography}

\printnomenclature
\phantomsection\addcontentsline{toc}{chapter}{List of Acronyms} 
\cleardoublepage

\dedicace{For Steven \& Alice}
\cleardoublepage

\epigraphe{A minute's success pays the failure of years.}{Robert Browning}
\cleardoublepage

%!TEX root = gabarit-doctorat.tex
\chapter*{Acknowledgements}         % ne pas numéroter
\phantomsection\addcontentsline{toc}{chapter}{Acknowledgements} % inclure dans TdM

Coming to study in Canada was the most important turning point of my life.

I would like to first give thanks my Ph.D supervisor, Xavier Maldague, who offered me the opportunity to come to Quebec and gave the chance to have three internships in Italy. I really appreciate his significant guidance, trustworthy support and advising encouragement during my Ph.D study. I want also to thank the member of thesis committee, Prof. Jean Dumoulin,  Prof. Pierre Servais and Dr. Stefano Sfarra, for spending part of their busy time on reading this manuscript and providing me beneficial comments and corrections.


Since one major part of this thesis is a joint project between the laboratory CNR-ITC of Padova in Italy and the Computer Vision and Systems Laboratory of University Laval, I’d like to express many thanks of gratitude to Dr. Paolo Bison and his research group—Alessandro Bortolin, Gianluca Cadelano, Giovanni Ferrarini, and the technical group—Stefano Rossi, Gianluca Cuccato, not only for all their kindness and help in preparation experiments and results discussions during my stay in Padova, but their suggestions and support of post-processing and redaction of conference papers as well. I really enjoyed the three stays in Italy, which made me learn more knowledge in research and certain applications in industry. In addition, the Italian foods have well refreshed my tastes.

On other hand, there was much help from the members of the Laboratory of Computer Vision and Systems. I thank all my colleges and friends from this laboratory. The union of discussing and spreading ideas, going to PEPS for sports, diner together at the weekend night, etc. all these have helped a lot to distract me out of some dull time during the research.

I would like also to thank all friends I've met in Quebec city. As known the winter here is much colder than one can imagine, but with them an home-feeling and warmhearted atmosphere has been created. 

Finally I'd like to express my gratitudes to my family, as they have always been a great source of mental energy for me. I have many reasons to say thank you to my parents, for their endless love, best wishes, unconditional support and so much confidence in me beyond my thought. I thank my wife Qilan for her patience, preparation of different delicious meals, and taking care of me considerately. I thank Steven, our one-year old son, for all the joy and happiness, lovely moments, invaluable experience that he has brought to us. 
         % remerciements
\chapter*{Preface}         % ne pas numéroter
\phantomsection\addcontentsline{toc}{chapter}{Preface} % inclure dans TdM

L'avant-propos est surtout nécessaire pour une thèse par article.
           % avant-propos

\mainmatter                     % corps du document

\chapter*{Introduction}         % ne pas numéroter
\phantomsection\addcontentsline{toc}{chapter}{Introduction} % inclure dans TdM

%Une thèse ou un mémoire devrait normalement débuter par une
%introduction. Celle-ci est traitée comme un chapitre normal, sauf
%qu'elle n'est pas numérotée.
Nowadays, the increasing cost of energy has made energy saving a vital necessity in the current world.
According to a research \citep{coulomb2008refrigeration}, about 15\% of all electricity consumed worldwide is devoured by refrigeration technology. Therefore, the correct transport of perishable foodstuffs in refrigerated vehicles, especially for dairy products, meat and frozen foods, known as ``Maintaining the cold chain", is one of the examples involved in. It has also become the key part fo every distributor's food safety program.

As one of the major applications of refrigeration technology, the cold chain concept has several main procedures as: production, processing, packaging, distribution and handling, retail and consumer's household \citep{Estrada-Flores2010}. It firstly starts at the procedure of production of raw materials, in which immediate cooling has been deployed to provide perfect quality. Then primary processing involves in with operations preparing the raw materials for a second processing stage.  In this stage, carcass chilling, precooling of fruits and vegetables or milk cooling and other cold chain operations are often represented. Transforming primary agricultural products into manufactured foods is performed in the secondary processing. After that, the cold chain continues in the form of freezing or chilling of foods, either in bulk or in packages. Food distribution and handling need to be performed always at controlled temperatures and planning of routes and schedules need to take into account the location and capacity of refrigerated distribution centres, the refrigerated modes (sea, air, land) and the volumes to be transported. Loading and unloading operations may be performed in refrigerated docks. Food at retail needs different levels of refrigeration: cold storage is required in supermarket distribution centres; preparation rooms  need to be air-conditioned; walk-ins and display cabinets need to be refrigerated. Finally, consumers store purchases of perishable products in domestic refrigerators.          % introduction
\chapter{State of the art}     % numéroté

Texte du chapitre.
             % chapitre 1
\part{Maintain the ``cold chain"}     % numéroté
%\phantomsection\addcontentsline{toc}{part}{Maintain the ``cold chain"} % inclure dans TdM
The following two chapters will present two published paper concerning the application of infrared thermography for Non-Destructive Testing \& Evaluation applied in ``Maintain the cold chain" procedure.

The first study mainly focused on mapping the heat flux on the external surface of an insulated roll-container by infrared thermography technique. The ATP standard measurement was performed to obtain the experimental results, meanwhile IR images of roll-container have been taken when the steady condition arrived, in order to analyze and compute the corresponding heat flux on entire surfaces. A simple thermal resistance model has been applied to realize the computation. Final temperature figures showed a good uniform distribution, and several defects in the structure like thermal bridges or air leakages have been identified. A reference zone of the external wall is measured by a thermal flux meter, then with that reference the whole surface heat flux map have been figured out. Besides, for a better view of the heat flux map, the homography technique has been performed into the raw images by applying a bilinear interpolation with the projective transformation matrix. The final corrected heat flux map has been demonstrated for each surface, in which the right one showed a smaller value than the others. 

In the second research, a panoramic view of the heat flux on the internal surface of an insulated vehicle by infrared thermography technique has been established in this study. The ATP measurement was performed to obtain the experimental results.  An IR camera was mounted on a pan-tilt head and automatically driven by a suitable software to map the temperature of the inner walls at the steady condition, in order to analyze and compute the corresponding heat flux on the entire surface.  The final result of the K-value obtained by IR thermography is accurate enough and compares well with that of the ATP test.

\chapter{Mapping of the heat flux of an insulated small container by infrared thermography}
The results of this study were presented at an oral session of the 24th IIR International Congress of Refrigeration 2015 Yokohama, Japan.

\section*{Résumé}
Cette étude portait principalement sur la cartographie du flux de chaleur sur la surface externe d'un conteneur-roulable isolé par thermographie infrarouge.
La mesure standard ATP a été réalisée pour obtenir les résultats expérimentaux, tandis que des images IR du conteneur-roulable ont été prises lorsque la condition stable a été atteinte, afin d'analyser et de calculer le flux de chaleur correspondant sur des surfaces entières. Un modèle simple de résistance thermique a été appliqué pour réaliser le calcul. Les chiffres de température finaux ont montré une bonne distribution uniforme, et plusieurs défauts dans la structure comme des ponts thermiques ou des fuites d'air ont été identifiés. Une zone de référence de la paroi externe est mesurée par un fluxmètre thermique, puis avec cette référence, la carte de flux de chaleur à la surface totale a été déterminée. Par ailleurs, pour une meilleure vision de la carte de flux thermique, la technique d'homographie a été réalisée sur les images brutes en appliquant une interpolation bilinéaire à la matrice de transformation projective. La carte de flux de chaleur corrigée finale a été démontrée pour chaque surface, dans laquelle la droite a montré une valeur plus petite que les autres.

\section*{Abstract}
This work mainly focused on mapping the heat flux on the external surface of an insulated roll-container by infrared thermography technique. The ATP standard measurement was performed to obtain the experimental results, meanwhile IR images of roll-container have been taken when the steady condition arrived, in order to analyze and compute the corresponding heat flux on entire surfaces. A simple thermal resistance model has been applied to realize the computation. Final temperature figures showed a good uniform distribution, and several defects in the structure like thermal bridges or air leakages have been identified. A reference zone of the external wall is measured by a thermal flux meter, then with that reference the whole surface heat flux map have been figured out. Besides, for a better view of the heat flux map, the homography technique has been performed into the raw images by applying a bilinear interpolation with the projective transformation matrix. The final corrected heat flux map has been demonstrated for each surface, in which the right one showed a smaller value than the others.


\textbf{\texttt{Contributing authors:}}

\textbf{Paolo Bison} (Research supervisor of CNR-ITC): developing protocol, student supervision, revision and correction of the manuscript. 

\textbf{Alessandro Bortolin} (Ph.D student of CNR-ITC): discussion and experiment preparation.

\textbf{Gianluca Cadelano} (Ph.D student of CNR-ITC): data collection, data analysis, discussion and experiment preparation.

\textbf{Giovanni Ferrarini} (Researcher of CNR-ITC): discussion in developing protocol, data collection, experiment preparation.

\textbf{\textsf{Lei Lei}} (Ph.D candidate):  experiment preparation, data analysis,  personnel coordination and manuscript preparation.

\textbf{Xavier Maldague} (Research director of LVSN in University Laval): student supervision, revision and correction of the manuscript.

\textbf{Stefano Rossi} (Ph.D, researcher of CNR-ITC): experiment planning and preparation.



\newpage
\section{Introduction}
Nowadays, the public is increasingly aware to the need of applying rigorous standards all along the ``food chain'', considering a better life with a good food supply.  Thus the transport of food in the refrigerated vehicles (such as trucks, trailers, containers, etc.), especially for dairy products, meat and frozen foods, is of great interest. Moreover, the ever increasing cost of energy incites limiting the minimum refrigeration. So it is essential to ensure a perfect thermal insulation at the vehicles inspection. 

There exits some agreements of the thermal insulation tests which ensures the suitability for the transport of food in refrigerated conditions---ATP.
``The Agreement on the International Carriage of Perishable Foodstuffs and on the Special Equipment to be Used for such Carriage (ATP) done at Geneva on 1 September 1970 entered into force on 21 November 1976'' \citep{Geneva1970}, which establishes standards for the international transport of perishable food between the states that ratify the treaty. It has been updated through amendment a number of times and as of 2013 has 48 state parties, most of which are in Europe or Central Asia. It is open to ratification by states that are members of the United Nations Economic Commission for Europe (UNECE) and states that otherwise participate in UNECE activities\citep{ATP_wiki}.
The details contents of the agreements can be found in \citep{ATP_wiki, rossi2009k}, therefore no more description in this report will be presented.

The institute (CNR-ITC) has extensive experience in the measurement of heat transfer coefficient $K$ applied to refrigerated vehicles \citep{rossi2009k,bison1993automatic,bison2012geometrical,dragano2009experimental,grinzato2010r}. It is also responsible for the certification of these vehicles throughout Italy.

In summary, the major methods and procedures for measuring and checking the insulation of the equipment during the ATP test, is following \citep{rossi2009k}: 
\begin{itemize}
	\item \textbf{insulated equipment} built with insulated envelope such that the heat exchanged between inside and outside is limited in such a way that the overall coefficient of heat transfer ($K$-value) is assignable into 2 classes: a)equal to or less than 0.7 $W/(m^2 K)$ for normally insulated equipment; b) equal to or less than 0.4 $W/(m^2 K)$ for heavily insulated equipment; 
	
	\item\textbf{refrigerated equipment} that are insulated equipment which utilize some source of cold like ice, eutectic plates, dry ice etc. This equipment, with an outside temperature of $30^{\circ}$C must be able to lower the inside temperature to: + $7^{\circ}$C (class A); $-10^{\circ}$C (class B); $-20^{\circ}$C (class C); $0^{\circ}$C (class D); 
	
	\item \textbf{mechanically refrigerated equipment} that  are  insulated  equipment  furnished  of  its  own  refrigerating appliance.  The  appliance,  with  an  outside  temperature  of  $30^{\circ}$C,  must  be  capable  of  lowering  the  inside temperature to: from $+12^{\circ}$C to $0^{\circ}$C (class A); from $+ 12^{\circ}$C to $-10^{\circ}$C (class B); from $+ 12^{\circ}$C to $- 20^{\circ}$C (class C); 
	
	\item \textbf{heated  equipment} that  can  heat  the  inside  (to  avoid  the  freezing  of  foodstuffs)  are  used  in  very  cold countries.
\end{itemize}
Especially, the overall coefficient of heat transfer ($K$) is defined as:
\begin{equation}
K = \frac{W}{S\cdot \Delta \theta}
\end{equation}
where W is  the  power  necessary  to  maintain  a  steady  temperature  difference $\Delta \theta$ between  the  mean internal and external air temperature of the equipment. $S$ is the mean surface of the equipment, given by the geometric mean of the inside and outside surface areas:
\begin{equation}
S = \sqrt{S_i \cdot S_e}
\end{equation}

The ATP standard test is a procedure to measure the insulating status of equipments with a global approach. Its robustness has been well demonstrated. While, on the other hand, some local defects in the structure of equipment, such as thermal bridges, air leakages or zones of anomalous aging, cannot be visualized in this procedure. Then the thermography technique could be particularly helpful to these issues. In fact, all the defects mentioned above lead to a variation of the heat flux and temperature on the surface of the equipment \citep{grinzatoquality,grinzato1comparison}. Therefore the local heat flux map of the equipment by infrared thermography could give a straightforward visualization of the structure, and also a local evaluation of the K-value.

The work in this study consists the modeling and simulation for an insulated roll container by COMSOL in theory part. In practice party, the experimentation will be performed to map the heat flux on its external surface by Infrared thermography. In addition, another test for a refrigerated truck will also be undertaken. Primary results will be analyzed and discussed. Some conclusions and perspectives come in the end.


\section{Theory \& Methods}
\subsection{The heat transfer model}
Analogy to an electrical circuit, the heat transfer can be modelled as the heat flux is represented by current, temperatures are represented by voltages \citep{Therm_Re}. Therefore, the resistors in the heat ``circuit" is then the thermal resistance. Symbolically Ohm’s law can be expressed as
\begin{equation}
I = \frac{\Delta V}{R_e}
\end{equation}
where $I$ is the current flowing through an element, $\Delta V$ is the voltage across the element, and $R_e$ is the electrical resistance across the element. With the observed analogy, Fourier’s law can be written similarly as
\begin{equation}
q = \frac{\Delta T}{R_t}
\end{equation}
where $q$ is the flux of heat conduction, $\Delta T$ is the temperature difference between the surfaces of a slab, and $R_t$ is the thermal resistance.

In this work, the thermal resistance model is applied to a roll-container, from inside to outside (More details about the container can be found in Chapter \ref{box_detail}). For the standard ATP requirement, a radiation heater is working inside the container to maintain a fixed a higher internal air temperature. After the steady conditions are reached, a heat power $W$ is delivered. Heat flow is transferred by convection from the hot inside air to the internal wall of the box, and then by conduction through internal wall to external wall, and again by convection from the external wall to the outside air, which is cooled by the ATP system. The total scheme is presented in Fig \ref{Therm_Res}
\begin{figure}[!htpb]
	\centering
	\includegraphics{mapping/Therm_Res}
	\caption{Overall thermal resistance of the roll-container}
	\label{Therm_Res}
\end{figure}

\noindent where $\theta_i$ and $\theta_e$ are the internal and external temperature of the container, respectively. $\theta_{wi}$ and $\theta_{we}$ are the internal and external wall temperature of the container. In essentially 1D hypothesis, $h_i$ and $h_e$ are respectively the convective heat exchange internal and external coefficients. $\lambda$ is the thermal conductivity of the container wall and $l$ its thickness, and $S$ is the mean surface of the box.


\subsection{Simulation by Comsol}

A simple simulation work for this model has been performed by COMSOL MultiPhysics, which helps to better understand the distribution of the temperature of final result. The measurements of box dimension can be found in Tab \ref{tab_box_dim}.

\begin{table}[h]
	\centering
	%\begin{tabular}{p{85pt}p{85pt}p{85pt}p{85pt}}
	\begin{tabular}{c|c|c|c||c}
		\hline
		Inside:  & $L_i=0.864$ m  & $D_i=0.613$ m  & $H_i=1.55$ m & Thickness\\
		\hline
		Outside:  & $L_o=0.988$ m & $D_o=0.74$ m & $H_o=1.67$ m  &  $50$ mm \\
		\hline
	\end{tabular}
	\caption{Roll container dimensions}
	\label{tab_box_dim}
\end{table}

Same as the ATP standard test, the box inside air temperature is set as $32.5^{\circ}$C, and the outside air temperature is kept as $7.2^{\circ}$C. 

The Heat Transfer in Solid was used during the simulation in this modelling. A heat flux is added inside the box between the air and the internal surface, with a free convection, then the coefficient is set as $h_i=15 W/(m^2 K)$. Another heat flux was added outside between the air and the external surface with a forced convection, by $h_e=25 W/(m^2 K)$ [referring \citep{airhe,htwiki}]. The material of the box is made by high density polyurethane foam, with a conductivity about 0.0026 $W/(mK)$ \citep{jarfelt2006thermal}. Two high conductive thin layer are added into the material, same as the experimental one. The study condition is set to the steady status, as in real test the whole period is more than 12 hours.

The simulation result is shown in Fig. \ref{3D_T}. 
\begin{figure}[!htbp]
	\centering
	\includegraphics[scale=0.65]{mapping/3D_T}
	\caption{Temperature distribution of the roll container in COMSOL}
	\label{3D_T}
\end{figure}

The distribution of temperature of the entire container presented above indicates an uniform result, since the model and simulation is in ideal condition. Neither air leakage no thermal bridge is found here.

The following two figures (\ref{cut_plane}) introduce two cut plane inside the box, which illustrate the temperature details in the layers. As one can see that, the heat transfers from inside to outside, with temperature decreasing from internal surface to external surface. Moreover, the high thickness in the corners lead to a less transferred heat, showing a lower corresponding temperature.
\begin{figure} [htbp]
	%	\centering
	\hspace{-20pt}
	\includegraphics[scale=0.40]{mapping/xy_plane}
	\includegraphics[scale=0.40]{mapping/2D_xy_plane}
	\vspace{5pt}
	\hspace{-20pt}
	\includegraphics[scale=0.40]{mapping/yz_plane}
	\includegraphics[scale=0.40]{mapping/2D_yz_plane}
	\caption{Temperature distribution of the cute planes}
	\label{cut_plane}
\end{figure}

\section{Experimental setup}
\subsection{Roll-container}
In this part the experimental installation will be presented. The aim of this work is to map the heat flux on the external surface of a roll-container, and Fig \ref{box} shows the installed probes on the surface of the roll container.

\begin{figure}[!htbp]
	\centering
	\includegraphics[scale=0.13]{mapping/DSC_0191}
	\includegraphics[scale=0.13]{mapping/DSC_0189}\\
	\vspace{4pt}
	\includegraphics[scale=0.13]{mapping/DC_51939}
	\includegraphics[scale=0.13]{mapping/DC_51941}
	\caption{The roll container used for the test[outside and inside(up-right)]}
	\label{box}
\end{figure}

This box is actually made by sandwich panels with three layers: two internal-external skins made by polyester-fiberglass and a core made by high density polyurethane foam, and total thickness is 50 $mm$. \label{box_detail}

12 points of measurements (thermal couple) are positioned at the 8 corners (inside and outside) and also at the center of 4 surfaces (front, left,back and right)[Fig \ref{therm_couple}]. All the thermal couples are at a distance of 10 cm from the wall.
\begin{figure}[!htbp]
	\centering
	\includegraphics[scale=0.40]{mapping/therm_couple}
	\caption{The position of the thermal couples}
	\label{therm_couple}
\end{figure}
In addition, on the upper-center of the front surface, a heat flux meter is set to measure the corresponding heat flux (Fig \ref{box} left). The acquisition system used in this work is an infrared camera--FLIR SC-660. It records the thermography image series during the whole test for the front and left surfaces of container (schema shown in Fig \ref{box} left). What's more, when the steady conditions were reached (a state period of time not less than 12 hours, according to the ATP standard \citep{rossi2009k}), several infrared images were captured for the external walls such as front, left, back, right and top one (impossible to capture the bottom surface). 

With standard ATP measurement during the test, a radiation heater is heating the inside the container to maintain a temperature about $32.5^{\circ}$C. For outside, with the air circulating a velocity between 1 and 2 $m s^{-1}$ in the tunnel, the temperature is maintained constant at about $7.2^{\circ}$C. Thus makes a temperature difference between inside air and outside of $\Delta \theta = 25.3^{\circ}$C.

Two hypothesis are proposed in this test: 1) the  heat exchange coefficients, inside ($h_i$) and outside ($h_e$) of the container are constant;\label{hyp1} 2) the heat diffusion is mainly 1D. These hypothesis serve the computation of heat flux for each external surface in the following chapter.

\subsection{Refrigerated Vehicle}

Even though the main work is testing on the roll container, one has also applied the same experimental setup to a refrigerated truck (Fig \ref{truck}), which refers to the main objective of the ATP agreements.
\begin{figure}[!htbp]
	\hspace{-10mm}
	\includegraphics[scale=0.12]{mapping/DSC_0210}
	\includegraphics[scale=0.12]{mapping/DSC_0211}\\
	
	\includegraphics[scale=0.12]{mapping/DSC_0209}
	\includegraphics[scale=0.12]{mapping/DSC_0213}
	\caption{The refrigerated vehicle used for the test}
	\label{truck}
\end{figure}

For this test, only thermography images of several surfaces (top, left, back and right) have been captured when the steady condition was reached. Besides, the heat flux meter was first set on the left surface (Fig \ref{truck} upper-left) and then was moved to test the back surface (Fig \ref{truck} lower-right). The former result contains the influence of air streaming whose velocity is between 1 and 2 $ms^{-1}$, while in the later one, the ventilation system had been switched off.

\section{Results \& Discussion}
\subsection{Heat flux map}
The idea is to measure the heat flux by a thermal flux meter in a reference zone outside of the insulated envelope and to build successively the whole map of the heat flux as a linear relation of the temperature difference between the outside wall and the air.

The specific heat flux at a reference zone of the external wall is measured by a thermal flux meter, then with the wall temperature in the proximity of the flux meter and by measuring the air temperature, one can draw the local approach like \citep{rossi2009k}:
\begin{equation}
q_r = \frac{\theta_{we}(x_r,y_r)-\theta_e}{1/h_e}
\end{equation}
which gives that:
\begin{equation}
h_e = \frac{q_r}{\theta_{we}(x_r,y_r)-\theta_e}
\end{equation}
where $q_r$ is the heat flux measured at the reference point ($x_r,y_r$) by the thermal flux meter. $\theta_{we}$ is the temperature measured by thermography at the reference point. With the hypothesis 1) mentioned in Section \ref{hyp1}, the heat flux map of the whole surface could be determined with the temperature map according to:
\begin{equation}
q(x,y) = \frac{q_r}{\theta_{we}(x_r,y_r)-\theta_e}(\theta_{we}(x,y)-\theta_e)
\label{eq_q}
\end{equation}
Where $\theta_{we}(x,y)$ is the temperature at each point of the external surface. This equation indicates that the  heat flux has a linear relation of the temperature difference between the outside wall and the air.

Finally the temperature of all the surfaces are distributed as Fig \ref{IR_box}:
\begin{figure}[!htbp]
	%	\hspace{-10mm}
	\centering
	\includegraphics[scale=0.50]{mapping/IR_front_m}
	\hspace{6pt}
	\includegraphics[scale=0.50]{mapping/IR_back_m}
	\vspace{3pt}
	\includegraphics[scale=0.50]{mapping/IR_left_m}
	\hspace{6pt}
	\includegraphics[scale=0.50]{mapping/IR_right_m}
	\includegraphics[scale=0.50]{mapping/IR_top}
	\caption{The temperature map of the roll container (front, back, left, right and top surfaces)}
	\label{IR_box}
\end{figure}
All these IR images were captured at steady condition. 
In the figure, a good uniform distribution of temperature was well shown, even though there are several abnormal lines which may be the thermal bridges or air leakages (at front, left, right and top surfaces). An attention might be paid is that there is an area of the light reflection on the container surface (on the second image left part). It could be no influence here as the back surface is the main part of the second figure.

Comparing all the surface temperature, one can see that on the right surface of container, the temperature value is a little lower than other surfaces. This result may conclude the Hypothesis 1) \ref{hyp1} is not fully correct, then the convective heat transfer coefficient could be different around the roll container during the test.

The temperature at the reference point around the thermal flux meter was presented in the figure (the first one), with that one could figure out the map of heat flux of the entire surface by Eq \ref{eq_q}. And the data measured by the thermal flux meter are presented in Fig \ref{flux_meter}.
\begin{figure}[!htbp]
%	\hspace{-35pt}
	\centering
	\includegraphics[scale=0.65]{mapping/It_project_2014_QProfile}
%	\vspace{20pt}
%	\hspace{-35pt}
	\includegraphics[scale=0.65]{mapping/It_project_2014_TProfile}
	\caption{Data from thermal flux meter (heat flux and temperature profiles)}
	\label{flux_meter}
\end{figure}
These data recorded the whole test. Thought the fluctuation is very high, ones took the steady period after 5 hours from the beginning. Then mean values are $q_r=9.73 W/(m^2 K)$ and $T_r = 7.83 ^\circ$C, which served the computation of heat flux.

Moreover, for a better view, some techniques are often applied for images corrections \citep{bison2012geometrical}. In this work, an easier way to realize the image correction is to apply the homography technique.

\subsection{Homography application}
In practical applications, such as image rectification, image registration, or computation of camera motion—rotation and translation—between two images, a homography is often performed in the field of computer vision \citep{homo_wiki}.

In this work, as one knows the radio between the length and the width of the roll container (or between the width and the height in same way), a projective transformation matrix in images has been obtained. Then applying a bilinear interpolation with the projective transformation matrix, into the raw images, one finally got the corrected heat flux mapping of each surfaces, demonstrated in Fig \ref{Q_box}.
\begin{figure}[!htbp]
	%	\hspace{-10mm}
	\centering
	\includegraphics[scale=0.55]{mapping/Q_front_m}
	\hspace{5pt}
	\includegraphics[scale=0.55]{mapping/Q_back_m}
	\includegraphics[scale=0.55]{mapping/Q_left_m}
	\hspace{5pt}
	\includegraphics[scale=0.55]{mapping/Q_right_m}
	\includegraphics[scale=0.55]{mapping/Q_top_m}
	\caption{The corrected heat flux map of the roll container (front,  back, left, right and top surfaces)}
	\label{Q_box}
\end{figure}

The Heat flux map exhibits that in right surface of the roll container, one gets smaller heat flux value rather than those on other surfaces, such as front, top, back and left.
On the other hand, the heat flux map of back, top and left surface show a little higher mean value than that of the front one, which is due to the air stream.

A contradiction is found that, normally the right surface's heat flux should be a bit greater than the front surface where the air stagnation is probable, as air stream flows the lateral surface, and that leads to a lower temperature in the right surface. While in the thermography images, one got the contrary result. The reason for this is due to the linear relationship between heat flux and the difference of temperature between external surface and the air. Since in our model Eq.\ref{eq_q}, lower $\theta_{we}(x,y)$ would lead to a lower $q(x,y)$. This might also indicate that the influence of the convective heat transfer in lateral surface is more important than imagining.
Comparing to the standard ATP measurement, the global K-value obtained is about $0.53 W/(m^2 K)$, then multiplied by the temperature difference $\Delta \theta =25.3^\circ$C, thus it gives us a global heat flux value of the roll container $13.41 W/m^2$. With thermography one obtained the local heat flux in the map about $10.7 W/m^2$, which is more or less a good result.

``A local variation of thermal conductivity leads to a local increase of the heat flux with a consequent variation of the local internal and external wall temperature"\citep{rossi2009k}, this phenomena can be well seen in all figures above.

\subsection{Vehicle results}
The temperature map of several surfaces of the refrigerated truck are distributed in Fig \ref{IR_truck}.
\begin{figure}[!htbp]
	%	\hspace{-10mm}
	\centering
	\includegraphics[scale=0.50]{mapping/IR_truck_lt_m}
	\hspace{6pt}
	\includegraphics[scale=0.50]{mapping/IR_truck_rt_m}
	\vspace{3pt}
	\includegraphics[scale=0.50]{mapping/IR_truck_bk11_m}
	\hspace{6pt}
	\includegraphics[scale=0.50]{mapping/IR_truck_bk12_m}
	\includegraphics[scale=0.50]{mapping/IR_truck_tp_m}
	\caption{The temperature map of the refrigerated vehicle (left, right, back1, back2 and top surfaces)}
	\label{IR_truck}
\end{figure}
From all the figures, one can see a good uniform of the temperature distribution on all the external surfaces of the vehicle. However, the reflection effects are very heavy on the left, right and top surfaces (which are not the abnormal lines in figures). Moreover, in this test, lower temperature are found on the lateral surfaces than the back one, thanks to the air streaming.

%Two tests of the thermal flux have been performed: one on the left surface with air streaming; another on the back surface without ventilation system. The corresponding two final data profiles can be found in Fig \ref{truck_meter}.
%\begin{figure}[!htbp]
%	\centering
%	\includegraphics[scale=0.5]{mapping/Q_truck_left_profil}
%	\includegraphics[scale=0.5]{mapping/T_truck_left_profil}
%	\includegraphics[scale=0.6]{mapping/Q_truck_back_profil}
%	\includegraphics[scale=0.6]{mapping/T_truck_back_profil}
%	\caption{Data from thermal flux meter for the refrigerated vehicle (up:left surface; down:back surface)}
%	\label{truck_meter}
%\end{figure}
%The mean values of all these data in steady status are in following table (Tab \ref{tab_truck}).
%\begin{table}[h]
%	\centering
%	%\begin{tabular}{p{85pt}p{85pt}p{85pt}p{85pt}}
%	\begin{tabular}{l|r}
%		\hline
%		left surface  & back surface  \\
%		\hline
%		$\overline{q}=10.34 W/(m^2 K)$ & $\overline{q}=5.62 W/(m^2 K)$  \\
%		\hline
%		$\overline{T}=7.84^\circ$C & $\overline{T}=8.97^\circ$C  \\
%		\hline
%	\end{tabular}
%	\caption{Thermal flux meter data for truck}
%	\label{tab_truck}
%\end{table}
%
%The huge difference (45\% error) between the heat flux of the back and left surface makes no sense, since two test were almost at the same condition. Therefore, that might because when switching the thermal flux meter from left surface to back surface, something wrong has been done to influence the accuracy of the equipment, like not attaching well to the surface. Another consideration is that, for the back surface test, the measuring time was not long, as it has already arrived in the steady condition. While in the profiles, a little increasing tendency in temperature could be observed. This may indicate again that the vehicle result of back surface was not so good.

As the main work of the roll container has been done, more tests on the truck will be undertaken in future measurements.


\section{Conclusion \& Perspectives}

%\addcontentsline{toc}{chapter}{Conclusion \& Perspectives}
This preliminary work mainly focused on mapping the heat flux on the external surface of an insulated roll-container using Infrared thermography technique. The ATP standard measurement was performed to obtain the experimental results, meanwhile IR images of the roll-container have been taken when the steady condition was reached, in order to analyze and compute the corresponding heat flux on the entire surface. A simple thermal resistance model has been applied to conduct the computation. Final temperature figures showed a good uniform distribution, and several defects in the structure such as thermal bridges or air leakages have been identified. A reference zone of the external wall is measured by a thermal flux meter, then with that reference the entire surface heat flux map have been figured out. Moreover, for a better view of the heat flux map, the homography technique has been performed into the raw images by applying a bilinear interpolation with the projective transformation matrix. The final corrected heat flux map has been demonstrated for each surface, in which the right surface showed a smaller value than the others. Due to the air streaming, temperatures in the lateral surfaces were a little smaller than other surfaces, thus this leads to the smaller heat flux values. That also indicated the convective heat transfer coefficient was not constant around the roll-container surfaces, contrary to our Hypothesis 1 in theory.

For the refrigerated vehicle test, two surfaces (left and back) have been taken into measurement with the thermal flux meter. The former one was tested with the air streaming, while the ventilation system was switched off for the later one. The final IR images presented a big reflection influence on the lateral surfaces. And a huge difference between the heat flux value on two test surfaces was found, which might be because the equipment was not attached well to the surface when moved. 

For future work, more tests will be taken place on refrigerated vehicle with two thermal flux meters measuring on the same time, to avoiding the problem encountered in this work. And the suppression of thermal reflections in thermal imaging \citep{vollmer2004identification} may be taken into consideration during the IR image processing. 

This project topic can be also extended to buildings where the monitoring of the effective transmittance is crucial for the energy saving \citep{grinzato2010r}.

\bibliographystyle{plainnat}              % style de la bibliographie
\bibliography{U:/Desktop/Bibliography/Biblio_th} 
             % chapitre 2, etc.

%\includepdf[pages=-]{Prop_these_lei}

%!TEX root = gabarit-doctorat.tex
\chapter{Panoramic view of the heat flux inside the vehicle}     % numéroté
(Published on-line  in the Quantitative InfraRed Thermography Journal, in February 2018).
The results of this study were presented at an oral session of the 1st QIRT Asia Conference 2015 Mamallapuram, India. % Then it was selected for the publication in Quantitative InfraRed Thermography Journal.
\section{Résumé}
Une caméra IR est montée sur une tête panoramique et automatiquement entraînée pour cartographier la température des parois internes du conteneur isolé sur un véhicule réfrigéré. Le véhicule est introduit dans une chambre d'essai où des conditions spéciales sont appliquées, de manière à maintenir une différence de température interne-externe d'environ 25 °C. L'objectif de ce travail est de compléter les résultats qualitatifs obtenus par thermographie infrarouge, avec une évaluation du flux de chaleur traversant la paroi.

\section{Abstract}
An IR camera is mounted on a pan-tilt head and automatically driven to map the temperature of the inner walls of the insulated container on a refrigerated vehicle. The vehicle is introduced in a test chamber where special conditions are applied, in such a way to maintain an inner-outer temperature difference of around 25 °C. The objective of this work is that of complementing the qualitative results obtained by infrared thermography, with an evaluation of the heat flux flowing through the wall.

\textbf{\texttt{Contributing authors:}}

\textbf{\textsf{Lei Lei}} (Ph.D candidate): developing protocol, experiment preparation, data analysis,  personnel coordination and manuscript preparation.

\textbf{Alessandro Bortolin} (Ph.D student of CNR-ITC): discussion and experiment preparation.

\textbf{Gianluca Cadelano} (Ph.D student of CNR-ITC): data collection, data analysis, discussion and experiment preparation.

\textbf{Giovanni Ferrarini} (Researcher of CNR-ITC): discussion in developing protocol, data collection, experiment preparation.

\textbf{Stefano Rossi} (Ph.D, researcher of CNR-ITC): experiment planning and preparation.

\textbf{Paolo Bison} (Research supervisor of CNR-ITC): student supervision, revision and correction of the manuscript. 

\textbf{Xavier Maldague} (Research director of LVSN in University Laval): student supervision, revision and correction of the manuscript.


% \phantomsection\addcontentsline{lot}{table}{3.1\quad Insulated container dimensions}
% \phantomsection\addcontentsline{lot}{table}{3.2\quad ATP test results}

% \phantomsection\addcontentsline{lof}{table}{3.1\quad Overall thermal behavior of the container being tested represented as an analogy of an electrical circuit.}
% \phantomsection\addcontentsline{lof}{table}{3.2\quad The inside of the insulated container used for the test.}
% \phantomsection\addcontentsline{lof}{table}{3.3\quad Spherical projection.}
% \phantomsection\addcontentsline{lof}{table}{3.4\quad Original IR image (left) and its spherical projection (right).}
% \phantomsection\addcontentsline{lof}{table}{3.5\quad Harris corners detected}
% \phantomsection\addcontentsline{lof}{table}{3.6\quad Estimation of translation between images.}
% \phantomsection\addcontentsline{lof}{table}{3.7\quad Stitching of images.}
% \phantomsection\addcontentsline{lof}{table}{3.8\quad Temperature panorama of the inside of the vehicle.}
% \phantomsection\addcontentsline{lof}{table}{3.9\quad Heat Flux meter measurement.}
% \phantomsection\addcontentsline{lof}{table}{3.10\enspace Heat Flux panorama of the inner part of the vehicle.}
% \phantomsection\addcontentsline{lof}{table}{3.11\enspace Schema of calculating real size of each image (Left: Side view; Right: Plan view).}
% \phantomsection\addcontentsline{lof}{table}{3.12\enspace One element surface map (up) and the area panorama of the inner part of the vehicle.}

% \includepdf[pages={2-14}]{Lei2018Panoramic}

\newpage
\section{Introduction}
According to a recent report \citet{Zion2016}, the global cold chain equipment (that is storage plus transport) market is growing rapidly and will reach an amount of about 120 billion of USD by 2021. In particular, the sector of refrigerated trailers is expected to expand at a $ 4.4\% $ CAGR (Compound Annual Growth Rate) in the period 2015-2021 \citet{RM2015}. Nowadays it could be estimated that there is a total of about four million refrigerated vehicles in use around the world \citet{UNEP2010}. It is thus clear that the demand for a high quality and safe transport is increasing worldwide, especially in the Asia Pacific region and this involves not only the perishable foodstuffs (such as frozen, dairy, meat, fish and seafood products), but also vaccines, drugs, chemical pharmaceuticals and in general all goods that are temperature-sensitive. 

Every actor (stakeholders, customers, transport companies, etc.) in the field of refrigerated transport is focusing increasingly on the idea of an efficient and energy saving transport. Energy consumption is, therefore, becoming one of the major aspects studied in the recent times \citet{Tassou2009,Cavalier2010,Adekomaya2017}. 

Therefore thermal insulation is a clearly key factor in the global account of the energy consumption applied to the system transport and recent research has attempted to obtain and utilize new materials in order to ensure a lower thermal conductivity of the walls of the insulated vehicles \citet{Tinti2014,Lawton2016}.

Moreover, there are many international regulations, standards, agreements related to the measure of the thermal insulation properties. Among others, there are ISO 1496/2 (thermal container), American Bureau of Shipping (certification of Cargo Container) and ATP (Agreement on Transport of Perishable Foodstuffs).

The ATP agreement \citet{Geneva1970}, even if it’s not a standard, can be considered as a reference “de facto” in the field of refrigerated transport in Europe, North Africa and Central Asia. The principal methodology to assess the quality of insulation of a refrigerated vehicle is the measure of the overall heat transfer coefficient (referred to as the K-value) defined as:
\begin{equation}
K=\frac{W}{S⋅\Delta \theta}
\end{equation}


where $ W $ is the power necessary to maintain a steady temperature difference $ \Delta \theta $ between the mean internal and external air temperature of the container and S is the mean surface of the container, given by the geometric mean of the inside ($ Si $) and outside ($ Se $) surface areas:

\begin{equation}
S=\sqrt{S_i∙S_e}
\end{equation}

CNR-ITC is one of the officially designated test station in Italy and has a consolidated experience (more than 30 years) in the measuring of the K-value \citet{rossi2009k}.

The result of this type of measure is a global indicator of the insulation quality of the container, but cannot give any information concerning potential local defects in the structure, such as thermal bridges, air leakages or zones of anomalous ageing. IR thermography can be particularly helpful regarding these issues. In fact, all of the defects mentioned above lead to a variation of the heat flux and temperature on the surface of the container \citet{grinzato2010r, grinzatoquality, grinzato1comparison}. Therefore, the local heat flux map obtained by infrared thermography could give a straightforward image of the structure, and also a local evaluation of the transmittance.

This work consists of the image processing applied to an insulated container on a refrigerated vehicle \citet{bison1993automatic,bison2012geometrical}. In practice, the experiments will be performed so as to map the heat flux on each one of its internal surfaces by infrared thermography. When considering all the surfaces together, this will provide a panoramic view of the heat flux.


\section{The heat transfer model}

In what follows we describe a simplified model of the heat transfer from inside and the outside of the equipment being tested. In order to meet the requirements of ATP a heater is working inside the insulated vehicle. When steady conditions are reached it delivers a power $ W $ in order to maintain a steady air temperature $ \theta_i $ inside the container. Since the external temperature is lower than the internal one, heat flows from the inside to the outside of the container where a steady air temperature is maintained at the level of $ \theta_e $. Heat is transferred by convection from the hot inside air to the internal surface of the wall, by conduction through the wall and again by convection from the external surface of the wall to the cold outside air according to the diagram shown in Figure~\ref{Therm_Res}. 

In the hypothesis that the heat flux is essentially 1D,  then $ h_e $ and $ h_i $ are respectively the effective external and internal heat exchange coefficients. $ \theta_{we} $ and $ \theta_{wi} $ are the average temperature of the external and internal walls respectively, while $\lambda $ is the thermal conductivity of the wall and $l$ is its thickness. The central thermal resistance of Figure~\ref{Therm_Res} is the sum of the individual thermal resistances given by the different layers of the insulated wall. In many cases three layers compose the walls of an insulated container mounted on a truck: two internal-external skins made of polyester-fiberglass and a core made of high density polyurethane foam.
\begin{figure}[ht]
    \centering
    \includegraphics{chp3/Fig.1.png}
    \caption{Overall thermal behavior of the container being tested represented as an analogy of an electrical circuit}
    \label{Therm_Res}
\end{figure}


What is seen by looking with thermography at the internal wall of the insulated container is a uniform temperature with possible local variations that could depend on the reduction of insulation. For instance, the joint zone of two insulation panel modules could produce a more or less severe thermal bridge due to the structural elements present in the joint zone connection that are typically more conductive than the core polyurethane foam. This means that a local variation of thermal conductivity leads to a local increase of the heat flux with a consequent variation of the local internal and external wall temperature. Since thermography can map the internal temperature of the insulated container, it is more convenient to shift the focus from a global view to a local view of the heat transfer. In fact, by measuring the specific heat flux in a reference zone of the internal wall using a thermal flux meter, and by looking at the wall temperature in the proximity of the flux meter and by measuring the air temperature, the following local equation can be drawn:

\begin{equation}
q_r=\frac{\theta_i-\theta_{wi}(x_r,y_r )}{1/h_i} = h_i*[\theta_i-\theta_{wi}(x_r,y_r )]
\end{equation}


where $ q_r $ is the specific heat flux measured at the reference point by the heat flux meter and $ \theta_{wi}(x_r,y_r) $ is the temperature measured on the internal wall by thermography at the coordinate of the reference point, or very close to it. The assumption that the internal heat exchange coefficient $ h_i $ does not vary significantly with the position allows the temperature and flux measurements in the reference point to be taken as a pivot, chosen after a preliminary thermographic scanning. The heat flux map may be determined based on the temperature map obtained by thermography according to the following equation:
\begin{equation}
q(x,y)=h_i*[\theta_i-\theta_{wi} (x,y)]=q_r-h_i*\Delta \theta_{x,y}
\end{equation}
where $\Delta \theta_{x,y}=\theta_{wi} (x,y)- \theta_{wi} (x_r,y_r) $.\\
Equation 4 gives $ q(x,y) $ as a linear relation depending on  $ \Delta \theta_{x,y} $, with intercept $ q_r $ and slope $ h_i $. It shows that the mapping of heat flux depends linearly on the measurement of the temperature difference between any point of the surface and the reference point (located close to the position of the heat flux meter). The mapping of the temperature difference is one of the strong points of IR thermography that, in high quality equipment, can reach a NETD (Noise Equivalent Temperature Difference) as low as 20 mK. 
In the case where the surface temperature variation is statistically distributed around the $\theta_{wi} (x_r,y_r)$ value, $\Delta \theta_{x,y}$ will be very close to zero on average. Consequently, an average value of the heat  flux very close to the reference value $q_r$ is expected. In such a case, the uncertainty of the average value of $q(x, y)$ depends essentially on the accuracy of the $q_r$ value, that we can estimate within ±5\%. When a thermal bridge is mapped, $\theta_{wi} (x,y)$ is typically lower than $\theta_{wi} (x_r,y_r)$ and $\Delta \theta_{x,y}$ becomes negative. Consequently, $q(x,y)$ becomes greater than $q_r$. In this case the correct evaluation of the slope parameter $h_i$ is also important since it depends both on the accuracy of the $q_r$ measurement and on that of $(\theta_i - \theta_{wi}(x_r,y_r))$.
% In the case where the surface temperature variation is statistically distributed around the $ \Delta \theta_{wi}(x_r,y_r) $ value, $ \Delta \theta_{x,y} $ will be very close to zero on average and we expect to obtain an average value of the heat flux very close to the reference value $ q_r $. In such a case the uncertainty of the average value of $ q(x,y) $ depends essentially on the accuracy of the $ q_r $ that we can estimate within $ ± 5\% $. In this case, the thermal bridges $ \Delta \theta_{x,y}$ becomes negative on average. Consequently, $ q(x,y) $  will be greater than $ q_r $. In this case the correct evaluation of the slope parameter $ h_i $ is also important since it depends both on the accuracy of the $ q_r $ measurement and on that of $ \Delta \theta_{r} $.

\section{Experimental setup}
An IR camera is mounted on a pan-tilt head and automatically driven by a suitable software to map the temperature of the inner walls of the insulated container on a refrigerated vehicle. Figure~\ref{Exp_setup} shows the Pan-Tilt camera installed inside the truck. The angles of the Pan-Tilt camera are set as follows: panning from 0° to 360° with 20° steps; tilting from 0° to 180° with 15° steps. A heat flux meter is set to measure the corresponding heat flux on a suitable zone, possibly far from thermal bridges. The acquisition system used in this work is a low-cost IR camera FLIR A320, with a field of view (FOV) $ 25°×19° $. It records IR images of all of the inner walls during the entire test. Furthermore, when the steady conditions are reached (which generally takes up to 8 hours, according to the ATP test), several rounds of IR images have been collected.

According to the ATP, the test is carried out by means of a combined convection-radiation heater that heats the inside of the insulated container at a temperature around 32.5 °C. The vehicle, with its insulated container is located inside an isothermal tunnel where the air is flowing at a velocity between 1 and 2 ms$^{−1}$ and the temperature is maintained constant at about 7.5 °C. This allows a temperature difference of $ ∆\theta  = 25 $ °C to be obtained between the inside and the outside of the insulated container. Two hypotheses are proposed in this test: 1) the heat exchange coefficients, inside ($ h_i $) and outside ($ h_e $) of the container are constant; 2) the heat diffusion is mainly 1D. These assumptions permit the computation of the heat flux for each surface. Moreover, for the correction of the emissivity with the angle, according to the Lambert cosine law \citet{dragano2009experimental,Hottel1967a}, the observed intensity is equal from a normal direction and an off-normal direction. Therefore, it is negligible within the angles used  in this inspection.
\begin{figure}[ht]
    \centering
    \includegraphics[scale=0.35]{chp3/Fig.2.jpg}
    \caption{The inside of the insulated container used for the test.}
    \label{Exp_setup}
\end{figure}


\section{Image processing}
The use of a pan-tilt head ensures a uniform coverage of the visual field with $ 320×240 $ pixels full view and allows the images to be stitched using cylindrical or spherical coordinates by pure translation. Normal panoramas involve complex image-processing including translation, rotation and perspective projection, etc. However, cylindrical and spherical panoramas are commonly used because of their ease of construction. In our case, a spherical panorama is more convenient. Szeliski and Shum [18] explain in detail how to compute an approximate spherical projection using the focal length. A brief conversion is shown in Figure~\ref{Sph_pro}. In which $X, Y, Z$ are the 3D point coordinates of the object, and $\hat{x}, \hat{y}, \hat{z}$ are the corresponding spherical coordinates. Then $\tilde{x}, \tilde{y}$ are the coordinates in spherical images. $f$ is the focal length, $\theta$ is polar angle and $\phi$ (or $\varphi$) is azimuthal angle. Therefore, given the focal length $ f $ and the image coordinates $ (x, y) $, the corresponding spherical coordinates $ (x', y') $ are:
\begin{align}
x'={} f·tan(\dfrac{x-x_c}{f})+x_c \notag \\
y'=f·\frac{tan(\dfrac{y-y_c}{f})}{cos(\dfrac{x-x_c}{f})} +y_c
\end{align}

where $ (x_c,y_c) $ are the center coordinates of the spherical image.
\begin{figure}[ht]
    \centering
    \includegraphics[scale=0.5]{chp3/Fig.3a.jpg}
    \includegraphics[scale=0.5]{chp3/Fig.3b.jpg}
    \caption{Spherical projection.}
    \label{Sph_pro}
\end{figure}

The original image and its spherical projection are shown in Figure~\ref{Orig_sph}.
\begin{figure}[ht]
    \centering
    \includegraphics[scale=0.44]{chp3/Fig.4a.jpg}
    \includegraphics[scale=0.44]{chp3/Fig.4b.jpg}
    \caption{Original IR image (left) and its spherical projection (right).}
    \label{Orig_sph}
\end{figure}

The next step involves stitching all of the spherical images together to create a panorama. Here one attempts to detect and extract the main features in each image by using the Harris corners \citet{Harris1988}. One result is shown in Figure~\ref{Harris} which indicates that the features obtained by automatic detectors between images fail to match. This might be due to the narrow view of the IR camera and the similarity of the internal surface of the vehicle. In this case, one has to proceed with the manual image stitching.  The spherical projection allows a full panorama to be created using only translations without rotation \citet{Szeliski1997}, which can help to create the full panorama efficiently and easily.
\begin{figure}[ht]
    %   \centering
    \includegraphics[scale=0.65]{chp3/Fig.5a.png}
    \includegraphics[scale=0.65]{chp3/Fig.5b.png}
    \caption{Harris corners detected (green crosses) in the Fig 4 and its precedent image of the whole series. In these two continuous images, the IR camera scanned from top to bottom inside the vehicle and there is a ``clear" (from our vision) overlapping part (left bottom part in the first and left top part in the second) in these two images. However, from the Harris corners detection results, all the features obtained could not be matched correctly.}
    \label{Harris}
\end{figure}

To estimate the translation between images, a scheme is elucidated in Figure~\ref{Trans}, where $ n $ is the width of each image, $ \Delta n $ is the displacement for image translation, for a given Pan angle $ (20°) $ and a field of view $ (25° \times 19°) $. Therefore $ \Delta n $ can be obtained by a simple calculation:
\begin{equation}
\Delta n=\dfrac{n}{2}[1-tan(20°-\dfrac{25°}{2})\cdot tan(90°-\dfrac{25°}{2})]
\end{equation}

\begin{figure}[ht]
    \centering
    \includegraphics[scale=0.45]{chp3/Fig.6.jpg}
    \caption{ Estimation of translation between images.}
    \label{Trans}
\end{figure}

Once the translations have been estimated, it is possible to overlay the two contiguous images together. In this case, for the blended regions between images, the \textit{max} function of Matlab, which returns the maximum value in the blended regions of two images, is used. The reason is that, since after the spherical projection of the original images, there are several regions with $ NaN $ values, it is not possible to obtain average from the two overlapping images. The iterate calculation is shown in Figure~\ref{img_sti} and the result in the temperature panorama of the inside of the vehicle is shown in Figure~\ref{Pano_T_Final}.
\begin{figure}[ht]
    \centering
    \includegraphics[scale=0.4]{chp3/Fig.7.jpg}
    \caption{Stitching of images.}
    \label{img_sti}
\end{figure}

\begin{figure}[ht]
%   \centering
    \hspace*{-20pt}
    \includegraphics[scale=0.3]{chp3/Fig.8.jpg}
    \caption{Temperature panorama of the inside of the vehicle.}
    \label{Pano_T_Final}
\end{figure}

\section{Heat flux panorama}
According to equations (3) and (4) the auxiliary measurement of the heat flux in a reference point allows the surface temperature image (panorama) to be transformed into a heat flux image (panorama). The measurement of the heat flux density in the reference point is shown in Figure~\ref{Flux_meter}. Once a steady condition has been reached (10 hours from the beginning of the test), the mean value is taken as the heat flux density reference. The panorama of the temperature is therefore transformed into the panorama of heat flux density by equation (4), which is displayed in Figure~\ref{Pano_Q_Final} . 
\begin{figure}[ht]
    \centering
    \includegraphics[scale=0.45]{chp3/Fig.9.png}
    \caption{Heat Flux meter measurement.}
    \label{Flux_meter}
\end{figure}

\begin{figure}[ht]
%   \centering
    \hspace*{-20pt}
    \includegraphics[scale=0.3]{chp3/Fig.10.jpg}
    \caption{Heat Flux panorama of the inner part of the vehicle.}
    \label{Pano_Q_Final}
\end{figure}



\section{Comparison with ATP test results }
The ATP test results can be found in Table~\ref{box_dim} and Table~\ref{ATP_res}.
\begin{table}[ht]
    \centering
    \scriptsize
    \caption{Insulated container dimensions.}
    \begin{tabular}{|l|c|c|c|c|c|}
        \toprule
        
         & \multirow{3}{*}{\centering LENGTH [m]} & \multirow{3}{*}{\centering WIDTH [m]} & \multirow{3}{*}{\centering HEIGHT [m]} & \multirow{3}{*}{\centering SURFACE [m$ ^2 $]} &  SURFACE \\
         & & & & &  GEOMETRIC\\
         & & & & &  MEAN [m$ ^2 $]\\
         \midrule
        INTERNAL & 7.890 & 2.455 & 2.705 & 94.71 & \multirow{2}{*}{98.98} \\
%       \hline
        EXTERNAL & 8.055 & 2.575 & 2.915 & 103.46 & \\
        \bottomrule
    \end{tabular}
    \label{box_dim}
\end{table}


\begin{table}[ht]
    \centering
    \small
    \caption{ATP test results.}
    \begin{tabular}{l|r}
        \toprule
        Fans Power [ W] (Mean over 6 hours) & 144 \\
        % \hline 
        Heaters Power [ W ] (Mean over 6 hours) &   988\\
        % \hline
        Internal temperature [ °C ] (Mean over 6 hours) &   32.5\\
        % \hline
        External temperature [ °C ] (Mean over 6 hours) &   7.5\\
        % \hline
        K-Value [ W/(K m$^2 $)] & 0.46 \\
        \bottomrule
    \end{tabular}
    \label{ATP_res}
\end{table}
The ATP K-value above is obtained using the Eq. (1). 
With the help of the spherical projection of each image, and from the specifications of the camera, the pan tilt movement pattern, the size of the truck, and the position of the apparatus inside the truck, one can eventually generated an Area Map Panorama, corresponding to the Temperature Panorama. The computation is illustrated in Figure~\ref{Image2}.
\begin{figure}[ht]
    \centering
    \includegraphics[scale=0.55]{chp3/Fig.11.png}
    \caption{ Schema of calculating real size of each image (Left: Side view; Right: Plan view).}
    \label{Image2}
\end{figure}

Note that in Figure~\ref{Image2}, $ L’ $ is the distance from the camera to the surface in each image, and $ S $ is the height (or width, depending on which angle was used in the FOV [$25°$ or $19°$]) of the image. $ D $ and $ L $ are the width and length of internal part of truck, respectively. Therefore, we have:
\begin{align}
H= & \frac{D}{2*cos\theta} \indent or\indent  H'=  \frac{L}{2*sin\theta} \\
L'= & \frac{H}{cos\varphi}   \indent or\indent  L' = \frac{H'}{cos\varphi}  \\
S= & 2*L'*tan⁡(\dfrac{v}{2})
\end{align}

Note here whether we use $ D $ or $ L $ to compute $ H $ or $ H’ $ depends on the angle between the camera direction and the vertical direction to the inside wall of the vehicle ($\varphi $).\\
The final result is shown in Figure~\ref{AM_map}.
\begin{figure}[ht]
    \centering
    \includegraphics[scale=0.8]{chp3/Fig.12a.png}\\
    \hspace*{-20pt}
    \includegraphics[scale=0.4]{chp3/Fig.12b.png}
    \caption{One element surface map (up) and the area panorama of the inner part of the vehicle.}
    \label{AM_map}
\end{figure}

Here one should note that the area map panorama obtained in Figure~\ref{AM_map} is created by stitching each image together, so the surface size varies from part to part, not pixel to pixel. This is due to the fact that the spherical projection is performed for each image separately.
Then starting with the heat flux and area map panoramas, the ‘Power map’ is determined by multiplying the former panoramas. Once the power value of each pixel has been obtained, the total power from thermography method is established:

\begin{align}
P_{map} =& Q_{map}×S_{map}  \\
P_{total} =& sum(P_{map})  
\end{align}

where $ × $ represents the dot multiplication for the matrix. The final result is 1212.1 $ W $.

In addition to this method, another technique is also derived from the IR thermography method. We can obtain the mean value of the internal surface heat flux from the final Heat Flux Panorama (Figure~\ref{Pano_Q_Final}): $q_{mean} = 11.724$  W/m$^2 $, which provides the entire internal heat power per square meters inside the truck. The mean value of the internal surface temperature (Figure~\ref{Pano_T_Final}) can be obtained in the same way: $ T_{mean} = 305.94$ K $= 32.79$ °C.
 Therefore, the final K-value from IR thermography is:

\begin{equation}
K_{th}=\frac{\bar{q}}{\Delta ̅\theta} =\frac{11.724 W/m^2}{(32.79-7.5) K}=0.464 W/K m^2 
\end{equation}
The error between these two results is then:

\begin{equation}
e=  \frac{|K_{th}-K|}{K}=\frac{0.464-0.46}{0.46}=0.0087=0.87\%
\end{equation}

This result reveals the accuracy of the IR thermography method, in comparison to the ATP test. Here it should be noted that the difference between the ATP test and the IR thermography methods is that, during the ATP calculation the mean surface of the truck, given by the geometric mean of the inside and outside surface areas, is taken into account, while in IR thermography, only the internal surface of the truck is considered. Moreover, the necessary power to maintain the ATP test was given by the heaters power in addition to the fans power (Table~\ref{box_dim} and Table~\ref{ATP_res}). In fact, the heat flux detected by the heat flux meter in the specific area on the internal surface of the truck, is the actual power per square meters implemented on the surface. The power corresponding to this value multiplied by the internal surface should be less than the total power:

\begin{align}
W_{th}= \bar{q}*S_i{in}=&11.724 W/m^2 *94.71 m^2 =1110.38 W \\
W_{total}= &144 W + 988 W = 1132 W
\end{align}


\section{Conclusion}

This work  mainly focused on mapping the heat flux on the internal surface of an insulated vehicle by infrared thermography technique. The ATP measurement was performed to obtain the experimental results. An IR camera was mounted on a pan-tilt head and automatically driven by a suitable software to map the temperature of the inner walls at the steady condition, in order to analyze and compute the corresponding heat flux on the entire surface. A simple thermal resistance model has been applied to achieve the computation. The final temperature figures showed a good uniform distribution, and several defects in the structure, such as thermal bridges or air leakages, could be identified. A reference zone of the internal wall was measured using a thermal flux meter, and this reference was used to determine the heat flux map of the entire surface. When a complete set of images of the entire inside surface of the vehicle was obtained, a panoramic view of the heat flux was produced. Image processing techniques such as spherical projection, image translation and stitching have been performed with the raw images. The preliminary results demonstrate that the algorithm performs well, however the narrow view of the IR camera and the similarity of the internal surface of the vehicle make it difficult to create the full-view panorama due to the automatic detectors between images, which leads to a manual detection that requires more time to complete. Nonetheless, the final result of the K-value obtained by IR thermography is accurate enough and compares well with that of the ATP test ($0.87\% $ of error).

%!TEX root = gabarit-doctorat.tex
\part{Exploration of cold approach}     % numéroté
%\phantomsection\addcontentsline{toc}{part}{Exploration of cold approach} % inclure dans TdM
The following two chapters will present two published paper concerning the exploration of cold approach in infrared thermography applied in Non-Destructive Testing \& Evaluation.

The first study concentrates on the detection of defects and thermal bridges in insulated truck box panels, by active infrared thermography. Comparison between heating and cooling approaches for experiments and models has been established. In addition, passive thermography detection in computational models has been presented. Results demonstrate that the compressed air spray is more rapid than the traditional heating method in providing successful detection. Even if the traditional heating approach provides clearer results, in reality it is not easy and practical to heat a whole truck box to conduct inspection: the compressed air spray approach is much more convenient.

The second research investigates an external stimulation–cooling instead of heating in IR Thermography for NDT \& E. A steel specimen is used to test three different stimulations for thermal images and also ROC analysis comparison. Results shows that all techniques highlight part of the flaws in the sample, whereas the LN$_2$ technique represents the defects only at the beginning; this maybe due to the high conductivity of steel. In thermal results, the PCT post-processing method displays a better results for all procedures. More defects are exhibited in Flash stimulation with PCT processing.  The results of this study were firstly presented at an oral session of SPIE Thermosense: Thermal Infrared Applications XXXIX , in 2017 in the United States.

\chapter{Detection of insulation flaws and thermal bridges in insulated truck box panels}
(Published on-line in the Quantitative InfraRed Thermography Journal, in May 2017. Cited 3 times up to now).

%
The results of this study were firstly presented at an oral session of the 13th Quantitative InfraRed Thermography Conference 2016 at Gdańsk University of Technology in Poland . Then it was selected for the publication in Quantitative InfraRed Thermography Journal.

\section{Résumé}
Cet article se concentre sur la détection des défauts et des ponts thermiques dans les panneaux de caisses de camions isolés, en utilisant la thermographie infrarouge. Contrairement à la méthode traditionnelle de thermographie passive, cette recherche utilise des méthodes de chauffage et de refroidissement dans des configurations de thermographie active. Le chauffage de la lampe est utilisé comme stimulation externe chaude, tandis qu'un jet d'air comprimé est appliqué comme stimulation externe froide. Une caméra thermique capture tout le processus. En outre, des simulations numériques sous la plate-forme COMSOL$^®$ sont également menées. Les résultats expérimentaux et de simulation pour deux situations sont comparés et discutés.

\section{Abstract}
This paper focuses on the detection of defects and thermal bridges in insulated truck box panels, utilizing infrared thermography. Unlike the traditional way in which passive thermography is applied, this research uses both heating and cooling methods in active thermography configurations. Lamp heating is used as the hot external stimulation, while a compressed air jet is applied as the cold external stimulation. A thermal camera captures the whole process. In addition, numerical simulations under COMSOL$^®$ platform are also conducted. Experimental and simulation results for two situations are compared and discussed.

\textbf{\texttt{Contributing authors:}}

\textbf{\textsf{Lei Lei}} (Ph.D candidate): developing protocol, experiment preparation and planning, data collection, personnel coordination and manuscript preparation.

\textbf{Alessandro Bortolin} (Ph.D student of CNR-ITC): data analysis, discussion and manuscript preparation.

\textbf{Paolo Bison} (Research supervisor of CNR-ITC): student supervision, revision and correction of the manuscript. 

\textbf{Xavier Maldague} (Research director of LVSN in University Laval): student supervision, revision and correction of the manuscript. 

\phantomsection\addcontentsline{lot}{table}{4.1\quad Specimen specification details}
\phantomsection\addcontentsline{lot}{table}{4.2\quad Air-cooling parameters}
\phantomsection\addcontentsline{lot}{table}{4.3\quad Materials properties}
\phantomsection\addcontentsline{lot}{table}{4.4\quad Thermal contrast peak talbe}

\phantomsection\addcontentsline{lof}{table}{4.1\quad Details of defects inside the specimen(left) and final specimen to test(right).}
\phantomsection\addcontentsline{lof}{table}{4.2\quad Experimental set-up.}
\phantomsection\addcontentsline{lof}{table}{4.3\quad Simulation 3D models transparency view.}
\phantomsection\addcontentsline{lof}{table}{4.4\quad Experimental results.}
\phantomsection\addcontentsline{lof}{table}{4.5\quad Simulation results.}
\phantomsection\addcontentsline{lof}{table}{4.6\quad Experimental quantitative results of panel surface.}
\phantomsection\addcontentsline{lof}{table}{4.7\quad Computational quantitative results of panel surface.}
\phantomsection\addcontentsline{lof}{table}{4.8\quad Temperature contrast profiles of simulation models.}
\phantomsection\addcontentsline{lof}{table}{4.9\quad Temperature contrast profiles of Lamp heating models.}


\includepdf[pages={2-11}]{Lei2017Detection}


%\includepdf[pages=-]{Revised_Version_2}
%\includepdf[pages={2-11}]{Lei2017Detection}
%\includepdf[pages=-]{Thermosense2017_Lei}

\chapter*{Conclusion \& Perspectives}         % ne pas numéroter
\phantomsection\addcontentsline{toc}{chapter}{Conclusion \& Perspectives} % dans TdM

\section{General conclusions}
The objectives of the present thesis are, the first part, to deploy the infrared thermography technique in the procedure of maintain the ``cold food chain'', especially in the insulated vehicle of ATP standard. The application of infrared thermography aims to identify thermal insulation anomalies, in which the standard ATP cannot localize. 

The preliminary work focused on mapping the heat flux on the external surface of an insulated roll-container by Infrared thermography technique. The ATP standard measurement was performed to obtain the experimental results, meanwhile IR images of roll-container have been taken when the steady condition arrived, in order to analyze and compute the corresponding heat flux on entire surfaces. A simple thermal resistance model has been applied to realize the computation. Final temperature figures showed a good uniform distribution, and several defects in the structure like thermal bridges or air leakages have been identified. A reference zone of the external wall is measured by a thermal flux meter, then with that reference the whole surface heat flux map have been figured out. Besides, for a better view of the heat flux map, the homography technique has been performed into the raw images by applying a bilinear interpolation with the projective transformation matrix. The final corrected heat flux map has been demonstrated for each surface, in which the right one showed a smaller value than the others.

When implemented into the internal surface of the insulated vehicle, a panoramic view was needed. With the help of an infrared camera mounted on a pan-tilt head and automatically driven by a suitable software, a series thermal images of the inner walls of the vehicle at a steady condition have been obtained. Proper methods such as inverse spherical projection, stitching images by translation helped make the final panorama. The same thermal resistance model was utilized to compute the corresponding heat flux map. 

Then based on the previous favorable results, the second part is to explore cold approaches (such as compressed air, liquid nitrogen, etc.) in infrared thermography for Non-Destructive Testing \& Evaluation. 



\section{Future perspectives}
            % conclusion


%\appendix                       % annexes le cas échéant

%%!TEX root = gabarit-doctorat.tex
\chapter*{Annexe}     % numérotée
\phantomsection\addcontentsline{toc}{chapter}{Annexe} 

\section*{Summary of contributions}
\phantomsection\addcontentsline{toc}{section}{Summary of contributions} 

    \begin{enumerate}
        \item \textbf{Lei, L.}, G. Ferrarini, A. Bortolin, G. Cadelano, P. Bison and X. Maldague. Thermography is cool: defect detection using liquid nitrogen as a thermal stimulus. (submitted to) NDT \& E International journal, 2018.

        \item \textbf{Lei, L.}, Bortolin, A., Cadelano, G., Ferrarini, G., Rossi, S., Maldague, X., and Bison, P. Panoramic view of the heat flux inside an insulated vehicle by infrared thermography, \textit{Quantitative InfraRed Thermography Journal, Taylor \& Francis}, 2018, 1-13.

        \item \textbf{Lei, L.}, Bortolin, A., Bison, P., and Maldague, X. 
        Detection of Insulation Flaws and Thermal Bridges in Insulated Truck Box Panels, \textit{Quantitative InfraRed Thermography Journal, Taylor \& Francis}, 2017, 1-10.

        \item \textbf{Lei, L.}, G. Ferrarini, A. Bortolin, G. Cadelano, P. Bison and X. Maldague.
        Liquid nitrogen cooling in IR thermography applied to steel specimen, \textit{SPIE Commercial+ Scientific Sensing and Imaging}, 2017, 102140T-102140T.

        \item \textbf{Lei, L.}; Ferrarini, G.; Bison, P. and Maldague, X. 
        Investigation on cooling methods in IR Thermography for Non-destructive Testing, \textit{NDT in Canada 2017 Conference}, 2017.

        \item Yousefi, B., Kalhor, D., Usamentiaga, R., \textbf{Lei, L.}, Castanedo, C. and Maldague, X.
        Application of Deep Learning in Infrared Non-Destructive Testing, \textit{Submitted to QIRT 2018 Conference, Berlin, Germany, 2018}

        \item Fleuret, J. Lei, L. Shui, C. Usamentiaga, R., Sfarra, S., Castanedo, C. and Maldague, X.
        A Robust Defect Detection and Segmentation Algorithm in Highly Noised Images, \textit{Submitted to QIRT 2018 Conference, Berlin, Germany, 2018}

        \item Fleuret, J., R., Yousefi, B., \textbf{Lei, L.}, Dizeu, F. B. D., Zhang, H., Sfarra, S., Ouellet, D. and Maldague, X. P. 
        Investigation of the influence of spatial degrees of freedom on thermal infrared measurement, \textit{SPIE Commercial+ Scientific Sensing and Imaging}, 2017, 1021418-1021418.

        \item Bortolin, A., Bison, P., Cadelano, G., Ferrarini, G., Rossi, S., \textbf{Lei, L.} and Maldague, X. 
        A Thermographic Approach to the Heat Flux Measurement of Insulated Containers, \textit{13th International Workshop on Advanced Infrared Technology \& Applications}, 2015, 45-48.

        \item Bison, P., Bortolin, A., Cadelano, G., Ferrarini, G., \textbf{Lei, L.}, Maldague, X. and Rossi, S. 
        Mapping of the heat flux of an insulated small container by infrared thermography, \textit{24th IIR International Congress of Refrigeration}, 2015, 2929-2935.

    \end{enumerate}

% \section*{List of experimental apparatuses}


\bigskip
\section*{First page of published papers issued in thesis}
\phantomsection\addcontentsline{toc}{section}{First page of published papers issued in thesis} 
The following are the first pages of published papers which are issued in the thesis:
\includepdf{Lei2018Panoramic_2}
\includepdf{Lei2017Detection_2}


\section*{Programming code}
\phantomsection\addcontentsline{toc}{section}{Programming code} 

\textbf{Computing Homography matrix:}
\lstinputlisting{computeH.m}

\textbf{Warping the images:}
\lstinputlisting{warpim_fb.m}

\textbf{Display results:}
\lstinputlisting{IR_display.m}
\lstinputlisting{Q_display.m}

\textbf{Spherical projection:}
\lstinputlisting{sphe_proj.m}

\textbf{Convert to Heat flux:}
\lstinputlisting{convertT2Q.m}

\textbf{ROC curves:}
\lstinputlisting{IT_2016_test.m}
                % annexe A

\bibliography{//gel.ulaval.ca/Vision/Usagers/lelei/Desktop/Bibliography/Biblio_th}                 % production de la bibliographie

\end{document}
