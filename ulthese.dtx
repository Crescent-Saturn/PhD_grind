% \iffalse meta-comment
%
% Copyright (C) 2017 Universite Laval
%
% This file may be distributed and/or modified under the conditions
% of the LaTeX Project Public License, either version 1.3c of this
% license or (at your option) any later version. The latest version
% of this license is in:
%
%   http://www.latex-project.org/lppl.txt
%
% and version 1.3c or later is part of all distributions of LaTeX
% version 2006/05/20 or later.
%
% This work has the LPPL maintenance status `maintained'.
%
% The Current Maintainer of this work is Universite Laval
% <ulthese-dev@bibl.ulaval.ca>.
%
% This work consists of the files ulthese.dtx and ulthese.ins
% and the derived files listed in the README.md file.
%
% \fi
%
% \iffalse
%<*dtx>
\ProvidesFile{ulthese.dtx}
%</dtx>
%<class>\NeedsTeXFormat{LaTeX2e}[2009/09/24]
%<class>\ProvidesClass{ulthese}%
%<*class>
  [2017/06/01 v4.4 Universite Laval thesis and memoir class]
%</class>
%<*driver>
\documentclass[11pt,english,french]{ltxdoc}
  \usepackage[utf8]{inputenc}
  \usepackage[T1]{fontenc}
  \usepackage{natbib}
  \usepackage{babel}
  \usepackage[autolanguage]{numprint}
  \usepackage{microtype}
  \usepackage[scaled=0.92]{helvet}
  \usepackage[scaled=1.02]{inconsolata}
  \usepackage[sc]{mathpazo}
  \usepackage{fontawesome}
  \usepackage{metalogo}
  \usepackage{tabularx,booktabs}
  \DisableCrossrefs
  \CodelineNumbered
  \RecordChanges
  \GlossaryPrologue{\section*{Historique des versions}%
    \addcontentsline{toc}{section}{Historique des versions}}

  %% Couleurs
  \usepackage{xcolor}
  \definecolor{link}{rgb}{0,0.4,0.6}   % ~RoyalBlue de dvips
  \definecolor{url}{rgb}{0.6,0,0}      % rouge foncé
  \definecolor{citation}{rgb}{0,0.5,0} % vert foncé

  %% Liste description alignée à gauche
  \usepackage{enumitem}
  \setlist[description]{leftmargin=*,align=left}

  %% Environnement pour remarques
  \usepackage{amsthm}
  \theoremstyle{definition}
  \newtheorem*{rem}{Remarque}

  %% Hyperliens
  \usepackage{hyperref}
  \hypersetup{%
    pdfauthor = {Faculté des études supérieures et postdoctorales},
    pdftitle = {Guide de l'utilisateur de la classe ulthese},
    colorlinks = {true},
    linktocpage = {true},
    urlcolor = {url},
    linkcolor = {link},
    citecolor = {citation},
    pdfpagemode = {UseOutlines},
    pdfstartview = {Fit},
    bookmarksopen = {true},
    bookmarksnumbered = {true},
    bookmarksdepth = {subsubsection}}
  \addto\extrasfrench{%
    \def\appendixautorefname{annexe}%
    \def\tableautorefname{tableau}%
    \def\subsectionautorefname{section}%
  }

  %% Paramétrage de babel
  \frenchbsetup{%
    StandardItemizeEnv=true,       % listes standards
    ThinSpaceInFrenchNumbers=true, % espace fine dans les nombres
    og=«, fg=»                     % « et » sont les guillemets
  }
  \def\frenchtablename{{\scshape Tab.}}

  %% Commandes de mise en forme spéciales
  \newcommand{\class}[1]{\textsf{#1}}
  \newcommand{\pkg}[1]{\textbf{#1}}

  %% Commandes pour les liens vers la documentation
  \newcommand{\link}[2]{\href{#1}{#2~\raisebox{-0.2ex}{\faExternalLink}}}
  \newcommand{\doc}[3][documentation]{%
    \link{#3}{#1}\marginpar{\hfill\faBookmark~\texttt{#2}}}

\begin{document}
  \DocInput{ulthese.dtx}
\end{document}
%</driver>
% \fi
% \CheckSum{0}
% \DoNotIndex{\',\^,\`,\ ,\ae}
% \DoNotIndex{\RequirePackage,\ExecuteOptions,\ProcessOptions}
% \DoNotIndex{\newcommand,\newcommand*}
% \DoNotIndex{\setlength}
% \changes{4.4}{2017-05-16}{Prise en charge des bibliographies
%   multiples pour les thèses et mémoires par articles.}
% \changes{4.4}{2017-05-16}{Nouveau gabarit pour une thèse par
%   articles.}
% \changes{4.4}{2017-05-16}{Documentation révisée, principalement ce
%   qui touche à la préparation de la bibliographie.}
% \changes{4.3}{2017-03-01}{Nouvelles options pour les titres de
%   docteur en musique, maître en architecture, maître en
%   ergothérapie, maître en physiothérapie et maître en
%   psychoéducation.}
% \changes{4.3}{2017-03-01}{Modifications à la composition des sigles de
%   grades.}
% \changes{4.3}{2017-03-01}{Vérification de la compatibilité entre le
%   grade et les options multifacultaire, cotutelle, bidiplomation et
%   extension.}
% \changes{4.3}{2017-03-01}{Améliorations à la documentation.}
% \changes{4.2}{2016-03-29}{Ajout de l'option \texttt{examen}. Nouvelles
%   options de \pkg{babel} dans les gabarits.}
% \changes{4.1}{2016-02-13}{Paquetage \pkg{geometry} déclaré incompatible
%   avec la classe.}
% \changes{4.0}{2015-06-12}{Nouvelles règles de présentation matérielle
%   de la FESP: recto seulement, page frontispice.}
% \changes{3.1}{2014-05-23}{Prise en charge de la maîtrise en bidiplomation}
% \changes{3.0a}{2014-03-24}{Modifications et corrections à la
%   documentation, notamment relativement à la configuration de \pkg{natbib}}
% \changes{3.0}{2014-01-06}{Déclaration du grade en option de la
%   classe.}
% \changes{3.0}{2014-01-06}{Moteur {\XeLaTeX} supporté}
% \changes{3.0}{2014-01-06}{Ajout de l'option \texttt{nobabel}}
% \changes{2.1}{2013-01-16}{Utilisation transparente de la police
%   Helvetica pour la page de titre}
% \changes{2.0}{2013-01-13}{Traitement automatique des longs titres}
% \changes{1.0b}{2012-11-11}{Ajouts et corrections mineures dans la
%   documentation}
% \changes{1.0a}{2012-10-17}{Précisions dans la documentation}
% \changes{1.0}{2012-09-30}{Version initiale}
% \GetFileInfo{ulthese.dtx}
% \title{\class{ulthese}: la classe pour les thèses et mémoires de
%   l'Université Laval\thanks{Ce document décrit la classe
%   \class{ulthese}~\fileversion, datée du \filedate.}}
% \author{Faculté des études supérieures et postdoctorales\thanks{%
%   Cette classe et sa documentation ont été rédigées par Vincent
%   Goulet~(Faculté des sciences et de génie) avec la collaboration de
%   Koassi D'Almeida~(Faculté des études supérieures et postdoctorales) et
%   de Pierre Lasou~(Bibliothèque).}}
% \date{}
% \maketitle
%
% \section{Introduction}
%
% La classe \class{ulthese} permet de composer avec {\LaTeX} ou
% {\XeLaTeX} des thèses et mémoires immédiatement conformes aux règles
% générales de présentation matérielle de la Faculté des études
% supérieures et postdoctorales (FESP) de l'Université Laval. Ces
% règles définissent principalement la présentation des pages de titre
% des thèses et mémoires ainsi que la disposition du texte sur la
% page. La classe en elle-même est donc relativement simple.
%
% Cependant, \class{ulthese} est basée sur la classe \class{memoir}
% \citep{memoir}, une extension de la classe standard \class{book}
% facilitant à plusieurs égards la préparation de documents d'allure
% professionnelle dans {\LaTeX}. La classe \class{memoir} incorpore
% d'office plus de 30 des paquetages les plus populaires\footnote{%
% Consulter la section~18.24 de la documentation de \class{memoir}
% pour la liste ou encore le journal de la compilation (\emph{log})
% d'un document utilisant la classe \class{ulthese}.}. %
% L'intégralité des fonctionnalités de \class{memoir} se retrouve donc
% dans \class{ulthese}.
%
% La classe \class{memoir} fait partie des distributions {\LaTeX}
% modernes; elle devrait donc être installée et disponible sur votre
% système. Elle est livrée avec une %
% \doc{memman}{http://texdoc.net/pkg/memoir}\footnote{%
% Première occurence d'une convention de ce document quand il s'agit
% de documentation d'un paquetage: un hyperlien mène vers une version
% en ligne dans le site \link{http://texdoc.net}{TeXdoc Online} et on
% trouve dans la marge le nom du fichier correspondant (sans
% l'extension \texttt{.pdf}) sur un système doté d'une installation de
% {\TeX}~Live.} %
% exhaustive: le guide de l'utilisateur fait près de 600~pages!
% N'hésitez pas à vous y référer pour réaliser une mise en page
% particulière.
%
% L'autre compagnon naturel de la présente documentation est
% \doc[Rédaction avec {\LaTeX}]{formation-latex-ul}{http://texdoc.net/pkg/formation-latex-ul} ^^A
% \citep{formation-latex-ul}, la formation {\LaTeX} de l'Université Laval.
% Vous y trouverez des informations additionnelles sur l'utilisation des
% classes \class{ulthese} et \class{memoir}.
%
% \section{Démarrage rapide (pour les impatients)}
% \label{sec:utilisation:rapide}
%
% La classe est livrée avec les distributions {\TeX}~Live, Mac{\TeX} et
% MiK{\TeX}. Si vous utilisez l'une ou l'autre de ces distributions et
% qu'elle est à jour\footnote{%
%   Validez à l'aide de l'assistant de mise à jour de votre
%   distribution.}, %
% vous devriez pouvoir utiliser \class{ulthese} sans autre
% intervention.
%
% Il est recommandé de segmenter tout document d'une certaine ampleur
% dans des fichiers |.tex| distincts pour chaque partie ---
% habituellement un fichier par chapitre. Le document complet est
% composé à l'aide d'un fichier maître qui contient le préambule
% {\LaTeX} et un ensemble de commandes \cmd{\include} pour réunir les
% parties dans un tout.
%
% La classe \class{ulthese} est livrée avec un ensemble de gabarits sur
% lesquels se baser pour:
% \begin{itemize}
% \item les fichiers maîtres de divers types de thèses et de mémoires
%   (standard, par articles, sur mesure, en cotutelle, en
%   bidiplomation, en extension, etc.);
% \item les fichiers des parties les plus usuelles (résumés français
%   et anglais, avant-propos, introduction, chapitres, conclusion,
%   etc.).
% \end{itemize}
% Les noms des fichiers devraient permettre de facilement identifier
% leur contenu (c'est là une bonne pratique: le nom |rappels.tex|
% parle de lui-même et résiste mieux aux changements à l'ordre des
% chapitres que |chapitre1.tex|).
%
% Dans {\TeX}~Live, les gabarits sont classés avec la documentation de
% la classe. Pour les utiliser, copiez les fichiers appropriés dans votre
% dossier de travail.
%
% Pour débuter la rédaction, renommez le gabarit de document maître
% approprié d'après votre numéro de dossier. Par exemple, l'étudiante
% dont le numéro de dossier est 900352789 et qui entame la rédaction
% d'une thèse multifacultaire renommera le fichier
% \begin{quote}
%   |gabarit-doctorat-multifacultaire.tex|
% \end{quote}
% en
% \begin{quote}
%   |900352789.tex|.
% \end{quote}
%
% Les gabarits comportent des commentaires succincts pour vous guider
% dans la préparation de votre document. La \autoref{sec:gabarits}
% fournit de plus amples informations.
%
% \section{Installation}
%
% Cette section explique comment installer la classe \class{ulthese}
% si elle n'est pas disponible sur votre système ou si la version
% n'est pas à jour.
%
% La classe est distribuée sous forme d'une archive |ulthese.zip| via
% le réseau de sites \emph{Comprehensive {\TeX} Archive Network} (CTAN):
% \begin{quote}
%   \url{https://www.ctan.org/pkg/ulthese}
% \end{quote}
%
% L'installation de la classe consiste à créer le fichier
% |ulthese.cls| et plusieurs gabarits |.tex| à partir du code source
% documenté se trouvant dans le fichier |ulthese.dtx|. Il est
% recommandé de simplement créer ces fichiers dans le dossier de
% travail de la thèse ou du mémoire.
%
% Pour procéder à l'installation, décompressez l'archive |ulthese.zip|
% dans votre dossier de travail, puis compilez avec {\LaTeX} le fichier
% |ulthese.ins| en exécutant
% \begin{quote}
%   |latex ulthese.ins|
% \end{quote}
% depuis une invite de commande. Il est aussi possible d'ouvrir le
% fichier |ulthese.ins| dans votre éditeur de texte favori et de
% lancer depuis celui-ci la compilation avec {\LaTeX}, pdf{\LaTeX},
% {\XeLaTeX} ou un autre moteur {\TeX}.
%
% \section{Utilisation}
%
% La classe est compatible avec les moteurs {\LaTeX} traditionnels
% ainsi qu'avec le plus récent moteur {\XeLaTeX}.
%
% On charge la classe avec la commande
% \begin{quote}
% |\documentclass|\oarg{options}|{ulthese}|
% \end{quote}
% Les marges, l'interligne et la numérotation des pages sont adaptées
% aux règles de présentation matérielle de la FESP. Les sections
% suivantes décrivent les options et les commandes définies par la
% classe.
%
% \subsection{Options de la classe}
% \label{sec:utilisation:options}
%
% Cette section passe en revue les \meta{options} que l'on peut
% spécifier au chargement de la classe. Les commandes mentionnées
% ci-dessous font quant à elles l'objet de la \autoref{sec:commandes}.
%
% \begin{DescribeMacro}{PhD,MSc,MA, ...}
%   Identifie le type de grade; consulter le \autoref{tab:grades}
%   pour la liste complète des options et les grades correspondants.
%   La déclaration d'un type de grade est obligatoire.
% \end{DescribeMacro}
%
% \begin{table}[t]
%   \centering
%   \begin{tabularx}{0.85\linewidth}{lX}
%     \toprule
%     Option & Nom du grade (sigle) \\
%     \midrule
%     |LLD|
%       & Docteur en droit (LL.~D.) \\
%     |DMus|
%       & Docteur en musique (D.~Mus.) \\
%     |DPsy|
%       & Docteur en psychologie (D.~Psy.) \\
%     |DThP|
%       & Docteur en théologie pratique (D.~Th.~P.) \\
%     |PhD|
%       & Philosophi{\ae} doctor (Ph.~D.) \\
%     \addlinespace[6pt]
%     |MATDR|
%       & Maître en aménagement du territoire et développement régional
%       (M.ATDR) \\
%     |MArch|
%       & Maître en architecture
%       (M.~Arch.) \\
%     |MA|
%       & Maître ès arts (M.A.) \\
%     |LLM|
%       & Maître en droit (LL.~M.) \\
%     |MErg|
%       & Maître en ergothérapie (M.~Erg.) \\
%     |MMus|
%       & Maître en musique (M.~Mus.) \\
%     |MPht|
%       & Maître en physiothérapie (M.~Pht.) \\
%     |MSc|
%       & Maître ès sciences (M.~Sc.) \\
%     |MScGeogr|
%       & Maître en sciences géographiques (M.~Sc.~géogr.) \\
%     |MServSoc|
%       & Maître en service social (M.~Serv.~soc.) \\
%     |MPsEd|
%       & Maître en psychoéducation (M.~Ps.~éd.) \\
%     \bottomrule
%   \end{tabularx}
%   \caption{Options de la classe pour la déclaration du grade et libellés
%     correspondants}
%   \label{tab:grades}
% \end{table}
%
% \begin{DescribeMacro}{multifacultaire}
%   Identifie une thèse multifacultaire. Valide uniquement avec un
%   grade de doctorat. Cette déclaration requiert ensuite d'utiliser
%   la commande \cmd{\faculteUL}.
% \end{DescribeMacro}
%
% \begin{DescribeMacro}{cotutelle}
%   Identifie une thèse effectuée en cotutelle avec une autre
%   université. Valide uniquement avec un grade de doctorat. Cette
%   déclaration requiert ensuite d'utiliser les commandes
%   \cmd{\univcotutelle} et \cmd{\gradecotutelle}.
% \end{DescribeMacro}
%
% \begin{DescribeMacro}{bidiplomation}
%   Identifie un mémoire effectué en bidiplomation avec une autre
%   université. Valide uniquement avec un grade de maîtrise. Cette
%   déclaration requiert ensuite d'utiliser les commandes
%   \cmd{\univbidiplomation} et \cmd{\gradebidiplomation}.
% \end{DescribeMacro}
%
% \begin{DescribeMacro}{extensionUdeS}
%   Identifie une thèse réalisée en extension à l'Université de
%   Sherbrooke. Valide uniquement avec un grade de doctorat. Cette
%   déclaration requiert ensuite d'utiliser les commandes
%   \cmd{\faculteUL} et \cmd{\faculteUdeS}.
% \end{DescribeMacro}
%
% \begin{DescribeMacro}{extensionUQO}
%   Identifie une thèse réalisée en extension à l'Université du
%   Québec en Outaouais (UQO). Valide uniquement avec un grade de doctorat.
%   Cette déclaration requiert ensuite d'utiliser les commandes
%   \cmd{\faculteUL} et \cmd{\faculteUQO}.
% \end{DescribeMacro}
%
% \begin{DescribeMacro}{extensionUQAC}
%   Identifie un mémoire réalisé en extension à l'Université du Québec
%   à Chicoutimi (UQAC). Valide uniquement avec un grade de maîtrise.
%   Cette déclaration requiert ensuite d'utiliser les commandes
%   \cmd{\faculteUL} et \cmd{\faculteUQAC}.
% \end{DescribeMacro}
%
% \begin{DescribeMacro}{examen}
%   Identifie un examen de doctorat. Cette option permet d'utiliser la
%   classe pour la rédaction d'un examen de doctorat respectant les
%   règles de présentation matérielle de la FESP. Elle a pour effet de
%   changer l'appellation «Thèse» sur la page couverture pour «Examen
%   de doctorat». Elle supprime également la page frontispice.
%   L'option n'est compatible qu'avec l'une des options de grade de
%   doctorat.
% \end{DescribeMacro}
%
% \begin{DescribeMacro}{10pt,11pt,12pt}
%   Sélectionne une taille de police de 10, 11 ou 12~points. Par
%   défaut la classe utilise une police de 11~points. Ces options
%   n'ont aucun effet sur la taille des polices des pages de titre.
% \end{DescribeMacro}
%
% \changes{4.4}{2017-05-15}{Ajout de l'option \texttt{bibchapitre}}
% \changes{4.4}{2017-05-15}{Ajout de l'option \texttt{bibsection}}
% \begin{DescribeMacro}{bibchapitre}
%   \begin{DescribeMacro}{bibsection}
%     Permettent de composer une bibliographie distincte après chaque
%     fichier inséré dans le document avec \cmd{\include},
%     habituellement un chapitre; voir la
%     \autoref{sec:bibliographie:multiples}. Ces options sont
%     principalement utiles pour les thèses et mémoires par articles.
%     La bibliographie se présente sous la forme d'un chapitre non
%     numéroté avec |bibchapitre|, et sous la forme d'une section
%     numérotée avec |bibsection|.
%   \end{DescribeMacro}
% \end{DescribeMacro}
%
% \begin{DescribeMacro}{nonatbib}
%   Empêche le chargement du paquetage \pkg{natbib}. La paquetage est
%   normalement chargé par la classe; voir la
%   \autoref{sec:bibliographie}. L'option |nonatbib| permet d'empêcher
%   le chargement pour modifier les options du paquetage ou en cas de
%   conflit avec un autre paquetage de mise en forme de la
%   bibliographie.
% \end{DescribeMacro}
%
% \begin{DescribeMacro}{nobabel}
%   Empêche le chargement du paquetage \pkg{babel}. La classe utilise
%   par défaut ce paquetage pour le traitement des langues dans le
%   document; voir la \autoref{sec:langues}. L'option |nobabel| permet
%   d'empêcher son chargement si un autre paquetage devait être
%   utilisé --- on pense ici principalement à \pkg{polyglossia} pour
%   un document produit avec le moteur {\XeLaTeX}.
% \end{DescribeMacro}
%
% \begin{DescribeMacro}{english, french, ...}
%   Déclare les langues utilisées dans le document. Ces options sont
%   transférées au paquetage \pkg{babel} (dans la mesure où |nobabel|
%   n'est pas spécifié, bien entendu). Le libellé des langues devrait
%   donc correspondre aux options de \pkg{babel}. La dernière langue
%   spécifiée est la langue active par défaut dans le document.
% \end{DescribeMacro}
%
% Toute autre option sera passée à la classe \class{memoir} dont,
% entre autres, le format du papier. Le format lettre nord-américain
% (option |letterpaper|) est utilisé par défaut. Si votre thèse doit
% être imprimée en format international A4, utilisez l'option
% |a4paper|. La classe \class{memoir} est toujours chargée avec
% l'option |oneside|.
%
% \subsection{Commandes de la classe}
% \label{sec:commandes}
%
% La classe \class{ulthese} définit quelques nouvelles commandes
% servant principalement à créer les pages de titre et des éléments
% des pages liminaires. On trouvera un sommaire des commandes au
% \autoref{tab:commandes} et leurs descriptions détaillées ci-dessous.
%
% \begin{table}[t]
%   \centering
%   \begin{tabularx}{\linewidth}{lX}
%     \toprule
%     Commande & Usage \\
%     \midrule
%     \cmd{\titre}$^\star$ & titre principal du document \\
%     \cmd{\soustitre} & sous-titre du document \\
%     \cmd{\auteur}$^\star$ & nom complet de l'auteur \\
%     \cmd{\annee}$^\star$ & année du dépôt final \\
%     \cmd{\programme}$^\star$ & nom officiel du programme d'études \\
%     \cmd{\direction}$^\star$ & nom du directeur (directrice) de recherche \\
%     \cmd{\codirection} & noms des codirecteurs de recherche \\
%     \cmd{\univcotutelle} & université de cotutelle \\
%     \cmd{\gradecotutelle} & grade conféré par l'université de cotutelle \\
%     \cmd{\univbidiplomation} & université de bidiplomation \\
%     \cmd{\gradebidiplomation} & grade conféré par l'université de bidiplomation \\
%     \cmd{\faculteUL} & noms des facultés de l'Université Laval \\
%     \cmd{\faculteUdeS} & nom de la faculté de l'Université de Sherbrooke \\
%     \cmd{\faculteUQO} & nom de la faculté de l'UQO \\
%     \cmd{\faculteUQAC} & nom de la faculté de l'UQAC \\
%     \cmd{\pagestitre}$^\star$ & production des pages de titre et frontispice \\
%     \cmd{\dedicace} & dédicace du document \\
%     \cmd{\epigraphe} & épigraphe du document \\
%     \bottomrule
%   \end{tabularx}
%   \caption{Sommaire des commandes de la classe \class{ulthese}.
%     Celles marquées d'une étoile $^\star$ sont obligatoires.}
%   \label{tab:commandes}
% \end{table}
%
% \begin{DescribeMacro}{\titre}
%   Titre principal de la thèse ou du mémoire. Ne pas utiliser la
%   commande \cmd{\title} de {\LaTeX} pour ce faire.
%
%   Un titre très long devra être coupé manuellement avec |\\| ou
%   \cmd{\newline}. Par exemple, la déclaration d'un titre d'une seule
%   ligne est:
%   \begin{quote}
%     |\titre{Ceci est un titre d'une seule ligne}|
%   \end{quote}
%   Pour un titre de deux lignes, utiliser:
%   \begin{quote}
%     |\titre{Ceci est la première ligne d'un long titre \\| \\
%     |       et ceci est la seconde}|
%   \end{quote}
% \end{DescribeMacro}
%
% \begin{DescribeMacro}{\soustitre}
%   Sous-titre de la thèse ou du mémoire, le cas échéant. Les remarques
%   sur un long titre principal s'appliquent également au sous-titre.
% \end{DescribeMacro}
%
% \begin{DescribeMacro}{\auteur}
%   Nom complet de l'auteur de la thèse ou du mémoire, sous la forme
%   |Prénom Nom| avec seulement des majuscules initiales. Ne pas
%   utiliser la commande \cmd{\author} de {\LaTeX} pour le nom de
%   l'auteur.
% \end{DescribeMacro}
%
% \begin{DescribeMacro}{\annee}
%   Année du dépôt final de la thèse ou du mémoire.
% \end{DescribeMacro}
%
% \begin{DescribeMacro}{\programme}
%   Nom complet officiel du programme d'études comme «Doctorat en
%   informatique» ou «Maîtrise en mathématiques». Si le programme
%   comporte une majeure, séparer sa mention de celle du programme
%   principal par un tiret demi-quadratin (obtenu avec |--|).
% \end{DescribeMacro}
%
% \begin{DescribeMacro}{\direction}
%   Nom complet du directeur ou de la directrice de recherche, sous la
%   forme |Prénom Nom| avec seulement des majuscules initiales, suivi
%   d'une virgule et de la mention «directeur de recherche» ou
%   «directrice de recherche».
%
%   Les thèses en cotutelle comportent un directeur ou une directrice
%   de recherche et un directeur ou une directrice de cotutelle.
%   Séparer chaque mention par |\\|, comme ceci:
%   \begin{quote}
%     |\direction{Prénom Nom, directrice de recherche \\| \\
%     |           Prénom Nom, directeur de cotutelle}|
%   \end{quote}
%
%   De même, les maîtrises en bidiplomation comptent deux directeurs
%   ou directrices de recherche. Séparer chaque mention comme
%   mentionné ci-dessus.
% \end{DescribeMacro}
%
% \begin{DescribeMacro}{\codirection}
%   Comme la commande \cmd{\direction}, mais pour le ou les
%   codirecteurs de recherche, s'il y a lieu. Lorsqu'il y a plus d'un
%   codirecteur de recherche, séparer chaque mention par |\\|, comme
%   ceci:
%   \begin{quote}
%     |\codirection{Prénom Nom, directrice de recherche \\| \\
%     |             Prénom Nom, directeur de recherche}|
%   \end{quote}
% \end{DescribeMacro}
%
% \begin{DescribeMacro}{\univcotutelle}
%   Nom, ville et pays de l'université de cotutelle, déclarés sous la
%   forme
%   \begin{quote}
%     |\univcotutelle{Nom de l'université \\ Ville, Pays}|
%   \end{quote}
%   Cette commande prend effet seulement lorsque la classe est chargée
%   avec l'option |cotutelle|.
% \end{DescribeMacro}
%
% \begin{DescribeMacro}{\gradecotutelle}
%   Grade conféré par l'université de cotutelle, déclaré
%   sous la forme
%   \begin{quote}
%     |\gradecotutelle{Nom du grade (sigle)}|
%   \end{quote}
%   Cette commande prend effet seulement lorsque la classe est chargée
%   avec l'option |cotutelle|.
% \end{DescribeMacro}
%
% \begin{DescribeMacro}{\univbidiplomation}
%   Nom, ville et pays de l'université de bidiplomation, déclarés sous
%   la forme
%   \begin{quote}
%     |\univbidiplomation{Nom de l'université \\ Ville, Pays}|
%   \end{quote}
%   Cette commande prend effet seulement lorsque la classe est chargée
%   avec l'option |bidiplomation|.
% \end{DescribeMacro}
%
% \begin{DescribeMacro}{\gradebidiplomation}
%   Grade conféré par l'université de bidiplomation, déclaré sous la
%   forme
%   \begin{quote}
%     |\gradebidiplomation{Nom du grade (sigle)}|
%   \end{quote}
%   Cette commande prend effet seulement lorsque la classe est chargée
%   avec l'option |bidiplomation|.
% \end{DescribeMacro}
%
% \begin{DescribeMacro}{\faculteUL}
%   La commande a deux usages:
%   \begin{enumerate}
%   \item noms des facultés pour les thèses et mémoires
%     multifacultaires, séparés par des commandes |\\|;
%   \item nom de la faculté de l'Université Laval où sont réalisés les
%     thèses et mémoires en extension à l'Université de Sherbrooke, à
%     l'UQO ou à l'UQAC.
%   \end{enumerate}
%   Cette commande prend effet seulement lorsque la classe est chargée
%   avec l'une ou l'autre des options |multifacultaire|,
%   |extensionUdeS|, |extensionUQO| ou |extensionUQAC|.
% \end{DescribeMacro}
%
% \begin{DescribeMacro}{\faculteUdeS}
%   Nom de la faculté de l'Université de Sherbrooke hébergeant la
%   thèse en extension. Cette commande prend effet seulement lorsque
%   la classe est chargée avec l'option |extensionUdeS|.
% \end{DescribeMacro}
%
% \begin{DescribeMacro}{\faculteUQO}
%   Nom de la faculté de l'Université du Québec en Outaouais
%   hébergeant la thèse en extension. Cette commande prend effet
%   seulement lorsque la classe est chargée avec l'option
%   |extensionUQO|.
% \end{DescribeMacro}
%
% \begin{DescribeMacro}{\faculteUQAC}
%   Nom de la faculté de l'Université du Québec à Chicoutimi
%   hébergeant le mémoire en extension. Cette commande prend effet
%   seulement lorsque la classe est chargée avec l'option
%   |extensionUQAC|.
% \end{DescribeMacro}
%
% \begin{DescribeMacro}{\pagestitre}
%   Création de la page de titre et de la page frontispice. Ne pas
%   utiliser la commande \cmd{\pagetitle} de {\LaTeX} pour ce faire.
%   De toutes les commandes ci-dessus, c'est la seule qui doit se
%   trouver dans le corps du document plutôt que dans le préambule.
% \end{DescribeMacro}
%
% \begin{DescribeMacro}{\dedicace}
%   Ajout d'une dédicace («À mes parents», «À Camille») à la thèse ou
%   au mémoire. La dédicace est disposée seule sur une page liminaire,
%   à une dizaine de lignes de la marge du haut et alignée à droite.
%   Par défaut, elle est composée en italique.
% \end{DescribeMacro}
%
% \begin{DescribeMacro}{\epigraphe}
%   Ajout d'une épigraphe au début du document. Comme la dédicace,
%   l'épigraphe est disposée seule sur une page liminaire, à une
%   dizaine de lignes de la marge du haut et alignée à droite. La
%   commande accepte deux arguments, soit le texte de la citation et
%   son auteur ou la source, dans l'ordre.
%
%   Pour ajouter une épigraphe au début d'un ou plusieurs chapitres,
%   utiliser directement la commande \cmd{\epigraph} de
%   \class{memoir}, sur laquelle \cmd{\dedicace} et \cmd{\epigraphe}
%   sont d'ailleurs basées.
% \end{DescribeMacro}
%
% \subsection{Citations}
% \label{sec:citations}
%
% {\LaTeX} offre deux environnements pour les citations dans le
% texte: |quote| et |quotation|.
%
% \begin{DescribeEnv}{quote}
%   L'environnement |quote| sert pour les citations «courtes»,
%   quelques lignes au plus. Dans la classe, le texte est alors placé
%   en retrait des marges normales de 10~mm à gauche et à droite.
% \end{DescribeEnv}
%
% \begin{DescribeEnv}{quotation}
%   L'environnement |quotation|, quant à lui, doit être utilisé pour
%   les citations «longues», celles qui peuvent s'étendre sur plus de
%   cinq lignes ou, surtout, plus d'un paragraphe. Dans la classe, le
%   texte est alors toujours placé en retrait de 10~mm, mais également
%   à interligne simple. De plus, un espace vertical sépare les
%   paragraphes, le cas échéant, afin de bien les distinguer les uns
%   des autres.
% \end{DescribeEnv}
%
% \subsection{Interligne}
% \label{sec:interligne}
%
% \begin{DescribeMacro}{\OnehalfSpacing}
%   L'espacement d'un interligne et demi utilisé dans la classe est
%   obtenu avec la commande \cmd{\OnehalfSpacing} de \class{memoir}.
%   L'interligne simple est automatiquement rétabli pour les pages
%   de titre, la table des matières, la liste des tableaux, la liste
%   des figures et les longues citations (\autoref{sec:citations}).
% \end{DescribeMacro}
%
% \begin{DescribeMacro}{\SingleSpacing}
%   Si ce devait être nécessaire ailleurs dans le document, la
%   commande \cmd{\SingleSpacing} permet de passer à l'interligne simple.
% \end{DescribeMacro}
%
%
% \subsection{Autres paquetages chargés}
% \label{sec:paquetages}
%
% Outre \class{memoir}, la classe \class{ulthese} charge quelques
% paquetages qui peuvent aussi vous être utiles. Il n'est donc pas
% nécessaire de charger de nouveau les paquetages suivants:
% \begin{description}
% \item[\normalfont\pkg{babel}] \citep{babel} gestion des documents
%   rédigés dans une ou plusieurs langues autres que l'anglais (si
%   l'option |nobabel| de la classe est absente; voir aussi la
%   \autoref{sec:langues});
% \item[\normalfont\pkg{numprint}] \citep{numprint} requis par la
%   commande \cmd{\nombre} de \pkg{babel}; le paquetage est donc
%   chargé uniquement si \pkg{babel} l'est. Permet de composer
%   automatiquement des nombres avec un séparateur toutes les trois
%   positions (une espace en français);
% \item[\normalfont\pkg{natbib}] \citep{natbib} gestion de la
%   bibliographie (si l'option |nonatbib| de la classe est absente;
%   voir aussi la \autoref{sec:bibliographie});
% \item[\normalfont\pkg{chapterbib}] \citep{chapterbib} gestion de
%   bibliographies multiples (si l'une ou l'autre des options
%   |bibchapitre| ou |bibsection| est spécifiée;
%   voir aussi la \autoref{sec:bibliographie:multiples});
% \item[\normalfont\pkg{fontspec}] \citep{fontspec} gestion des
%   polices OpenType sous {\XeLaTeX} (chargé avec ce moteur
%   seulement);
% \item[\normalfont\pkg{unicode-math}] \citep{unicode-math} gestion
%   des polices mathématiques {\XeLaTeX} (chargé avec ce moteur
%   seulement);
% \item[\normalfont\pkg{graphicx}] \citep{graphicx} insertion et
%   manipulation de graphiques;
% \item[\normalfont\pkg{xcolor}] \citep{xcolor} gestion des couleurs
%   dans le document;
% \item[\normalfont\pkg{textcomp}] multitude de symboles spéciaux,
%   dont un beau symbole de copyright, \textcopyright.
% \end{description}
% L'\autoref{sec:meo} sur la mise en {\oe}uvre de la classe fournit
% plus de détails sur la liste des paquetages chargés et les raisons
% pour lesquelles ils sont requis dans la classe.
%
% \subsection{Paquetage incompatible}
% \label{sec:incompatible}
%
% Le paquetage \pkg{geometry} est incompatible avec la classe à cause
% de sa mauvaise interaction avec \class{memoir}. Son chargement dans
% le préambule du document cause une erreur lors de la compilation.
%
% \section{Français et autres langues}
% \label{sec:langues}
%
% Une complication additionnelle pour les auteurs rédigeant dans une
% langue autre que l'anglais consiste à adapter {\LaTeX} à leur
% langue, qu'il s'agisse des mots clés, de la typographie ou de la
% césure des mots. La solution standard à ce problème provient du
% paquetage \pkg{babel}. Celui-ci permet de combiner plusieurs
% langues dans un même document et de passer de l'une à l'autre
% facilement. Il est chargé par défaut par la classe \class{ulthese}.
%
% Aucune langue n'est spécifiée dans la classe. La plupart des auteurs
% auront recours à l'anglais et au français, ne serait-ce que pour les
% deux résumés demandés par la FESP. Les langues utilisées dans le
% document doivent être spécifiées comme options à la classe, tel que
% mentionné à la \autoref{sec:utilisation:options}. La
% \emph{dernière} langue spécifiée devient par défaut la langue active
% du document.
%
% \begin{DescribeMacro}{\selectlanguage}
%   La commande \cmd{\selectlanguage} de \pkg{babel} permet de passer de
%   la langue courante à la langue spécifiée en argument.
% \end{DescribeMacro}
%
% \begin{DescribeEnv}{otherlanguage}
%   L'environnement |otherlanguage| de \pkg{babel} permet de faire
%   la même chose que la commande \cmd{\selectlanguage}, sauf que le
%   changement de langue est local à l'environnement --- utile pour
%   les brefs changements de langue.
% \end{DescribeEnv}
%
% Si vous n'êtes pas autrement familier avec le paquetage
% \pkg{babel}, consultez sa %
% \doc{babel}{http://texdoc.net/pkg/babel/}. %
% Celle-ci est éclatée en un
% document principal, pour le c{\oe}ur du paquetage et plusieurs
% autres pour les fonctionnalités propres à une langue: %
% \doc[anglais]{english}{http://texdoc.net/pkg/babel-english}, %
% \doc[français]{frenchb}{http://texdoc.net/pkg/babel-french}, %
% etc. Consultez au moins les documents consacrés aux
% langues utilisées dans votre thèse ou mémoire. Le plus simple
% consiste sans doute à consulter en ligne sur CTAN les
% \link{http://mirrors.ctan.org/tex-archive/macros/latex/required/babel/contrib/}{%
% documents spécifiques par langue}.
% \changes{4.4}{2017-05-14}{Correction d'une url vers la documentation
%   de babel.}
%
% \begin{DescribeMacro}{\nombre}
%   Le paquetage \pkg{numprint} étant chargé dans la classe avec
%   \pkg{babel}, vous pouvez utiliser la commande \cmd{\nombre} pour
%   formater automatiquement les nombres. Par exemple, le résultat de
%   |\nombre{123456789}| est \nombre{123456789}.
% \end{DescribeMacro}
%
% Les utilisateurs de {\XeLaTeX} qui souhaiteraient plutôt utiliser le
% plus récent paquetage \pkg{polyglossia} \citep{polyglossia} peuvent
% empêcher le chargement de \pkg{babel} avec l'option |nobabel| de la
% classe. Ils devront toutefois charger et configurer
% \pkg{polyglossia} eux-mêmes dans l'entête de leur document. Ce
% paquetage est moins évolué que \pkg{babel} pour la typographie
% française.
%
% \section{Bibliographie}
% \label{sec:bibliographie}
%
% \begin{DescribeMacro}{\bibliography}
%   Tel qu'expliqué au chapitre~8 de \citet{formation-latex-ul},
%   il est fortement recommandé d'utiliser {\BibTeX} pour la
%   préparation de la bibliographie d'un document. Celle-ci est
%   insérée dans le document à l'endroit où apparait la commande
%   \cmd{\bibliography} dans le code source. Cette commande prend en
%   arguments les noms des bases de données bibliographiques séparés
%   par des virgules.
% \end{DescribeMacro}
%
% \subsection{Mise en forme des citations}
% \label{sec:bibliographie:options}
%
% Dans la classe \class{ulthese}, la mise en forme des citations est
% confiée au paquetage \pkg{natbib} (à moins que l'option |nonatbib|
% ne soit spécifiée). Le paquetage est chargé avec les options par
% défaut, soit |round|, |semicolon| et |authoryear|. Pour spécifier
% d'autres options, vous avez deux possibilités:
% \begin{enumerate}
% \item utiliser l'option |nonatbib| de la classe et ensuite charger
%   explicitement \pkg{natbib} avec ses options;
% \item
%   \begin{DescribeMacro}{\setcitestyle}
%     utiliser la commande \cmd{\setcitestyle} pour passer de
%     nouvelles options à \pkg{natbib}.
%   \end{DescribeMacro}
% \end{enumerate}
%
% Par exemple, pour utiliser un style de citation numérique où le
% numéro de la référence se trouve entre crochets, vous pouvez
% procéder de l'une ou l'autre des deux manières suivantes:
% \begin{quote}
%   |\documentclass[nonatbib]{ulthese}| \\
%   |\usepackage[numbers,square]{natbib}| \\
%   |...| \\
%   |\bibliographystyle{plain-fr}|
% \end{quote}
% ou
% \begin{quote}
%   |\documentclass{ulthese}| \\
%   |...| \\
%   |\setcitestyle{numbers,square}| \\
%   |\bibliographystyle{plain-fr}|
% \end{quote}
%
% Consultez la %
% \doc{natbib}{http://texdoc.net/pkg/natbib/} %
% de \pkg{natbib} pour les détails.
%
% \subsection{Bibliographies multiples}
% \label{sec:bibliographie:multiples}
%
% La rédaction de la thèse ou du mémoire par articles est une pratique
% qui gagne en popularité. Elle consiste à remplacer le corps
% principal du manuscrit par un ou des articles scientifiques,
% habituellement à raison d'un par chapitre. Dans de tels cas, on
% souhaitera utiliser une bibliographie distincte pour chacun de ces
% chapitres. L'ajout de |bibchapitre| ou de |bibsection| dans les
% options de la classe permet de réaliser cette mise en forme
% particulière.
%
% La classe fait appel au paquetage \pkg{chapterbib}
% \citep{chapterbib} pour composer des bibliographies multiples. La
% \doc{chapterbib}{http://texdoc.net/pkg/chapterbib/} du paquetage
% explique la procédure à suivre pour obtenir une bibliographie par
% chapitre avec {\BibTeX}. En résumé:
% \begin{enumerate}
% \item chaque chapitre comportant sa propre bibliographie doit
%   absolument être inséré dans le document principal avec la commande
%   \cmd{\include} \citep[section~3.5]{formation-latex-ul};
% \item chaque fichier visé doit contenir une commande
%   \cmd{\bibliographystyle} (\autoref{sec:bibliographie:style}) et
%   une commande \cmd{\bibliography};
% \item on obtient les diverses bibliographies en compilant avec
%   {\BibTeX} les fichiers individuels, et non le document maître.
% \end{enumerate}
%
% La plupart des auteurs préféreront sans doute l'effet de l'option
% |bibsection|, où la bibliographie apparait comme une section normale
% à la fin d'un chapitre. Lorsque \pkg{babel} est chargé, la section
% sera numérotée. Pour supprimer la numérotation, insérer dans le
% préambule du document la commande
% \begin{quote}
%   |\addto\extras|\meta{langue}|{\renewcommand{\bibsection}{%| \\
%   |  \section*{\bibname}\prebibhook}}|
% \end{quote}
% où \meta{langue} est la langue par défaut du document
% (\autoref{sec:langues}).
%
% Le gabarit |gabarit-doctorat-articles| fournit la structure de base
% d'une thèse par articles.
%
% \subsection{Style de la bibliographie}
% \label{sec:bibliographie:style}
%
% \begin{DescribeMacro}{\bibliographystyle}
%   Le format général de la bibliographie est contrôlé par un
%   \emph{style} choisi avec la commande \cmd{\bibliographystyle} dans
%   le préambule du document ou dans les fichiers de chaque chapitre
%   dans une thèse ou un mémoire par articles. Les styles standards de
%   {\LaTeX} sont |plain|, |unsrt|, |alpha| et |abbrv|.
% \end{DescribeMacro}
%
% Le paquetage \pkg{natbib} chargé par défaut par la classe supporte
% le style de citation auteur-année fréquemment employé en sciences
% naturelles, plusieurs commandes de citation, un grand nombre de
% styles de bibliographie ainsi que des entrées spécifiques pour les
% numéros ISBN et les URL. Le paquetage fournit des styles de
% bibliographie |plainnat|, |unsrtnat| et |abbrvnat| similaires aux
% styles standards, mais plus complets. Il existe des
% \link{http://mirrors.ctan.org/biblio/bibtex/contrib/bib-fr/}{versions
% francisées} de ces styles (et de quelques autres) dans CTAN.
%
% Le paquetage \pkg{francais-bst} \citep{francais-bst} fournit une
% feuille de style compatible avec \pkg{natbib} permettant de composer
% des bibliographies auteur-année respectant les normes de typographie
% française proposées dans \cite{Malo:1996}. Pour utiliser ce style,
% spécifier dans le préambule du document LaTeX
% \begin{quote}
%   |\bibliographystyle{francais}|
% \end{quote}
%
% Autrement, la FESP n'a pas d'exigences particulières quant à la
% présentation de la bibliographie (présentation du titre, des auteurs
% et autres informations bibliographiques).
%
% \section{Police de caractères du document}
% \label{sec:police}
%
% Les documents {\LaTeX} sont facilement reconnaissables par leur
% police de caractères par défaut, {\fontfamily{cmr}\selectfont
% Computer Modern}. Avec toute distribution {\LaTeX} moderne, il est
% maintenant simple d'utiliser l'une ou l'autre des polices PostScript
% standards. D'ailleurs la classe \class{ulthese} utilise la police
% sans empattements \textsf{Helvetica} pour composer les pages de
% titre.
%
% La FESP permet l'utilisation des polices Times, Palatino (la police
% du présent document) et Lucida~Bright dans les thèse et mémoires.
% \citet[section~10.2]{formation-latex-ul} explique comment utiliser ces
% polices dans votre document.
%
% \section{Gabarits}
% \label{sec:gabarits}
%
% Les gabarits livrés avec la classe comportent des commentaires
% succincts pour vous guider dans la préparation de votre manuscrit.
% Les sections suivantes fournissent des détails additionnels, et ce,
% dans l'ordre où les commandes apparaissent dans les gabarits.
%
% \begin{rem}
%   Il n'y a pas de gabarit spécifique pour un examen de doctorat. On
%   utilise le gabarit de thèse approprié en ajoutant simplement
%   l'option |examen| dans la commande \cmd{\documentclass}.
% \end{rem}
%
% \subsection{Encodage des fichiers}
%
% Composer en {\LaTeX} de longs textes dans une langue ayant recours
% aux signes diacritiques devient rapidement pénible si l'on utilise
% des commandes telles que |\'e|, |\`a| ou |\"o| pour entrer des lettres
% accentuées. Afin de pouvoir plutôt entrer directement |é|, |à| ou
% |ö|, {\LaTeX} doit être configuré pour reconnaître les lettres
% accentuées. C'est le rôle du paquetage \pkg{inputenc}
% \citep{inputenc}.
%
% Il existe plusieurs manières différentes d'encoder --- ou
% d'enregistrer --- les lettres accentuées et autres caractères
% spéciaux (comme, par exemple, le symbole de l'euro) dans un
% ordinateur. La méthode la plus répandue et celle standard sur les
% versions récentes des systèmes d'exploitation Linux et macOS est
% l'UTF-8 de la norme
% \link{http://fr.wikipedia.org/wiki/Unicode}{Unicode}. Les gabarits
% sont livrés dans ce type d'encodage.
%
% La déclaration
% \begin{quote}
%   |\usepackage[utf8]{inputenc}|
% \end{quote}
% dans le préambule assure que {\LaTeX} traitera correctement des
% fichiers source encodés en UTF-8.
%
% La norme Unicode n'est pas aussi uniformément supportée par Windows.
% Selon l'éditeur de texte employé et la version du système
% d'exploitation, il peut être nécessaire d'utiliser les normes
% d'encodage %
% \link{http://fr.wikipedia.org/wiki/ISO_8859-1}{ISO~8859-1} %
% (ou Latin-1; option |latin1| de \pkg{inputenc}), %
% \link{http://fr.wikipedia.org/wiki/ISO_8859-15}{ISO~8859-15} %
% (ou Latin-9; option |latin9|) ou %
% \link{http://fr.wikipedia.org/wiki/Windows-1252}{Windows-1252} %
% (options |cp1252| ou |ansinew|).
%
% La situation est plus simple avec {\XeLaTeX} puisqu'il gère
% nativement Unicode. Le paquetage \pkg{inputenc} est non seulement
% inutile, mais incompatible avec {\XeLaTeX}. C'est pourquoi, dans les
% gabarits, \pkg{inputenc} est chargé seulement lorsque {\XeTeX}
% n'est pas le moteur employé pour compiler le document.
%
% \subsection{Paquetages additionnels}
%
% Tel qu'expliqué à la \autoref{sec:paquetages}, la classe
% charge déjà quelques paquetages. Cependant, il est fort probable que
% vous devrez en charger d'autres pour composer votre document. Les
% gabarits prévoient un endroit pour le chargement de paquetages
% additionnels. Il est recommandé d'inscrire vos commandes
% \cmd{\usepackage} à cet endroit afin de respecter un certain ordre de
% chargement; voir ci-dessous.
%
% Si vous utilisez un paquetage non standard dans les distributions
% courantes (\TeX~Live, Mac\TeX, MiK\TeX), vous devez le fournir avec
% le code source de votre document lors du dépôt final.
%
% \subsection{Changement de police de caractères}
%
% Les gabarits comportent des déclarations types pour utiliser les
% polices Palatino ou Times sous {\LaTeX} ou, sous {\XeLaTeX}, leurs
% équivalents Pagella et Termes du projet
% \link{http://www.gust.org.pl/projects/e-foundry/tex-gyre/}{TeX~Gyre}.
%
% \subsection{Hyperliens}
% \label{sec:hyperref}
%
% Le paquetage \pkg{hyperref} \citep{hyperref} permet de transformer
% toutes les références en hyperliens cliquables lorsque le document
% est produit avec pdf{\LaTeX} ou {\XeLaTeX}. L'interaction de ce
% paquetage avec les autres est parfois (voire souvent) délicate. Pour
% cette raison, il est habituellement nécessaire de charger
% \pkg{hyperref} en tout dernier. C'est pourquoi il n'est pas chargé
% dans la classe, mais plutôt dans les gabarits. Prenez soin de
% maintenir le dernier rang de chargement lors de l'édition d'un
% gabarit.
%
% \begin{DescribeMacro}{\hyperrefsetup}
%   La configuration du paquetage dans les gabarits fait en sorte que
%   les liens sont simplement signalés par une couleur de texte
%   légèrement contrastante. L'utilisation de couleurs dans un
%   document requiert le paquetage |xcolor|, chargé par la classe. La
%   couleur de lien par défaut, |ULlinkcolor|, est définie dans la
%   classe; voir la \autoref{sec:couleurs}.
% \end{DescribeMacro}
%
% \subsection{Options de \pkg{babel}}
%
% \begin{DescribeMacro}{\frenchbsetup}
%   La commande \cmd{\frenchbsetup} de \pkg{babel} permet de contrôler
%   certains ajustements typographiques apportés par le paquetage en
%   mode français. Consultez la documentation de \pkg{babel} pour la
%   liste des options de configuration disponibles.
% \end{DescribeMacro}
%
% Les concepteurs de la classe \class{ulthese}  proposent trois
% ajustements dans les gabarits:
% \begin{enumerate}
% \item l'option |StandardItemizeEnv=true| évite que le mode français de
%   \pkg{babel} ne diminue l'espacement vertical dans les listes;
% \item l'option |ThinSpaceInFrenchNumbers=true| fait en sorte qu'une espace fine
%   sera utilisée comme séparateur des milliers dans les nombres plutôt
%   qu'une espace pleine;
% \item les options |og=«| et |fg=»| déclarent que les caractères « et
%   » utilisés dans le code source représentent les guillemets ouvrant
%   et fermant, respectivement. Cela évite de devoir utiliser les
%   commandes \cmd{\og} et \cmd{\fg} de \pkg{babel} tout en bénéficiant de
%   l'ajutement automatique des espaces autour des symboles.
% \end{enumerate}
% Vous devez évidemment désactiver ces ajustements si l'option
% |nobabel| est spécifiée au chargement de la classe.
%
% \subsection{Style de la bibliographie}
% \label{sec:bibliographystyle}
%
% Tel qu'expliqué à la \autoref{sec:bibliographie:style}, la commande
% \cmd{\bibliographystyle} permet de définir le mode de citation dans
% le texte et la présentation des notices bibliographiques. Cette
% commande devrait figurer dans les fichiers des chapitres
% individuels dans une thèse ou un mémoire par articles.
%
% \subsection{Déclarations des pages de titre}
%
% Les gabarits comportent toutes les déclarations nécessaires pour
% composer les pages de titre des divers types de thèse ou de mémoires.
% Vous devez remplacer les éléments se trouvant entre crochets <~> en
% respectant la forme indiquée. Assurez-vous de supprimer les
% caractères < et > afin qu'ils n'apparaissent pas sur les pages de titre de
% votre document.
%
% \subsection{Pages liminaires}
%
% \begin{DescribeMacro}{\frontmatter}
%   La commande \cmd{\frontmatter} déclare que {\LaTeX} doit considérer le
%   matériel qui suit comme des pages liminaires. En pratique, cela
%   résulte essentiellement en une numérotation des pages en chiffres
%   romains.
% \end{DescribeMacro}
%
% Les normes de présentation de la FESP édictent que les thèses et
% mémoires devraient comporter les pages liminaires suivantes, dans
% l'ordre:
% \begin{enumerate}
% \item la page de titre (obligatoire);
% \item la page frontispice (obligatoire);
% \item un résumé en français (obligatoire);
% \item un résumé en anglais (recommandé, mais non obligatoire);
% \item une table des matières (obligatoire);
% \item une liste des tableaux;
% \item une liste des figures;
% \item une liste des abbréviations et des sigles;
% \item une dédicace;
% \item une épigraphe;
% \item des remerciements;
% \item un avant-propos (obligatoire dans le cas d'une thèse ou d'un
%   mémoire par articles).
% \end{enumerate}
%
% \begin{rem}
%   Pour un examen de doctorat, on ne considère que la page de titre
%   comme obligatoire. L'option de classe |examen| supprime d'ailleurs
%   la page frontispice. Il est laissé aux auteurs le soin de
%   supprimer les autres pages liminaires des gabarits.
% \end{rem}
%
% Les commandes
% \begin{quote}
%   \cmd{\pagestitre} \\
%   \cmd{\tableofcontents} \\
%   \cmd{\listoftables} \\
%   \cmd{\listoffigures} \\
%   \cmd{\dedicace}\marg{texte} \\
%   \cmd{\epigraphe}\marg{texte}\marg{auteur}
% \end{quote}
% permettent de générer les pages correspondantes. Seules les deux
% dernières commandes admettent des arguments.
%
% \begin{DescribeMacro}{\chapter*}
%   Les résumés, la liste des abbréviations et des sigles, les
%   remerciements et l'avant-propos sont composés comme des chapitres
%   normaux, mais sans être numérotés. Il faut donc définir ces éléments
%   avec la commande \cmd{\chapter*}.
% \end{DescribeMacro}
%
% \begin{DescribeMacro}{\phantomsection}
%   \begin{DescribeMacro}{\addcontentsline}
%     Les sections declarées avec la commande \cmd{\chapter*}
%     n'apparaissent pas dans la table des matières. Comme les normes
%     de présentation de la FESP exigent que toutes les pages
%     liminaires y figurent, on fait suivre les commandes
%     \cmd{\chapter*}\marg{Titre} des commandes
%   \end{DescribeMacro}
% \end{DescribeMacro}
% \begin{quote}
%   |\phantomsection\addcontentsline{toc}{chapter}|\marg{Titre}
% \end{quote}
% Celles-ci ajoutent à la table des matières (|toc|) une section de
% niveau |chapter| dont le titre est \meta{Titre}. La commande
% \cmd{\phantomsection} est rendue nécessaire (ou recommandée) par le
% paquetage \pkg{hyperref}.
%
% \subsection{Corps du document}
%
% \begin{DescribeMacro}{\mainmatter}
%   La commande \cmd{\mainmatter} délimite le début du corps du document. La
%   numérotation des pages passe en chiffres arabes.
% \end{DescribeMacro}
%
% Le corps du document devrait normalement compter une introduction (non
% numérotée), un développement divisé en chapitres (numérotés) et une
% conclusion (non numérotée).
%
% \subsection{Annexes}
%
% \begin{DescribeMacro}{\appendix}
%   Si la thèse ou le mémoire comporte une ou plusieurs annexes,
%   composer celles-ci comme des chapitres normaux insérés dans le
%   document maître après la commande \cmd{\appendix}. Cette commande a
%   pour effet de passer d'un mode de numération numérique (1, 1.1, 2, 2.1,
%   \dots) à un mode alphanumérique (A, A.1, B, B.1, \dots).
% \end{DescribeMacro}
%
% \subsection{Bibliographie}
%
% La bibliographie figure normalement à la toute fin du document, sous
% forme de chapitre non numéroté. Tel qu'expliqué à la
% \autoref{sec:bibliographie:multiples}, la commande
% \cmd{\bibliography} doit plutôt se trouver dans les fichiers des
% chapitres dans une thèse ou un mémoire par articles.
%
% \section{Aide additionnelle}
%
% Pour obtenir de l'aide additionnelle sur l'utilisation de la classe
% \class{ulthese} (et non sur celle de {\LaTeX} en général), prière
% de consulter d'abord
% \begin{enumerate}
% \item le \link{http://www.theses.ulaval.ca/wiki/}{WikiThèse} de
%   l'Université Laval, en particulier la
%   \link{http://www.theses.ulaval.ca/wiki/index.php?title=FAQ}{Foire aux questions};
% \item les
%   \link{http://listes.ulaval.ca/listserv/archives/ulthese-aide.html}{archives}
%   de la liste de distribution \texttt{ulthese-aide}.
% \end{enumerate}
% Si la réponse à votre question ne se trouve ni dans le wiki, ni dans
% les archives, alors écrire à l'adresse
% \href{mailto:ulthese-aide@listes.ulaval.ca}{\url{ulthese-aide@listes.ulaval.ca}}.
%
% \StopEventually{
%   \begin{thebibliography}{15}
%     \addcontentsline{toc}{section}{Références}
%   \bibitem[{Arseneau(2010)}]{chapterbib}
%     Arseneau, D. 2010,
%     {\selectlanguage{english}\emph{chapterbib. Multiple
%     bibliographies in {\LaTeX}}}.
%     URL~\url{http://www.ctan.org/pkg/chapterbib/}.
%
%   \bibitem[{Braams et Bezos(2016)}]{babel}
%     Braams, J. et J.~Bezos. 2016,
%     {\selectlanguage{english}\emph{Babel}}.
%     URL~\url{http://www.ctan.org/pkg/babel/}.
%
%   \bibitem[{Carlisle et {The \LaTeX3\ Project}(2016)}]{graphicx}
%     Carlisle, D. et {The \LaTeX3\ Project}. 2016,
%     {\selectlanguage{english}\emph{Packages in the `graphics'
%     Bundle}}. URL~\url{http://www.ctan.org/pkg/graphics/}.
%
%   \bibitem[{Charette(2015)}]{polyglossia}
%     Charette, F. 2015, {\selectlanguage{english}\emph{Polyglossia:
%     An Alternative to Babel for {\XeLaTeX} and {\LuaLaTeX}}}.
%     URL~\url{http://www.ctan.org/pkg/polyglossia/}, current
%     maintainer Arthur Reutenauer.
%
%   \bibitem[{Daly(2010)}]{natbib}
%     Daly, P.~W. 2010, {\selectlanguage{english}\emph{Natural
%     Sciences Citations and References}}.
%     URL~\url{http://www.ctan.org/pkg/natbib/}.
%
%   \bibitem[{Goulet(2013)}]{francais-bst}
%     Goulet, V. 2013, {\selectlanguage{french}«Paquetage
%     \pkg{francais-bst}»},
%     URL~\url{http://www.ctan.org/pkg/francais-bst/}.
%
%   \bibitem[{Goulet(2016)}]{formation-latex-ul}
%     Goulet, V. 2016, {\selectlanguage{french}\emph{Rédaction avec
%     {\LaTeX}}}, document libre sous contrat Creative Commons.
%     ISBN~978-2-9811416-7-5.
%     URL~\url{https://ctan.org/pkg/formation-latex-ul}.
%
%   \bibitem[{Harders(2012)}]{numprint}
%     Harders, H. 2012, {\selectlanguage{english}\emph{The
%     \pkg{numprint} package}}.
%     URL~\url{http://www.ctan.org/pkg/numprint/}.
%
%   \bibitem[{Jeffrey et Mittelbach(2015)}]{inputenc}
%     Jeffrey, A. et F.~Mittelbach. 2015,
%     {\selectlanguage{english}\emph{inputenc.sty}}.
%     URL~\url{http://www.ctan.org/pkg/inputenc/}.
%
%   \bibitem[{Kern(2016)}]{xcolor}
%     Kern, D.~U. 2016, {\selectlanguage{english}\emph{Extending
%     {\LaTeX}’s color facilities: the \pkg{xcolor} package}}.
%     URL~\url{http://www.ctan.org/pkg/xcolor/}.
%
%   \bibitem[{Malo(1996)}]{Malo:1996}
%     Malo, M. 1996, {\selectlanguage{french}\emph{Guide de la
%     communication écrite au cégep, à l'université et en
%     entreprise}}, Québec Amérique. ISBN~978-2-8903-7875-9.
%
%   \bibitem[{Rahtz et Oberdiek(2017)}]{hyperref}
%     Rahtz, S. et H.~Oberdiek. 2017,
%     {\selectlanguage{english}\emph{Hypertext marks in {\LaTeX}: a
%     manual for \pkg{hyperref}}}.
%     URL~\url{http://www.ctan.org/pkg/hyperref/}.
%
%   \bibitem[{Robertson et Hosny(2017)}]{fontspec}
%     Robertson, W. et K.~Hosny. 2017,
%     {\selectlanguage{english}\emph{The \pkg{fontspec} package: Font
%     selection for {\XeLaTeX} and {\LuaLaTeX}}}.
%     URL~\url{http://www.ctan.org/pkg/fontspec/}.
%
%   \bibitem[{Robertson et coll.(2017)Robertson, Stephani et
%     Hosny}]{unicode-math}
%     Robertson, W., P.~Stephani et K.~Hosny. 2017,
%     {\selectlanguage{english}\emph{Experimental {U}nicode
%     Mathematical Typesetting: The \pkg{unicode-math} Package}}.
%     ULR~\url{http://www.ctan.org/pkg/unicode-math/}.
%
%   \bibitem[{Wilson(2016)}]{memoir}
%     Wilson, P. 2016, {\selectlanguage{english}\emph{The Memoir Class
%     for Configurable Typesetting}}, 8{\ieme} éd., The Herries Press.
%     URL~\url{http://www.ctan.org/pkg/memoir/}, maintained by Lars
%     Madsen.
%   \end{thebibliography}
%   \PrintChanges
% }
%
% ^^A Début du code de la classe
%
% \appendix
% \section{Mise en {\oe}uvre}
% \label{sec:meo}
%
% Cette annexe passe en revue le code {\TeX} et {\LaTeX} de la
% classe. Elle ne risque d'intéresser que les personnes qui souhaitent
% explorer comment la classe est programmée.
%
% \subsection{Tests et valeurs booléennes}
% \changes{4.4}{2017-06-01}{Tests et valeurs booléennes réalisés sans ifthen}
%
% Le paquetage \pkg{ifxetex} permet de tester si un document est
% compilé avec \XeTeX.
%    \begin{macrocode}
%<*class>
\RequirePackage{ifxetex}
%    \end{macrocode}
% Nous définissons ici toutes les valeurs booléennes requises par la
% classe.
%    \begin{macrocode}
\newif\ifUL@babel       \UL@babeltrue        % charger babel?
\newif\ifUL@natbib      \UL@natbibtrue       % charger natbib?
\newif\ifUL@chapterbib  \UL@chapterbibfalse  % charger chapterbib?
\newif\ifUL@sectionbib  \UL@sectionbibfalse  % option sectionbib de chapterbib?
\newif\ifUL@isthesis                         % programme est une thèse?
\newif\ifUL@iscotutelle \UL@iscotutellefalse % thèse en cotutelle?
\newif\ifUL@isexam      \UL@isexamfalse      % examen de doctorat?
\newif\ifUL@hassubtitle \UL@hassubtitlefalse % document a un sous-titre?
%    \end{macrocode}
%
% \subsection{Options de la classe}
%
% Il y a cinq grandes catégories d'options propres à la classe: la
% possibilité d'empêcher le chargement du paquetage \pkg{natbib};
% la possibilité d'empêcher le chargement du paquetage \pkg{babel};
% la taille de la police de caractères en points; le type de grade; la
% déclaration qu'il s'agit d'un examen de doctorat.
%
% \begin{macro}{nonatbib}
%   L'option |nonatbib| permet d'empêcher la classe de charger le
%   paquetage \pkg{natbib} en cas d'incompatibilité avec d'autres
%   paquetages spécialisés de mise en forme de la bibliographie.
%    \begin{macrocode}
\DeclareOption{nonatbib}{\UL@natbibfalse}
%    \end{macrocode}
% \end{macro}
%
% \begin{macro}{bibchapitre}
%   \begin{macro}{bibsection}
%     L'option |bibchapitre| entraine le chargement du paquetage
%     \pkg{chapterbib} qui permet de composer des bibliographies
%     multiples dans un document, habituellement une par
%     chapitre. Avec l'option |bibsection|, la bibliographie se
%     présente sous la forme d'une section plutôt que d'un chapitre.
%     La seconde option implique la première.
%    \begin{macrocode}
\DeclareOption{bibchapitre}{\UL@chapterbibtrue}
\DeclareOption{bibsection}{\UL@chapterbibtrue\UL@sectionbibtrue}
%    \end{macrocode}
%   \end{macro}
% \end{macro}
%
% \begin{macro}{nobabel}
%   L'option |nobabel| permet d'empêcher la classe de charger le
%   paquetage \pkg{babel}. Cette option peut s'avérer utile pour
%   les utilisateurs de {\XeLaTeX} qui souhaitent plutôt utiliser
%   \pkg{poyglossia} pour le traitement des langues dans leur
%   document.
%    \begin{macrocode}
\DeclareOption{nobabel}{\UL@babelfalse}
%    \end{macrocode}
% \end{macro}
%
% \begin{macro}{10pt,11pt,12pt}
%   Les valeurs possibles pour la taille de la police de caractères
%   sont |10pt|, |11pt| et |12pt|. Cette option est gérée au niveau de
%   la classe afin de s'assurer que les divers éléments sur les pages
%   de titre sont toujours de la même taille. La taille de la police
%   par défaut permet de déterminer si, par exemple, le titre du
%   document doit être dans la taille \cmd{\Huge}, \cmd{\huge} ou \cmd{\LARGE} de
%   \class{memoir}.
%
%   La taille de la police est passée à \class{memoir} et la macro
%   \cmd{\UL@ptsize} stocke la taille des caractères pour usage futur.
%    \begin{macrocode}
\newcommand*{\UL@ptsize}{}
\DeclareOption{10pt}{%
  \PassOptionsToClass{10pt}{memoir}
  \renewcommand*{\UL@ptsize}{10}}
\DeclareOption{11pt}{%
  \PassOptionsToClass{11pt}{memoir}
  \renewcommand*{\UL@ptsize}{11}}
\DeclareOption{12pt}{%
  \PassOptionsToClass{12pt}{memoir}
  \renewcommand*{\UL@ptsize}{12}}
%    \end{macrocode}
% \end{macro}
%
% \begin{macro}{PhD,MSc,MA,...}
%   Définition du type de grade et si la thèse ou le mémoire est
%   multifacultaire, effectué en cotutelle, en bidiplomation ou en
%   extension. Certaines options ne sont valides que pour une thèse ou
%   que pour un mémoire. La page de titre des programmes en extension
%   comporte une mention «offert en extension» ou «offerte en
%   extension» selon qu'il s'agit d'un doctorat ou d'une maîtrise; le
%   bon terme est défini avec l'option correspondante.
%    \begin{macrocode}
\newcommand*{\UL@typenum}{}
\DeclareOption{LLD}{%
  \UL@isthesistrue
  \renewcommand*{\UL@typenum}{0}
  \newcommand*{\UL@degree}{Docteur en droit (LL.~D.)}}
\DeclareOption{DMus}{%
  \UL@isthesistrue
  \renewcommand*{\UL@typenum}{0}
  \newcommand*{\UL@degree}{Docteur en musique (D.~Mus.)}}
\DeclareOption{DPsy}{%
  \UL@isthesistrue
  \renewcommand*{\UL@typenum}{0}
  \newcommand*{\UL@degree}{Docteur en psychologie (D.~Psy.)}}
\DeclareOption{DThP}{%
  \UL@isthesistrue
  \renewcommand*{\UL@typenum}{0}
  \newcommand*{\UL@degree}{Docteur en th\'eologie pratique (D.~Th.~P.)}}
\DeclareOption{PhD}{%
  \UL@isthesistrue
  \renewcommand*{\UL@typenum}{0}
  \newcommand*{\UL@degree}{Philosophi{\ae} doctor (Ph.~D.)}}
\DeclareOption{MATDR}{%
  \UL@isthesisfalse
  \renewcommand*{\UL@typenum}{0}
  \newcommand*{\UL@degree}{Ma\^itre en am\'enagement du territoire %
    et d\'eveloppement r\'egional (M.ATDR)}}
\DeclareOption{MArch}{%
  \UL@isthesisfalse
  \renewcommand*{\UL@typenum}{0}
  \newcommand*{\UL@degree}{Ma\^itre en architecture (M.~Arch.)}}
\DeclareOption{MA}{%
  \UL@isthesisfalse
  \renewcommand*{\UL@typenum}{0}
  \newcommand*{\UL@degree}{Ma\^itre \`es arts (M.A.)}}
\DeclareOption{LLM}{%
  \UL@isthesisfalse
  \renewcommand*{\UL@typenum}{0}
  \newcommand*{\UL@degree}{Ma\^itre en droit (LL.~M.)}}
\DeclareOption{MErg}{%
  \UL@isthesisfalse
  \renewcommand*{\UL@typenum}{0}
  \newcommand*{\UL@degree}{Ma\^itre en ergoth\'erapie (M.~Erg.)}}
\DeclareOption{MMus}{%
  \UL@isthesisfalse
  \renewcommand*{\UL@typenum}{0}
  \newcommand*{\UL@degree}{Ma\^itre en musique (M.~Mus.)}}
\DeclareOption{MPht}{%
  \UL@isthesisfalse
  \renewcommand*{\UL@typenum}{0}
  \newcommand*{\UL@degree}{Ma\^itre en physioth\'erapie (M.~Pht.)}}
\DeclareOption{MSc}{%
  \UL@isthesisfalse
  \renewcommand*{\UL@typenum}{0}
  \newcommand*{\UL@degree}{Ma\^itre \`es sciences (M.~Sc.)}}
\DeclareOption{MScGeogr}{%
  \UL@isthesisfalse
  \renewcommand*{\UL@typenum}{0}
  \newcommand*{\UL@degree}{Ma\^itre en sciences g\'eographiques (M.~Sc.~g\'eogr.)}}
\DeclareOption{MServSoc}{%
  \UL@isthesisfalse
  \renewcommand*{\UL@typenum}{0}
  \newcommand*{\UL@degree}{Ma\^itre en service social (M.~Serv.~soc.)}}
\DeclareOption{MPsEd}{%
  \UL@isthesisfalse
  \renewcommand*{\UL@typenum}{0}
  \newcommand*{\UL@degree}{Ma\^itre en psycho\'education (M.~Ps.~\'ed.)}}
\DeclareOption{multifacultaire}{%
  \ifUL@isthesis
    \renewcommand*{\UL@typenum}{1}
  \else
    \ClassError{ulthese}{%
      Incompatible option multifacultaire}
    {Use this option with a doctorate degree only.}
  \fi}
\DeclareOption{cotutelle}{%
  \ifUL@isthesis
    \renewcommand*{\UL@typenum}{2}
    \UL@iscotutelletrue
  \else
    \ClassError{ulthese}{%
      Incompatible option cotutelle}
    {Use this option with a doctorate degree only.}
  \fi}
\DeclareOption{bidiplomation}{%
  \ifUL@isthesis
    \ClassError{ulthese}{%
      Incompatible option bidiplomation}
    {Use this option with a master degree only.}
  \else
    \renewcommand*{\UL@typenum}{2}
  \fi}
\DeclareOption{extensionUdeS}{%
  \ifUL@isthesis
    \renewcommand*{\UL@typenum}{3}
    \newcommand*{\UL@offered}{offert}
    \newcommand*{\UL@extensionat}{Universit\'e de Sherbrooke}
    \newcommand*{\UL@extensionloc}{Sherbrooke, Canada}
  \else
    \ClassError{ulthese}{%
      Incompatible option extensionUdeS}
    {Use this option with a doctorate degree only.}
  \fi}
\DeclareOption{extensionUQO}{%
  \ifUL@isthesis
    \renewcommand*{\UL@typenum}{3}
    \newcommand*{\UL@offered}{offert}
    \newcommand*{\UL@extensionat}{Universit\'e du Qu\'ebec en Outaouais}
    \newcommand*{\UL@extensionloc}{Gatineau, Canada}
  \else
    \ClassError{ulthese}{%
      Incompatible option extensionUQO}
    {Use this option with a doctorate degree only.}
  \fi}
\DeclareOption{extensionUQAC}{%
  \ifUL@isthesis
    \ClassError{ulthese}{%
      Incompatible option extensionUQAC}
    {Use this option with a master degree only.}
  \else
    \renewcommand*{\UL@typenum}{3}
    \newcommand*{\UL@offered}{offerte}
    \newcommand*{\UL@extensionat}{Universit\'e du Qu\'ebec \`a Chicoutimi}
    \newcommand*{\UL@extensionloc}{Chicoutimi, Canada}
  \fi}
%    \end{macrocode}
% \end{macro}
%
% \begin{macro}{examen}
%   L'option |examen| change l'appellation «Thèse» sur la couverture
%   pour «Examen de doctorat» et supprime la page frontispice. Elle
%   n'est compatible qu'avec l'une des options de thèse, autrement un
%   message d'erreur est émis.
%    \begin{macrocode}
\DeclareOption{examen}{%
  \ifUL@isthesis
    \UL@isexamtrue
  \else
    \ClassError{ulthese}{%
      Incompatible option examen}
    {Use this option with a thesis type only.}
  \fi}
%    \end{macrocode}
% \end{macro}
%
% \subsection{Chargement de la classe \class{memoir}}
%
% Toutes les options de la classe sont passées à \class{memoir}. Le
% format de papier et la taille de police par défaut sont, dans
% l'ordre, |letterpaper| et |11pt|. On vérifie qu'un type de grade a
% bien été déclaré. L'option de \class{memoir} |oneside| est
% explicitement déclarée afin d'éviter toute tentative de passer outre
% à cette exigence de la FESP.
%    \begin{macrocode}
\DeclareOption*{\PassOptionsToClass{\CurrentOption}{memoir}}
\ExecuteOptions{11pt,letterpaper}
\ProcessOptions\relax
\ifx\UL@typenum\empty
  \ClassError{ulthese}{%
    No thesis type specified}
    {Declare the thesis type as a class option.}
\fi
\LoadClass[oneside]{memoir}
%    \end{macrocode}
%
% \subsection{Paquetages requis}
%
% La classe s'efforce de charger un minimum de paquetages afin d'éviter
% les conflits potentiels.
%
% {\XeLaTeX} requiert le paquetage \pkg{fontspec} pour le
% traitement des polices. Le paquetage \pkg{unicode-math} facilite
% également le traitement des polices et des symboles mathématiques
% avec ce moteur. Sous {\LaTeX}, il est aujourd'hui préférable
% d'utiliser les polices T1.
%    \begin{macrocode}
\ifxetex
  \RequirePackage{fontspec}
  \RequirePackage{unicode-math}
  \defaultfontfeatures{Ligatures=TeX}
\else
  \RequirePackage[T1]{fontenc}
\fi
%    \end{macrocode}
%
% Les paquetages \pkg{natbib} et \pkg{chapterbib} doivent être chargés
% avant \pkg{babel} pour bien fonctionner, le cas échéant. Tel que
% précisé dans la documentation de \pkg{natbib}, l'option |sectionbib|
% est passée à ce paquetage lorsqu'il est chargé, ou à
% \pkg{chapterbib} autrement.
%    \begin{macrocode}
\ifUL@natbib
  \ifUL@sectionbib
    \PassOptionsToPackage{sectionbib}{natbib}
  \fi
  \RequirePackage[round,semicolon,authoryear]{natbib}
\fi
\ifUL@chapterbib
  \ifUL@sectionbib
     \ifUL@natbib\else
       \PassOptionsToPackage{sectionbib}{chapterbib}
     \fi
  \fi
  \RequirePackage{chapterbib}
\fi
%    \end{macrocode}
%
% Le support pour les langues autres que l'anglais est offert par le
% paquetage \pkg{babel} --- à moins que l'option |nobabel| n'ait
% été spécifiée au chargement de la classe. Les langues sont passées
% en option de la classe, et non du paquetage. Le paquetage
% \pkg{numprint} est requis par \pkg{babel} pour la définition
% de la commande de mise en forme des nombres \cmd{\nombre}.
%    \begin{macrocode}
\ifUL@babel
  \RequirePackage{babel}
  \RequirePackage[autolanguage]{numprint}
\fi
%    \end{macrocode}
%
% L'insertion du logo de l'Université sur la page de titre requiert
% \pkg{graphicx}. Les coloration des hyperliens requiert \pkg{xcolor}.
% \autoref{sec:hyperref}).
%    \begin{macrocode}
\RequirePackage{graphicx}
\RequirePackage{xcolor}
%    \end{macrocode}
%
% La commande \cmd{\textcopyright} utilisée sur la page de titre requiert le
% paquetage \pkg{textcomp} pour obtenir un beau signe de copyright.
%    \begin{macrocode}
\RequirePackage{textcomp}
%    \end{macrocode}
%
% \subsection{Paquetage incompatible}
%
% Le chargement du paquetage \pkg{geometry} avec la classe
% \class{memoir} modifie les marges du document. Pour cette raison,
% \pkg{geometry} est déclaré incompatible avec la classe.
%    \begin{macrocode}
\AtBeginDocument{%
  \@ifpackageloaded{geometry}{%
    \ClassError{ulthese}{%
      Package geometry is incompatible with this class}
    {Use the memoir class facilities to change the page layout.}}{\relax}}
%    \end{macrocode}
%
% \subsection{Couleur des hyperliens}
% \label{sec:couleurs}
%
% Le paquetage \pkg{hyperref} est chargé dans les gabarits afin de
% demeurer le dernier paquetage chargé; voir la
% \autoref{sec:hyperref}. La classe définit néanmoins une couleur standard pour
% les hyperliens, une teinte de bleu assez foncée pour être à fois
% visible en couleur et peu contrastante si le document est imprimé en
% noir et blanc.
%    \begin{macrocode}
\definecolor{ULlinkcolor}{rgb}{0,0,0.3}
%    \end{macrocode}
%
% \subsection{Marges}
%
% Les marges exigées par les normes de présentation de la FESP sont de
% 30~mm pour les marges gauche et droite et 25~mm pour les marges
% supérieure et inférieure. Le pied de page est placé de sorte que le
% folio de page se retrouve à 10~mm du bas de la page.
%    \begin{macrocode}
\setlrmarginsandblock{30mm}{30mm}{*}
\setulmarginsandblock{25mm}{25mm}{*}
\checkandfixthelayout[nearest]
\setlength{\footskip}{\lowermargin}
\addtolength{\footskip}{-10mm}
%    \end{macrocode}
%
% Comme les thèses et mémoires comportent normalement plusieurs pages
% liminaires, il arrive que des folios (en chiffres romains) dépassent
% dans la marge de droite dans la table des matières. Pour régler ce
% problème, nous augmentont la largeur de la boîte prévue pour les
% imprimer.
%    \begin{macrocode}
\renewcommand{\@pnumwidth}{3em}
\renewcommand{\@tocrmarg}{4em}
%    \end{macrocode}
%
% \subsection{Interligne}
%
% L'espacement entre les lignes est d'un interligne et demi.
% L'espacement «double» entre les paragraphes est fixé à
% |0.5\baselineskip| afin d'en arriver à une disposition agréable à
% l'{\oe}il. Le retrait de première ligne est supprimé puisque plus
% nécessaire suite à l'ajout de l'espacement entre les paragraphes.
%    \begin{macrocode}
\OnehalfSpacing
\setlength{\parskip}{0.5\baselineskip}
\setlength{\parindent}{0em}
%    \end{macrocode}
%
% La table des matières, la liste des tableaux et la liste des figures
% sont composées à interligne simple.
%    \begin{macrocode}
\renewcommand{\tocheadstart}{\SingleSpacing\chapterheadstart}
\renewcommand{\lotheadstart}{\SingleSpacing\chapterheadstart}
\renewcommand{\lofheadstart}{\SingleSpacing\chapterheadstart}
%    \end{macrocode}
%
% \subsection{Entêtes et pieds de page}
%
% Les règles pour les entêtes et pieds de page sont uniformes pour
% tout le document: aucun entête et folio au centre du pied de page.
% Ceci correspond au style standard |plain|.
%    \begin{macrocode}
\pagestyle{plain}
%    \end{macrocode}
%
% \subsection{Pages de titre}
%
% Le code pour traiter et composer la page de titre et la page
% frontispice constitue l'essentiel de la classe.
%
% \subsubsection{Famille et style de la police de caractères}
%
% Les pages de titre sont composées avec la police \textsf{Helvetica}
% (famille \texttt{phv} dans la classification NFSS) dans les
% tailles\footnote{%
%   La police Helvetica produite par {\LaTeX} est plus grande que
%   celle utilisée par Microsoft Word. Pour cette raison, les tailles
%   utilisées dans la classe sont toutes quelques points inférieures à
%   celles des gabarits Word.}%
% et les graisses présentées au \autoref{tab:polices}. La
% déclaration |\fontencoding{T1}| est nécessaire avec {\XeLaTeX} pour
% explicitement charger la même police que sous {\LaTeX}.
%
% \begin{table}
%   \centering
%   \begin{tabular}{ll}
%     \toprule
%     Élément          & Police \\
%     \midrule
%     Titre            & 17~points gras \\
%     Sous-titre       & 14~points gras \\
%     Auteur           & 12~points gras \\
%     Nom du programme & 12~points gras \\
%     Autres éléments  & 12~points normal \\
%     \bottomrule
%   \end{tabular}
%   \caption{Tailles et graisses de la police Helvetica des éléments
%     de la page de titre}
%   \label{tab:polices}
% \end{table}
%
%    \begin{macrocode}
\newcommand*{\UL@phvfamily}{\fontencoding{T1}\fontfamily{phv}\selectfont}
%    \end{macrocode}
% Les commandes sélectionnant ces polices sont adaptées selon la
% taille de police choisie pour le document afin d'être toujours
% identiques. Nous utilisons les déclarations de taille de police de
% la classe \class{memoir}, présentées au tableau~3.9 de sa
% documentation.
%    \begin{macrocode}
\ifnum\UL@ptsize=10\relax
  \newcommand*{\UL@fonttitle}{\normalfont\huge\bfseries\UL@phvfamily}
  \newcommand*{\UL@fontsubtitle}{\normalfont\LARGE\bfseries\UL@phvfamily}
  \newcommand*{\UL@fontauthor}{\normalfont\Large\bfseries\UL@phvfamily}
  \newcommand*{\UL@fontprogram}{\UL@fontauthor}
  \newcommand*{\UL@fontbase}{\normalfont\Large\UL@phvfamily}
\fi
\ifnum\UL@ptsize=11\relax
  \newcommand*{\UL@fonttitle}{\normalfont\LARGE\bfseries\UL@phvfamily}
  \newcommand*{\UL@fontsubtitle}{\normalfont\Large\bfseries\UL@phvfamily}
  \newcommand*{\UL@fontauthor}{\normalfont\large\bfseries\UL@phvfamily}
  \newcommand*{\UL@fontprogram}{\UL@fontauthor}
  \newcommand*{\UL@fontbase}{\normalfont\large\UL@phvfamily}
\fi
\ifnum\UL@ptsize=12\relax
  \newcommand*{\UL@fonttitle}{\normalfont\Large\bfseries\UL@phvfamily}
  \newcommand*{\UL@fontsubtitle}{\normalfont\large\bfseries\UL@phvfamily}
  \newcommand*{\UL@fontauthor}{\normalfont\normalsize\bfseries\UL@phvfamily}
  \newcommand*{\UL@fontprogram}{\UL@fontauthor}
  \newcommand*{\UL@fontbase}{\normalfont\normalsize\UL@phvfamily}
\fi
%    \end{macrocode}
%
% \subsubsection{Interfaces interne et externe}
%
% Définition des commandes permettant de construire les pages de titre.
% L'interface utilisateur est basée sur un ensemble de commandes
% internes. On commence par celles-ci.
%    \begin{macrocode}
\newcommand{\UL@maintitle}{}
\newcommand{\UL@subtitle}{}
\newcommand*{\UL@author}{}
\newcommand*{\UL@year}{}
\newcommand*{\UL@program}{}
\newcommand*{\UL@director}{}
\newcommand*{\UL@codirector}{}
\newcommand*{\UL@nameother}{}
\newcommand*{\UL@degreeother}{}
\newcommand*{\UL@facUL}{}
\newcommand*{\UL@facother}{}
%    \end{macrocode}
% Puis les commandes visibles pour les utilisateurs, qui redéfinissent
% les commandes internes. Voir la \autoref{sec:commandes} pour leur
% signification.
%    \begin{macrocode}
\newcommand{\titre}[1]{\renewcommand{\UL@maintitle}{#1}}
\newcommand{\soustitre}[1]{%
  \UL@hassubtitletrue
  \renewcommand{\UL@subtitle}{#1}}
\newcommand*{\auteur}[1]{\renewcommand*{\UL@author}{#1}}
\newcommand*{\annee}[1]{\renewcommand*{\UL@year}{#1}}
\newcommand*{\programme}[1]{\renewcommand*{\UL@program}{#1}}
\newcommand*{\direction}[1]{\renewcommand*{\UL@director}{#1}}
\newcommand*{\codirection}[1]{\renewcommand*{\UL@codirector}{#1}}
\newcommand*{\univcotutelle}[1]{\renewcommand*{\UL@nameother}{#1}}
\newcommand*{\gradecotutelle}[1]{\renewcommand*{\UL@degreeother}{#1}}
\newcommand*{\univbidiplomation}[1]{\renewcommand*{\UL@nameother}{#1}}
\newcommand*{\gradebidiplomation}[1]{\renewcommand*{\UL@degreeother}{#1}}
\newcommand{\faculteUL}[1]{\renewcommand*{\UL@facUL}{#1}}
\newcommand*{\faculteUdeS}[1]{\renewcommand*{\UL@facother}{#1}}
\newcommand*{\faculteUQO}[1]{\renewcommand*{\UL@facother}{#1}}
\newcommand*{\faculteUQAC}[1]{\renewcommand*{\UL@facother}{#1}}
%    \end{macrocode}
%
% \subsubsection{Titre et sous-titre}
%
% Le titre et le sous-titre peuvent s'étendre sur plus d'une ligne.
% Sans traitement spécial, un long titre ou sous-titre aurait pour
% impact de décaler vers le bas tous les autres éléments de la page
% de titre. Pour contrer ce phénomène, nous devrons mesurer la hauteur du
% titre et du sous-titre pour ensuite ajuster en conséquence la
% distance entre ce bloc et les éléments qui suivent.
%
% \begin{macro}{\UL@measuretitle}
%   On place le titre et le sous-titre centrés dans des boîtes
%   \cmd{\UL@titlebox} et \cmd{\UL@subtitlebox}. La commande
%   \cmd{\UL@measuretitle} permettra de mesurer leur hauteur lorsque le
%   titre sera créé avec \cmd{\pagestitre}, plus loin. Un espacement
%   vertical d'un demi interligne est ajouté entre le titre et le
%   sous-titre, le cas échéant.
%    \begin{macrocode}
\newsavebox{\UL@titlebox}
\newsavebox{\UL@subtitlebox}
\newlength{\UL@titleboxtotht}
\newlength{\UL@subtitleboxtotht}
\newcommand{\UL@measuretitle}{%
  \setbox\UL@titlebox=\vbox{%
    \centering\UL@fonttitle\UL@maintitle}
  \setlength{\UL@titleboxtotht}{%
    \dimexpr\ht\UL@titlebox+\dp\UL@titlebox}
  \ifUL@hassubtitle
    \setbox\UL@subtitlebox=\vbox{%
      \centering\vspace*{0.5\baselineskip}\UL@fontsubtitle\UL@subtitle}
    \setlength{\UL@subtitleboxtotht}{%
      \dimexpr\ht\UL@subtitlebox+\dp\UL@subtitlebox}
  \fi}
%    \end{macrocode}
% \end{macro}
%
% \subsubsection{Type de document}
%
% \begin{macro}{\UL@typeofdoc}
%   La commande |\UL@typeofdoc| contient le type de document qui est
%   produit: thèse, thèse en cotutelle, maîtrise
%   ou examen de doctorat.
%    \begin{macrocode}
\ifUL@isthesis
  \ifUL@iscotutelle
    \newcommand*{\UL@typeofdoc}{Th\`ese en cotutelle}
  \else
    \newcommand*{\UL@typeofdoc}{Th\`ese}
  \fi
\else
  \newcommand*{\UL@typeofdoc}{M\'emoire}
\fi
\ifUL@isexam
  \renewcommand*{\UL@typeofdoc}{Examen de doctorat}
\fi
%    \end{macrocode}
% \end{macro}
%
% \begin{macro}{\UL@docid}
%   La commande |\UL@docid| prépare ensuite la mention mise en forme
%   du type de document pour les pages de titre. La thèse ou le
%   mémoire en cotutelle ou en bidiplomation requiert un traitement
%   différent puisque le programme d'étude apparaît immédiatement sous
%   la mention.
%    \begin{macrocode}
\newcommand{\UL@docid}{%
  {\UL@fontprogram\UL@typeofdoc\par
  \ifnum\UL@typenum=2 \UL@program\par \fi}}
%    \end{macrocode}
% \end{macro}
%
% \subsubsection{Détails sur les facultés et universités d'attache}
%
% \begin{macro}{\Ul@details}
%   La commande |\Ul@details| est la plus complexe puisque la
%   disposition des informations additionnelles sur le document varie
%   beaucoup selon le type de thèse ou de mémoire. Il existe quatre
%   grandes catégories de disposition des éléments sur la page de titre:
%   standard; multifacultaire; en cotutelle ou en bidiplomation
%   (disposition identique); en extension.
%
%   Tel qu'expliqué à la \autoref{sec:utilisation:options}, certains
%   types de grade requièrent expressément que certaines informations
%   soient fournies. Si un élément d'information manque, un
%   avertissement est émis.
%    \begin{macrocode}
\newcommand{\UL@details}{%
  \ifcase\UL@typenum\relax% 0 standard
    \vspace{96pt}
    {\UL@fontprogram\UL@program}\par
    \UL@degree\par
    \vspace{112pt}
    Qu\'ebec, Canada\par
  \or%                      1 multifacultaire
    \vspace{96pt}
    {\UL@fontprogram\UL@program}\par
    \UL@degree\par
    \vspace{36pt}
    \ifx\UL@facUL\empty
      \ClassWarningNoLine{ulthese}{UL faculty names missing.}
    \else
      \UL@facUL\par
    \fi
    \vspace{48pt}
    Qu\'ebec, Canada\par
  \or%                      2 cotutelle et bidiplomation
    \vspace{72pt}
    Universit\'e Laval\par Qu\'ebec, Canada\par
    \UL@degree\par
    \vspace{\baselineskip} et\par \vspace{\baselineskip}
    \ifx\UL@nameother\empty
      \ClassWarningNoLine{ulthese}{Other university name and location missing}
    \else
      \UL@nameother\par
    \fi
    \ifx\UL@degreeother\empty
      \ClassWarningNoLine{ulthese}{Other university degree missing}
    \else
      \UL@degreeother\par
    \fi
  \or%                      3 extension
    \vspace{48pt}
    {\UL@fontprogram\UL@program\ de l'Universit\'e Laval\par
      \UL@offered\ en extension \`a l'\UL@extensionat}\par
    \vspace{36pt}
    \UL@degree\par
    \vspace{36pt}
    \ifx\UL@facother\empty
      \ClassWarningNoLine{ulthese}{Other university faculty name missing}
    \else
      \UL@facother\par
    \fi
    \UL@extensionat\par
    \UL@extensionloc\par
    \vspace{\baselineskip}
    \ifx\UL@facUL\empty
      \ClassWarningNoLine{ulthese}{UL faculty name missing}
    \else
      \UL@facUL\par
    \fi
    Universit\'e Laval\par Qu\'ebec, Canada\par
  \fi}
%    \end{macrocode}
% \end{macro}
%
% \subsubsection{Conception des pages de titre}
%
% \begin{macro}{\pagestitre}
%   Les thèses et mémoires comportent une page de titre et une page
%   frontispice. La première comporte:
%   \begin{enumerate}
%   \item le logo de l'Université Laval (sauf pour les thèses ou
%     mémoires réalisés en cotutelle, en bidiplomation ou en
%     extension);
%   \item le titre et le sous-titre, le cas échéant;
%   \item le type de document (thèse, thèse en cotutelle,
%     mémoire, etc.);
%   \item le nom complet de l'auteur;
%   \item une description du programme, du grade obtenu et des
%     facultés ou universités d'attache, le cas échéant;
%   \item la mention «Québec, Canada» si le logo de l'Université
%     Laval est présent;
%   \item la notice de copyright.
%   \end{enumerate}
%   La page frontispice reprend les éléments 2--4 et ajoute les
%   noms des directeur et codirecteurs de recherche.
%
%   On doit rétablir pour les pages de titre l'interligne simple et
%   l'espacement nul entre les paragraphes (\cmd{\parskip}). Ensuite, on
%   doit ajuster la distance entre le bloc de titre et le type de
%   document (\cmd{\UL@docidspacing}) et celle entre ce dernier et le nom
%   de l'auteur (\cmd{\UL@authorspacing}). Cela fait en sorte que les
%   éléments des pages de titre se retrouvent (presque) toujours au même
%   endroit sur la page. Une distance minimale d'un interligne est
%   conservée entre le bloc de titre et le type de document
%   (précaution nécessaire pour l'éventuel cas d'un bloc de titre
%   s'étendant sur plusieurs lignes).
%
%   Sur la page frontispice, le logo de l'Université est remplacé par
%   une boîte de réglure invisible (\emph{strut}) de la même hauteur.
%
%   Le nom de l'auteur, les directeurs et codirecteurs de recherche et
%   la notice de copyright sont insérées directement dans le code de
%   la commande \cmd{\pagestitre}.
%    \begin{macrocode}
\newlength{\UL@docidspacing}
\setlength{\UL@docidspacing}{82pt}
\newlength{\UL@authorspacing}
\setlength{\UL@authorspacing}{72pt}
\newcommand{\pagestitre}{{%
    \clearpage
    \pagestyle{empty}
    \SingleSpacing\setlength{\parskip}{0pt}
    \centering
    \UL@fontbase
    \UL@measuretitle
    \addtolength{\UL@docidspacing}{-\UL@titleboxtotht}
    \addtolength{\UL@docidspacing}{-\UL@subtitleboxtotht}
    \ifdim\UL@docidspacing<\baselineskip\relax
      \setlength{\UL@docidspacing}{\baselineskip}
      \addtolength{\UL@authorspacing}{-\baselineskip}
    \fi
    \ifnum\UL@typenum>1\relax
      \vspace*{0pt}\par
    \else
      \includegraphics[height=15mm,keepaspectratio=true]{ul_p}\par
    \fi
    \vspace{82pt}
    \copy\UL@titlebox
    \copy\UL@subtitlebox
    \vspace{\UL@docidspacing}
    \UL@docid
    \vspace{\UL@authorspacing}
    {\UL@fontauthor\UL@author}\par
    \UL@details
    \vfill
    {\textcopyright} \UL@author, \UL@year\par
    \ifUL@isexam\else
      \clearpage
      \ifnum\UL@typenum>1\relax
        \vspace*{0pt}\par
      \else
        \rule{0mm}{15mm}\par    % strut
      \fi
      \vspace{82pt}
      \box\UL@titlebox
      \box\UL@subtitlebox
      \vspace{\UL@docidspacing}
      \UL@docid
      \vspace{\UL@authorspacing}
      {\UL@fontauthor\UL@author}\par
      \vspace{72pt}
      Sous la direction de:\par
      \vspace{\baselineskip}
      \UL@director\par
      \UL@codirector
    \fi
    \clearpage}}
%    \end{macrocode}
% \end{macro}
%
% \begin{macro}{\pagetitre}
%   La page de titre était produite avec la commande \cmd{\pagetitre} dans
%   les versions précédentes de la classe. Émettre un avertissement et
%   utiliser plutôt \cmd{\pagestitre} si la commande obsolète est utilisée.
%    \begin{macrocode}
\newcommand{\pagetitre}{
  \ClassWarning{ulthese}{Command \protect\pagetitre\space is obsolete.\MessageBreak
    Using \protect\pagestitre\space instead}\pagestitre}
%    \end{macrocode}
% \end{macro}
%
% \subsection{Listes des figures et des tableaux}
%
% \begin{macro}{\listfigurename}
%   Le paquetage \pkg{babel} définit comme titre pour la liste des
%   figures «Table des figures», alors que la liste des tableaux est
%   «Liste des tableaux». Pour une plus grande symétrie, la classe
%   redéfinit le titre correspondant à \cmd{\listoffigures}. La commande
%   \cmd{\addto} est nécessaire pour éviter que \pkg{babel} redéfinisse
%   le titre à |\begin{document}|.
%    \begin{macrocode}
\ifUL@babel
  \addto\captionsfrench{\renewcommand{\listfigurename}{Liste des figures}}
\fi
%    \end{macrocode}
%   Si \pkg{babel} n'est pas chargé, ce sera à l'utilisateur de faire
%   une correction équivalente. Avec \pkg{polyglossia}, la commande à
%   insérer dans l'entête du document est la même que ci-dessus.
% \end{macro}
%
% \subsection{Dédicace et épigraphe}
%
% La dédicace et l'épigraphe sont mises en forme avec la commande
% \cmd{\epigraph} de \class{memoir}.
% \begin{macro}{\dedicace}
%   La dédicace est une épigraphe simplifiée placée seule sur une
%   page, alignée à droite à une dizaine de lignes de la marge
%   supérieure, sans auteur ou source et sans ligne de démarcation.
%    \begin{macrocode}
\newcommand{\dedicace}[1]{{%
    \clearpage
    \pagestyle{empty}
    \setlength{\beforeepigraphskip}{10\baselineskip}
    \setlength{\epigraphrule}{0pt}
    \epigraphtextposition{flushright}
    \mbox{}\epigraph{\itshape #1}{}}}
%    \end{macrocode}
% \end{macro}
% \begin{macro}{\epigraphe}
%   L'épigraphe de début de document est placée seule sur une page à
%   une dizaine de lignes de la marge supérieure. Pour le reste, on
%   s'en remet à la commande \cmd{\epigraph} de \class{memoir}.
%    \begin{macrocode}
\newcommand{\epigraphe}[2]{{%
    \clearpage
    \pagestyle{empty}
    \setlength{\beforeepigraphskip}{10\baselineskip}
    \mbox{}\epigraph{#1}{#2}}}
%    \end{macrocode}
% \end{macro}
%
% \subsection{Citations}
%
% \begin{environment}{quote}
%   La classe redéfinit l'environnement |quote| de \class{memoir}
%   afin que le texte des citations se trouve en retrait de 10~mm à
%   gauche et à droite, conformément aux règles de présentation de la
%   FESP.
%    \begin{macrocode}
\renewenvironment{quote}{%
  \list{}{\rightmargin 10mm \leftmargin 10mm}%
  \item[]}{\endlist}
%    \end{macrocode}
%   \end{environment}
%
% \begin{environment}{quotation}
%   Il en va de même de l'environnement |quotation|. Cependant, cet
%   environnement passe également à l'interligne simple et la classe
%   ajuste l'espacement vertical entre les paragraphes afin que
%   ceux-ci soient bien distincts les uns des autres tout en demeurant
%   raisonnablement compacts. Cet espacement est ici fixé à 6~points.
%    \begin{macrocode}
\renewenvironment{quotation}{%
  \list{}{%
    \SingleSpacing
    \listparindent 0em
    \itemindent    \listparindent
    \leftmargin    10mm
    \rightmargin   \leftmargin
    \parsep        6\p@ \@plus\p@}%
  \item[]}{\endlist}
%    \end{macrocode}
% \end{environment}
%
% \subsection{Numérotation des divisions du document}
%
% Par défaut, \class{memoir} numérote les divisions du document
% seulement jusqu'au niveau des sections. La classe étend la
% numérotation aux sous-sections.
%    \begin{macrocode}
\setsecnumdepth{subsection}
%</class>
%    \end{macrocode}
% ^^A Fin du code de la classe
%
% \Finale
%
% \iffalse
% ^^A Gabarits du document maître
%<*gabarit>
%<phd&standard>%% GABARIT POUR THÈSE STANDARD
%<phd&mesure>%% GABARIT POUR THÈSE SUR MESURE
%<phd&articles>%% GABARIT POUR THÈSE PAR ARTICLES
%<phd&multifac>%% GABARIT POUR THÈSE MULTIFACULTAIRE
%<phd&cotutelle>%% GABARIT POUR THÈSE EN COTUTELLE
%<phd&UdeS>%% GABARIT POUR THÈSE EN EXTENSION À L'UNIVERSITÉ DE SHERBROOKE
%<phd&UQO>%% GABARIT POUR THÈSE EN EXTENSION À L'UQO
%<m&standard>%% GABARIT POUR MÉMOIRE STANDARD
%<m&mesure>%% GABARIT POUR MÉMOIRE SUR MESURE
%<m&bidiplomation>%% GABARIT POUR MÉMOIRE EN BIDIPLOMATION
%<m&UQAC>%% GABARIT POUR MÉMOIRE EN EXTENSION À L'UQAC
%%
%% Consulter la documentation de la classe ulthese pour une
%% description détaillée de la classe, de ce gabarit et des options
%% disponibles.
%%
%% [Ne pas hésiter à supprimer les commentaires après les avoir lus.]
%%
%% Déclaration de la classe avec le type de grade
%<phd>%%   [l'un de LLD, DMus, DPsy, DThP, PhD]
%<m>%%   [l'un de MATDR, MArch, MA, LLM, MErg, MMus, MPht, MSc, MScGeogr,
%<m>%%    MServSoc, MPsEd]
%% et les langues les plus courantes. Le français sera la langue par
%<!articles>%% défaut du document.
%<articles>%% défaut du document. L'option 'bibsection' permet de créer des
%<articles>%% bibliographies par chapitre présentées sous forme de section
%<articles>%% numérotée.
%<phd&(standard|mesure)>\documentclass[PhD,english,french]{ulthese}
%<phd&articles>\documentclass[PhD,bibsection,english,french]{ulthese}
%<phd&multifac>\documentclass[PhD,multifacultaire,english,french]{ulthese}
%<phd&cotutelle>\documentclass[PhD,cotutelle,english,french]{ulthese}
%<phd&UdeS>\documentclass[PhD,extensionUdeS,english,french]{ulthese}
%<phd&UQO>\documentclass[PhD,extensionUQO,english,french]{ulthese}
%<m&(standard|mesure)>\documentclass[MSc,english,french]{ulthese}
%<m&bidiplomation>\documentclass[MA,bidiplomation,english,french]{ulthese}
%<m&UQAC>\documentclass[MSc,extensionUQAC,english,french]{ulthese}
  %% Encodage utilisé pour les caractères accentués dans les fichiers
  %% source du document. Les gabarits sont encodés en UTF-8. Inutile
  %% avec XeLaTeX, qui gère Unicode nativement.
  \ifxetex\else \usepackage[utf8]{inputenc} \fi

  %% Charger ici les autres paquetages nécessaires pour le document.
  %% Quelques exemples; décommenter au besoin.
  %\usepackage{amsmath}       % recommandé pour les mathématiques
  %\usepackage{ncccomma}      % gestion de la virgule dans les nombres

  %% Utilisation d'une autre police de caractères pour le document.
  %% - Sous LaTeX
  %\usepackage{mathpazo}      % texte et mathématiques en Palatino
  %\usepackage{mathptmx}      % texte et mathématiques en Times
  %% - Sous XeLaTeX
  %\setmainfont{TeX Gyre Pagella}      % texte en Pagella (Palatino)
  %\setmathfont{TeX Gyre Pagella Math} % mathématiques en Pagella (Palatino)
  %\setmainfont{TeX Gyre Termes}       % texte en Termes (Times)
  %\setmathfont{TeX Gyre Termes Math}  % mathématiques en Termes (Times)

  %% Gestion des hyperliens dans le document. S'assurer que hyperref
  %% est le dernier paquetage chargé.
  \usepackage{hyperref}
  \hypersetup{colorlinks,allcolors=ULlinkcolor}

  %% Options de mise en forme du mode français de babel. Consulter la
  %% documentation du paquetage babel pour les options disponibles.
  %% Désactiver (effacer ou mettre en commentaire) si l'option
  %% 'nobabel' est spécifiée au chargement de la classe.
  \frenchbsetup{%
    StandardItemizeEnv=true,       % format standard des listes
    ThinSpaceInFrenchNumbers=true, % espace fine dans les nombres
    og=«, fg=»                     % caractères « et » sont les guillemets
  }

%<!articles>  %% Style de la bibliographie.
%<!articles> \bibliographystyle{}
%<articles>  %% Suppression du numéro de section de la bibliographie. Utilisation
%<articles>  %% de \extrasfrench parce que c'est la dernière langue déclarée dans
%<articles>  %% \documentclass, ci-dessus.
%<articles>  %\addto\extrasfrench{%
%<articles>  %  \renewcommand{\bibsection}{\section*{\bibname}\prebibhook}}

  %% Déclarations des pages de titre. Remplacer les éléments entre < >.
  %% Supprimer les caractères < >. Couper un long titre ou un long
  %% sous-titre manuellement avec \\.
  \titre{<Titre principal>}
  % \titre{Ceci est un exemple de long titre \\
  %   avec saut de ligne manuel}
  % \soustitre{Sous-titre le cas échéant}
  % \soustitre{Ceci est un exemple de long sous-titre \\
  %   avec saut de ligne manuel}
  \auteur{<Prénom Nom>}
  \annee{<20xx>}
%<phd&(standard|multifac|cotutelle)>  \programme{Doctorat en <discipline> <-- majeure, s'il y a lieu>}
%<phd&(mesure)>  \programme{Doctorat sur mesure en <discipline> <-- majeure, s'il y a lieu>}
%<phd&(UdeS|UQO)>  \programme{Doctorat en <discipline>}
%<m&standard>  \programme{Maîtrise en <discipline> <-- majeure, s'il y a lieu>}
%<m&mesure>  \programme{Maîtrise sur mesure en <discipline> <-- majeure, s'il y a lieu>}
%<m&bidiplomation>  \programme{Maîtrise en <discipline>}
%<m&UQAC>  \programme{Maîtrise en <discipline>}
%<!(cotutelle|bidiplomation)>  \direction{<Prénom Nom>, <directeur ou directrice> de recherche}
%<cotutelle>  \direction{<Prénom Nom>, <directeur ou directrice> de recherche \\
%<cotutelle>             <Prénom Nom>, <directeur ou directrice> de cotutelle}
%<bidiplomation>  \direction{<Prénom Nom>, <directeur ou directrice> de recherche \\
%<bidiplomation>             <Prénom Nom>, <directeur ou directrice> de recherche}
  % \codirection{<Prénom Nom>, <codirecteur ou codirectrice> de recherche}
  % \codirection{<Prénom Nom>, <codirecteur ou codirectrice> de recherche \\
  %              <Prénom Nom>, <codirecteur ou codirectrice> de recherche}
%<cotutelle>  \univcotutelle{<Université de cotutelle> \\ <Ville>, <Pays>}
%<bidiplomation>  \univcotutelle{<Université de bidiplomation> \\ <Ville>, <Pays>}
%<cotutelle|bidiplomation>  \gradecotutelle{<Nom du grade> (<sigle du grade>)}
%<multifac>  \faculteUL{<Faculté 1> \\ <Faculté 2>}
%<UdeS|UQO|UQAC>  \faculteUL{<Nom de la faculté UL>}
%<UdeS>  \faculteUdeS{<Nom de la faculté UdeS>}
%<UQO>  \faculteUQO{<Nom de la faculté UQO>}
%<UQAC>  \faculteUQAC{<Nom de la faculté UQAC>}

\begin{document}

\frontmatter                    % pages liminaires

\pagestitre                     % production des pages de titre

%!TEX root = gabarit-doctorat.tex
\chapter*{Résumé}                      % ne pas numéroter
\phantomsection\addcontentsline{toc}{chapter}{Résumé} % inclure dans TdM

\begin{otherlanguage*}{french}
  Le coût croissant de l'énergie a fait de l'économie d'énergie une nécessité vitale dans le monde actuel. Un des exemples consiste à ``maintenir la chaîne du froid", c'est-à-dire le transport correct des aliments périssables dans les véhicules réfrigérés, en particulier pour les produits laitiers, la viande et les aliments congelés. Tout en conservant une conservation appropriée des denrées alimentaires, l'ATP (Agreement on Transport of Perishable Foodstuffs) est l'un des accords concernant les essais d'isolation thermique qui déterminent l'adéquation du transport.
  
  Le test standard ATP est une procédure pour mesurer l'état isolant des équipements avec une approche globale. Néanmoins, certains défauts locaux dans la structure de l'équipement ne peuvent pas être visualisés dans cette procédure. Dans ce contexte, la technique de thermographie pourrait être particulièrement utile à ces problèmes. Deux exemples de cette application sont présentés dans cette thèse, l'un d'eux se concentre sur la cartographie du flux de chaleur sur la surface externe d'un rouleau-conteneur isolé par la technique de thermographie infrarouge. La seconde tente d'établir une vue panoramique du flux de chaleur sur la surface interne d'un véhicule isolé.
  
  Encouragé par les résultats favorables précédents, une exploration de l'approche à froid dans la thermographie infrarouge pour les Tests Non-Destructifs et l'Évaluation est introduite et réalisée dans ce qui suit. Une approche se concentre sur la détection des défauts isolés et des ponts thermiques dans les panneaux de caisses de camions isolés par chauffage à lampe et refroidissement par air, deux moyens d'excitation opposés. L'autre examine un refroidissement à l'azote liquide appliqué à un échantillon d'acier avec des trous à fond plat de différentes profondeurs et tailles.
  
  Différentes méthodes de traitement des données et de modélisation et de simulation sont effectuées dans des chapitres connexes.
\end{otherlanguage*}
                % résumé français
\chapter*{Abstract}                      % ne pas numéroter
\phantomsection\addcontentsline{toc}{chapter}{Abstract} % inclure dans TdM

\begin{otherlanguage*}{english}
   The increasing cost of energy has made energy saving a vital necessity in the current world. One of the examples involves, ``Maintaining the cold chain", which is the correct transport of perishable foodstuffs in refrigerated vehicles, especially for dairy products, meat and frozen foods.  In this respect a suitable thermal insulation implemented in refrigerated vehicles is essential for saving energy while maintaining an appropriate conservation of the foodstuffs. ATP is one of the agreements concerning thermal insulation tests ensuing the suitability of the transport.
   
   The ATP standard test is a procedure to measure the insulating status of equipment with a global approach. Nonetheless, some local defects in the structure of equipment cannot be visualized in this procedure. The thermography technique could be particularly helpful for these issues. Two examples of this application are presented in this thesis, one focuses on mapping the heat flux on the external surface of an insulated roll-container by infrared thermography technique. The second one attempts to establish a panoramic view of the heat flux on the internal surface of an insulated vehicle. 
   
   Encouraged by previous favorable results, an exploration of the cold approach in infrared thermography for Non-Destructive Testing \& Evaluation is introduced and performed herein. One approach focuses on the detection of insulated flaws and thermal bridges in insulated truck box panels by lamp heating and air cooling, two opposite means of excitation. The other approach investigates the application of  liquid nitrogen cooling to a steel specimen with flat-bottom holes of different depths and sizes.
   
   Different data processing methods and modeling and simulation are also carried out.% in related chapters. 
\end{otherlanguage*}
              % résumé anglais
\cleardoublepage

\tableofcontents                % production de la TdM
\cleardoublepage

\listoftables                   % production de la liste des tableaux
\cleardoublepage

\listoffigures                  % production de la liste des figures
\cleardoublepage

\dedicace{Dédicace si désiré}
\cleardoublepage

\epigraphe{Texte de l'épigraphe}{Source ou auteur}
\cleardoublepage

\chapter*{Acknowledgements}         % ne pas numéroter
\phantomsection\addcontentsline{toc}{chapter}{Acknowledgements} % inclure dans TdM

I would like to first thank 
         % remerciements
\chapter*{Preface}         % ne pas numéroter
\phantomsection\addcontentsline{toc}{chapter}{Preface} % inclure dans TdM

This thesis is submitted to the ``Faculté des études supérieures de l'Université Laval" to obtain the degree of Philosophiae Doctor of Science (Ph.D.). The current thesis is composed of six chapters. The first chapter is a literature review of maintaining the ``cold chain" in industry field, as well as cold approaches that are used nowadays in infrared thermography for Non-Destructive Testing \& Evaluation. This chapter concludes with the issue, hypothesis and objectives. In addition, several common data analyzing and processing methods, including modeling and simulation applied in whole research studies, are also presented and highlighted. Chapters two and three are presented in form of papers and describe the application of infrared thermogaphy in the ``cold chain" and corresponding discussion. Chapters four and five are composed of manuscripts and describe the exploration of cold approaches in infrared thermography including several experimental results and discussions. Finally, in order to provide a closure on the results obtained in the current thesis, chapter six presents a general conclusions and proposed perspectives of the performed studies.

The first manuscript is entitled ``Mapping of the heat flux of an insulated small container by infrared thermography" and was presented in the 24th IIR International Congress of Refrigeration. Authors: Paolo Bison, Alessandro Bortolin, Gianluca Cadelano, Giovanni Ferrarini, Lei Lei, Xavier Maldague, Stefano Rossi.

The second paper is entitled ``Panoramic View of the Heat Flux Inside an Insulated Vehicle by Infrared Thermography" and was first presented at an oral session of the 1st QIRT Asia Conference 2015 Mamallapuram, India. Then it was selected for publication in the Quantitative InfraRed Thermography Journal. Authors: Lei Lei, Alessandro Bortolin, Gianluca Cadelano, Giovanni Ferrarini, Stefano Rossi, Paolo Bison, Xavier Maldague. 

Then, the third article is entitled ``Detection of insulation flaws and thermal bridges in insulated truck box panels" and was first presented at an oral session of the 13th Quantitative InfraRed Thermography Conference 2016 at Gdańsk University of Technology in Poland. Then this paper was invited for publication in the Quantitative InfraRed Thermography Journal. Authors: Lei Lei, Alessandro Bortolin, Paolo Bison, Xavier Maldague.

The last manuscript is entitled ``Liquid nitrogen cooling in IR thermography applied to steel specimen" and it was accepted for publication in the Proceedings Volume 10214 of Thermosense: Thermal Infrared Applications XXXIX; 102140T (2017). Authors: Lei Lei, Giovanni Ferrarini, Alessandro Bortolin, Gianluca Cadelano, Paolo Bison, Xavier Maldague.

The main objectives of the present thesis are first to deploy the infrared thermography technique in the procedure of maintaining the ``cold food chain'', especially in insulated vehicles of ATP standards. The benefits of time-saving and the accuracy of the results in the determination of K-value could be applied for assessment at a commercial level. Then based on the previous favorable results, another objective is to explore cold approaches (such as compressed air, liquid nitrogen, etc.) in infrared thermography for Non-Destructive Testing \& Evaluation. The first part of this research was a cooperative project supported by the governments of Italy and Quebec (Ministère des Relations internationales et de la Francophonie) through the Joint Subcommittee Québec-Italy. Our collaborator, the Construction Technologies Institute of the Italian National Research Council (ITC-CNR), has made a significant contribution to this research. Three main experiments were performed at ITC-CNR, and Dr. Paolo Bison oversaw all steps of the work there.

For the exploration of cold approaches, the research supervisor, prof. Xavier Maldague collaborated in the set up and work planning of experiments, equipment and participated in the interpretation of the results and the revision of the manuscripts.

\textbf{TO BE COMPLETED}


%Finally, the candidate has been invited to present the results of this thesis in a good management practices guide for pigs from farm to slaughter. This guide will be developed for use by the pork meat industry and the Canadian hog producers.
           % avant-propos

\mainmatter                     % corps du document

\chapter*{Introduction}         % ne pas numéroter
\phantomsection\addcontentsline{toc}{chapter}{Introduction} % inclure dans TdM

Une thèse ou un mémoire devrait normalement débuter par une
introduction. Celle-ci est traitée comme un chapitre normal, sauf
qu'elle n'est pas numérotée.
          % introduction
%<!articles>\chapter{State of the art}     % numéroté
\section*{Maintain the ``cold chain"}
\phantomsection\addcontentsline{toc}{section}{Maintain the ``cold chain"}


\section*{Infrared thermography for NDT \& E}
\phantomsection\addcontentsline{toc}{section}{Infrared thermography for NDT \& E} % inclure dans TdM

In the past few decades, a Nondestructive Testing \& Evaluation (NDT\&E) technique: Infrared Thermography (IT), also commonly referred as \textit{Thermal imaging}, or \textit{thermography}, has received growing attention and applications for diagnostics. This is due to its characteristics that ``it allows the mapping of thermal patterns, i.e. thermograms, on the surface of objects, bodies, or systems through the use of an infrared imaging instrument, such as an infrared camera" \citep{maldague3introduction}. Besides, its impressive quickness and convenience, the inspection attracted a wide variety of applications including biological, civil engineering, aerospace, cultural heritage, etc \citep{2007-Ibarra-Castanedo,2000-Li,cielo1987thermographie,shoja2011inspection,pradere2009microscale,avdelidis2004applications,maierhofer2005quantitative}. Other utilization such evaluation or measurements of heat transfer coefficients (or absorption) can be found in \citep{dragano2009experimental,grinzato2010r,grinzatoquality,grinzato1comparison,rossi2009k}.

Infrared thermography is often divided into two approaches: passive thermography, in which materials and structures are naturally at different temperature than the environment; and active thermography, in which an external simulation is added to induce a thermal response \citep{Maldague2001theory}. The two approaches are described in next two sections.
\section{Passive thermography}
This approach is due to the principle of energy conservation and states that an important quantity of heat is released by any process consuming energy because of the law of entropy, which is known as \textit{the First law of thermodynamics} \citep{thdy1}. 

Therefore, the passive thermography is often qualitative, such as the diagnosis of the presence of a given abnormality or hot spot with respect to the immediate surroundings.  A delta-T of a few degrees (> 5°C) is generally
found suspicious while greater values indicate strong evidences of abnormal behaviors. Some applications in civil engineering can be found in \citep{2000-Li,stanley1994non,lo2004building}, where the thermography approach was proved to be reliable in detecting the debonded ceramic tiles on a building finish, and a temperature contours over the surface of a target object to provide an appropriate measure of the damaged building or structure has been mapped. The inspection of buildings, for different purposes such as energy monitoring, moisture mapping and many others, has also been widely deployed \citep{laranjeirapassive,bison1993automatic,bison2012geometrical}. 

\section{Active thermography}
Contrary to ``\textit{Passive}", the word ``\textit{Active}" seems more progressive. Thus might be the reason why this technique finds a large number of applications in NDT. The experimental configuration for active infrared thermography can be seen in Fig \ref{exp_active} \citep{sfarra2010comparative}. 
\begin{figure}[!htbp]
	\centering
	\includegraphics[scale=0.4]{art/exp_active}
	\caption{Experimental setup for the active thermography}
	\label{exp_active}
\end{figure}
where \textcircled{1} are two different stimulation sources. They can be located in the same side of the camera, which is known as \textit{reflection} mode; or in the opposite side of the camera, which is known as \textit{transmission} mode.  \textcircled{2} is the specimen under test, \textcircled{3} is the IR camera for recording, and \textcircled{4} is the PC for processing. The whole system is connected by a synchronization control.

Practically, \textit{any} form of energy can be used to produce a measurable thermal contrast. In addition, since the external stimulation can be accurately controlled, quantitative procedure is possible. 

\subsection*{Excitation methods}
Generally, energy used as the external resources can be delivered by the following mechanisms:
\begin{itemize}
	\item Conduction: heating blanket, hot bag or cool bags (as snow or ice);
	\item Convection: Hot (or cold) water (or gas(air));
	\item Radiation: Lamps, flashes, infrared heaters;
	\item Mechanical stimulation: ultrasonic vibration;
\end{itemize}
Even though ``any" form of energy can be used, the heating sources are often preferred. They can be commonly divided as {ibarra2013infrared}:
\begin{itemize}
	\item optical: Photographic flashes or lamps are utilized, known as \textit{Pulsed Thermography} (sometimes even using laser as heating resource \citep{suzuki2002application,burrows2007combined}). When using periodic heating at a given frequency to measure the amplitude and (or) phase delay  of the thermal response, that is known as \textit{Lock-in thermography} \citep{wu1998lock,duan2013quantitative,2007-Ibarra-Castanedo};
	\item mechanical: Sound or ultrasound waves are injected to the specimen to produce heat by friction, known as \textit{Vibrothermography} \citep{2007-ClementeIbarra-Castanedo,2007-Ibarra-Castanedo};
	\item induction: Eddy currents are generated by a coil inside the specimen \citep{riegert2004lockin,zenzinger2007thermographic}. This type of heating source is limited to conductive materials;
	\item microwave: Heat is introduced into the specimen by a time-gated microwave source \citep{myers1979microwave,land1987clinical};
\end{itemize}
Since the main technique in this work is using the excitation method, \textit{Pulsed Thermography} will be presented in details in the following sections \citep{Maldague2001theory,ibarra2013infrared}.
\subsection{Pulsed Thermography}
%In this technique, the specimen surface is submitted to a heat pulse generated by a high power and fast heat source such as flashes. Then the thermal front propagates under the surface by diffusion. Then as time passes, the surface temperature decreases uniformly for a piece without any defects. While on the contrary,  the diffusion rate can be changed and abnormal temperature patterns will be produced at the subsurface, by any kind of surface discontinuities (such as air leakage, delaminations, thermal bridges, porosity, inclusions, etc.). The detection of these discontinuities  with an IR camera depends on their sizes.

Pulsed thermography (PT) is one of the most popular thermal stimulation method in IR thermography. One reason for this popularity is the quickness of the inspection relying on a thermal stimulation pulse, with duration going from a few ms for high thermal conductivity material inspection (such as metal parts) to a few seconds for low thermal conductivity specimens (such as plastics, graphite epoxy components \citep{Maldague1993Nondestructive,Maldague1994bInfra}). Such quick thermal stimulation allows direct deployment on the plant floor with convenient heating sources. Moreover, the brief heating prevents damage to the component (heating is generally limited to a few degrees above the initial component temperature).

Basically, PT consists of briefly heating the specimen and then recording its temperature decay curve. Qualitatively, the phenomenon is as follows. The temperature of the material first rises during the pulse. After the pulse, it then decays because the energy - the thermal front - propagates by diffusion under the surface. Later, the presence of a subsurface defect (example: a disbonding) reduces the diffusion rate so that when observing the surface temperature, such a subsurface defect appears as an area of higher temperature with respect to the surrounding sound area. In fact in such a case the reduced diffusion rate caused by the subsurface defect presence translates into “heat accumulation” and hence higher surface temperature just over the defect. Moreover, such phenomenon occurs in time so that, deeper defects are observed later and with a reduced “diluted” or “spread” thermal contrast.

Theoretically, the 1D solution of the \textit{Fourier} equation for the propagation of a \textit{Dirac} heat pulse through a semi-infinite homogeneous material is given by \citep{carslaw1986heat}:
\begin{equation}
T(z,t) = T_0 + \frac{Q}{\sqrt{k\rho C_p \pi t}}exp(-\frac{z^2}{4\alpha t})
\end{equation}
where $Q$ is the energy absorbed, $T_0$  is the initial temperature, $k$ the conductivity of the material, $C_p$ the heat capacity at constant pressure and $\alpha$ thermal diffusivity.

Thus, at the surface ($z=0$), one has:
\begin{equation}
T(0,t) = T_0 + \frac{Q}{\sqrt{k\rho C_p \pi t}}=T_0 + \frac{Q}{e\sqrt{\pi t}}
\label{PT_eq}
\end{equation}
where $e=\sqrt{k\rho C_p}$ is defined as the thermal effusivity, which measures the material ability to exchange heat with its surroundings. Therefore, surface temperature will decay as a function of $t^{1/2}$.

\noindent The energy source can be applied in various ways:
\begin{description}
	\item \textbf{Point inspection}: heating with a laser or a focused light beam; advantages: repeatable heating, uniformity; drawback: the necessity to move the inspection head to fully inspect a surface slows down the inspection process.
	\item \textbf{Line inspection}: heating using line lamps, heated wire, scanning laser, line of air jets (cool or hot); advantages: fast inspection rate (up to 1 $m^2/s$) and good uniformity thanks to the lateral motion; drawback: only part of the temperature history curve is available due to the lateral motion of the specimen and the fixed distance between thermal stimulation and temperature signal pick-up. Projection of a series of line heating strips is also used to detect surface cracks.
	\item \textbf{Surface inspection}: heating using lamps, flash lamps, scanning laser; advantages: the complete analysis of the phenomenon is possible since the whole temperature history curve is recorded; drawback: concerns about non-uniformity of the heating (lamps, flashes, heat gun, laser, microwave).
\end{description}

If the temperature of the part to inspect is already higher than ambient temperature, it can be of interest to make use of a cold thermal source such as a line of air jets (or water jets; sudden contact with ice, snow, etc.). In fact, a thermal front propagates the same way whether being hot or cold: what is important is the temperature differential between the thermal source and the specimen. An advantage of a cold thermal source is that it does not induce spurious thermal reflections into the IR camera as in the case of a hot thermal source. The main limitations of cold stimulation sources are related to practical considerations as for instance it is generally easier and more efficient, to heat rather then to cool a part. For this reason, our work on cold stimulation needs to be investigated in detail and better understood.

\subsection{Step Heating}
When the specimen is  continuously heated, the increase of surface temperature is monitored during the application of a stepped heating pulse, thus the principle of \textit{Step Heating}. Same as in pulsed thermography, variations of surface temperature with time are related to specimen features. So it can be called \textit{Long pulse}. Besides, it is also referred as  time-resolved  infrared radiometry (TRIR) \citep{spicer1992time}.  More details can be found in \citep{ibarra2013infrared,osiander1998thermal}

\subsection{Lock-in Thermography}
Lock-in thermography (LT) is based on thermal waves generated inside the inspected specimen and detected remotely. Wave generation is for instance performed by periodically depositing heat on the specimen surface (e.g. through sine-modulated lamp heating) while the resulting oscillating temperature field in the stationary regime is remotely recorded through its thermal infrared emission \citep{wu1998lock}.

The lock-in terminology refers to the necessity to monitor the exact time dependence between the output signal and the reference input signal (i.e. the modulated heating). This is done with a locking amplifier in a point by point laser heating or by computer in full-field (lamp) deployment so that both phase and magnitude images becomes available. Phase images are related to the propagation time and since they are relatively insensitive to local optical surface features (such as non uniform heating), they are interesting for NDE purposes. The depth range of images is inversely proportional to the modulation frequency so that higher modulation frequencies restrict the analysis in a near surface region \citep{Maldague2001theory}.

\subsection{Vibrothermography}
Known as \textit{Ultrasound thermography} \citep{dillenz2001progress}, vibrothemography (VT) is an active Infrared Thermography technique based on that principle: under the effect of mechanical vibrations (0 to 25 kHz) induced externally to the structure, thanks to direct conversion from mechanical to thermal energy, heat is released by friction precisely at locations where defects such as cracks and delaminations are located.

Ultrasonic waves are ideal for NDT in the sense that, defect detection is independent from of its orientation inside the specimen, and both internal and open surface defects can be detected. Thus, VT is very useful for the detection of cracks and delaminations. The range for ultrasonic waves is often between 20 kHz and 1 MHz. Unlike electromagnetic waves, mechanical elastic waves such as sonic and ultrasonic waves can not propagate in a vacuum. They require a medium for traveling. They travel faster in solids and liquids than the air. This indicates an important aspect of VT: the common approach in VT is to use a coupling media between a transducer and the specimen to reduce losses. Therefore, one of the Infrared thermography characteristic ``contactless" is meaningless in this technique. However, image acquisition can still be carried out from a distance using an IR camera.

\section{Advantages and limitations of infrared thermography}
As every coin has two sides, all technique has its strengths and weaknesses. For Infrared thermography, its advantages are as follows \citep{maldague3introduction, Maldague2001theory}:
\begin{itemize}
	\item Fast, surface inspection
	\item Ease of deployment
	\item Contactles, no coupling needed as in the case of conventional ultrasounds.
	\item Security, there's no damaging radiation.
	\item Easy access to results thanks to the imaging capabilities.
	\item Numerous applications.
	\item Unique inspection tool in some tasks.
\end{itemize}
While, on the other hand, some limitations specific are evident:% \citep{maldague3introduction, Maldague2001theory}:
\begin{itemize}
	\item Non-uniform heating, when over a large surface.
	\item Emissivity differs from materials.
	\item Defects detected are generally shallow.
	\item Thermal losses, absorption by the environment.
	\item Inspected thickness of material under the surface has a limitation.
	\item Cost of the apparatus.
\end{itemize}

\section{Recent research on cold approach}
The first mention of this technique was found in \citep{Maldague1993Nondestructive,Maldague1994bInfra}, where a filament-wound graphite-epoxy tank was tested by TNDT using a liquid nitrogen spray to identify defects and bonded Al structures were inspected  in a thermographical \textit{reflection} procedure by propagation of a cold front (cool air jet) Shown in Fig \ref{Al_stru_cold}. 
\begin{figure}[!htbp]
	\centering
	\includegraphics[scale=0.81]{art/Al_stru_cold}
	\caption{Thermographic inspection by propagation of a cold front for the inspection of bonded Al structures.}
	\label{Al_stru_cold}
\end{figure}
In the latter's thermal image, the structure is initially at a temperature of some 10°C above room temperature and a line
of air jets is used to quickly cool the inspected area. The image shows the hotter central area which corresponds to the bonded area (oriented vertically on the thermogram). The cooler region at the center of the bonding line reveals a bonding defect (lack of adhesive).
Another example of stimulation by propagation of a cool front in the same mode on an Al-foam panel was also presented in \citep{Maldague1993Nondestructive}(Fig \ref{Al_foam_hc}).
\begin{figure}[!htbp]
	\centering
	\includegraphics[scale=0.90]{art/Al_foam_hc}
	\caption{Thermographic inspection using a line of air jets with pre-heating then cooling.}
	\label{Al_foam_hc}
\end{figure}
In this case, it is necessary to detect an non-bonded area in a Al-foam laminate. The panel is first uniformly heated at a temperature of about 10$°C$ above room temperature. The surface is then cooled by a line of cool air jets during the lateral moving of the panel (at a constant speed of 2.4 $cm/s$). In the figure, two thermograms are shown, one for a sound area and one for an area where a circular-shaped disbonding (4 cm diameter) is present. The defect is clearly visible at the image
center. 
Besides, after comparing  the radiative-heat injection and convection-heat removal approach in \citep{Maldague1993Nondestructive}. They found that, a reduced thermal contrast is obtained with the cool air stimulation due to both the reduced heat capacity of air and the smaller temperature differential between the thermal perturbing source and the inspected surface (case of air versus high temperature radiative source). 

Other applications have then been made during the past decades in several domains. \citep{endohdynamical2012} proposed constructing an active thermographic imaging system, in which a cooling material contacts the surface and scans over a welded zone of two stainless plates (Fig \ref{endoh_fig}).
\begin{figure}[!htbp]
	\centering
	\includegraphics[scale=1]{art/endoh_axe}\\
	\includegraphics[scale=1]{art/endoh_lat}
	\caption{ Experimental setup (a) Axial movement, (b) lateral movement of a coolant material.}
	\label{endoh_fig}
\end{figure}
They recorded both a time-varying thermographic image and a real-time response of each pixel. The results are exhibited in Fig \ref{endoh_res_1}, \ref{endoh_res_2}.
\begin{figure}[!htbp]
	\centering
	\includegraphics[scale=0.9]{art/endoh_res_1}
	\caption{(a)Thermal image, (b) temperature profile along A-A’ line, (c) temperature profile along B-B’ line (v=10$mm/s$).}
	\label{endoh_res_1}
\end{figure}
\begin{figure}[!htbp]
	\centering
	\includegraphics[scale=1.0]{art/endoh_res_2}
	\caption{(a)Thermal image, (b) temperature profile along A-A’ line (v=50$ mm/s$).}
	\label{endoh_res_2}
\end{figure}

In both two cases of scanning (v= 10 and v=50 $mm/s$), it is observed that the locally high-temperature appears in low-temperature region of the simulated welding region. For axial movement of coolant material, the heat inside and the surface of the specimen transfers toward the interface surface of the specimen to the coolant, due to conservation of the axial symmetry of heat transfer. However, for lateral movement of coolant, it was squeezed for the spatial distribution of the temperature along the direction of coolant movement. Therefore, an elliptical shape of temperature distribution was obtained.

In \citep{2012-LewisHom} a custom, microfluidic  heat  sink  with  an  IR-transparent working  fluid  (0.75  LPM)  was  manufactured  to  cool  an instrumented test chip while permitting optical access for IR thermal imaging.  Then a detailed system calibration was conducted to account for the temperature-dependent optical properties of the chip and heat sink. Their experiments  confirm  that  the  dynamic  range  of  the infrared  microscope  can  be  greatly  enhanced  by decreasing the  thickness  of  the  liquid  layer. \citep{rodriguez2014cooling} deployed an thermographic test to the detection of subsurface cracks in welding. The procedure started with the thermal excitation of the material, following with the monitoring of the cooling process with infraRed thermography. They used the natural convection between the material and the environment, but two method of heating were deployed in to same specimen. One is heating from the front surface (Method 1), the other is heating from back surface (Method 2). 

\begin{figure}
	\hspace{-45pt}
%	\centering
	\includegraphics[scale=0.65]{art/cooling_res}
	\caption{Thermal images of each defect for two methods Natural convection cooling}
	\label{cooling_res}
\end{figure}

The experimental results (Fig \ref{cooling_res}) have shown that the technique has a potential to detect hidden flaws due to the influence of the physical differences between the defect and the non-defect area in the cooling parameters. But the results depended on the depth of defect: ``Greater depths require larger heating and greater ranges of precision in thermographic data acquisitions". %Still, the limitation of that proposed method in this literature is only by natural convection. %Our work will apply the forced convection by spraying the nitrogen gas.

\section{Issue, hypothesis and objectives}
\subsection{Issue}

\subsection{Hypothesis}

\subsection{Objectives}




\section{Methodology}
Thought infrared thermography has the advantages such as fast inspection, ease of deployment, contactless, security and easy access to results with the imaging capabilities, raw infrared thermography results is difficult to handle and analyze in case of reflections and non-uniform external stimulation. To improve the inspection results, there are various post-processing techniques which have been developed. The recently popular data analyzing and processing methods are presented in detail in the following subsections.
\subsection{Fourier Transform (FT)}
Among all data processing methods, Fourier Transform (FT) is especial because it helps retrieve phase and amplitude data from raw results, since our principal results are images, which can be seen as signals with two dimensions.

It is well-known that any wave form, periodic or not, can be approximated by the sum of purely harmonic waves oscillating at different frequencies. The Continuous Fourier Transform (CFT) is then given by:
\begin{equation}
F(\omega) = \int_{-\infty}^{\infty}f(t)e^{-j\omega t}dt = A(\omega)e^{i\phi(\omega)}
\end{equation}
where $\omega = 2 \pi f$. the FT technique serves to transform the perception of signal from a time-based domain to a frequency-based domain.

In case of discrete situation, the Discrete Fourier Transform (DFT) is possible to analyze the data in the frequency domain:
\begin{equation}
F_n = \Delta t \sum_{k=0}^{N-1}T(k\Delta t)e^{-\tfrac{i2\pi nk}{N}} = \Re(F_n) + i\Im(F_n)
\label{DFT}
\end{equation}
where $n$ designates the frequency increment ($n=0, 1, ..., N$), $\Delta t$ is the sampling interval, $N$ is the total number of infrared images, and $\Re$ and $\Im$ are the real and the imaginary parts of the transform, respectively.
DFT is often applied in Pulsed Phase Thermography (PPT), which analyzes phase data obtained from PT results. In addition, the Fast Fourier Transform (FFT) algorithm is often applied to reduce computation time.

\subsection{PPT}
From Eq.\ref{DFT}, amplitude $A_n$ and phase delay $\Phi_n$ are given by:
\begin{equation}
A_n = \sqrt{\Re(F_n)^2 + \Im(F_n)^2} \qquad \Phi_n = \tan^{-1}\frac{\Re(F_n)}{\Im(F_n)}
\end{equation}
It should be noted here that The processed sequence is less affected than the original data by undesired noise
sources such as environmental reflections, emissivity variations, non-uniform heating.

By selecting two pixels, the first in correspondence of a reference zone, the second in correspondence of a possible defect zone, and following both of them in time, after the pulse, the two profiles shown in Fig. \ref{T_profile_PPT} are obtained. By taking the FFT of the two signals, the two profiles of amplitude (Fig. \ref{FT_AM_profile_PPT}) and phase (Fig. \ref{FT_PH_profile_PPT})
as a function of frequency are obtained.
\begin{figure}[!h]
	\centering
	\includegraphics[scale=0.35]{art/T_profile_PPT}
	\caption{Temperature profiles of a reference zone (blue) and a defect zone (red)}
	\label{T_profile_PPT}
\end{figure}

\begin{figure}[!h]
	\centering
	\includegraphics[scale=0.35]{art/FT_AM_T_profile_PPT}
	\caption{Amplitude as a function of frequency: blue reference, red defect}
	\label{FT_AM_profile_PPT}
\end{figure}

\begin{figure}[!h]
	\centering
	\includegraphics[scale=0.35]{art/FT_PH_T_profile_PPT}
	\caption{Phase as a function of frequency: blue reference, red defect}
	\label{FT_PH_profile_PPT}
\end{figure}

\subsection{Differential Absolute Contrast (DAC)}
Traditionally, once the temperature of a sound area (reference zone)$T_s(t) $ and that of a defect zone $T_{def}(t) $ are known, contrast methods can be applied by simply:
\begin{equation}
C_{ac}(t) = T_{def}(t) - T_s(t)
\label{AC_eq}
\end{equation}
which is known as the absolute contrast. However, this method becomes inconvenient when the sound area cannot be pratical defined. In addition, the common case of non-uniform heating has a strong effect on the results.

To improve this, the Differential Absolute Contrast (DAC) has proven its amelioration for non-uniform heating situations\citep{Benitez2008, pilla2002new}.

DAC method starts from Eq. \ref{PT_eq}, the surface temperature increase based on one-dimensional model of the Fourier equation after an instantaneous Dirac heating pulse is applied:
\begin{equation}
\Delta T = T(0,t) - T_0  = \frac{Q}{e\sqrt{\pi t}}
\label{PT_eq_2}
\end{equation}
The temperature of the sound area at the surface ($z=0$) $T_s$ at time $t_1$ is given by:
\begin{equation}
\Delta T_s(t_1) = \frac{Q}{e\sqrt{\pi t_1}}
\end{equation}
Then at time $t$, the temperature can be written as:
\begin{equation}
\Delta T_s(t) = \frac{Q}{e\sqrt{\pi t_1}} = \sqrt{\frac{t_1}{t}}\cdot \Delta T(t_1)
\end{equation}
where $t_1$ is a time between the thermal pulse lasting and the time at which the first temperature spot of the subsurface defects appear.\\
The absolute temperature contrast (Eq. \ref{AC_eq}) can be rewritten as:
\begin{align}
C_{ac}(t) = & [T_{def}(t) -T_0] - [T_s(t) - T_0] \\ 
		  = & \Delta T_{def}(t) - \sqrt{\frac{t_1}{t}}\cdot \Delta T(t_1)
\end{align}
DAC Applications on different type of material samples can be found in \citep{pilla2002new}, which has proven this method has an effective improvement on the signal to noise ratio.

\subsection{Temperature Signal Reconstruction (TSR)}
The thermographic signal reconstruction (TSR) data processing technique is one of the most recent improvements which raise thermography to the level of the most established NDE techniques \citep{Balageas2015}.

Same as DAC method, Eq. \ref{PT_eq_2} can be rewritten  in logarithmic way as:
\begin{equation}
\log (\Delta T) = \log (\frac{Q}{e}) - \frac{1}{2}\log (\pi t)
\end{equation}
In a general way, as: %with degree $n$:
\begin{equation}
\log (\Delta T) = a_0 + a_1\log (t) + a_2[\log (t)]^2 +...+ [a_n\log(t)]^n
\end{equation}
This technique usually consists in the fitting of the experimental log-log plot by a polynomial of degree $n$, and also the computation of the 1$^{st}$ and 2$^{nd}$ derivatives of the thermograms.



\subsection{Principal Component Thermography (PCT)}

\subsection{Receiver Operating 	Characteristic (ROC)}
The Receiver operating characteristic (ROC) curve is a technique in statistics which helps visualize, organize and select classifiers based on their performance. The curve graph is created by plotting the the true positive rate (TPR) against the false positive rate (FPR) at various threshold settings. More details about concepts and definitions can be found in \citep{Fawcett2006}. While being frequently chosen a standard method in several scientific fields, ROC are rarely applied in the thermographic field.\citep{Bison2014a} 

In the ROC curves definitions, the \textit{sensitivity} is the True positive rate (\textit{tp rate}), and \textit{1-specificity} is the False positive rate (\textit{fp rate}).


\section{Modeling and simulation}
In this research, all modelling and simulation work are undertaken in the platform COMSOL Multiphysics$^{\textregistered}$, which might be helpful when comparing with the experimental results.
%\section{Introduction to COMSOL Multiphysics}

COMSOL Multiphysics is a general-purpose software platform, based on advanced numerical methods, for modeling and simulating physics-based problems. It is a finite element analysis, solver and Simulation software / FEA Software package for various physics and engineering applications, especially coupled phenomena, or multiphysics.

The advantages of using COMSOL Multiphysics for modeling:
\begin{itemize}
	\item COMSOL Multiphysics has an integrated modeling environment.
	\item COMSOL Multiphysics takes a semi-analytic approach: once questions specified, COMSOL symbolically assembles finite-element method matrices and organizes the bookkeeping.
	\item COMSOL Multiphysics is fully compatible with MATLAB, so user could define programming for the modeling,organizing the computation, or the post-processing has full functionality. COMSOL Script is a MATLAB-like integrated programming environment that can also provide these facilities.
	\item COMSOL Multiphysics provides pre-built templates as Application Modes and in the Model Library for common modeling applications.
	\item COMSOL Multiphysics provides multiphysics modeling--linking well known ``application modes" transparently.
	\item COMSOL Multiphysics innovated extended multiphysics--coupling between logically distinct domains and models that permits simultaneous solution.
\end{itemize}

All the simulation parameters and conditions can be found in corresponding chapters.             % chapitre 1
%<articles>\include{chapitre1-articles}    % chapitre 1
%<!articles>\chapter*{Maintain the ``cold chain"}     % numéroté
\phantomsection\addcontentsline{toc}{chapter}{Maintain the ``cold chain"} % inclure dans TdM
The following two chapters will present two published paper concerning the application of infrared thermography for Non-Destructive Testing \& Evaluation applied in ``Maintain the cold chain" procedure.

The first study mainly focused on mapping the heat flux on the external surface of an insulated roll-container by Infrared thermography technique. The ATP standard measurement was performed to obtain the experimental results, meanwhile IR images of roll-container have been taken when the steady condition arrived, in order to analyze and compute the corresponding heat flux on entire surfaces. A simple thermal resistance model has been applied to realize the computation. Final temperature figures showed a good uniform distribution, and several defects in the structure like thermal bridges or air leakages have been identified. A reference zone of the external wall is measured by a thermal flux meter, then with that reference the whole surface heat flux map have been figured out. Besides, for a better view of the heat flux map, the homography technique has been performed into the raw images by applying a bilinear interpolation with the projective transformation matrix. The final corrected heat flux map has been demonstrated for each surface, in which the right one showed a smaller value than the others. 

In the second research, a panoramic view of the heat flux on the internal surface of an insulated vehicle by infrared thermography technique has been established in this study. The ATP measurement was performed to obtain the experimental results.  An IR camera was mounted on a pan-tilt head and automatically driven by a suitable software to map the temperature of the inner walls at the steady condition, in order to analyze and compute the corresponding heat flux on the entire surface.  The final result of the K-value obtained by IR thermography is accurate enough and compares well with that of the ATP test.

\chapter{Mapping of the heat flux of an insulated small container by infrared thermography}
\section{Introduction}
Nowadays, the public is increasingly aware to the need of applying rigorous standards all along the ``food chain'', considering a better life with a good food supply.  Thus the transport of food in the refrigerated vehicles (such as trucks, trailers, containers, etc.), especially for dairy products, meat and frozen foods, is of great interest. Moreover, the ever increasing cost of energy incites limiting the minimum refrigeration. So it is essential to ensure a perfect thermal insulation at the vehicles inspection. 

There exits some agreements of the thermal insulation tests which ensures the suitability for the transport of food in refrigerated conditions---ATP.
``The Agreement on the International Carriage of Perishable Foodstuffs and on the Special Equipment to be Used for such Carriage (ATP) done at Geneva on 1 September 1970 entered into force on 21 November 1976'' \citep{Geneva1970}, which establishes standards for the international transport of perishable food between the states that ratify the treaty. It has been updated through amendment a number of times and as of 2013 has 48 state parties, most of which are in Europe or Central Asia. It is open to ratification by states that are members of the United Nations Economic Commission for Europe (UNECE) and states that otherwise participate in UNECE activities\citep{ATP_wiki}.
The details contents of the agreements can be found in \citep{ATP_wiki, rossi2009k}, therefore no more description in this report will be presented.

The institute (CNR-ITC) has extensive experience in the measurement of heat transfer coefficient $K$ applied to refrigerated vehicles \citep{rossi2009k,bison1993automatic,bison2012geometrical,dragano2009experimental,grinzato2010r}. It is also responsible for the certification of these vehicles throughout Italy.

In summary, the major methods and procedures for measuring and checking the insulation of the equipment during the ATP test, is following \citep{rossi2009k}: 
\begin{itemize}
	\item \textbf{insulated equipment} built with insulated envelope such that the heat exchanged between inside and outside is limited in such a way that the overall coefficient of heat transfer ($K$-value) is assignable into 2 classes: a)equal to or less than 0.7 $W/(m^2 K)$ for normally insulated equipment; b) equal to or less than 0.4 $W/(m^2 K)$ for heavily insulated equipment; 
	
	\item\textbf{refrigerated equipment} that are insulated equipment which utilize some source of cold like ice, eutectic plates, dry ice etc. This equipment, with an outside temperature of $30^{\circ}$C must be able to lower the inside temperature to: + $7^{\circ}$C (class A); $-10^{\circ}$C (class B); $-20^{\circ}$C (class C); $0^{\circ}$C (class D); 
	
	\item \textbf{mechanically refrigerated equipment} that  are  insulated  equipment  furnished  of  its  own  refrigerating appliance.  The  appliance,  with  an  outside  temperature  of  $30^{\circ}$C,  must  be  capable  of  lowering  the  inside temperature to: from $+12^{\circ}$C to $0^{\circ}$C (class A); from $+ 12^{\circ}$C to $-10^{\circ}$C (class B); from $+ 12^{\circ}$C to $- 20^{\circ}$C (class C); 
	
	\item \textbf{heated  equipment} that  can  heat  the  inside  (to  avoid  the  freezing  of  foodstuffs)  are  used  in  very  cold countries.
\end{itemize}
Especially, the overall coefficient of heat transfer ($K$) is defined as:
\begin{equation}
K = \frac{W}{S\cdot \Delta \theta}
\end{equation}
where W is  the  power  necessary  to  maintain  a  steady  temperature  difference $\Delta \theta$ between  the  mean internal and external air temperature of the equipment. $S$ is the mean surface of the equipment, given by the geometric mean of the inside and outside surface areas:
\begin{equation}
S = \sqrt{S_i \cdot S_e}
\end{equation}

The ATP standard test is a procedure to measure the insulating status of equipments with a global approach. Its robustness has been well demonstrated. While, on the other hand, some local defects in the structure of equipment, such as thermal bridges, air leakages or zones of anomalous aging, cannot be visualized in this procedure. Then the thermography technique could be particularly helpful to these issues. In fact, all the defects mentioned above lead to a variation of the heat flux and temperature on the surface of the equipment \citep{grinzatoquality,grinzato1comparison}. Therefore the local heat flux map of the equipment by Infrared thermography could give a straightforward visualization of the structure, and also a local evaluation of the K-value.

The work in this internship consists the modeling and simulation for an insulated roll container by COMSOL in theory part. In practice party, the experimentation will be performed to map the heat flux on its external surface by Infrared thermography. Besides, another test for a refrigerated truck will also be done. These primary results would be obtained and discussed. Some conclusions and perspectives come in the end.


\section{Theory \& Methods}
\subsection{The heat transfer model}
Analogy to an electrical circuit, the heat transfer can be modelled as the heat flux is represented by current, temperatures are represented by voltages \citep{Therm_Re}. Therefore, the resistors in the heat ``circuit" is then the thermal resistance. Symbolically Ohm’s law can be expressed as
\begin{equation}
I = \frac{\Delta V}{R_e}
\end{equation}
where $I$ is the current flowing through an element, $\Delta V$ is the voltage across the element, and $R_e$ is the electrical resistance across the element. With the observed analogy, Fourier’s law can be written similarly as
\begin{equation}
q = \frac{\Delta T}{R_t}
\end{equation}
where $q$ is the flux of heat conduction, $\Delta T$ is the temperature difference between the surfaces of a slab, and $R_t$ is the thermal resistance.

In this work, the thermal resistance model is applied to a roll-container, from inside to outside (More details about the container can be found in Chapter \ref{box_detail}). For the standard ATP requirement, a radiation heater is working inside the container to maintain a fixed a higher internal air temperature. After the steady conditions are reached, a heat power $W$ is delivered. Heat flow is transferred by convection from the hot inside air to the internal wall of the box, and then by conduction through internal wall to external wall, and again by convection from the external wall to the outside air, which is cooled by the ATP system. The total scheme is presented in Fig \ref{Therm_Res}
\begin{figure}[!htpb]
	\centering
	\includegraphics{mapping/Therm_Res}
	\caption{Overall thermal resistance of the roll-container}
	\label{Therm_Res}
\end{figure}

\noindent where $\theta_i$ and $\theta_e$ are the internal and external temperature of the container, respectively. $\theta_{wi}$ and $\theta_{we}$ are the internal and external wall temperature of the container. In essentially 1D hypothesis, $h_i$ and $h_e$ are respectively the convective heat exchange internal and external coefficients. $\lambda$ is the thermal conductivity of the container wall and $l$ its thickness, and $S$ is the mean surface of the box.


\subsection{Simulation by Comsol}

A simple simulation work for this model has been performed by COMSOL MultiPhysics, which helps to better understand the distribution of the temperature of final result. The measurements of box dimension can be found in Tab \ref{tab_box_dim}.

\begin{table}[h]
	\centering
	%\begin{tabular}{p{85pt}p{85pt}p{85pt}p{85pt}}
	\begin{tabular}{c|c|c|c||c}
		\hline
		Inside:  & $L_i=0.864$ m  & $D_i=0.613$ m  & $H_i=1.55$ m & Thickness\\
		\hline
		Outside:  & $L_o=0.988$ m & $D_o=0.74$ m & $H_o=1.67$ m  &  $50$ mm \\
		\hline
	\end{tabular}
	\caption{Roll container dimensions}
	\label{tab_box_dim}
\end{table}

Same as the ATP standard test, the box inside air temperature is set as $32.5^{\circ}$C, and the outside air temperature is kept as $7.2^{\circ}$C. 

The Heat Transfer in Solid was used during the simulation in this modelling. A heat flux is added inside the box between the air and the internal surface, with a free convection, then the coefficient is set as $h_i=15 W/(m^2 K)$. Another heat flux was added outside between the air and the external surface with a forced convection, by $h_e=25 W/(m^2 K)$ [referring \citep{airhe,htwiki}]. The material of the box is made by high density polyurethane foam, with a conductivity about 0.0026 $W/(mK)$ \citep{jarfelt2006thermal}. Two high conductive thin layer are added into the material, same as the experimental one. The study condition is set to the steady status, as in real test the whole period is more than 12 hours.

The simulation result is shown in Fig. \ref{3D_T}. 
\begin{figure}[!htbp]
	\centering
	\includegraphics[scale=0.65]{mapping/3D_T}
	\caption{Temperature distribution of the roll container in COMSOL}
	\label{3D_T}
\end{figure}

The distribution of temperature of the entire container presented above indicates an uniform result, since the model and simulation is in ideal condition. Neither air leakage no thermal bridge is found here.

The following two figures (\ref{cut_plane}) introduce two cut plane inside the box, which illustrate the temperature details in the layers. As one can see that, the heat transfers from inside to outside, with temperature decreasing from internal surface to external surface. Moreover, the high thickness in the corners lead to a less transferred heat, showing a lower corresponding temperature.
\begin{figure} [htbp]
	%	\centering
	\hspace{-20pt}
	\includegraphics[scale=0.40]{mapping/xy_plane}
	\includegraphics[scale=0.40]{mapping/2D_xy_plane}
	\vspace{5pt}
	\hspace{-20pt}
	\includegraphics[scale=0.40]{mapping/yz_plane}
	\includegraphics[scale=0.40]{mapping/2D_yz_plane}
	\caption{Temperature distribution of the cute planes}
	\label{cut_plane}
\end{figure}

\section{Experimental setup}
\subsection{Roll-container}
In this part the experimental installation will be presented. The aim of this work is to map the heat flux on the external surface of a roll-container, and Fig \ref{box} shows the installed probes on the surface of the roll container.

\begin{figure}[!htbp]
	\centering
	\includegraphics[scale=0.13]{mapping/DSC_0191}
	\includegraphics[scale=0.13]{mapping/DSC_0189}\\
	\vspace{4pt}
	\includegraphics[scale=0.13]{mapping/DC_51939}
	\includegraphics[scale=0.13]{mapping/DC_51941}
	\caption{The roll container used for the test[outside and inside(up-right)]}
	\label{box}
\end{figure}

This box is actually made by sandwich panels with three layers: two internal-external skins made by polyester-fiberglass and a core made by high density polyurethane foam, and total thickness is 50 $mm$. \label{box_detail}

12 points of measurements (thermal couple) are positioned at the 8 corners (inside and outside) and also at the center of 4 surfaces (front, left,back and right)[Fig \ref{therm_couple}]. All the thermal couples are at a distance of 10 cm from the wall.
\begin{figure}[!htbp]
	\centering
	\includegraphics[scale=0.40]{mapping/therm_couple}
	\caption{The position of the thermal couples}
	\label{therm_couple}
\end{figure}
Besides, on the upper-center of the front surface, a heat flux meter is set to measure the corresponding heat flux (Fig \ref{box} left). The acquisition system used in this work is an infrared camera--FLIR SC-660. It records the thermography image series during the whole test for the front and left surfaces of container (schema shown in Fig \ref{box} left). What's more, when the steady conditions were reached (a state period of time not less than 12 hours, according to the ATP standard \citep{rossi2009k}), several infrared images were captured for the external walls such as front, left, back, right and top one (impossible to capture the bottom surface). 

With standard ATP measurement during the test, a radiation heater is heating the inside the container to maintain a temperature about $32.5^{\circ}$C. For outside, with the air circulating a velocity between 1 and 2 $m s^{-1}$ in the tunnel, the temperature is maintained constant at about $7.2^{\circ}$C. Thus makes a temperature difference between inside air and outside of $\Delta \theta = 25.3^{\circ}$C.

Two hypothesis are proposed in this test: 1) the  heat exchange coefficients, inside ($h_i$) and outside ($h_e$) of the container are constant;\label{hyp1} 2) the heat diffusion is mainly 1D. These hypothesis serve the computation of heat flux for each external surface in the following chapter.

\subsection{Refrigerated Vehicle}

Even though the main work is testing on the roll container, one has also applied the same experimental setup to a refrigerated truck (Fig \ref{truck}), which refers to the main objective of the ATP agreements.
\begin{figure}[!htbp]
	\hspace{-10mm}
	\includegraphics[scale=0.12]{mapping/DSC_0210}
	\includegraphics[scale=0.12]{mapping/DSC_0211}\\
	
	\includegraphics[scale=0.12]{mapping/DSC_0209}
	\includegraphics[scale=0.12]{mapping/DSC_0213}
	\caption{The refrigerated vehicle used for the test}
	\label{truck}
\end{figure}

For this test, only thermography images of several surfaces (top, left, back and right) have been captured when the steady condition was reached. Besides, the heat flux meter was first set on the left surface (Fig \ref{truck} upper-left) and then was moved to test the back surface (Fig \ref{truck} lower-right). The former result contains the influence of air streaming whose velocity is between 1 and 2 $ms^{-1}$, while in the later one, the ventilation system had been switched off.

\section{Results \& Discussion}
\subsection{Heat flux map}
The idea is to measure the heat flux by a thermal flux meter in a reference zone outside of the insulated envelope and to build successively the whole map of the heat flux as a linear relation of the temperature difference between the outside wall and the air.

The specific heat flux at a reference zone of the external wall is measured by a thermal flux meter, then with the wall temperature in the proximity of the flux meter and by measuring the air temperature, one can draw the local approach like (\citep{rossi2009k}):
\begin{equation}
q_r = \frac{\theta_{we}(x_r,y_r)-\theta_e}{1/h_e}
\end{equation}
which gives that:
\begin{equation}
h_e = \frac{q_r}{\theta_{we}(x_r,y_r)-\theta_e}
\end{equation}
where $q_r$ is the heat flux measured at the reference point ($x_r,y_r$) by the thermal flux meter. $\theta_{we}$ is the temperature measured by thermography at the reference point. With the hypothesis 1) mentioned in Section \ref{hyp1}, the heat flux map of the whole surface could be determined with the temperature map according to:
\begin{equation}
q(x,y) = \frac{q_r}{\theta_{we}(x_r,y_r)-\theta_e}(\theta_{we}(x,y)-\theta_e)
\label{eq_q}
\end{equation}
Where $\theta_{we}(x,y)$ is the temperature at each point of the external surface. This equation indicates that the  heat flux has a linear relation of the temperature difference between the outside wall and the air.

Finally the temperature of all the surfaces are distributed as Fig \ref{IR_box}:
\begin{figure}[!htbp]
	%	\hspace{-10mm}
	\centering
	\includegraphics[scale=0.50]{mapping/IR_front_m}
	\hspace{6pt}
	\includegraphics[scale=0.50]{mapping/IR_back_m}
	\vspace{3pt}
	\includegraphics[scale=0.50]{mapping/IR_left_m}
	\hspace{6pt}
	\includegraphics[scale=0.50]{mapping/IR_right_m}
	\includegraphics[scale=0.50]{mapping/IR_top}
	\caption{The temperature map of the roll container (front, back, left, right and top surfaces)}
	\label{IR_box}
\end{figure}
All these IR images were captured at steady condition. 
In the figure, a good uniform distribution of temperature was well shown, even though there are several abnormal lines which may be the thermal bridges or air leakages (at front, left, right and top surfaces). An attention might be paid is that there is an area of the light reflection on the container surface (on the second image left part). It could be no influence here as the back surface is the main part of the second figure.

Comparing all the surface temperature, one can see that on the right surface of container, the temperature value is a little lower than other surfaces. This result may conclude the Hypothesis 1) \ref{hyp1} is not fully correct, then the convective heat transfer coefficient could be different around the roll container during the test.

The temperature at the reference point around the thermal flux meter was presented in the figure (the first one), with that one could figure out the map of heat flux of the entire surface by Eq \ref{eq_q}. And the data measured by the thermal flux meter are presented in Fig \ref{flux_meter}.
\begin{figure}[!htbp]
%	\hspace{-35pt}
	\centering
	\includegraphics[scale=0.65]{mapping/It_project_2014_QProfile}
%	\vspace{20pt}
%	\hspace{-35pt}
	\includegraphics[scale=0.65]{mapping/It_project_2014_TProfile}
	\caption{Data from thermal flux meter (heat flux and temperature profiles)}
	\label{flux_meter}
\end{figure}
These data recorded the whole test. Thought the fluctuation is very high, ones took the steady period after 5 hours from the beginning. Then mean values are $q_r=9.73 W/(m^2 K)$ and $T_r = 7.83 ^\circ$C, which served the computation of heat flux.

Moreover, for a better view, some techniques are often applied for images corrections \citep{bison2012geometrical}. In this work, an easier way to realize the image correction is to apply the homography technique.

\subsection{Homography application}
In practical applications, such as image rectification, image registration, or computation of camera motion—rotation and translation—between two images, a homography is often performed in the field of computer vision \citep{homo_wiki}.

In this work, as one knows the radio between the length and the width of the roll container (or between the width and the height in same way), a projective transformation matrix in images has been obtained. Then applying a bilinear interpolation with the projective transformation matrix, into the raw images, one finally got the corrected heat flux mapping of each surfaces, demonstrated in Fig \ref{Q_box}.
\begin{figure}[!htbp]
	%	\hspace{-10mm}
	\centering
	\includegraphics[scale=0.55]{mapping/Q_front_m}
	\hspace{5pt}
	\includegraphics[scale=0.55]{mapping/Q_back_m}
	\includegraphics[scale=0.55]{mapping/Q_left_m}
	\hspace{5pt}
	\includegraphics[scale=0.55]{mapping/Q_right_m}
	\includegraphics[scale=0.55]{mapping/Q_top_m}
	\caption{The corrected heat flux map of the roll container (front,  back, left, right and top surfaces)}
	\label{Q_box}
\end{figure}

The Heat flux map exhibits that in right surface of the roll container, one gets smaller heat flux value rather than those on other surfaces, such as front, top, back and left.
On the other hand, the heat flux map of back, top and left surface show a little higher mean value than that of the front one, which is due to the air stream.

A contradiction is found that, normally the right surface's heat flux should be a bit greater than the front surface where the air stagnation is probable, as air stream flows the lateral surface, and that leads to a lower temperature in the right surface. While in the thermography images, one got the contrary result. The reason for this is due to the linear relationship between heat flux and the difference of temperature between external surface and the air. Since in our model Eq.\ref{eq_q}, lower $\theta_{we}(x,y)$ would lead to a lower $q(x,y)$. This might also indicate that the influence of the convective heat transfer in lateral surface is more important than imagining.
Comparing to the standard ATP measurement, the global K-value obtained is about $0.53 W/(m^2 K)$, then multiplied by the temperature difference $\Delta \theta =25.3^\circ$C, thus it gives us a global heat flux value of the roll container $13.41 W/m^2$. With thermography one obtained the local heat flux in the map about $10.7 W/m^2$, which is more or less a good result.

``A local variation of thermal conductivity leads to a local increase of the heat flux with a consequent variation of the local internal and external wall temperature"\citep{rossi2009k}, this phenomena can be well seen in all figures above.

\subsection{Vehicle results}
The temperature map of several surfaces of the refrigerated truck are distributed in Fig \ref{IR_truck}.
\begin{figure}[!htbp]
	%	\hspace{-10mm}
	\centering
	\includegraphics[scale=0.50]{mapping/IR_truck_lt_m}
	\hspace{6pt}
	\includegraphics[scale=0.50]{mapping/IR_truck_rt_m}
	\vspace{3pt}
	\includegraphics[scale=0.50]{mapping/IR_truck_bk11_m}
	\hspace{6pt}
	\includegraphics[scale=0.50]{mapping/IR_truck_bk12_m}
	\includegraphics[scale=0.50]{mapping/IR_truck_tp_m}
	\caption{The temperature map of the refrigerated vehicle (left, right, back1, back2 and top surfaces)}
	\label{IR_truck}
\end{figure}
From all the figures, one can see a good uniform of the temperature distribution on all the external surfaces of the vehicle. However, the reflection effects are very heavy on the left, right and top surfaces (which are not the abnormal lines in figures). Besides, in this test, lower temperature are found on the lateral surfaces than the back one, thanks to the air streaming.

%Two tests of the thermal flux have been performed: one on the left surface with air streaming; another on the back surface without ventilation system. The corresponding two final data profiles can be found in Fig \ref{truck_meter}.
%\begin{figure}[!htbp]
%	\centering
%	\includegraphics[scale=0.5]{mapping/Q_truck_left_profil}
%	\includegraphics[scale=0.5]{mapping/T_truck_left_profil}
%	\includegraphics[scale=0.6]{mapping/Q_truck_back_profil}
%	\includegraphics[scale=0.6]{mapping/T_truck_back_profil}
%	\caption{Data from thermal flux meter for the refrigerated vehicle (up:left surface; down:back surface)}
%	\label{truck_meter}
%\end{figure}
%The mean values of all these data in steady status are in following table (Tab \ref{tab_truck}).
%\begin{table}[h]
%	\centering
%	%\begin{tabular}{p{85pt}p{85pt}p{85pt}p{85pt}}
%	\begin{tabular}{l|r}
%		\hline
%		left surface  & back surface  \\
%		\hline
%		$\overline{q}=10.34 W/(m^2 K)$ & $\overline{q}=5.62 W/(m^2 K)$  \\
%		\hline
%		$\overline{T}=7.84^\circ$C & $\overline{T}=8.97^\circ$C  \\
%		\hline
%	\end{tabular}
%	\caption{Thermal flux meter data for truck}
%	\label{tab_truck}
%\end{table}
%
%The huge difference (45\% error) between the heat flux of the back and left surface makes no sense, since two test were almost at the same condition. Therefore, that might because when switching the thermal flux meter from left surface to back surface, something wrong has been done to influence the accuracy of the equipment, like not attaching well to the surface. Another consideration is that, for the back surface test, the measuring time was not long, as it has already arrived in the steady condition. While in the profiles, a little increasing tendency in temperature could be observed. This may indicate again that the vehicle result of back surface was not so good.

As the main work of the roll container has been done, more tests on the truck will be undertaken in future measurements.


\section{Conclusion \& Perspectives}

%\addcontentsline{toc}{chapter}{Conclusion \& Perspectives}
This preliminary work mainly focused on mapping the heat flux on the external surface of an insulated roll-container by Infrared thermography technique. The ATP standard measurement was performed to obtain the experimental results, meanwhile IR images of roll-container have been taken when the steady condition arrived, in order to analyze and compute the corresponding heat flux on entire surfaces. A simple thermal resistance model has been applied to realize the computation. Final temperature figures showed a good uniform distribution, and several defects in the structure like thermal bridges or air leakages have been identified. A reference zone of the external wall is measured by a thermal flux meter, then with that reference the whole surface heat flux map have been figured out. Besides, for a better view of the heat flux map, the homography technique has been performed into the raw images by applying a bilinear interpolation with the projective transformation matrix. The final corrected heat flux map has been demonstrated for each surface, in which the right one showed a smaller value than the others. Due to the air streaming, one got a bit smaller temperatures in the lateral surfaces than other surfaces, thus making the smaller heat flux values. That also indicated the convective heat transfer coefficient was not constant around the roll-container surfaces, contrary to our Hypothesis 1 in theory.

For the refrigerated vehicle test, two surfaces (left and back) have been taken into measurement with the thermal flux meter. The former one was tested with the air streaming, while the ventilation system was switched off for the later one. The final IR images presented a big reflection influence on the lateral surfaces. And a huge difference between the heat flux value on two test surfaces was found, which might be because the equipment was not attached well to the surface when moved. 

For future work, more tests will be taken place on refrigerated vehicle with two thermal flux meters measuring on the same time, to avoiding the problem encountered in this work. And the suppression of thermal reflections in thermal imaging \citep{vollmer2004identification} may be taken into consideration during the IR image processing. 

This project topic can be also extended to buildings where the monitoring of the effective transmittance is crucial for the energy saving \citep{grinzato2010r}.             % chapitre 2, etc.
%<articles>\include{chapitre2-articles}    % chapitre 2, etc.
\chapter{Conclusion \& Perspectives}         % ne pas numéroter
%\phantomsection\addcontentsline{toc}{chapter}{Conclusion \& Perspectives} % dans TdM

\section{General conclusions}
The objectives of the present thesis were, first, to deploy the infrared thermography technique in the procedure of maintaining the ``cold food chain'', especially in insulated vehicles of ATP standards. The application of infrared thermography aims to identify thermal insulation anomalies, which the standard ATP test cannot localize. 

The preliminary work focused on mapping the heat flux on the external surface of an insulated roll-container using the Infrared thermography technique. The ATP standard measurement was performed to obtain the experimental results, meanwhile IR images of the roll-container were obtained when the steady condition was reached, in order to analyze and compute the corresponding heat flux on the entire surface. A simple thermal resistance model was applied to conduct the computation. Final temperature figures showed a good uniform distribution, and several defects in the structure such as thermal bridges or air leakages were identified. A reference zone of the external wall was measured by a thermal flux meter, then with that reference the entire surface heat flux map was determined. In addition, for a better view of the heat flux map, the homography, one of the computer vision techniques, was performed in the raw images by applying a bilinear interpolation with the projective transformation matrix. The final corrected heat flux map was demonstrated for each surface, in which the right surface showed a smaller value than the others.  Due to the air streaming, temperatures in the lateral surfaces were a little smaller than other surfaces, thus this leads to the smaller heat flux values.

When implemented into the internal surface of the insulated vehicle, a panoramic view was needed, since the field of view (FOV) of the infrared camera could not capture the entire surface of insulated vehicle. With the help of an infrared camera mounted on a pan-tilt head and automatically driven by a suitable software, a series of thermal images of the inner walls of the vehicle under steady condition have been captured. Proper computer vision techniques such as inverse spherical projection and stitching images by translation helped to generate the final panorama. The same thermal resistance model was utilized to compute the corresponding heat flux map. Results demonstrated a good performance of the algorithm, though the manual creation of the panoramic view required more time for completion. Compared with the standard ATP test, the final K-value obtained by infrared thermography showed a good accuracy (0.87\% of error).

Then based on the previous favorable results, the second aspect of this project was to explore cold approaches (such as compressed air, liquid nitrogen, etc.) in infrared thermography for Non-Destructive Testing \& Evaluation. The first interesting idea of detection of insulation flaws and thermal bridges in insulated truck box panels then emerged. As it is not convenient to heat the entire vehicle panels for the detection, cooling them by compressed air therefore can be a better solution. The study then focused on the cooling approach for the truck box panels inspection by infrared thermography. Both heating and cooling methods were applied by lamp and compressed air respectively. Numerical simulations under COMSOL Multiphysics{\textregistered} platform were conducted as well. For a comprehensive analysis, passive thermography detection in computational models has been presented at the same time. Results demonstrate that the compressed air spray is more rapid than the traditional heating method in providing successful detection.

A consideration of replacing compressed air by liquid nitrogen was then explored in more detail. Thus a study was performed, in which a steel specimen was used to test three different stimulations for thermal images and also Receiver Operating Characteristic (ROC) analysis comparison. Results showed that all techniques highlighted part of the flaws in the sample, whereas the liquid nitrogen technique represented the defects only at the beginning; this may be due to the high conductivity of steel. In thermal results, the PCT post-processing method displayed better results for all procedures. More defects were exhibited in Flash stimulation with PCT processing. ROC curve analysis has elucidated a straightforward classification comparison, in which the best curve was obtained using the Flash technique with PCT processing. 



\section{Future perspectives}
Generally, the infrared thermography technique has been applied with promising results in ATP standard insulated vehicle tests, for the goal of maintaining the ``cold food chain". The results have demonstrated that the benefits of time-saving and the accuracy in the determination of the K-value could be applied for assessment at a commercial level.  For the panoramic view of the insulated vehicle internal surface, due to the repeated structure on the internal surface of vehicle, algorithms of automatic creation which have better feature detection and comparison remain to be explored.  A suitable software package may be created to simplify the post-processing of thermal images. This will facilitate the computation of the K-value.

On the other hand, the exploration of cold approaches in infrared thermography for Non-Destructive Testing and Evaluation has also shown several advantages. Compressed air cooling can be a good replacement of heating in the detection of insulation flaws for truck box panels. The strategy of heating one side and cooling another side can be deployed in practice, since ideal cases in simulation show favorable results.

The use of ROC curves to compare the different methods in infrared thermography is an interesting approach. This technique would benefit from more complete application and discussion, which would favor a more harmonious implementation in traditional techniques of NDT.



            % conclusion

\appendix                       % annexes le cas échéant

\chapter{Titre de l'annexe}     % numérotée

Texte de l'annexe.
                % annexe A

%<!articles>\bibliography{}                 % production de la bibliographie
%<!articles>
\end{document}
%</gabarit>
%
% ^^A Gabarits des parties du document
%<*resume>
\chapter*{Résumé}                      % ne pas numéroter
\phantomsection\addcontentsline{toc}{chapter}{Résumé} % inclure dans TdM

\begin{otherlanguage*}{french}
  Texte du résumé en français.
\end{otherlanguage*}
%</resume>
%
%<*abstract>
\chapter*{Abstract}                      % ne pas numéroter
\phantomsection\addcontentsline{toc}{chapter}{Abstract} % inclure dans TdM

\begin{otherlanguage*}{english}
  Text of English abstract.
\end{otherlanguage*}
%</abstract>
%
%<*remerciements>
\chapter*{Remerciements}         % ne pas numéroter
\phantomsection\addcontentsline{toc}{chapter}{Remerciements} % inclure dans TdM

Texte des remerciements en prose.
%</remerciements>
%
%<*avantpropos>
\chapter*{Avant-propos}         % ne pas numéroter
\phantomsection\addcontentsline{toc}{chapter}{Avant-propos} % inclure dans TdM

L'avant-propos est surtout nécessaire pour une thèse par article.
%</avantpropos>
%
%<*introduction>
\chapter*{Introduction}         % ne pas numéroter
\phantomsection\addcontentsline{toc}{chapter}{Introduction} % inclure dans TdM

Une thèse ou un mémoire devrait normalement débuter par une
introduction. Celle-ci est traitée comme un chapitre normal, sauf
qu'elle n'est pas numérotée.
%</introduction>
%
%<*chapitre>
\chapter{Titre du chapitre}     % numéroté

Texte du chapitre.
%<articles>
%<articles>\bibliographystyle{}              % style de la bibliographie
%<articles>\bibliography{}                   % production de la bibliographie
%</chapitre>
%
%<*conclusion>
\chapter*{Conclusion}         % ne pas numéroter
\phantomsection\addcontentsline{toc}{chapter}{Conclusion} % dans TdM

Une thèse ou un mémoire devrait normalement se terminer par une
conclusion, placée avant les annexes, le cas échéant. Celle-ci est
traitée comme un chapitre normal, sauf qu'elle n'est pas numérotée.
%</conclusion>
%
%<*annexe>
\chapter{Titre de l'annexe}     % numérotée

Texte de l'annexe.
%</annexe>
% Local Variables:
% mode: doctex
% coding: utf-8
% TeX-master: t
% TeX-engine: xetex
% End:
% \fi
