\chapter{Conclusion \& Perspectives}         % ne pas numéroter
%\phantomsection\addcontentsline{toc}{chapter}{Conclusion \& Perspectives} % dans TdM

\section{General conclusions}
The objectives of the present thesis were, first, to deploy the infrared thermography technique in the procedure of maintaining the ``cold food chain'', especially in insulated vehicles of ATP standards. The application of infrared thermography aims to identify thermal insulation anomalies, which the standard ATP test cannot localize. 

The preliminary work focused on mapping the heat flux on the external surface of an insulated roll-container using the Infrared thermography technique. The ATP standard measurement was performed to obtain the experimental results, meanwhile IR images of the roll-container were obtained when the steady condition was reached, in order to analyze and compute the corresponding heat flux on the entire surface. A simple thermal resistance model was applied to conduct the computation. Final temperature figures showed a good uniform distribution, and several defects in the structure such as thermal bridges or air leakages were identified. A reference zone of the external wall was measured by a thermal flux meter, then with that reference the entire surface heat flux map was determined. In addition, for a better view of the heat flux map, the homography, one of the computer vision techniques, was performed in the raw images by applying a bilinear interpolation with the projective transformation matrix. The final corrected heat flux map was demonstrated for each surface, in which the right surface showed a smaller value than the others.  Due to the air streaming, temperatures in the lateral surfaces were a little smaller than other surfaces, thus this leads to the smaller heat flux values.

When implemented into the internal surface of the insulated vehicle, a panoramic view was needed, since the field of view (FOV) of the infrared camera could not capture the entire surface of insulated vehicle. With the help of an infrared camera mounted on a pan-tilt head and automatically driven by a suitable software, a series of thermal images of the inner walls of the vehicle under steady condition have been captured. Proper computer vision techniques such as inverse spherical projection and stitching images by translation helped to generate the final panorama. The same thermal resistance model was utilized to compute the corresponding heat flux map. Results demonstrated a good performance of the algorithm, though the manual creation of the panoramic view required more time for completion. Compared with the standard ATP test, the final K-value obtained by infrared thermography showed a good accuracy (0.87\% of error).

Then based on the previous favorable results, the second aspect of this project was to explore cold approaches (such as compressed air, liquid nitrogen, etc.) in infrared thermography for Non-Destructive Testing \& Evaluation. The first interesting idea of detection of insulation flaws and thermal bridges in insulated truck box panels then emerged. As it is not convenient to heat the entire vehicle panels for the detection, cooling them by compressed air therefore can be a better solution. The study then focused on the cooling approach for the truck box panels inspection by infrared thermography. Both heating and cooling methods were applied by lamp and compressed air respectively. Numerical simulations under COMSOL Multiphysics{\textregistered} platform were conducted as well. For a comprehensive analysis, passive thermography detection in computational models has been presented at the same time. Results demonstrate that the compressed air spray is more rapid than the traditional heating method in providing successful detection.

A consideration of replacing compressed air by liquid nitrogen was then explored in more detail. Thus a study was performed, in which a steel specimen was used to test three different stimulations for thermal images and also Receiver Operating Characteristic (ROC) analysis comparison. Results showed that all techniques highlighted part of the flaws in the sample, whereas the liquid nitrogen technique represented the defects only at the beginning; this may be due to the high conductivity of steel. In thermal results, the PCT post-processing method displayed better results for all procedures. More defects were exhibited in Flash stimulation with PCT processing. ROC curve analysis has elucidated a straightforward classification comparison, in which the best curve was obtained using the Flash technique with PCT processing. 



\section{Future perspectives}
Generally, the infrared thermography technique has been applied with promising results in ATP standard insulated vehicle tests, for the goal of maintaining the ``cold food chain". The results have demonstrated that the benefits of time-saving and the accuracy in the determination of the K-value could be applied for assessment at a commercial level.  For the panoramic view of the insulated vehicle internal surface, due to the repeated structure on the internal surface of vehicle, algorithms of automatic creation which have better feature detection and comparison remain to be explored.  A suitable software package may be created to simplify the post-processing of thermal images. This will facilitate the computation of the K-value.

On the other hand, the exploration of cold approaches in infrared thermography for Non-Destructive Testing and Evaluation has also shown several advantages. Compressed air cooling can be a good replacement of heating in the detection of insulation flaws for truck box panels. The strategy of heating one side and cooling another side can be deployed in practice, since ideal cases in simulation show favorable results.

The use of ROC curves to compare the different methods in infrared thermography is an interesting approach. This technique would benefit from more complete application and discussion, which would favor a more harmonious implementation in traditional techniques of NDT.



