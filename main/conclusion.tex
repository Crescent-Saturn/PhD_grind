\chapter{Conclusion \& Perspectives}         % ne pas numéroter
%\phantomsection\addcontentsline{toc}{chapter}{Conclusion \& Perspectives} % dans TdM

\section{General conclusions}
The objectives of the present thesis are, the first part, to deploy the infrared thermography technique in the procedure of maintain the ``cold food chain'', especially in the insulated vehicle of ATP standard. The application of infrared thermography aims to identify thermal insulation anomalies, in which the standard ATP cannot localize. 

The preliminary work focused on mapping the heat flux on the external surface of an insulated roll-container by Infrared thermography technique. The ATP standard measurement was performed to obtain the experimental results, meanwhile IR images of roll-container have been taken when the steady condition arrived, in order to analyze and compute the corresponding heat flux on entire surfaces. A simple thermal resistance model has been applied to realize the computation. Final temperature figures showed a good uniform distribution, and several defects in the structure like thermal bridges or air leakages have been identified. A reference zone of the external wall is measured by a thermal flux meter, then with that reference the whole surface heat flux map have been figured out. In addition, for a better view of the heat flux map, the homography, one of the computer vision technique, has been performed into the raw images by applying a bilinear interpolation with the projective transformation matrix. The final corrected heat flux map has been demonstrated for each surface, in which the right one showed a smaller value than the others.

When implemented into the internal surface of the insulated vehicle, a panoramic view was needed, since the field of view (FOV) of infrared camera cannot undertake the entire surface of insulated vehicle. With the help of an infrared camera mounted on a pan-tilt head and automatically driven by a suitable software, a series thermal images of the inner walls of the vehicle at a steady condition have been captured. Proper computer vision techniques such as inverse spherical projection, stitching images by translation helped make the final panorama. The same thermal resistance model was utilized to compute the corresponding heat flux map. Results demonstrated a good performance of the algorithm, though the manual creation of panoramic required more time for completion. Compared with the standard ATP test, the final K-value obtained by infrared thermography showed a good accuracy (0.87\% of error).

Then based on the previous favorable results, the second part is to explore cold approaches (such as compressed air, liquid nitrogen, etc.) in infrared thermography for Non-Destructive Testing \& Evaluation. The first pertinent idea of detection of insulation flaws and thermal bridges in insulated truck box panels then came up. As it is not convenient to heat the entire vehicle panels for the detection, cooling them by compressed air therefore can be a better solution. The following study focused on the cooling approach for the truck box panels inspection by infrared thermography. Both heating and cooling methods have been applied by lamp and compressed air respectively. Numerical simulations under COMSOL platform have been conducted as well. For a comprehensive analysis, passive thermography detection in computational models has been presented at the same time. Results demonstrate that the compressed air spray is more rapid than the traditional heating method in providing successful detection.

A consideration of replacing compressed air by liquid nitrogen is then undertaken for exploration. Thus a following study has been performed, in which a steel specimen is used to test three different stimulations for thermal images and also Receiver Operating Characteristic (ROC) analysis comparison. Results shows that all techniques highlight part of the flaws in the sample, whereas the liquid nitrogen technique represents the defects only at the beginning; this maybe due to the high conductivity of steel. In thermal results, the PCT post-processing method displays a better results for all procedures. More defects are exhibited in Flash stimulation with PCT processing. ROC curve analysis has elucidated a straightforward classification comparison, in which the best curve obtained is by the Flash technique with PCT processing. 



\section{Future perspectives}
Generally, we have well applied the infrared thermography technique in to ATP standard insulated vehicle tests, for the goal of maintaining the ``cold food chain". The results have demonstrated that the benefits of time-saving and the accuracy in determination of K-value could be applied to assess at commercial level.  For the panoramic view of the insulated vehicle internal surface, due to the repeated structure on the internal surface of vehicle, algorithms of automatic creation which have a better features detection and comparison remain to be explored.  A suitable software package may be created to simplify the post-processing of thermal images. This will facilitate the computation of K-value.

On the other hand, the exploration of cold approaches in infrared thermography for Non-Destructive Testing and Evaluation has also shown several advantages, compressed air cooling can be a good replacement of heating in detection of insulation flaws for truck box panels. The strategy of heating one side and cooling another side can be deployed in practice, since ideal cases in simulation show favorable results.

The use of ROC curves to compare the different methods in infrared thermography is an interesting approach. More complete application and discussion can be implemented into traditional techniques of NDT.



