\chapter*{Conclusion \& Perspectives}         % ne pas numéroter
\phantomsection\addcontentsline{toc}{chapter}{Conclusion \& Perspectives} % dans TdM

\section{General conclusions}
The objectives of the present thesis are, the first part, to deploy the infrared thermography technique in the procedure of maintain the ``cold food chain'', especially in the insulated vehicle of ATP standard. The application of infrared thermography aims to identify thermal insulation anomalies, in which the standard ATP cannot localize. 

The preliminary work focused on mapping the heat flux on the external surface of an insulated roll-container by Infrared thermography technique. The ATP standard measurement was performed to obtain the experimental results, meanwhile IR images of roll-container have been taken when the steady condition arrived, in order to analyze and compute the corresponding heat flux on entire surfaces. A simple thermal resistance model has been applied to realize the computation. Final temperature figures showed a good uniform distribution, and several defects in the structure like thermal bridges or air leakages have been identified. A reference zone of the external wall is measured by a thermal flux meter, then with that reference the whole surface heat flux map have been figured out. Besides, for a better view of the heat flux map, the homography technique has been performed into the raw images by applying a bilinear interpolation with the projective transformation matrix. The final corrected heat flux map has been demonstrated for each surface, in which the right one showed a smaller value than the others.

When implemented into the internal surface of the insulated vehicle, a panoramic view was needed. With the help of an infrared camera mounted on a pan-tilt head and automatically driven by a suitable software, a series thermal images of the inner walls of the vehicle at a steady condition have been obtained. Proper methods such as inverse spherical projection, stitching images by translation helped make the final panorama. The same thermal resistance model was utilized to compute the corresponding heat flux map. 

Then based on the previous favorable results, the second part is to explore cold approaches (such as compressed air, liquid nitrogen, etc.) in infrared thermography for Non-Destructive Testing \& Evaluation. 



\section{Future perspectives}
