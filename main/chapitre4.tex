\chapter*{Exploration of cold approach}     % numéroté
\phantomsection\addcontentsline{toc}{chapter}{Exploration of cold approach} % inclure dans TdM
The following two chapters will present two published paper concerning the exploration of cold approach in infrared thermography applied in Non-Destructive Testing \& Evaluation.

The first study concentrates on the detection of defects and thermal bridges in insulated truck box panels, by active infrared thermography. Comparison between heating and cooling approaches for experiments and models has been established. In addition, passive thermography detection in computational models has been presented. Results demonstrate that the compressed air spray is more rapid than the traditional heating method in providing successful detection. Even if the traditional heating approach provides clearer results, in reality it is not easy and practical to heat a whole truck box to conduct inspection: the compressed air spray approach is much more convenient.

The second research investigates an external stimulation–cooling instead of heating in IR Thermography for NDT \& E. A steel specimen is used to test three different stimulations for thermal images and also ROC analysis comparison. Results shows that all techniques highlight part of the flaws in the sample, whereas the LN$_2$ technique represents the defects only at the beginning; this maybe due to the high conductivity of steel. In thermal results, the PCT post-processing method displays a better results for all procedures. More defects are exhibited in Flash stimulation with PCT processing.  The results of this study were firstly presented at an oral session of SPIE Thermosense: Thermal Infrared Applications XXXIX 2017, in the United States.

\chapter{Detection of insulation flaws and thermal bridges in insulated truck box panels}
(Accepted for publication in the Quantitative InfraRed Thermography Journal, in May 2017).

%
The results of this study were firstly presented at an oral session of the 13th Quantitative InfraRed Thermography Conference 2016 at Gdańsk University of Technology in Poland . Then it was selected for the publication in Quantitative InfraRed Thermography Journal.

\section*{Résumé}
Cette étude se concentre sur la détection des défauts et des ponts thermiques dans les caissons isothermes, par thermographie infrarouge active. La comparaison entre les approches de chauffage et de refroidissement pour les expériences et les modèles a été établie. En outre, la détection par thermographie passive dans les modèles informatiques a été présentée. Les résultats démontrent que le spray d'air comprimé est plus rapide que la méthode de chauffage traditionnelle pour assurer une détection réussie. Même si l'approche de chauffage traditionnelle donne des résultats plus clairs, en réalité, il n'est pas facile ni pratique de chauffer une boîte de camion entière pour effectuer une inspection: l'approche par le spray d'air comprimé est beaucoup plus pratique.

\textbf{\texttt{Contributing authors:}}

\textbf{\textsf{Lei Lei}} (Ph.D candidate): developing protocol, experiment preparation and planning, data collection, personnel coordination and manuscript preparation.

\textbf{Alessandro Bortolin} (Ph.D student of CNR-ITC): data analysis, discussion and manuscript preparation.

\textbf{Paolo Bison} (Research supervisor of CNR-ITC): student supervision, revision and correction of the manuscript. 

\textbf{Xavier Maldague} (Research director of LVSN in University Laval): student supervision, revision and correction of the manuscript. 

\phantomsection\addcontentsline{lot}{table}{4.1\quad Specimen specification details}
\phantomsection\addcontentsline{lot}{table}{4.2\quad Air-cooling parameters}
\phantomsection\addcontentsline{lot}{table}{4.3\quad Materials properties}
\phantomsection\addcontentsline{lot}{table}{4.4\quad Thermal contrast peak talbe}

\phantomsection\addcontentsline{lof}{table}{4.1\quad Details of defects inside the specimen(left) and final specimen to test(right).}
\phantomsection\addcontentsline{lof}{table}{4.2\quad Experimental set-up.}
\phantomsection\addcontentsline{lof}{table}{4.3\quad Simulation 3D models transparency view.}
\phantomsection\addcontentsline{lof}{table}{4.4\quad Experimental results.}
\phantomsection\addcontentsline{lof}{table}{4.5\quad Simulation results.}
\phantomsection\addcontentsline{lof}{table}{4.6\quad Experimental quantitative results of panel surface.}
\phantomsection\addcontentsline{lof}{table}{4.7\quad Computational quantitative results of panel surface.}
\phantomsection\addcontentsline{lof}{table}{4.8\quad Temperature contrast profiles of simulation models.}
\phantomsection\addcontentsline{lof}{table}{4.9\quad Temperature contrast profiles of Lamp heating models.}


\includepdf[pages={2-11}]{Lei2017Detection}

\chapter{Liquid Nitrogen Cooling in IR Thermography applied to steel specimen}
(Accepted for publication in Proceedings Volume 10214, Thermosense: Thermal Infrared Applications XXXIX; 102140T (2017)).


%
\section*{Résumé}
Cette recherche étudie une stimulation externe - refroidissement au lieu de chauffer en thermographie infrarouge pour NDT \& E. Un spécimen en acier est utilisé afin de  tester trois stimulations différentes sur les images thermiques et également une comparaison d'analyse ROC. Les résultats montrent que toutes les techniques mettent en évidence une partie des défauts de l'échantillon, alors que la technique LN $ _2 $ ne représente les défauts qu'au début; ceci peut être dû à la conductivité élevée de l'acier. Dans les résultats thermiques, la méthode de post-traitement PCT affiche de meilleurs résultats pour toutes les procédures. Plus de défauts sont exposés dans la stimulation Flash avec le traitement PCT.
% Les résultats de cette étude ont d'abord été présentés lors d'une session orale de SPIE Thermosense: Thermal Infrared Applications XXXIX 2017, aux Etats-Unis.

\textbf{\texttt{Contributing authors:}}

\textbf{\textsf{Lei Lei}} (Ph.D candidate): developing protocol, experiment preparation and planning, data analysis,  personnel coordination and manuscript preparation.

\textbf{Giovanni Ferrarini} (Researcher of CNR-ITC): discussion in developing protocol, experiment preparation.

\textbf{Alessandro Bortolin} (Ph.D student of CNR-ITC): discussion and experiment preparation.

\textbf{Gianluca Cadelano} (Ph.D student of CNR-ITC): data collection, discussion and experiment preparation.

\textbf{Paolo Bison} (Research supervisor of CNR-ITC): student supervision, revision and correction of the manuscript. 

\textbf{Xavier Maldague} (Research director of LVSN in University Laval): student supervision, revision and correction of the manuscript.

\phantomsection\addcontentsline{lof}{table}{5.1\quad Experimental set-up in the reflection mode}
\phantomsection\addcontentsline{lof}{table}{5.2\quad  Steel sample dimension details with Flat-Bottom Holes of different depths and sizes.}
\phantomsection\addcontentsline{lof}{table}{5.3\quad Experimental set-up for LN$_2$ cooling}
\phantomsection\addcontentsline{lof}{table}{5.4\quad One example of ROC analysis (LN$_2$ results) and binary map of defect locations}
\phantomsection\addcontentsline{lof}{table}{5.5\quad Thermal Raw Images of PT and LN$_2$ stimulation techniques}
\phantomsection\addcontentsline{lof}{table}{5.6\quad FFT in amplitude and Phase results for LIT}
\phantomsection\addcontentsline{lof}{table}{5.7\quad PCT results of corresponding technique}
\phantomsection\addcontentsline{lof}{table}{5.8\quad ROC curves obtained from above results}

\includepdf[pages=-]{Thermosense2017_Lei}
