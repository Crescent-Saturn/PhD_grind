%!TEX root = gabarit-doctorat.tex
\part{Exploration of cold approach}     % numéroté
%\phantomsection\addcontentsline{toc}{part}{Exploration of cold approach} % inclure dans TdM
The following two chapters will present two published paper concerning the exploration of cold approach in infrared thermography applied in Non-Destructive Testing \& Evaluation.

The first study concentrates on the detection of defects and thermal bridges in insulated truck box panels, by active infrared thermography. Comparison between heating and cooling approaches for experiments and models has been established. In addition, passive thermography detection in computational models has been presented. Results demonstrate that the compressed air spray is more rapid than the traditional heating method in providing successful detection. Even if the traditional heating approach provides clearer results, in reality it is not easy and practical to heat a whole truck box to conduct inspection: the compressed air spray approach is much more convenient.

The second research investigates an external stimulation–cooling instead of heating in IR Thermography for NDT \& E. A steel specimen is used to test three different stimulations for thermal images and also ROC analysis comparison. Results shows that all techniques highlight part of the flaws in the sample, whereas the LN$_2$ technique represents the defects only at the beginning; this maybe due to the high conductivity of steel. In thermal results, the PCT post-processing method displays a better results for all procedures. More defects are exhibited in Flash stimulation with PCT processing.  The results of this study were firstly presented at an oral session of SPIE Thermosense: Thermal Infrared Applications XXXIX , in 2017 in the United States.

\chapter{Detection of insulation flaws and thermal bridges in insulated truck box panels}
(Published on-line in the Quantitative InfraRed Thermography Journal, in May 2017. Cited 3 times up to now).

%
The results of this study were firstly presented at an oral session of the 13th Quantitative InfraRed Thermography Conference 2016 at Gdańsk University of Technology in Poland . Then it was selected for the publication in Quantitative InfraRed Thermography Journal.

\section{Résumé}
Cet article se concentre sur la détection des défauts et des ponts thermiques dans les panneaux de caisses de camions isolés, en utilisant la thermographie infrarouge. Contrairement à la méthode traditionnelle de thermographie passive, cette recherche utilise des méthodes de chauffage et de refroidissement dans des configurations de thermographie active. Le chauffage de la lampe est utilisé comme stimulation externe chaude, tandis qu'un jet d'air comprimé est appliqué comme stimulation externe froide. Une caméra thermique capture tout le processus. En outre, des simulations numériques sous la plate-forme COMSOL$^®$ sont également menées. Les résultats expérimentaux et de simulation pour deux situations sont comparés et discutés.

\section{Abstract}
This paper focuses on the detection of defects and thermal bridges in insulated truck box panels, utilizing infrared thermography. Unlike the traditional way in which passive thermography is applied, this research uses both heating and cooling methods in active thermography configurations. Lamp heating is used as the hot external stimulation, while a compressed air jet is applied as the cold external stimulation. A thermal camera captures the whole process. In addition, numerical simulations under COMSOL$^®$ platform are also conducted. Experimental and simulation results for two situations are compared and discussed.

\textbf{\texttt{Contributing authors:}}

\textbf{\textsf{Lei Lei}} (Ph.D candidate): developing protocol, experiment preparation and planning, data collection, personnel coordination and manuscript preparation.

\textbf{Alessandro Bortolin} (Ph.D student of CNR-ITC): data analysis, discussion and manuscript preparation.

\textbf{Paolo Bison} (Research supervisor of CNR-ITC): student supervision, revision and correction of the manuscript. 

\textbf{Xavier Maldague} (Research director of LVSN in University Laval): student supervision, revision and correction of the manuscript. 

\phantomsection\addcontentsline{lot}{table}{4.1\quad Specimen specification details}
\phantomsection\addcontentsline{lot}{table}{4.2\quad Air-cooling parameters}
\phantomsection\addcontentsline{lot}{table}{4.3\quad Materials properties}
\phantomsection\addcontentsline{lot}{table}{4.4\quad Thermal contrast peak talbe}

\phantomsection\addcontentsline{lof}{table}{4.1\quad Details of defects inside the specimen(left) and final specimen to test(right).}
\phantomsection\addcontentsline{lof}{table}{4.2\quad Experimental set-up.}
\phantomsection\addcontentsline{lof}{table}{4.3\quad Simulation 3D models transparency view.}
\phantomsection\addcontentsline{lof}{table}{4.4\quad Experimental results.}
\phantomsection\addcontentsline{lof}{table}{4.5\quad Simulation results.}
\phantomsection\addcontentsline{lof}{table}{4.6\quad Experimental quantitative results of panel surface.}
\phantomsection\addcontentsline{lof}{table}{4.7\quad Computational quantitative results of panel surface.}
\phantomsection\addcontentsline{lof}{table}{4.8\quad Temperature contrast profiles of simulation models.}
\phantomsection\addcontentsline{lof}{table}{4.9\quad Temperature contrast profiles of Lamp heating models.}


\includepdf[pages={2-11}]{Lei2017Detection}

