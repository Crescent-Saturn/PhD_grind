\part{Exploration of cold approach}     % numéroté
%\phantomsection\addcontentsline{toc}{part}{Exploration of cold approach} % inclure dans TdM
The following two chapters will present two published paper concerning the exploration of cold approach in infrared thermography applied in Non-Destructive Testing \& Evaluation.

The first study concentrates on the detection of defects and thermal bridges in insulated truck box panels, by active infrared thermography. Comparison between heating and cooling approaches for experiments and models has been established. In addition, passive thermography detection in computational models has been presented. Results demonstrate that the compressed air spray is more rapid than the traditional heating method in providing successful detection. Even if the traditional heating approach provides clearer results, in reality it is not easy and practical to heat a whole truck box to conduct inspection: the compressed air spray approach is much more convenient.

The second research investigates an external stimulation–cooling instead of heating in IR Thermography for NDT \& E. A steel specimen is used to test three different stimulations for thermal images and also ROC analysis comparison. Results shows that all techniques highlight part of the flaws in the sample, whereas the LN$_2$ technique represents the defects only at the beginning; this maybe due to the high conductivity of steel. In thermal results, the PCT post-processing method displays a better results for all procedures. More defects are exhibited in Flash stimulation with PCT processing.  The results of this study were firstly presented at an oral session of SPIE Thermosense: Thermal Infrared Applications XXXIX , in 2017 in the United States.

\chapter{Detection of insulation flaws and thermal bridges in insulated truck box panels}
(Published on-line in the Quantitative InfraRed Thermography Journal, in May 2017. Cited 3 times up to now).

%
The results of this study were firstly presented at an oral session of the 13th Quantitative InfraRed Thermography Conference 2016 at Gdańsk University of Technology in Poland . Then it was selected for the publication in Quantitative InfraRed Thermography Journal.

\section*{Résumé}
Cet article se concentre sur la détection des défauts et des ponts thermiques dans les panneaux de caisses de camions isolés, en utilisant la thermographie infrarouge. Contrairement à la méthode traditionnelle de thermographie passive, cette recherche utilise des méthodes de chauffage et de refroidissement dans des configurations de thermographie active. Le chauffage de la lampe est utilisé comme stimulation externe chaude, tandis qu'un jet d'air comprimé est appliqué comme stimulation externe froide. Une caméra thermique capture tout le processus. En outre, des simulations numériques sous la plate-forme COMSOL$^®$ sont également menées. Les résultats expérimentaux et de simulation pour deux situations sont comparés et discutés.

\section*{Abstract}
This paper focuses on the detection of defects and thermal bridges in insulated truck box panels, utilizing infrared thermography. Unlike the traditional way in which passive thermography is applied, this research uses both heating and cooling methods in active thermography configurations. Lamp heating is used as the hot external stimulation, while a compressed air jet is applied as the cold external stimulation. A thermal camera captures the whole process. In addition, numerical simulations under COMSOL$^®$ platform are also conducted. Experimental and simulation results for two situations are compared and discussed.

\textbf{\texttt{Contributing authors:}}

\textbf{\textsf{Lei Lei}} (Ph.D candidate): developing protocol, experiment preparation and planning, data collection, personnel coordination and manuscript preparation.

\textbf{Alessandro Bortolin} (Ph.D student of CNR-ITC): data analysis, discussion and manuscript preparation.

\textbf{Paolo Bison} (Research supervisor of CNR-ITC): student supervision, revision and correction of the manuscript. 

\textbf{Xavier Maldague} (Research director of LVSN in University Laval): student supervision, revision and correction of the manuscript. 

\phantomsection\addcontentsline{lot}{table}{4.1\quad Specimen specification details}
\phantomsection\addcontentsline{lot}{table}{4.2\quad Air-cooling parameters}
\phantomsection\addcontentsline{lot}{table}{4.3\quad Materials properties}
\phantomsection\addcontentsline{lot}{table}{4.4\quad Thermal contrast peak talbe}

\phantomsection\addcontentsline{lof}{table}{4.1\quad Details of defects inside the specimen(left) and final specimen to test(right).}
\phantomsection\addcontentsline{lof}{table}{4.2\quad Experimental set-up.}
\phantomsection\addcontentsline{lof}{table}{4.3\quad Simulation 3D models transparency view.}
\phantomsection\addcontentsline{lof}{table}{4.4\quad Experimental results.}
\phantomsection\addcontentsline{lof}{table}{4.5\quad Simulation results.}
\phantomsection\addcontentsline{lof}{table}{4.6\quad Experimental quantitative results of panel surface.}
\phantomsection\addcontentsline{lof}{table}{4.7\quad Computational quantitative results of panel surface.}
\phantomsection\addcontentsline{lof}{table}{4.8\quad Temperature contrast profiles of simulation models.}
\phantomsection\addcontentsline{lof}{table}{4.9\quad Temperature contrast profiles of Lamp heating models.}


\includepdf[pages={2-11}]{Lei2017Detection}

\chapter{Liquid Nitrogen Cooling in IR Thermography applied to steel specimen}
The results of this study were firstly presented at an oral session of SPIE Commercial+ Scientific Sensing and Imaging. International Society for Optics and Photonics, 2017.
Then it was accepted for publication in Proceedings Volume 10214, Thermosense: Thermal Infrared Applications XXXIX; 102140T (2017).


%
\section*{Résumé}
Thermographie pulsée (PT) est l'une des méthodes les plus courantes dans les procédures de thermographie active de la thermographie pour NDT \& E ​​(essais non destructifs  \& évaluation), en raison de la rapidité et la commodité de cette technique d'inspection. Des éclairs ou des lampes sont souvent utilisés pour chauffer les échantillons dans le PT traditionnel. Cet article explore principalement exactement la stimulation externe opposée en IR Thermographie: refroidissement au lieu de chauffage. Un échantillon d'acier avec des trous à fond plat de différentes profondeurs et tailles a été testé. De l'azote liquide (LN $_2$) est répandu sur la surface de l'échantillon et l'ensemble du processus est capturé par une caméra thermique. Pour obtenir une bonne comparaison, deux autres techniques classiques de CND, la thermographie pulsée et la thermographie verrouillée, sont également utilisées. En particulier, la méthode Lock-in est implémentée avec trois fréquences différentes. Dans la procédure de traitement d'image, la méthode de thermographie en composantes principales (PCT) a été effectuée sur toutes les images thermiques. Pour les résultats Lock-in, les images de phase et d'amplitude sont générées par la transformée de Fourier rapide (FFT). Les résultats montrent que toutes les techniques présentaient une partie des défauts tandis que la technique LN $_2$ affiche les défauts seulement au début du test. De plus, un poste-traitement de seuil binaire est appliqué aux images thermiques, et en comparant ces images à une carte binaire de l'emplacement des défauts, les courbes caractéristiques de fonctionnement du récepteur (ROC) correspondantes sont établies et discutées. Une comparaison des résultats indique que la meilleure courbe ROC est obtenue en utilisant la technique flash avec la méthode de traitement PCT.

%Cette recherche étudie une stimulation externe - refroidissement au lieu de chauffer en thermographie infrarouge pour NDT \& E. Un spécimen en acier est utilisé afin de  tester trois stimulations différentes sur les images thermiques et également une comparaison d'analyse ROC. Les résultats montrent que toutes les techniques mettent en évidence une partie des défauts de l'échantillon, alors que la technique LN $ _2 $ ne représente les défauts qu'au début; ceci peut être dû à la conductivité élevée de l'acier. Dans les résultats thermiques, la méthode de post-traitement PCT affiche de meilleurs résultats pour toutes les procédures. Plus de défauts sont exposés dans la stimulation Flash avec le traitement PCT.
% Les résultats de cette étude ont d'abord été présentés lors d'une session orale de SPIE Thermosense: Thermal Infrared Applications XXXIX 2017, aux Etats-Unis.

\section*{Abstract}
Pulsed Thermography (PT) is one of the most common methods in Active Thermography procedures of the Thermography for NDT \& E (Nondestructive Testing \& Evaluation), due to the rapidity and convenience of this inspection technique. Flashes or lamps are often used to heat the samples in the traditional PT. This paper mainly explores exactly the opposite external stimulation in IR Thermography: cooling instead of heating. A steel sample with flat-bottom holes of different depths and sizes has been tested. Liquid nitrogen (LN$_2$) is sprinkled on the surface of the specimen and the whole process is captured by a thermal camera. To obtain a good comparison, two other classic NDT techniques, Pulsed Thermography and Lock-In Thermography, are also employed. In particular, the  Lock-in  method  is  implemented  with  three  different  frequencies.  In  the  image  processing  procedure,  the Principal Component Thermography (PCT) method has been performed on all thermal images. For Lock-In results, both Phase and Amplitude images are generated by Fast Fourier Transform (FFT). Results show that all techniques presented part of the defects while the LN$_2$ technique displays the flaws only at the beginning of the test. Moreover, a binary threshold post-processing is applied to the thermal images, and by comparing these images to a binary map of the location of the defects, the corresponding Receiver Operating Characteristic (ROC) curves are established and discussed. A comparison of the results indicates that the better ROC curve is obtained using the Flash technique with PCT processing method.  

\newpage
\textbf{\texttt{Contributing authors:}}

\textbf{\textsf{Lei Lei}} (Ph.D candidate): developing protocol, experiment preparation and planning, data analysis,  personnel coordination and manuscript preparation.

\textbf{Giovanni Ferrarini} (Researcher of CNR-ITC): discussion in developing protocol, experiment preparation.

\textbf{Alessandro Bortolin} (Ph.D student of CNR-ITC): discussion and experiment preparation.

\textbf{Gianluca Cadelano} (Ph.D student of CNR-ITC): data collection, discussion and experiment preparation.

\textbf{Paolo Bison} (Research supervisor of CNR-ITC): student supervision, revision and correction of the manuscript. 

\textbf{Xavier Maldague} (Research director of LVSN in University Laval): student supervision, revision and correction of the manuscript.

\phantomsection\addcontentsline{lof}{table}{5.1\quad Experimental set-up in the reflection mode}
\phantomsection\addcontentsline{lof}{table}{5.2\quad  Steel sample dimension details with Flat-Bottom Holes of different depths and sizes.}
\phantomsection\addcontentsline{lof}{table}{5.3\quad Experimental set-up for LN$_2$ cooling}
\phantomsection\addcontentsline{lof}{table}{5.4\quad One example of ROC analysis (LN$_2$ results) and binary map of defect locations}
\phantomsection\addcontentsline{lof}{table}{5.5\quad Thermal Raw Images of PT and LN$_2$ stimulation techniques}
\phantomsection\addcontentsline{lof}{table}{5.6\quad FFT in amplitude and Phase results for LIT}
\phantomsection\addcontentsline{lof}{table}{5.7\quad PCT results of corresponding technique}
\phantomsection\addcontentsline{lof}{table}{5.8\quad ROC curves obtained from above results}

\includepdf[pages=-]{Thermosense2017_Lei}
