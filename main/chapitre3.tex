\chapter{Panoramic view of the heat flux inside the vehicle}     % numéroté
(Published on-line  in the Quantitative InfraRed Thermography Journal, in February 2018).
The results of this study were presented at an oral session of the 1st QIRT Asia Conference 2015 Mamallapuram, India. % Then it was selected for the publication in Quantitative InfraRed Thermography Journal.
\section*{Résumé}
Une caméra IR est montée sur une tête panoramique et automatiquement entraînée pour cartographier la température des parois internes du conteneur isolé sur un véhicule réfrigéré. Le véhicule est introduit dans une chambre d'essai où des conditions spéciales sont appliquées, de manière à maintenir une différence de température interne-externe d'environ 25 °C. L'objectif de ce travail est de compléter les résultats qualitatifs obtenus par thermographie infrarouge, avec une évaluation du flux de chaleur traversant la paroi.

\section*{Abstract}
An IR camera is mounted on a pan-tilt head and automatically driven to map the temperature of the inner walls of the insulated container on a refrigerated vehicle. The vehicle is introduced in a test chamber where special conditions are applied, in such a way to maintain an inner-outer temperature difference of around 25 °C. The objective of this work is that of complementing the qualitative results obtained by infrared thermography, with an evaluation of the heat flux flowing through the wall.

\textbf{\texttt{Contributing authors:}}

\textbf{\textsf{Lei Lei}} (Ph.D candidate): developing protocol, experiment preparation, data analysis,  personnel coordination and manuscript preparation.

\textbf{Alessandro Bortolin} (Ph.D student of CNR-ITC): discussion and experiment preparation.

\textbf{Gianluca Cadelano} (Ph.D student of CNR-ITC): data collection, data analysis, discussion and experiment preparation.

\textbf{Giovanni Ferrarini} (Researcher of CNR-ITC): discussion in developing protocol, data collection, experiment preparation.

\textbf{Stefano Rossi} (Ph.D, researcher of CNR-ITC): experiment planning and preparation.

\textbf{Paolo Bison} (Research supervisor of CNR-ITC): student supervision, revision and correction of the manuscript. 

\textbf{Xavier Maldague} (Research director of LVSN in University Laval): student supervision, revision and correction of the manuscript.


\phantomsection\addcontentsline{lot}{table}{3.1\quad Insulated container dimensions}
\phantomsection\addcontentsline{lot}{table}{3.2\quad ATP test results}

\phantomsection\addcontentsline{lof}{table}{3.1\quad Overall thermal behavior of the container being tested represented as an analogy of an electrical circuit.}
\phantomsection\addcontentsline{lof}{table}{3.2\quad The inside of the insulated container used for the test.}
\phantomsection\addcontentsline{lof}{table}{3.3\quad Spherical projection.}
\phantomsection\addcontentsline{lof}{table}{3.4\quad Original IR image (left) and its spherical projection (right).}
\phantomsection\addcontentsline{lof}{table}{3.5\quad Harris corners detected}
\phantomsection\addcontentsline{lof}{table}{3.6\quad Estimation of translation between images.}
\phantomsection\addcontentsline{lof}{table}{3.7\quad Stitching of images.}
\phantomsection\addcontentsline{lof}{table}{3.8\quad Temperature panorama of the inside of the vehicle.}
\phantomsection\addcontentsline{lof}{table}{3.9\quad Heat Flux meter measurement.}
\phantomsection\addcontentsline{lof}{table}{3.10\enspace Heat Flux panorama of the inner part of the vehicle.}
\phantomsection\addcontentsline{lof}{table}{3.11\enspace Schema of calculating real size of each image (Left: Side view; Right: Plan view).}
\phantomsection\addcontentsline{lof}{table}{3.12\enspace One element surface map (up) and the area panorama of the inner part of the vehicle.}

\includepdf[pages={2-14}]{Lei2018Panoramic}

