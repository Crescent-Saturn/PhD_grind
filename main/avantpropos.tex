\chapter*{Preface}         % ne pas numéroter
\phantomsection\addcontentsline{toc}{chapter}{Preface} % inclure dans TdM

This thesis is submitted to the ``Faculté des études supérieures de l'Université Laval" to obtain the degree of Philosophiae Doctor of Science (Ph.D.). The current thesis is composed of six chapters. The first chapter is a literature review of maintaining the ``cold chain" in industry field, as well as cold approaches that are used nowadays in infrared thermography for Non-Destructive Testing \& Evaluation. This chapter is ended by the issue, hypothesis and objectives which are presented and highlighted. The chapter two, three are presented in form of papers and describe the application of infrared thermogaphy in the ``cold chain" and corresponding discussion. The chapter four and five are presented in the form of manuscripts and describe the exploration of cold approaches in infrared thermography and several experimental results and discussions. Finally, in order to provide a closure on the results obtained in the current thesis, the chapter six presents a general conclusion, implications and proposed perspectives of the performed studies.

The first manuscript is entitled ``Mapping of the heat flux of an insulated small container by infrared thermography" and was presented in the 24th IIR International Congress of Refrigeration. Authors: Paolo Bison, Alessandro Bortolin, Gianluca Cadelano, Giovanni Ferrarini, Lei Lei, Xavier Maldague, Stefano Rossi

The second paper is entitled ``Panoramic View of the Heat Flux Inside an Insulated Vehicle by Infrared Thermography" and was firstly presented at an oral session of the 1st QIRT Asia Conference 2015 Mamallapuram, India. Then it was selected for the publication in Quantitative InfraRed Thermography Journal. Authors: Lei Lei, Alessandro Bortolin, Gianluca Cadelano, Giovanni Ferrarini, Stefano Rossi, Paolo Bison, Xavier Maldague. 

Then, the third article is entitled ``Detection of insulation flaws and thermal bridges in insulated truck box panels" and was firstly presented at an oral session of the 13th Quantitative InfraRed Thermography Conference 2016 at Gdańsk University of Technology in Poland . Then it was invited for the publication in Quantitative InfraRed Thermography Journal. Authors: Lei Lei, Alessandro Bortolin, Paolo Bison, Xavier Maldague.

The last manuscript is entitled ``Liquid nitrogen cooling in IR thermography applied to steel specimen" and it was accepted for publication in Proceedings Volume 10214, Thermosense: Thermal Infrared Applications XXXIX; 102140T (2017). Authors: Lei Lei, Giovanni Ferrarini, Alessandro Bortolin, Gianluca Cadelano, Paolo Bison, Xavier Maldague.

The main objectives of the present thesis ware firstly to deploy the infrared thermography technique in the procedure of maintain the ``cold chain'', especially in the insulated vehicle of ATP standard. The benefits of time-saving and the accuracy of results in determination of K-value could be applied to assess at commercial level. Then based on the previous favorable results, another objective is to explore cold approaches (such as compressed air, liquid nitrogen, etc.) in infrared thermography for Non-Destructive Testing \& Evaluation. The first part of research was a cooperative project supported the governments of Italy and Quebec (Ministère des Relations internationales et de la Francophonie)through the Joint Subcommittee Québec-Italy. Our collaborator the Construction Technologies Institute of the Italian National Research Council (ITC-CNR), had made a huge favor for this research. Three main experiments were performed at ITC-CNR, and Dr. Paolo Bison oversaw all steps of the work there.

For exploration of cold approaches, research supervisor, prof. Xavier Maldague collaborated in the set up and work planning of expriemnts, equipments and participated in the interpretation of the results and manuscripts revision.


%Finally, the candidate has been invited to present the results of this thesis in a good management practices guide for pigs from farm to slaughter. This guide will be developed for use by the pork meat industry and the Canadian hog producers.
