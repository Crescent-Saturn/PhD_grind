\chapter{Liquid Nitrogen Cooling in IR Thermography applied to steel specimen}
The results of this study were firstly presented at an oral session of SPIE Commercial+ Scientific Sensing and Imaging. International Society for Optics and Photonics, 2017.
Then it was accepted for publication in Proceedings Volume 10214, Thermosense: Thermal Infrared Applications XXXIX; 102140T (2017).


%
\section*{Résumé}
La thermographie pulsée (PT) est l'une des méthodes les plus courantes dans les procédures de thermographie active de la thermographie pour NDT \& E ​​(essais non destructifs  \& évaluation), en raison de la rapidité et la commodité de cette technique d'inspection. Des éclairs ou des lampes sont souvent utilisés pour chauffer les échantillons dans la PT traditionnelle. Cet article explore principalement exactement la stimulation externe opposée en IR Thermographie: refroidissement au lieu de chauffage. Un échantillon d'acier avec des trous à fond plat de différentes profondeurs et tailles a été testé. De l'azote liquide (LN $_2$) est répandu sur la surface de l'échantillon et l'ensemble du processus est capturé par une caméra thermique. Pour obtenir une bonne comparaison, deux autres techniques classiques de CND, la thermographie pulsée et la thermographie verrouillée, sont également utilisées. En particulier, la méthode Lock-in est implémentée avec trois fréquences différentes. Dans la procédure de traitement d'image, la méthode de thermographie en composantes principales (PCT) a été effectuée sur toutes les images thermiques. Pour les résultats Lock-in, les images de phase et d'amplitude sont générées par la transformée de Fourier rapide (FFT). Les résultats montrent que toutes les techniques présentaient en partie les défauts tandis que la technique LN $_2$ affichait les défauts seulement au début du test. De plus, un poste-traitement de seuil binaire est appliqué aux images thermiques, et en comparant ces images à une carte binaire de l'emplacement des défauts, les courbes caractéristiques de fonctionnement du récepteur (ROC) correspondantes sont établies et discutées. Une comparaison des résultats indique que la meilleure courbe ROC est obtenue en utilisant la technique flash avec la méthode de traitement PCT.

%Cette recherche étudie une stimulation externe - refroidissement au lieu de chauffer en thermographie infrarouge pour NDT \& E. Un spécimen en acier est utilisé afin de  tester trois stimulations différentes sur les images thermiques et également une comparaison d'analyse ROC. Les résultats montrent que toutes les techniques mettent en évidence une partie des défauts de l'échantillon, alors que la technique LN $ _2 $ ne représente les défauts qu'au début; ceci peut être dû à la conductivité élevée de l'acier. Dans les résultats thermiques, la méthode de post-traitement PCT affiche de meilleurs résultats pour toutes les procédures. Plus de défauts sont exposés dans la stimulation Flash avec le traitement PCT.
% Les résultats de cette étude ont d'abord été présentés lors d'une session orale de SPIE Thermosense: Thermal Infrared Applications XXXIX 2017, aux Etats-Unis.

\section*{Abstract}
Pulsed Thermography (PT) is one of the most common methods in Active Thermography procedures of the Thermography for NDT \& E (Nondestructive Testing \& Evaluation), due to the rapidity and convenience of this inspection technique. Flashes or lamps are often used to heat the samples in the traditional PT. This paper mainly explores exactly the opposite external stimulation in IR Thermography: cooling instead of heating. A steel sample with flat-bottom holes of different depths and sizes has been tested. Liquid nitrogen (LN$_2$) is sprinkled on the surface of the specimen and the whole process is captured by a thermal camera. To obtain a good comparison, two other classic NDT techniques, Pulsed Thermography and Lock-In Thermography, are also employed. In particular, the  Lock-in  method  is  implemented  with  three  different  frequencies.  In  the  image  processing  procedure,  the Principal Component Thermography (PCT) method has been performed on all thermal images. For Lock-In results, both Phase and Amplitude images are generated by Fast Fourier Transform (FFT). Results show that all techniques presented part of the defects while the LN$_2$ technique displays the flaws only at the beginning of the test. Moreover, a binary threshold post-processing is applied to the thermal images, and by comparing these images to a binary map of the location of the defects, the corresponding Receiver Operating Characteristic (ROC) curves are established and discussed. A comparison of the results indicates that the better ROC curve is obtained using the Flash technique with PCT processing method.  

\newpage
\textbf{\texttt{Contributing authors:}}

\textbf{\textsf{Lei Lei}} (Ph.D candidate): developing protocol, experiment preparation and planning, data analysis,  personnel coordination and manuscript preparation.

\textbf{Giovanni Ferrarini} (Researcher of CNR-ITC): discussion in developing protocol, experiment preparation.

\textbf{Alessandro Bortolin} (Ph.D student of CNR-ITC): discussion and experiment preparation.

\textbf{Gianluca Cadelano} (Ph.D student of CNR-ITC): data collection, discussion and experiment preparation.

\textbf{Paolo Bison} (Research supervisor of CNR-ITC): student supervision, revision and correction of the manuscript. 

\textbf{Xavier Maldague} (Research director of LVSN in University Laval): student supervision, revision and correction of the manuscript.

\phantomsection\addcontentsline{lof}{table}{5.1\quad Experimental set-up in the reflection mode}
\phantomsection\addcontentsline{lof}{table}{5.2\quad  Steel sample dimension details with Flat-Bottom Holes of different depths and sizes.}
\phantomsection\addcontentsline{lof}{table}{5.3\quad Experimental set-up for LN$_2$ cooling}
\phantomsection\addcontentsline{lof}{table}{5.4\quad One example of ROC analysis (LN$_2$ results) and binary map of defect locations}
\phantomsection\addcontentsline{lof}{table}{5.5\quad Thermal Raw Images of PT and LN$_2$ stimulation techniques}
\phantomsection\addcontentsline{lof}{table}{5.6\quad FFT in amplitude and Phase results for LIT}
\phantomsection\addcontentsline{lof}{table}{5.7\quad PCT results of corresponding technique}
\phantomsection\addcontentsline{lof}{table}{5.8\quad ROC curves obtained from above results}

\includepdf[pages=-]{Thermosense2017_Lei}
