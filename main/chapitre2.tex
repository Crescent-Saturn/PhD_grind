\chapter*{Maintain the ``cold chain"}     % numéroté
\phantomsection\addcontentsline{toc}{chapter}{Maintain the ``cold chain"} % inclure dans TdM
The following two chapters will present two published paper concerning the application of infrared thermography for Non-Destructive Testing \& Evaluation applied in ``Maintain the cold chain" procedure.

The first study mainly focused on mapping the heat flux on the external surface of an insulated roll-container by infrared thermography technique. The ATP standard measurement was performed to obtain the experimental results, meanwhile IR images of roll-container have been taken when the steady condition arrived, in order to analyze and compute the corresponding heat flux on entire surfaces. A simple thermal resistance model has been applied to realize the computation. Final temperature figures showed a good uniform distribution, and several defects in the structure like thermal bridges or air leakages have been identified. A reference zone of the external wall is measured by a thermal flux meter, then with that reference the whole surface heat flux map have been figured out. Besides, for a better view of the heat flux map, the homography technique has been performed into the raw images by applying a bilinear interpolation with the projective transformation matrix. The final corrected heat flux map has been demonstrated for each surface, in which the right one showed a smaller value than the others. 

In the second research, a panoramic view of the heat flux on the internal surface of an insulated vehicle by infrared thermography technique has been established in this study. The ATP measurement was performed to obtain the experimental results.  An IR camera was mounted on a pan-tilt head and automatically driven by a suitable software to map the temperature of the inner walls at the steady condition, in order to analyze and compute the corresponding heat flux on the entire surface.  The final result of the K-value obtained by IR thermography is accurate enough and compares well with that of the ATP test.

\chapter{Mapping of the heat flux of an insulated small container by infrared thermography}
The results of this study were firstly presented at an oral session of the 24th IIR International Congress of Refrigeration 2015 Yokohama, Japan.

\section*{Résumé}
Cette étude portait principalement sur la cartographie du flux de chaleur sur la surface externe d'un rouleau-conteneur isolé par thermographie infrarouge.
La mesure standard ATP a été réalisée pour obtenir les résultats expérimentaux, tandis que des images IR de rouleau-conteneur ont été prises lorsque la condition stable est arrivée, afin d'analyser et de calculer le flux de chaleur correspondant sur des surfaces entières. Un modèle simple de résistance thermique a été appliqué pour réaliser le calcul. Les chiffres de température finaux ont montré une bonne distribution uniforme, et plusieurs défauts dans la structure comme des ponts thermiques ou des fuites d'air ont été identifiés. Une zone de référence de la paroi externe est mesurée par un fluxmètre thermique, puis avec cette référence, la carte de flux de chaleur à la surface totale a été déterminée. Par ailleurs, pour une meilleure vision de la carte de flux thermique, la technique d'homographie a été réalisée sur les images brutes en appliquant une interpolation bilinéaire à la matrice de transformation projective. La carte de flux de chaleur corrigée finale a été démontrée pour chaque surface, dans laquelle la droite a montré une valeur plus petite que les autres.

\textbf{\texttt{Contributing authors:}}

\textbf{Paolo Bison} (Research supervisor of CNR-ITC): developing protocol, student supervision, revision and correction of the manuscript. 

\textbf{Alessandro Bortolin} (Ph.D student of CNR-ITC): discussion and experiment preparation.

\textbf{Gianluca Cadelano} (Ph.D student of CNR-ITC): data collection, data analysis, discussion and experiment preparation.

\textbf{Giovanni Ferrarini} (Researcher of CNR-ITC): discussion in developing protocol, data collection, experiment preparation.

\textbf{\textsf{Lei Lei}} (Ph.D candidate):  experiment preparation, data analysis,  personnel coordination and manuscript preparation.

\textbf{Xavier Maldague} (Research director of LVSN in University Laval): student supervision, revision and correction of the manuscript.

\textbf{Stefano Rossi} (Ph.D, researcher of CNR-ITC): experiment planning and preparation.



\newpage
\section{Introduction}
Nowadays, the public is increasingly aware to the need of applying rigorous standards all along the ``food chain'', considering a better life with a good food supply.  Thus the transport of food in the refrigerated vehicles (such as trucks, trailers, containers, etc.), especially for dairy products, meat and frozen foods, is of great interest. Moreover, the ever increasing cost of energy incites limiting the minimum refrigeration. So it is essential to ensure a perfect thermal insulation at the vehicles inspection. 

There exits some agreements of the thermal insulation tests which ensures the suitability for the transport of food in refrigerated conditions---ATP.
``The Agreement on the International Carriage of Perishable Foodstuffs and on the Special Equipment to be Used for such Carriage (ATP) done at Geneva on 1 September 1970 entered into force on 21 November 1976'' \citep{Geneva1970}, which establishes standards for the international transport of perishable food between the states that ratify the treaty. It has been updated through amendment a number of times and as of 2013 has 48 state parties, most of which are in Europe or Central Asia. It is open to ratification by states that are members of the United Nations Economic Commission for Europe (UNECE) and states that otherwise participate in UNECE activities\citep{ATP_wiki}.
The details contents of the agreements can be found in \citep{ATP_wiki, rossi2009k}, therefore no more description in this report will be presented.

The institute (CNR-ITC) has extensive experience in the measurement of heat transfer coefficient $K$ applied to refrigerated vehicles \citep{rossi2009k,bison1993automatic,bison2012geometrical,dragano2009experimental,grinzato2010r}. It is also responsible for the certification of these vehicles throughout Italy.

In summary, the major methods and procedures for measuring and checking the insulation of the equipment during the ATP test, is following \citep{rossi2009k}: 
\begin{itemize}
	\item \textbf{insulated equipment} built with insulated envelope such that the heat exchanged between inside and outside is limited in such a way that the overall coefficient of heat transfer ($K$-value) is assignable into 2 classes: a)equal to or less than 0.7 $W/(m^2 K)$ for normally insulated equipment; b) equal to or less than 0.4 $W/(m^2 K)$ for heavily insulated equipment; 
	
	\item\textbf{refrigerated equipment} that are insulated equipment which utilize some source of cold like ice, eutectic plates, dry ice etc. This equipment, with an outside temperature of $30^{\circ}$C must be able to lower the inside temperature to: + $7^{\circ}$C (class A); $-10^{\circ}$C (class B); $-20^{\circ}$C (class C); $0^{\circ}$C (class D); 
	
	\item \textbf{mechanically refrigerated equipment} that  are  insulated  equipment  furnished  of  its  own  refrigerating appliance.  The  appliance,  with  an  outside  temperature  of  $30^{\circ}$C,  must  be  capable  of  lowering  the  inside temperature to: from $+12^{\circ}$C to $0^{\circ}$C (class A); from $+ 12^{\circ}$C to $-10^{\circ}$C (class B); from $+ 12^{\circ}$C to $- 20^{\circ}$C (class C); 
	
	\item \textbf{heated  equipment} that  can  heat  the  inside  (to  avoid  the  freezing  of  foodstuffs)  are  used  in  very  cold countries.
\end{itemize}
Especially, the overall coefficient of heat transfer ($K$) is defined as:
\begin{equation}
K = \frac{W}{S\cdot \Delta \theta}
\end{equation}
where W is  the  power  necessary  to  maintain  a  steady  temperature  difference $\Delta \theta$ between  the  mean internal and external air temperature of the equipment. $S$ is the mean surface of the equipment, given by the geometric mean of the inside and outside surface areas:
\begin{equation}
S = \sqrt{S_i \cdot S_e}
\end{equation}

The ATP standard test is a procedure to measure the insulating status of equipments with a global approach. Its robustness has been well demonstrated. While, on the other hand, some local defects in the structure of equipment, such as thermal bridges, air leakages or zones of anomalous aging, cannot be visualized in this procedure. Then the thermography technique could be particularly helpful to these issues. In fact, all the defects mentioned above lead to a variation of the heat flux and temperature on the surface of the equipment \citep{grinzatoquality,grinzato1comparison}. Therefore the local heat flux map of the equipment by infrared thermography could give a straightforward visualization of the structure, and also a local evaluation of the K-value.

The work in this study consists the modeling and simulation for an insulated roll container by COMSOL in theory part. In practice party, the experimentation will be performed to map the heat flux on its external surface by Infrared thermography. In addition, another test for a refrigerated truck will also be undertaken. Primary results will be analyzed and discussed. Some conclusions and perspectives come in the end.


\section{Theory \& Methods}
\subsection{The heat transfer model}
Analogy to an electrical circuit, the heat transfer can be modelled as the heat flux is represented by current, temperatures are represented by voltages \citep{Therm_Re}. Therefore, the resistors in the heat ``circuit" is then the thermal resistance. Symbolically Ohm’s law can be expressed as
\begin{equation}
I = \frac{\Delta V}{R_e}
\end{equation}
where $I$ is the current flowing through an element, $\Delta V$ is the voltage across the element, and $R_e$ is the electrical resistance across the element. With the observed analogy, Fourier’s law can be written similarly as
\begin{equation}
q = \frac{\Delta T}{R_t}
\end{equation}
where $q$ is the flux of heat conduction, $\Delta T$ is the temperature difference between the surfaces of a slab, and $R_t$ is the thermal resistance.

In this work, the thermal resistance model is applied to a roll-container, from inside to outside (More details about the container can be found in Chapter \ref{box_detail}). For the standard ATP requirement, a radiation heater is working inside the container to maintain a fixed a higher internal air temperature. After the steady conditions are reached, a heat power $W$ is delivered. Heat flow is transferred by convection from the hot inside air to the internal wall of the box, and then by conduction through internal wall to external wall, and again by convection from the external wall to the outside air, which is cooled by the ATP system. The total scheme is presented in Fig \ref{Therm_Res}
\begin{figure}[!htpb]
	\centering
	\includegraphics{mapping/Therm_Res}
	\caption{Overall thermal resistance of the roll-container}
	\label{Therm_Res}
\end{figure}

\noindent where $\theta_i$ and $\theta_e$ are the internal and external temperature of the container, respectively. $\theta_{wi}$ and $\theta_{we}$ are the internal and external wall temperature of the container. In essentially 1D hypothesis, $h_i$ and $h_e$ are respectively the convective heat exchange internal and external coefficients. $\lambda$ is the thermal conductivity of the container wall and $l$ its thickness, and $S$ is the mean surface of the box.


\subsection{Simulation by Comsol}

A simple simulation work for this model has been performed by COMSOL MultiPhysics, which helps to better understand the distribution of the temperature of final result. The measurements of box dimension can be found in Tab \ref{tab_box_dim}.

\begin{table}[h]
	\centering
	%\begin{tabular}{p{85pt}p{85pt}p{85pt}p{85pt}}
	\begin{tabular}{c|c|c|c||c}
		\hline
		Inside:  & $L_i=0.864$ m  & $D_i=0.613$ m  & $H_i=1.55$ m & Thickness\\
		\hline
		Outside:  & $L_o=0.988$ m & $D_o=0.74$ m & $H_o=1.67$ m  &  $50$ mm \\
		\hline
	\end{tabular}
	\caption{Roll container dimensions}
	\label{tab_box_dim}
\end{table}

Same as the ATP standard test, the box inside air temperature is set as $32.5^{\circ}$C, and the outside air temperature is kept as $7.2^{\circ}$C. 

The Heat Transfer in Solid was used during the simulation in this modelling. A heat flux is added inside the box between the air and the internal surface, with a free convection, then the coefficient is set as $h_i=15 W/(m^2 K)$. Another heat flux was added outside between the air and the external surface with a forced convection, by $h_e=25 W/(m^2 K)$ [referring \citep{airhe,htwiki}]. The material of the box is made by high density polyurethane foam, with a conductivity about 0.0026 $W/(mK)$ \citep{jarfelt2006thermal}. Two high conductive thin layer are added into the material, same as the experimental one. The study condition is set to the steady status, as in real test the whole period is more than 12 hours.

The simulation result is shown in Fig. \ref{3D_T}. 
\begin{figure}[!htbp]
	\centering
	\includegraphics[scale=0.65]{mapping/3D_T}
	\caption{Temperature distribution of the roll container in COMSOL}
	\label{3D_T}
\end{figure}

The distribution of temperature of the entire container presented above indicates an uniform result, since the model and simulation is in ideal condition. Neither air leakage no thermal bridge is found here.

The following two figures (\ref{cut_plane}) introduce two cut plane inside the box, which illustrate the temperature details in the layers. As one can see that, the heat transfers from inside to outside, with temperature decreasing from internal surface to external surface. Moreover, the high thickness in the corners lead to a less transferred heat, showing a lower corresponding temperature.
\begin{figure} [htbp]
	%	\centering
	\hspace{-20pt}
	\includegraphics[scale=0.40]{mapping/xy_plane}
	\includegraphics[scale=0.40]{mapping/2D_xy_plane}
	\vspace{5pt}
	\hspace{-20pt}
	\includegraphics[scale=0.40]{mapping/yz_plane}
	\includegraphics[scale=0.40]{mapping/2D_yz_plane}
	\caption{Temperature distribution of the cute planes}
	\label{cut_plane}
\end{figure}

\section{Experimental setup}
\subsection{Roll-container}
In this part the experimental installation will be presented. The aim of this work is to map the heat flux on the external surface of a roll-container, and Fig \ref{box} shows the installed probes on the surface of the roll container.

\begin{figure}[!htbp]
	\centering
	\includegraphics[scale=0.13]{mapping/DSC_0191}
	\includegraphics[scale=0.13]{mapping/DSC_0189}\\
	\vspace{4pt}
	\includegraphics[scale=0.13]{mapping/DC_51939}
	\includegraphics[scale=0.13]{mapping/DC_51941}
	\caption{The roll container used for the test[outside and inside(up-right)]}
	\label{box}
\end{figure}

This box is actually made by sandwich panels with three layers: two internal-external skins made by polyester-fiberglass and a core made by high density polyurethane foam, and total thickness is 50 $mm$. \label{box_detail}

12 points of measurements (thermal couple) are positioned at the 8 corners (inside and outside) and also at the center of 4 surfaces (front, left,back and right)[Fig \ref{therm_couple}]. All the thermal couples are at a distance of 10 cm from the wall.
\begin{figure}[!htbp]
	\centering
	\includegraphics[scale=0.40]{mapping/therm_couple}
	\caption{The position of the thermal couples}
	\label{therm_couple}
\end{figure}
In addition, on the upper-center of the front surface, a heat flux meter is set to measure the corresponding heat flux (Fig \ref{box} left). The acquisition system used in this work is an infrared camera--FLIR SC-660. It records the thermography image series during the whole test for the front and left surfaces of container (schema shown in Fig \ref{box} left). What's more, when the steady conditions were reached (a state period of time not less than 12 hours, according to the ATP standard \citep{rossi2009k}), several infrared images were captured for the external walls such as front, left, back, right and top one (impossible to capture the bottom surface). 

With standard ATP measurement during the test, a radiation heater is heating the inside the container to maintain a temperature about $32.5^{\circ}$C. For outside, with the air circulating a velocity between 1 and 2 $m s^{-1}$ in the tunnel, the temperature is maintained constant at about $7.2^{\circ}$C. Thus makes a temperature difference between inside air and outside of $\Delta \theta = 25.3^{\circ}$C.

Two hypothesis are proposed in this test: 1) the  heat exchange coefficients, inside ($h_i$) and outside ($h_e$) of the container are constant;\label{hyp1} 2) the heat diffusion is mainly 1D. These hypothesis serve the computation of heat flux for each external surface in the following chapter.

\subsection{Refrigerated Vehicle}

Even though the main work is testing on the roll container, one has also applied the same experimental setup to a refrigerated truck (Fig \ref{truck}), which refers to the main objective of the ATP agreements.
\begin{figure}[!htbp]
	\hspace{-10mm}
	\includegraphics[scale=0.12]{mapping/DSC_0210}
	\includegraphics[scale=0.12]{mapping/DSC_0211}\\
	
	\includegraphics[scale=0.12]{mapping/DSC_0209}
	\includegraphics[scale=0.12]{mapping/DSC_0213}
	\caption{The refrigerated vehicle used for the test}
	\label{truck}
\end{figure}

For this test, only thermography images of several surfaces (top, left, back and right) have been captured when the steady condition was reached. Besides, the heat flux meter was first set on the left surface (Fig \ref{truck} upper-left) and then was moved to test the back surface (Fig \ref{truck} lower-right). The former result contains the influence of air streaming whose velocity is between 1 and 2 $ms^{-1}$, while in the later one, the ventilation system had been switched off.

\section{Results \& Discussion}
\subsection{Heat flux map}
The idea is to measure the heat flux by a thermal flux meter in a reference zone outside of the insulated envelope and to build successively the whole map of the heat flux as a linear relation of the temperature difference between the outside wall and the air.

The specific heat flux at a reference zone of the external wall is measured by a thermal flux meter, then with the wall temperature in the proximity of the flux meter and by measuring the air temperature, one can draw the local approach like \citep{rossi2009k}:
\begin{equation}
q_r = \frac{\theta_{we}(x_r,y_r)-\theta_e}{1/h_e}
\end{equation}
which gives that:
\begin{equation}
h_e = \frac{q_r}{\theta_{we}(x_r,y_r)-\theta_e}
\end{equation}
where $q_r$ is the heat flux measured at the reference point ($x_r,y_r$) by the thermal flux meter. $\theta_{we}$ is the temperature measured by thermography at the reference point. With the hypothesis 1) mentioned in Section \ref{hyp1}, the heat flux map of the whole surface could be determined with the temperature map according to:
\begin{equation}
q(x,y) = \frac{q_r}{\theta_{we}(x_r,y_r)-\theta_e}(\theta_{we}(x,y)-\theta_e)
\label{eq_q}
\end{equation}
Where $\theta_{we}(x,y)$ is the temperature at each point of the external surface. This equation indicates that the  heat flux has a linear relation of the temperature difference between the outside wall and the air.

Finally the temperature of all the surfaces are distributed as Fig \ref{IR_box}:
\begin{figure}[!htbp]
	%	\hspace{-10mm}
	\centering
	\includegraphics[scale=0.50]{mapping/IR_front_m}
	\hspace{6pt}
	\includegraphics[scale=0.50]{mapping/IR_back_m}
	\vspace{3pt}
	\includegraphics[scale=0.50]{mapping/IR_left_m}
	\hspace{6pt}
	\includegraphics[scale=0.50]{mapping/IR_right_m}
	\includegraphics[scale=0.50]{mapping/IR_top}
	\caption{The temperature map of the roll container (front, back, left, right and top surfaces)}
	\label{IR_box}
\end{figure}
All these IR images were captured at steady condition. 
In the figure, a good uniform distribution of temperature was well shown, even though there are several abnormal lines which may be the thermal bridges or air leakages (at front, left, right and top surfaces). An attention might be paid is that there is an area of the light reflection on the container surface (on the second image left part). It could be no influence here as the back surface is the main part of the second figure.

Comparing all the surface temperature, one can see that on the right surface of container, the temperature value is a little lower than other surfaces. This result may conclude the Hypothesis 1) \ref{hyp1} is not fully correct, then the convective heat transfer coefficient could be different around the roll container during the test.

The temperature at the reference point around the thermal flux meter was presented in the figure (the first one), with that one could figure out the map of heat flux of the entire surface by Eq \ref{eq_q}. And the data measured by the thermal flux meter are presented in Fig \ref{flux_meter}.
\begin{figure}[!htbp]
%	\hspace{-35pt}
	\centering
	\includegraphics[scale=0.65]{mapping/It_project_2014_QProfile}
%	\vspace{20pt}
%	\hspace{-35pt}
	\includegraphics[scale=0.65]{mapping/It_project_2014_TProfile}
	\caption{Data from thermal flux meter (heat flux and temperature profiles)}
	\label{flux_meter}
\end{figure}
These data recorded the whole test. Thought the fluctuation is very high, ones took the steady period after 5 hours from the beginning. Then mean values are $q_r=9.73 W/(m^2 K)$ and $T_r = 7.83 ^\circ$C, which served the computation of heat flux.

Moreover, for a better view, some techniques are often applied for images corrections \citep{bison2012geometrical}. In this work, an easier way to realize the image correction is to apply the homography technique.

\subsection{Homography application}
In practical applications, such as image rectification, image registration, or computation of camera motion—rotation and translation—between two images, a homography is often performed in the field of computer vision \citep{homo_wiki}.

In this work, as one knows the radio between the length and the width of the roll container (or between the width and the height in same way), a projective transformation matrix in images has been obtained. Then applying a bilinear interpolation with the projective transformation matrix, into the raw images, one finally got the corrected heat flux mapping of each surfaces, demonstrated in Fig \ref{Q_box}.
\begin{figure}[!htbp]
	%	\hspace{-10mm}
	\centering
	\includegraphics[scale=0.55]{mapping/Q_front_m}
	\hspace{5pt}
	\includegraphics[scale=0.55]{mapping/Q_back_m}
	\includegraphics[scale=0.55]{mapping/Q_left_m}
	\hspace{5pt}
	\includegraphics[scale=0.55]{mapping/Q_right_m}
	\includegraphics[scale=0.55]{mapping/Q_top_m}
	\caption{The corrected heat flux map of the roll container (front,  back, left, right and top surfaces)}
	\label{Q_box}
\end{figure}

The Heat flux map exhibits that in right surface of the roll container, one gets smaller heat flux value rather than those on other surfaces, such as front, top, back and left.
On the other hand, the heat flux map of back, top and left surface show a little higher mean value than that of the front one, which is due to the air stream.

A contradiction is found that, normally the right surface's heat flux should be a bit greater than the front surface where the air stagnation is probable, as air stream flows the lateral surface, and that leads to a lower temperature in the right surface. While in the thermography images, one got the contrary result. The reason for this is due to the linear relationship between heat flux and the difference of temperature between external surface and the air. Since in our model Eq.\ref{eq_q}, lower $\theta_{we}(x,y)$ would lead to a lower $q(x,y)$. This might also indicate that the influence of the convective heat transfer in lateral surface is more important than imagining.
Comparing to the standard ATP measurement, the global K-value obtained is about $0.53 W/(m^2 K)$, then multiplied by the temperature difference $\Delta \theta =25.3^\circ$C, thus it gives us a global heat flux value of the roll container $13.41 W/m^2$. With thermography one obtained the local heat flux in the map about $10.7 W/m^2$, which is more or less a good result.

``A local variation of thermal conductivity leads to a local increase of the heat flux with a consequent variation of the local internal and external wall temperature"\citep{rossi2009k}, this phenomena can be well seen in all figures above.

\subsection{Vehicle results}
The temperature map of several surfaces of the refrigerated truck are distributed in Fig \ref{IR_truck}.
\begin{figure}[!htbp]
	%	\hspace{-10mm}
	\centering
	\includegraphics[scale=0.50]{mapping/IR_truck_lt_m}
	\hspace{6pt}
	\includegraphics[scale=0.50]{mapping/IR_truck_rt_m}
	\vspace{3pt}
	\includegraphics[scale=0.50]{mapping/IR_truck_bk11_m}
	\hspace{6pt}
	\includegraphics[scale=0.50]{mapping/IR_truck_bk12_m}
	\includegraphics[scale=0.50]{mapping/IR_truck_tp_m}
	\caption{The temperature map of the refrigerated vehicle (left, right, back1, back2 and top surfaces)}
	\label{IR_truck}
\end{figure}
From all the figures, one can see a good uniform of the temperature distribution on all the external surfaces of the vehicle. However, the reflection effects are very heavy on the left, right and top surfaces (which are not the abnormal lines in figures). Moreover, in this test, lower temperature are found on the lateral surfaces than the back one, thanks to the air streaming.

%Two tests of the thermal flux have been performed: one on the left surface with air streaming; another on the back surface without ventilation system. The corresponding two final data profiles can be found in Fig \ref{truck_meter}.
%\begin{figure}[!htbp]
%	\centering
%	\includegraphics[scale=0.5]{mapping/Q_truck_left_profil}
%	\includegraphics[scale=0.5]{mapping/T_truck_left_profil}
%	\includegraphics[scale=0.6]{mapping/Q_truck_back_profil}
%	\includegraphics[scale=0.6]{mapping/T_truck_back_profil}
%	\caption{Data from thermal flux meter for the refrigerated vehicle (up:left surface; down:back surface)}
%	\label{truck_meter}
%\end{figure}
%The mean values of all these data in steady status are in following table (Tab \ref{tab_truck}).
%\begin{table}[h]
%	\centering
%	%\begin{tabular}{p{85pt}p{85pt}p{85pt}p{85pt}}
%	\begin{tabular}{l|r}
%		\hline
%		left surface  & back surface  \\
%		\hline
%		$\overline{q}=10.34 W/(m^2 K)$ & $\overline{q}=5.62 W/(m^2 K)$  \\
%		\hline
%		$\overline{T}=7.84^\circ$C & $\overline{T}=8.97^\circ$C  \\
%		\hline
%	\end{tabular}
%	\caption{Thermal flux meter data for truck}
%	\label{tab_truck}
%\end{table}
%
%The huge difference (45\% error) between the heat flux of the back and left surface makes no sense, since two test were almost at the same condition. Therefore, that might because when switching the thermal flux meter from left surface to back surface, something wrong has been done to influence the accuracy of the equipment, like not attaching well to the surface. Another consideration is that, for the back surface test, the measuring time was not long, as it has already arrived in the steady condition. While in the profiles, a little increasing tendency in temperature could be observed. This may indicate again that the vehicle result of back surface was not so good.

As the main work of the roll container has been done, more tests on the truck will be undertaken in future measurements.


\section{Conclusion \& Perspectives}

%\addcontentsline{toc}{chapter}{Conclusion \& Perspectives}
This preliminary work mainly focused on mapping the heat flux on the external surface of an insulated roll-container using Infrared thermography technique. The ATP standard measurement was performed to obtain the experimental results, meanwhile IR images of the roll-container have been taken when the steady condition was reached, in order to analyze and compute the corresponding heat flux on the entire surface. A simple thermal resistance model has been applied to conduct the computation. Final temperature figures showed a good uniform distribution, and several defects in the structure such as thermal bridges or air leakages have been identified. A reference zone of the external wall is measured by a thermal flux meter, then with that reference the entire surface heat flux map have been figured out. Moreover, for a better view of the heat flux map, the homography technique has been performed into the raw images by applying a bilinear interpolation with the projective transformation matrix. The final corrected heat flux map has been demonstrated for each surface, in which the right surface showed a smaller value than the others. Due to the air streaming, temperatures in the lateral surfaces were a little smaller than other surfaces, thus this leads to the smaller heat flux values. That also indicated the convective heat transfer coefficient was not constant around the roll-container surfaces, contrary to our Hypothesis 1 in theory.

For the refrigerated vehicle test, two surfaces (left and back) have been taken into measurement with the thermal flux meter. The former one was tested with the air streaming, while the ventilation system was switched off for the later one. The final IR images presented a big reflection influence on the lateral surfaces. And a huge difference between the heat flux value on two test surfaces was found, which might be because the equipment was not attached well to the surface when moved. 

For future work, more tests will be taken place on refrigerated vehicle with two thermal flux meters measuring on the same time, to avoiding the problem encountered in this work. And the suppression of thermal reflections in thermal imaging \citep{vollmer2004identification} may be taken into consideration during the IR image processing. 

This project topic can be also extended to buildings where the monitoring of the effective transmittance is crucial for the energy saving \citep{grinzato2010r}.