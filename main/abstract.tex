\chapter*{Abstract}                      % ne pas numéroter
\phantomsection\addcontentsline{toc}{chapter}{Abstract} % inclure dans TdM

\begin{otherlanguage*}{english}
   The increasing cost of energy has made energy saving a vital necessity in the current world. One of the examples involves, ``Maintaining the cold chain", which is the correct transport of perishable foodstuffs in refrigerated vehicles, especially for dairy products, meat and frozen foods.  Then a suitable thermal insulation implemented in refrigerated vehicles is essential for saving energy while maintaining an appropriate conservation of the foodstuffs. ATP is one of the agreements concerning thermal insulation tests ensuing the suitability of the transport.
   
   The ATP standard test is a procedure to measure the insulating status of equipments with a global approach. Nonetheless, some local defects in the structure of equipment cannot be visualized in this procedure. Then the thermography technique could be particularly helpful to these issues. Two examples of this application are presented in this thesis, one focuses on mapping the heat flux on the external surface of an insulated roll-container by infrared thermography technique. The second one attempts to establish a panoramic view of the heat flux on the internal surface of an insulated vehicle. 
   
   Encouraged by previous favorable results, an exploration of cold approach in infrared thermography for Non-Destructive Testing \& Evaluation are introduced and performed in the following. One concentrates on the detection of insulated flaws and thermal bridges in insulated truck box panels by lamp heating and air cooling, two opposite means of excitation. The other one investigates a liquid nitrogen cooling applied to a steel specimen with flat-bottom holes of different depths and sizes.
   
   Different data processing methods and modeling and simulation are carried out in related chapters. 
\end{otherlanguage*}
