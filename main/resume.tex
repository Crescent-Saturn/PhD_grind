\chapter*{Résumé}                      % ne pas numéroter
\phantomsection\addcontentsline{toc}{chapter}{Résumé} % inclure dans TdM

\begin{otherlanguage*}{french}
  Le coût croissant de l'énergie a fait de l'économie d'énergie une nécessité vitale dans le monde actuel. Un des exemples consiste à ``maintenir la chaîne du froid", c'est-à-dire le transport correct des aliments périssables dans les véhicules réfrigérés, en particulier pour les produits laitiers, la viande et les aliments congelés. Tout en conservant une conservation appropriée des denrées alimentaires, l'ATP est l'un des accords concernant les essais d'isolation thermique qui déterminent l'adéquation du transport.
  
  Le test standard ATP est une procédure pour mesurer l'état isolant des équipements avec une approche globale. Néanmoins, certains défauts locaux dans la structure de l'équipement ne peuvent pas être visualisés dans cette procédure. Ensuite, la technique de thermographie pourrait être particulièrement utile à ces problèmes. Deux exemples de cette application sont présentés dans cette thèse, l'un d'eux se concentre sur la cartographie du flux de chaleur sur la surface externe d'un rouleau-conteneur isolé par la technique de thermographie infrarouge. La seconde tente d'établir une vue panoramique du flux de chaleur sur la surface interne d'un véhicule isolé.
  
  Encouragé par les résultats favorables précédents, une exploration de l'approche à froid dans la thermographie infrarouge pour les Tests Non-Destructifs et l'Évaluation est introduite et réalisée dans ce qui suit. L'un concentre sur la détection des défauts isolés et des ponts thermiques dans les panneaux de caisses de camions isolés par chauffage à lampe et refroidissement par air, deux moyens d'excitation opposés. L'autre examine un refroidissement à l'azote liquide appliqué à un échantillon d'acier avec des trous à fond plat de différentes profondeurs et tailles.
  
  Différentes méthodes de traitement des données et de modélisation et de simulation sont effectuées dans des chapitres connexes.
\end{otherlanguage*}
