\chapter{State of the art}     % numéroté
\section*{Maintain the ``cold chain"}
\phantomsection\addcontentsline{toc}{section}{Maintain the ``cold chain"}

\section{ATP background}
As one major procedure of the cold chain concept, the correct transport of perishable foodstuffs in refrigerated vehicles is a critical link in the food chain not only in terms of maintaining the temperature integrity of the transported products but also its impact on energy consumption and CO$_2$ emissions. Referring this, the  commercial food sector, including agriculture, food manufacture transport and retail are responsible for the total greenhouse gas emissions in whole world.

According to a report (\citet{watkiss2005validity}), food manufacture distribution estimated to be responsible for 1.8\% of the total emission. Due to a variety of operating conditions and constraints, transport refrigeration equipment operation needs more rigorous environment, such as suitable temperature controlled handling and storage facilities that can maintain food at appropriate temperatures and enable these temperatures to be monitored controlled and recorded. There are also specific temperature requirements for certain categories of food, one example is given in Fig.~\ref{tem_reqs}. %(\citet{tem_reqs}.
\begin{figure}[ht]
	\centering
	\includegraphics[scale=0.9]{art/temp_reqs}
	\label{tem_reqs}
	\caption{Transport temperature requirements of food products}
\end{figure}

Therefore cooling is essential to avoid food deterioration and market losses.  During transport and storage, the quality of these products might change rapidly as they confront a variety of risks leading to material quality losses. In particular, metabolic activities generally increase as storing temperatures are elevated. Short interruptions during the cold chain control may result in immediate deterioration of product quality (\citet{Nunes2003Quality}). In addition, local temperature can occur in almost any transport situation. There are several reports from literature which indicate that gradients of 5 degrees or more, deviations of only a few degrees can lead to spoiled products and thousands of dollars lost (\citet{Tanner2003Modelling, Nunes2006Brief, Rodriguez-Bermejo2007Thermal}).

Above all, the transport of perishable food products, other than fruit and vegetables, and the equipment used for the carriage of these products is governed by an agreement drawn by The Inland Transport Committee of the United Nations Economic Committee for Europe (UNECE) in 1970-1971 (\citet{Geneva1970}). The aim is to facilitate international traffic by setting common internationally recognized standards. The agreement is known as the ATP agreement and was firstly signed by Austria, West Germany, Italy, Luxembourg, Netherlands, Portugal, and Switzerland.  It provides common standards for temperature controlled transport vehicles such as road vehicles, railway wagons and sea containers and sets down the tests to be done on such equipment for certification purposes. Up to 2016, 50 states are party to ATP (\citet{ATP_wiki}).


According to ATP classification, insulated vehicle bodies are classified as: motor vehicle/panel van, rigid box/lorry, semi-trailer, container or tanker. And its overall coefficient of heat exchange value (known as $K$-value or $K$ coefficient) is given by:
\begin{equation}
K = \frac{W}{S\cdot \Delta \theta}
\end{equation}
where W is  the  power  necessary  to  maintain  a  steady  temperature  difference $\Delta \theta$ between  the  mean internal and external air temperature of the equipment. $S$ is the mean surface of the equipment, given by the geometric mean of the inside and outside surface areas:
\begin{equation}
S = \sqrt{S_i \cdot S_e}
\end{equation}

\begin{quote}
	\textit{The ATP classifies insulated vehicles and bodies as either Normally Insulated Equipment (\textbf{IN}, isotherme normal: K coefficient equal or less than $ 0.7 W/(m^2 K) $) or Heavily Insulated Equipment (\textbf{IR}, isotherme renforcé: K coefficient equal or less than $ 0.4 W/(m^2 K )$ (\citet{Tassou2009}).}
\end{quote}

After each ATP standard test, a valid ATP capacity report is delivered to the refrigeration equipment, and according to the agreements, a refrigerated vehicle installing that equipment must possess the report during the transportation. Usually, the ATP certificate ensures that the insulated body and the refrigeration unit have been tested by a third party and that the two have been appropriately matched.

\section{Application of infrared thermography in ATP}
Until now, the ATP prescribes a procedure to measure the insulating status of equipments with a global approach. This approach is robust and furnishes an average behavior of the insulated box. On the other hand, this procedure is not able to locate departures from the average conditions that may be localized somewhere in the structure, like thermal bridges due to structural elements, lack of insulation due to a bad foam distribution inside the panel, zones of anomalous aging due to a localized degradation of the polyurethane, air leakage around the opening gaskets. 

A technique like infrared thermography could be particularly useful in giving a local approach to the identification of these anomalies. One example is found in (\citet{Estrada-Flores1996use}), where a thermographic survey was performed on two types of insulated bodies: (a) truck bodies with a difference of 64 °C between internal and ambient temperature; and (b) insulated panel vans with a difference of 28 °C between internal and ambient temperature. The thermographic images can be found in \ref{Estrada-Flores1996use}. Heat leakage, thermal expansion and thermal bridges have been well observed and discussed in relevant results. 

\begin{figure}[htbp]
	\centering
	\subfloat[Side view of an insulated van]
	{
		\includegraphics[scale=0.9]{art/Estrada-Flores1996use_1}
	}
	\\
	\subfloat[Rear doors of an insulated van]
	{
		\includegraphics[scale=0.9]{art/Estrada-Flores1996use_2}
	}
	\caption{The thermographic images show 12 sections of each visible light image (\citet{Estrada-Flores1996use}).}
	\label{Estrada-Flores1996use}
\end{figure}

Indeed, all the localized lack of insulation described above lead to a variation of the heat flux and therefore to a variation of the temperature on the surface of the insulated equipments that in general may be detected by thermography. Some papers on this subject have been published (\citet{grinzatoquality,grinzato1comparison}). There, the expertise was used to weight the local thermal anomalies in order to rank the final effect on the K-value. %Here, a more straightforward and quantitative approach is used to evaluate the local K-value, mainly by infrared thermography.

Recently, (\citet{rossi2009k}) has applied infrared thermography to a simplified thermal model for evaluating thermal performance of insulated vehicles. ATP standard tests have been performed in parallel for comparison. In their results, defective zones are identified and the local heat flux mapped basing on a heat flux meter measurement in a reference point and on the thermographic temperature maps.

Moreover, infrared thermography can also help to evaluate the thermal coefficients of the refrigerated vehicle panels. One example is found in (\citet{dragano2009experimental}), where the absorption coefficient of an sandwich panel of the standard insulated vehicle, which is used in refrigerated transport, has been evaluated in transient conditions by a thermographic camera.



\newpage
\section*{Infrared thermography for NDT \& E}
\phantomsection\addcontentsline{toc}{section}{Infrared thermography for NDT \& E} % inclure dans TdM

In the past few decades, a Nondestructive Testing \& Evaluation (NDT\&E) technique: Infrared Thermography (IT), also commonly referred as \textit{Thermal imaging}, or \textit{thermography}, has received growing attention and applications for diagnostics. This is due to its characteristics that ``it allows the mapping of thermal patterns, i.e. thermograms, on the surface of objects, bodies, or systems through the use of an infrared imaging instrument, such as an infrared camera" (\citet{maldague3introduction}). Besides, its impressive quickness and convenience, the inspection attracted a wide variety of applications including biological, civil engineering, aerospace, cultural heritage, etc (\citet{2007-Ibarra-Castanedo,2000-Li,cielo1987thermographie,shoja2011inspection,pradere2009microscale,avdelidis2004applications,maierhofer2005quantitative}). Other utilization such evaluation or measurements of heat transfer coefficients (or absorption) can be found in (\citet{dragano2009experimental,grinzato2010r,grinzatoquality,grinzato1comparison,rossi2009k}).

Infrared thermography is often divided into two approaches: passive thermography, in which materials and structures are naturally at different temperature than the environment; and active thermography, in which an external simulation is added to induce a thermal response (\citet{Maldague2001theory}). The two approaches are described in next two sections.
\section{Passive thermography}
This approach is due to the principle of energy conservation and states that an important quantity of heat is released by any process consuming energy because of the law of entropy, which is known as \textit{the First law of thermodynamics} (\citet{thdy1}). 

Therefore, the passive thermography is often qualitative, such as the diagnosis of the presence of a given abnormality or hot spot with respect to the immediate surroundings.  A delta-T of a few degrees (> 5°C) is generally
found suspicious while greater values indicate strong evidences of abnormal behaviors. Some applications in civil engineering can be found in (\citet{2000-Li,stanley1994non,lo2004building}), where the thermography approach was proved to be reliable in detecting the debonded ceramic tiles on a building finish, and a temperature contours over the surface of a target object to provide an appropriate measure of the damaged building or structure has been mapped. The inspection of buildings, for different purposes such as energy monitoring, moisture mapping and many others, has also been widely deployed (\citet{laranjeirapassive,bison1993automatic,bison2012geometrical}). 

\section{Active thermography}
Contrary to ``\textit{Passive}", the word ``\textit{Active}" seems more progressive. Thus might be the reason why this technique finds a large number of applications in NDT. The experimental configuration for active infrared thermography can be seen in Fig \ref{exp_active} (\citet{sfarra2010comparative}). 
\begin{figure}[!htbp]
	\centering
	\includegraphics[scale=0.4]{art/exp_active}
	\caption{Experimental setup for the active thermography}
	\label{exp_active}
\end{figure}
where \textcircled{\small 1} are two different stimulation sources. They can be located in the same side of the camera, which is known as \textit{reflection} mode; or in the opposite side of the camera, which is known as \textit{transmission} mode.  \textcircled{\small 2} is the specimen under test, \textcircled{\small 3} is the IR camera for recording, and \textcircled{\small 4} is the PC for processing. The whole system is connected by a synchronization control.

Practically, \textit{any} form of energy can be used to produce a measurable thermal contrast. In addition, since the external stimulation can be accurately controlled, quantitative procedure is possible. 

\subsection*{Excitation methods}
Generally, energy used as the external resources can be delivered by the following mechanisms:
\begin{itemize}
	\item Conduction: heating blanket, hot bag or cool bags (as snow or ice);
	\item Convection: Hot (or cold) water [or gas(air)];
	\item Radiation: Lamps, flashes, infrared heaters;
	\item Mechanical stimulation: ultrasonic vibration;
\end{itemize}
Even though ``any" form of energy can be used, the heating sources are often preferred. They can be commonly divided as (\citet{ibarra2013infrared}):
\begin{itemize}
	\item optical: Photographic flashes or lamps are utilized, known as \textit{Pulsed Thermography} (sometimes even using laser as heating resource (\citet{suzuki2002application,burrows2007combined}). When using periodic heating at a given frequency to measure the amplitude and (or) phase delay  of the thermal response, that is known as \textit{Lock-in thermography} (\citet{wu1998lock,duan2013quantitative,2007-Ibarra-Castanedo});
	\item mechanical: Sound or ultrasound waves are injected to the specimen to produce heat by friction, known as \textit{Vibrothermography} (\citet{2007-ClementeIbarra-Castanedo,2007-Ibarra-Castanedo});
	\item induction: Eddy currents are generated by a coil inside the specimen (\citet{riegert2004lockin,zenzinger2007thermographic}). This type of heating source is limited to conductive materials;
	\item microwave: Heat is introduced into the specimen by a time-gated microwave source (\citet{myers1979microwave,land1987clinical});
\end{itemize}
Since the main technique in this work is using the excitation method, different methods will be presented in details in the following sections (\citet{Maldague2001theory,ibarra2013infrared}).
\subsection{Pulsed Thermography}
%In this technique, the specimen surface is submitted to a heat pulse generated by a high power and fast heat source such as flashes. Then the thermal front propagates under the surface by diffusion. Then as time passes, the surface temperature decreases uniformly for a piece without any defects. While on the contrary,  the diffusion rate can be changed and abnormal temperature patterns will be produced at the subsurface, by any kind of surface discontinuities (such as air leakage, delaminations, thermal bridges, porosity, inclusions, etc.). The detection of these discontinuities  with an IR camera depends on their sizes.

Pulsed thermography (PT) is one of the most popular thermal stimulation method in IR thermography. One reason for this popularity is the quickness of the inspection relying on a thermal stimulation pulse, with duration going from a few ms for high thermal conductivity material inspection (such as metal parts) to a few seconds for low thermal conductivity specimens (such as plastics, graphite epoxy components (\citet{Maldague1993Nondestructive,Maldague1994bInfra}). Such quick thermal stimulation allows direct deployment on the plant floor with convenient heating sources. Moreover, the brief heating prevents damage to the component (heating is generally limited to a few degrees above the initial component temperature).

Basically, PT consists of briefly heating the specimen and then recording its temperature decay curve. Qualitatively, the phenomenon is as follows. The temperature of the material first rises during the pulse. After the pulse, it then decays because the energy - the thermal front - propagates by diffusion under the surface. Later, the presence of a subsurface defect (example: a disbonding) reduces the diffusion rate so that when observing the surface temperature, such a subsurface defect appears as an area of higher temperature with respect to the surrounding sound area. In fact in such a case the reduced diffusion rate caused by the subsurface defect presence translates into “heat accumulation” and hence higher surface temperature just over the defect. Moreover, such phenomenon occurs in time so that, deeper defects are observed later and with a reduced “diluted” or “spread” thermal contrast.

Theoretically, the 1D solution of the \textit{Fourier} equation for the propagation of a \textit{Dirac} heat pulse through a semi-infinite homogeneous material is given by (\citet{carslaw1986heat}):
\begin{equation}
T(z,t) = T_0 + \frac{Q}{\sqrt{k\rho C_p \pi t}}exp(-\frac{z^2}{4\alpha t})
\end{equation}
where $Q$ is the energy absorbed, $T_0$  is the initial temperature, $k$ the conductivity of the material, $C_p$ the heat capacity at constant pressure and $\alpha$ thermal diffusivity.

Thus, at the surface ($z=0$), one has:
\begin{equation}
T(0,t) = T_0 + \frac{Q}{\sqrt{k\rho C_p \pi t}}=T_0 + \frac{Q}{e\sqrt{\pi t}}
\label{PT_eq}
\end{equation}
where $e=\sqrt{k\rho C_p}$ is defined as the thermal effusivity, which measures the material ability to exchange heat with its surroundings. Therefore, surface temperature will decay as a function of $t^{1/2}$.

\noindent The energy source can be applied in various ways:
\begin{description}
	\item \textbf{Point inspection}: heating with a laser or a focused light beam; advantages: repeatable heating, uniformity; drawback: the necessity to move the inspection head to fully inspect a surface slows down the inspection process.
	\item \textbf{Line inspection}: heating using line lamps, heated wire, scanning laser, line of air jets (cool or hot); advantages: fast inspection rate (up to 1 $m^2/s$) and good uniformity thanks to the lateral motion; drawback: only part of the temperature history curve is available due to the lateral motion of the specimen and the fixed distance between thermal stimulation and temperature signal pick-up. Projection of a series of line heating strips is also used to detect surface cracks.
	\item \textbf{Surface inspection}: heating using lamps, flash lamps, scanning laser; advantages: the complete analysis of the phenomenon is possible since the whole temperature history curve is recorded; drawback: concerns about non-uniformity of the heating (lamps, flashes, heat gun, laser, microwave).
\end{description}

If the temperature of the part to inspect is already higher than ambient temperature, it can be of interest to make use of a cold thermal source such as a line of air jets (or water jets; sudden contact with ice, snow, etc.). In fact, a thermal front propagates the same way whether being hot or cold: what is important is the temperature differential between the thermal source and the specimen. An advantage of a cold thermal source is that it does not induce spurious thermal reflections into the IR camera as in the case of a hot thermal source. The main limitations of cold stimulation sources are related to practical considerations as for instance it is generally easier and more efficient, to heat rather then to cool a part. For this reason, our work on cold stimulation needs to be investigated in detail and better understood.

\subsection{Step Heating}
When the specimen is  continuously heated, the increase of surface temperature is monitored during the application of a stepped heating pulse, thus the principle of \textit{Step Heating}. Same as in pulsed thermography, variations of surface temperature with time are related to specimen features. So it can be called \textit{Long pulse}. Besides, it is also referred as  time-resolved  infrared radiometry (TRIR) (\citet{spicer1992time}).  More details can be found in (\citet{ibarra2013infrared,osiander1998thermal}).

\subsection{Lock-in Thermography}
Lock-in thermography (LT) is based on thermal waves generated inside the inspected specimen and detected remotely. Wave generation is for instance performed by periodically depositing heat on the specimen surface (e.g. through sine-modulated lamp heating) while the resulting oscillating temperature field in the stationary regime is remotely recorded through its thermal infrared emission (\citet{wu1998lock}).

The lock-in terminology refers to the necessity to monitor the exact time dependence between the output signal and the reference input signal (i.e. the modulated heating). This is done with a locking amplifier in a point by point laser heating or by computer in full-field (lamp) deployment so that both phase and magnitude images becomes available. Phase images are related to the propagation time and since they are relatively insensitive to local optical surface features (such as non uniform heating), they are interesting for NDE purposes. The depth range of images is inversely proportional to the modulation frequency so that higher modulation frequencies restrict the analysis in a near surface region (\citet{Maldague2001theory}).

\subsection{Vibrothermography}
Known as \textit{Ultrasound thermography} (\citet{dillenz2001progress}), vibrothemography (VT) is an active Infrared Thermography technique based on that principle: under the effect of mechanical vibrations (0 to 25 kHz) induced externally to the structure, thanks to direct conversion from mechanical to thermal energy, heat is released by friction precisely at locations where defects such as cracks and delaminations are located.

Ultrasonic waves are ideal for NDT in the sense that, defect detection is independent from of its orientation inside the specimen, and both internal and open surface defects can be detected. Thus, VT is very useful for the detection of cracks and delaminations. The range for ultrasonic waves is often between 20 kHz and 1 MHz. Unlike electromagnetic waves, mechanical elastic waves such as sonic and ultrasonic waves can not propagate in a vacuum. They require a medium for traveling. They travel faster in solids and liquids than the air. This indicates an important aspect of VT: the common approach in VT is to use a coupling media between a transducer and the specimen to reduce losses. Therefore, one of the Infrared thermography characteristic ``contactless" is meaningless in this technique. However, image acquisition can still be carried out from a distance using an IR camera.

\section{Advantages and limitations of infrared thermography}
As every coin has two sides, all technique has its strengths and weaknesses. For Infrared thermography, its advantages are as follows (\citet{maldague3introduction, Maldague2001theory}):
\begin{itemize}
	\item Fast, surface inspection
	\item Ease of deployment
	\item Contactles, no coupling needed as in the case of conventional ultrasounds.
	\item Security, there's no damaging radiation.
	\item Easy access to results thanks to the imaging capabilities.
	\item Numerous applications.
	\item Unique inspection tool in some tasks.
\end{itemize}
While, on the other hand, some limitations specific are evident:% (\citet{maldague3introduction, Maldague2001theory}:
\begin{itemize}
	\item Non-uniform heating, when over a large surface.
	\item Emissivity differs from materials.
	\item Defects detected are generally shallow.
	\item Thermal losses, absorption by the environment.
	\item Inspected thickness of material under the surface has a limitation.
	\item Cost of the apparatus.
\end{itemize}

\section{Recent research on cold approach}
The first mention of cold excitation technique was found in (\citet{Burleigh1989Thermographic}), where a filament-wound graphite-epoxy tank was tested by thermographic methods and compared to an ultrasonic inspection. Two cold excitation were performed to identify defects and bonded Al structures during the test. The first one was an evaporative cooing of Freon, and the second trial used a liquid nitrogen spray. According to their findings, cold excitation had proven to be an effective fast NDT technique. The freon method is able to help visualize the delaminations, however it is undesirable on a large scale because of the evolution of high concentrations of fumes. This made inconvenient during the test. On the other hand, the use of liquid nitrogen for thermal excitation is low-cost, non-toxic, and only simple safety protection was required.


One other similar is found in (\citet{Maldague1994bInfra}), where the thermographic inspection was performed in a \textit{reflection} procedure by propagation of a cold front for the detection of bonded Al structures. Their results are shown in Fig \ref{Al_stru_cold}. 
\begin{figure}[!htbp]
	\centering
	\includegraphics[scale=0.81]{art/Al_stru_cold}
	\caption{Thermographic inspection by propagation of a cold front for the inspection of bonded Al structures.}
	\label{Al_stru_cold}
\end{figure}
In the study, the structure was initially at a temperature of some 10$°C$ above room temperature and a line of air jets was used to quickly cool the inspected area. The thermal image shows the hotter central area which corresponds to the bonded area. And a bonding defect, which is lack of adhesive, is then revealed by the cooler region at the center of thermogram.

Another example of stimulation by propagation of a cool front in the same mode on an Al-foam panel was presented in (\citet{Maldague1993Nondestructive}(Fig \ref{Al_foam_hc})).
\begin{figure}[!htbp]
	\centering
	\includegraphics[scale=0.90]{art/Al_foam_hc}
	\caption{Thermographic inspection using a line of air jets with pre-heating then cooling.}
	\label{Al_foam_hc}
\end{figure}
In this study, an non-bonded area in a Al-foam laminate was to be detected. The panel was first uniformly heated at a temperature of about 10$°C$ above room temperature (same to the previous study). The surface was then cooled by a line of cool air jets during the lateral moving of the panel, and it was at a constant speed of 2.4 $cm/s$. In the figure of two shown thermograms, one is without defect and the other is for an area where a circular-shaped disbonding (4 cm diameter) is present. The defect is clearly visible at the image center. 

In addition, after comparing  the radiative-heat injection and convection-heat removal approach in (\citet{Maldague1993Nondestructive}). They found that, a reduced thermal contrast is obtained with the cool air stimulation due to both the reduced heat capacity of air and the smaller temperature differential between the thermal perturbing source and the inspected surface (case of air versus high temperature radiative source).  Moreover, both two studies have firstly heated the specimen and then cooled it by air jets. This indicates that a higher initial temperature difference is essential for a better result in thermograms, especially when applying the cold approach by air jets. Unlike traditional heating method, which can heat the specimen to a suitable high temperature, cooling method by air jets cannot cool the specimen to an adequate low temperature. 


Other applications have then been made during the past decades in several domains. (\citet{Swiatczak2008Evaluation}) evaluates convection cooling conditions of electronic devices by thermal wave method using Fourier and wavelet analysis in lock-in thermography. One study of applying cold excitation in thermal tomography can be found in (\citet{Bajorek2010Analysis}), where a stream of carbon dioxide of different duration as the cooling excitation was applied by a cryogenic instrument. possibilities of reconstruction object parameters were analyzed in their results. The quality of the results (not only the defects locations, but also their depth) would be improved with increasing cold excitation duration.

(\citet{endohdynamical2012}) proposed constructing an active thermographic imaging system, in which a cooling material contacts the surface and scans over a welded zone of two stainless plates (Fig \ref{endoh_fig}).
\begin{figure}[!htbp]
	\centering
	\includegraphics[scale=1]{art/endoh_axe}\\
	\includegraphics[scale=1]{art/endoh_lat}
	\caption{ Experimental setup (a) Axial movement, (b) lateral movement of a coolant material.}
	\label{endoh_fig}
\end{figure}
They recorded both a time-varying thermographic image and a real-time response of each pixel. The results are exhibited in Fig \ref{endoh_res_1}, \ref{endoh_res_2}.
\begin{figure}[!htbp]
	\centering
	\includegraphics[scale=0.9]{art/endoh_res_1}
	\caption{(a)Thermal image, (b) temperature profile along A-A’ line, (c) temperature profile along B-B’ line (v=10$mm/s$).}
	\label{endoh_res_1}
\end{figure}
\begin{figure}[!htbp]
	\centering
	\includegraphics[scale=1.0]{art/endoh_res_2}
	\caption{(a)Thermal image, (b) temperature profile along A-A’ line (v=50$ mm/s$).}
	\label{endoh_res_2}
\end{figure}

In both two cases of scanning (v= 10 and v=50 $mm/s$), it is observed that the locally high-temperature appears in low-temperature region of the simulated welding region. For axial movement of coolant material, the heat inside and the surface of the specimen transfers toward the interface surface of the specimen to the coolant, due to conservation of the axial symmetry of heat transfer. However, for lateral movement of coolant, it was squeezed for the spatial distribution of the temperature along the direction of coolant movement. Therefore, an elliptical shape of temperature distribution was obtained.

In (\citet{Gradeck2012Solution}),  a jet cooling experiment is used to study a two-dimensional inverse heat conduction problem. To estimate a transient cooling flux, a nickel disk is heated up to 600 °C then cooled by a water jet squirting on its front side and a infrared camera captures its rear side temperature. Thermograms of cooling procedure can be found in Fig \ref{Gradeck2012Solution_1}.
\begin{figure}
	%\hspace{-45pt}
	\centering
	\includegraphics[scale=1.0]{art/Gradeck2012Solution_1}
	\caption{Raw thermographic frames during cooling of the disk by a 1.74 $ m/s $ water jet.}
	\label{Gradeck2012Solution_1}
\end{figure}
The disk temperature stays almost homogeneous at the beginning (Fig \ref{Gradeck2012Solution_1}a). Circular isotherms then are observed in the central region of the disk after jet impingement (Fig \ref{Gradeck2012Solution_1}b). Due to the forced convection, the cold zone propagates to the external regions rapidly (Fig \ref{Gradeck2012Solution_1}c-f), with a boiling zone where heat flux is important emerges (Fig \ref{Gradeck2012Solution_1}c), becomes film (Fig \ref{Gradeck2012Solution_1}e) and finally vanishes (Fig \ref{Gradeck2012Solution_1}f).

In (\citet{2012-LewisHom}) a custom, microfluidic  heat  sink  with  an  IR-transparent working  fluid  (0.75  LPM)  was  manufactured  to  cool  an instrumented test chip while permitting optical access for IR thermal imaging.  Then a detailed system calibration was conducted to account for the temperature-dependent optical properties of the chip and heat sink. Their experiments  confirm  that  the  dynamic  range  of  the infrared  microscope  can  be  greatly  enhanced  by decreasing the  thickness  of  the  liquid  layer. 

(\citet{rodriguez2014cooling}) deployed an thermographic test to the detection of subsurface cracks in welding. The procedure started with the thermal excitation of the material, following with the monitoring of the cooling process with infrared thermography. They used the natural convection between the material and the environment, but two method of heating were deployed in to same specimen. One is heating from the front surface (Method 1), the other is heating from back surface (Method 2). 

\begin{figure}
	\hspace{-45pt}
%	\centering
	\includegraphics[scale=0.65]{art/cooling_res}
	\caption{Thermal images of each defect for two methods Natural convection cooling}
	\label{cooling_res}
\end{figure}
The experimental results (Fig \ref{cooling_res}) have shown that the technique has a potential to detect hidden flaws due to the influence of the physical differences between the defect and the non-defect area in the cooling parameters. But the results depended on the depth of defect: ``Greater depths require larger heating and greater ranges of precision in thermographic data acquisitions".

 
%Still, the limitation of that proposed method in this literature is only by natural convection. %Our work will apply the forced convection by spraying the nitrogen gas.
%\newpage
%\section{Issue, hypothesis and objectives}
\subsection{Issue}

\subsection{Hypothesis}

\subsection{Objectives}




\section{Methodology}
Thought infrared thermography has the advantages such as fast inspection, ease of deployment, contactless, security and easy access to results with the imaging capabilities, raw infrared thermography results is difficult to handle and analyze in case of reflections and non-uniform external stimulation. To improve the inspection results, there are various post-processing techniques which have been developed. The recently popular data analyzing and processing methods are presented in detail in the following subsections.
\subsection{Fourier Transform (FT)}
Among all data processing methods, Fourier Transform (FT) is especial because it helps retrieve phase and amplitude data from raw results, since our principal results are images, which can be seen as signals with two dimensions.

It is well-known that any wave form, periodic or not, can be approximated by the sum of purely harmonic waves oscillating at different frequencies. The Continuous Fourier Transform (CFT) is then given by:
\begin{equation}
F(\omega) = \int_{-\infty}^{\infty}f(t)e^{-j\omega t}dt = A(\omega)e^{i\phi(\omega)}
\end{equation}
where $\omega = 2 \pi f$. the FT technique serves to transform the perception of signal from a time-based domain to a frequency-based domain.

In case of discrete situation, the Discrete Fourier Transform (DFT) is possible to analyze the data in the frequency domain:
\begin{equation}
F_n = \Delta t \sum_{k=0}^{N-1}T(k\Delta t)e^{-\tfrac{i2\pi nk}{N}} = \Re(F_n) + i\Im(F_n)
\label{DFT}
\end{equation}
where $n$ designates the frequency increment ($n=0, 1, ..., N$), $\Delta t$ is the sampling interval, $N$ is the total number of infrared images, and $\Re$ and $\Im$ are the real and the imaginary parts of the transform, respectively.
DFT is often applied in Pulsed Phase Thermography (PPT), which analyzes phase data obtained from PT results. In addition, the Fast Fourier Transform (FFT) algorithm is often applied to reduce computation time.

\subsection{PPT}
From Eq.\ref{DFT}, amplitude $A_n$ and phase delay $\Phi_n$ are given by:
\begin{equation}
A_n = \sqrt{\Re(F_n)^2 + \Im(F_n)^2} \qquad \Phi_n = \tan^{-1}\frac{\Re(F_n)}{\Im(F_n)}
\end{equation}
It should be noted here that The processed sequence is less affected than the original data by undesired noise
sources such as environmental reflections, emissivity variations, non-uniform heating.

By selecting two pixels, the first in correspondence of a reference zone, the second in correspondence of a possible defect zone, and following both of them in time, after the pulse, the two profiles shown in Fig. \ref{T_profile_PPT} are obtained. By taking the FFT of the two signals, the two profiles of amplitude (Fig. \ref{FT_AM_profile_PPT}) and phase (Fig. \ref{FT_PH_profile_PPT})
as a function of frequency are obtained.
\begin{figure}[!h]
	\centering
	\includegraphics[scale=0.35]{art/T_profile_PPT}
	\caption{Temperature profiles of a reference zone (blue) and a defect zone (red)}
	\label{T_profile_PPT}
\end{figure}

\begin{figure}[!h]
	\centering
	\includegraphics[scale=0.35]{art/FT_AM_T_profile_PPT}
	\caption{Amplitude as a function of frequency: blue reference, red defect}
	\label{FT_AM_profile_PPT}
\end{figure}

\begin{figure}[!h]
	\centering
	\includegraphics[scale=0.35]{art/FT_PH_T_profile_PPT}
	\caption{Phase as a function of frequency: blue reference, red defect}
	\label{FT_PH_profile_PPT}
\end{figure}

\subsection{Differential Absolute Contrast (DAC)}
Traditionally, once the temperature of a sound area (reference zone)$T_s(t) $ and that of a defect zone $T_{def}(t) $ are known, contrast methods can be applied by simply:
\begin{equation}
C_{ac}(t) = T_{def}(t) - T_s(t)
\label{AC_eq}
\end{equation}
which is known as the absolute contrast. However, this method becomes inconvenient when the sound area cannot be pratical defined. In addition, the common case of non-uniform heating has a strong effect on the results.

To improve this, the Differential Absolute Contrast (DAC) has proven its amelioration for non-uniform heating situations\citep{Benitez2008, pilla2002new}.

DAC method starts from Eq. \ref{PT_eq}, the surface temperature increase based on one-dimensional model of the Fourier equation after an instantaneous Dirac heating pulse is applied:
\begin{equation}
\Delta T = T(0,t) - T_0  = \frac{Q}{e\sqrt{\pi t}}
\label{PT_eq_2}
\end{equation}
The temperature of the sound area at the surface ($z=0$) $T_s$ at time $t_1$ is given by:
\begin{equation}
\Delta T_s(t_1) = \frac{Q}{e\sqrt{\pi t_1}}
\end{equation}
Then at time $t$, the temperature can be written as:
\begin{equation}
\Delta T_s(t) = \frac{Q}{e\sqrt{\pi t_1}} = \sqrt{\frac{t_1}{t}}\cdot \Delta T(t_1)
\end{equation}
where $t_1$ is a time between the thermal pulse lasting and the time at which the first temperature spot of the subsurface defects appear.\\
The absolute temperature contrast (Eq. \ref{AC_eq}) can be rewritten as:
\begin{align}
C_{ac}(t) = & [T_{def}(t) -T_0] - [T_s(t) - T_0] \\ 
		  = & \Delta T_{def}(t) - \sqrt{\frac{t_1}{t}}\cdot \Delta T(t_1)
\end{align}
DAC Applications on different type of material samples can be found in \citep{pilla2002new}, which has proven this method has an effective improvement on the signal to noise ratio.

\subsection{Temperature Signal Reconstruction (TSR)}
The thermographic signal reconstruction (TSR) data processing technique is one of the most recent improvements which raise thermography to the level of the most established NDE techniques \citep{Balageas2015}.

Same as DAC method, Eq. \ref{PT_eq_2} can be rewritten  in logarithmic way as:
\begin{equation}
\log (\Delta T) = \log (\frac{Q}{e}) - \frac{1}{2}\log (\pi t)
\end{equation}
In a general way, as: %with degree $n$:
\begin{equation}
\log (\Delta T) = a_0 + a_1\log (t) + a_2[\log (t)]^2 +...+ [a_n\log(t)]^n
\end{equation}
This technique usually consists in the fitting of the experimental log-log plot by a polynomial of degree $n$, and also the computation of the 1$^{st}$ and 2$^{nd}$ derivatives of the thermograms.



\subsection{Principal Component Thermography (PCT)}

\subsection{Receiver Operating 	Characteristic (ROC)}
The Receiver operating characteristic (ROC) curve is a technique in statistics which helps visualize, organize and select classifiers based on their performance. The curve graph is created by plotting the the true positive rate (TPR) against the false positive rate (FPR) at various threshold settings. More details about concepts and definitions can be found in \citep{Fawcett2006}. While being frequently chosen a standard method in several scientific fields, ROC are rarely applied in the thermographic field.\citep{Bison2014a} 

In the ROC curves definitions, the \textit{sensitivity} is the True positive rate (\textit{tp rate}), and \textit{1-specificity} is the False positive rate (\textit{fp rate}).


\section{Modeling and simulation}
In this research, all modelling and simulation work are undertaken in the platform COMSOL Multiphysics$^{\textregistered}$, which might be helpful when comparing with the experimental results.
%\section{Introduction to COMSOL Multiphysics}

COMSOL Multiphysics is a general-purpose software platform, based on advanced numerical methods, for modeling and simulating physics-based problems. It is a finite element analysis, solver and Simulation software / FEA Software package for various physics and engineering applications, especially coupled phenomena, or multiphysics.

The advantages of using COMSOL Multiphysics for modeling:
\begin{itemize}
	\item COMSOL Multiphysics has an integrated modeling environment.
	\item COMSOL Multiphysics takes a semi-analytic approach: once questions specified, COMSOL symbolically assembles finite-element method matrices and organizes the bookkeeping.
	\item COMSOL Multiphysics is fully compatible with MATLAB, so user could define programming for the modeling,organizing the computation, or the post-processing has full functionality. COMSOL Script is a MATLAB-like integrated programming environment that can also provide these facilities.
	\item COMSOL Multiphysics provides pre-built templates as Application Modes and in the Model Library for common modeling applications.
	\item COMSOL Multiphysics provides multiphysics modeling--linking well known ``application modes" transparently.
	\item COMSOL Multiphysics innovated extended multiphysics--coupling between logically distinct domains and models that permits simultaneous solution.
\end{itemize}

All the simulation parameters and conditions can be found in corresponding chapters.


%%%%%%%%%%%%%%%%%%%%%%%%%%%%%%%%%%%%%%%%%%%%%%%%%%%%%%%%%%%%%%%%%%%%%%%%%

\section*{Issue, hypothesis and objectives}
\phantomsection\addcontentsline{toc}{section}{Issue, hypothesis and objectives}
\subsection{Issue}

Following recent decades, the public is becoming increasingly aware of the need to apply rigorous standards throughout the food chain.

In this project, we want to improve these tests by developing quantitative methods for identifying thermal insulation anomalies in refrigerated vehicles for the transport of food. We also want to develop an aging model of the panels constituting the refrigerated "box" of the vehicle to predict its remaining useful life. The implementation of improved tests will ensure food safety during the transportation of refrigerated food.
Indeed, the coefficient k of a vehicle with a tendency in time to decrease because of the aging of the insulating panels (migration of the insulating gases out of the foam panels, detachments of the walls following the mechanical stresses undergone by the vehicle, etc.) . Currently the standards only require testing at commissioning and afterwards at six years. This duration is probably too long and a quantitative model allowing a prediction of the evolution of the coefficient $K$ would add rigor to the current process. In addition, the imposed measurement tests of $K$ are essentially done on each of the faces, without taking into account the corner zones which are especially at risk. Infrared thermography would identify and quantify local thermal anomalies so as to establish an overall measure of $K$ more accurate while "sounding the alarm" in case of serious failure.

On the other hand, tradition active infrared thermography with heating stimulation device can be impossible to detect thermal insulation anomalies directly in refrigerated vehicle, since it would be too difficult to heat the entire vehicle ``box". Therefore, the idea of cold approach (compressed air, liquid nitrogen, etc.) comes up having its own advantages under this circumstance: it will be more economical than the heating approach, and more beneficial since there is less thermal noises.

\subsection{Hypothesis}
For the part of the application of infrared thermography in ``cold food chain", the heat exchange coefficients inside and outside of the insulated ``box" will be constant when the ATP standard test arrives at a stable condition. Moreover, the heat diffusion will be mainly one dimension.

For the part of the exploration of cold approaches in infrared thermography, heat propagation will have the same performance, since the material conductivity will not change whether in hot approach or the cold one.
\subsection{Objectives}
In this project, we are particularly interested in the ``cold chain" and more particularly the transport of food in refrigerated vehicles (ie trucks, trailers, containers). The transportation of dairy products, meat and frozen foods are especially at risk. In addition, the ever-increasing cost of energy makes it possible to limit refrigeration to a minimum. It is therefore essential to ensure an irreproachable thermal insulation during the inspection of refrigerated vehicles. There are thermal insulation compliance tests to ensure satisfactory food transportability (\citet{Geneva1970}). During these tests, measurements of heat transfer coefficients are measured (coefficient $ K $). Nonetheless, if there are some local flaws or thermal bridges inside the panels, which could not be measured by ATP tests, then the insulation can no longer be guaranteed. Infrared thermography can be particularly helpful regarding these issues. %On the other hand this measure proves rather rough. In this project, we want to improve these tests.

To this end, three objectives will be pursued in this thesis:
\begin{itemize}
	\item \textbf{Objective 1$ ^\circ $ } - Development of quantitative methods for identifying thermal insulation anomalies in refrigerated vehicles for the transport of food.
	\item \textbf{Objective 2$ ^\circ $ } - Development of an aging model of panels constituting the refrigerated ``box" of the vehicle in order to predict its objective remaining useful lives. 
	\item \textbf{Objective 3$ ^\circ $ } - Development of a model using a cold stimulation device to identify thermal insulation anomalies in insulated materials.
\end{itemize}

The achievement of these three objectives of this thesis requires the following specific tasks (which can be carried out in parallel):

\S 1.1 Establishment of a thermal image database on a maximum of refrigerated vehicles. §1.2 In parallel, laboratory tests on a refrigerated "box" of reduced size placed in a climatic chamber will allow the elaboration of a first model of thermal behavior. §1.3 First approaches to the treatment of infrared images (identification of anomalies - at the junctions of the panels).

\S 2.1 Stripping of images from the bank will be done and image processing approaches will be developed to identify areas of anomalies. The images will be segmented and the thermal parameters will be calculated since the images will have been acquired under specified conditions (\citet{Ibarra-Castanedo2013Methods}). §2.2 The refrigerated "box" will be subjected to mechanical and thermal stresses in order to accelerate its aging which will allow a validation of the model of thermal behavior. Realistic artificial defects (based on the knowledge acquired in §1.1 will also be realized in the ``box"). §2.3 A test campaign for final validation (it should be possible to validate a certain aging by comparing the current results with previous measurements by re-testing some vehicles measured in advance. § 1.1). §2.4 Latest improvements to the thermal prediction model.

\S 3.1 Attempt of identifying defects in insulated material with infrared thermography by a cold stimulation (compressed air, liquid nitrogen, etc.) \S 3.2 Comparison with traditional heating approaches in infrared thermography for Non-Destructive Testing. \S 3.3 Comparison in modeling and simulation tests.

\section{Methodology}
Thought infrared thermography has the advantages such as fast inspection, ease of deployment, contactless, security and easy access to results with the imaging capabilities, raw infrared thermography results is difficult to handle and analyze in case of reflections and non-uniform external stimulation. To improve the inspection results, there are various post-processing techniques which have been developed. The recently popular data analyzing and processing methods are presented in detail in the following subsections.
\subsection{Fourier Transform (FT)}
Among all data processing methods, Fourier Transform (FT) is particularly interesting because it helps retrieve phase and amplitude data from raw results, since our principal results are images, which can be seen as signals with two dimensions.

It is well-known that any wave-form, periodic or not, can be approximated by the sum of purely harmonic waves oscillating at different frequencies. The Continuous Fourier Transform (CFT) is then given by:
\begin{equation}
F(\omega) = \int_{-\infty}^{\infty}f(t)e^{-j\omega t}dt = A(\omega)e^{i\phi(\omega)}
\end{equation}
where $\omega = 2 \pi f$. the FT technique serves to transform the perception of signal from a time-based domain to a frequency-based domain.

In case of discrete situation, the Discrete Fourier Transform (DFT) is possible to analyze the data in the frequency domain:
\begin{equation}
F_n = \Delta t \sum_{k=0}^{N-1}T(k\Delta t)e^{-\tfrac{i2\pi nk}{N}} = \Re(F_n) + i\Im(F_n)
\label{DFT}
\end{equation}
where $n$ designates the frequency increment ($n=0, 1, ..., N$), $\Delta t$ is the sampling interval, $N$ is the total number of infrared images, and $\Re$ and $\Im$ are the real and the imaginary parts of the transform, respectively.
DFT is often applied in Pulsed Phase Thermography (PPT), which analyzes phase data obtained from PT results. In addition, the Fast Fourier Transform (FFT) algorithm is often applied to reduce computation time.

\subsection{PPT}
From Eq.\ref{DFT}, amplitude $A_n$ and phase delay $\Phi_n$ are given by:
\begin{equation}
A_n = \sqrt{\Re(F_n)^2 + \Im(F_n)^2} \qquad \Phi_n = \tan^{-1}\frac{\Re(F_n)}{\Im(F_n)}
\end{equation}
It should be noted here that The processed sequence is less affected than the original data by undesired noise
sources such as environmental reflections, emissivity variations, non-uniform heating.

By selecting two pixels, the first in correspondence of a reference zone, the second in correspondence of a possible defect zone, and following both of them in time, after the pulse, the two profiles shown in Fig. \ref{T_profile_PPT} are obtained. By taking the FFT of the two signals, the two profiles of amplitude (Fig. \ref{FT_AM_profile_PPT}) and phase (Fig. \ref{FT_PH_profile_PPT})
as a function of frequency are obtained.
\begin{figure}[!h]
	\centering
	\includegraphics[scale=0.35]{art/T_profile_PPT}
	\caption{Temperature profiles of a reference zone (blue) and a defect zone (red)}
	\label{T_profile_PPT}
\end{figure}

\begin{figure}[!h]
	\centering
	\includegraphics[scale=0.35]{art/FT_AM_T_profile_PPT}
	\caption{Amplitude as a function of frequency: blue reference, red defect}
	\label{FT_AM_profile_PPT}
\end{figure}

\begin{figure}[!h]
	\centering
	\includegraphics[scale=0.35]{art/FT_PH_T_profile_PPT}
	\caption{Phase as a function of frequency: blue reference, red defect}
	\label{FT_PH_profile_PPT}
\end{figure}

\subsection{Differential Absolute Contrast (DAC)}
Traditionally, once the temperature of a sound area (reference zone)$T_s(t) $ and that of a defect zone $T_{def}(t) $ are known, contrast methods can be applied by simply:
\begin{equation}
C_{ac}(t) = T_{def}(t) - T_s(t)
\label{AC_eq}
\end{equation}
which is known as the absolute contrast. However, this method becomes inconvenient when the sound area cannot be practical defined. In addition, the common case of non-uniform heating has a strong effect on the results.

To improve this, the Differential Absolute Contrast (DAC) has proven its amelioration for non-uniform heating situations(\citet{Benitez2008, pilla2002new}).

DAC method starts from Eq. \ref{PT_eq}, the surface temperature increase based on one-dimensional model of the Fourier equation after an instantaneous Dirac heating pulse is applied:
\begin{equation}
\Delta T = T(0,t) - T_0  = \frac{Q}{e\sqrt{\pi t}}
\label{PT_eq_2}
\end{equation}
The temperature of the sound area at the surface ($z=0$) $T_s$ at time $t_1$ is given by:
\begin{equation}
\Delta T_s(t_1) = \frac{Q}{e\sqrt{\pi t_1}}
\end{equation}
Then at time $t$, the temperature can be written as:
\begin{equation}
\Delta T_s(t) = \frac{Q}{e\sqrt{\pi t_1}} = \sqrt{\frac{t_1}{t}}\cdot \Delta T(t_1)
\end{equation}
where $t_1$ is a time between the thermal pulse lasting and the time at which the first temperature spot of the subsurface defects appear.\\
The absolute temperature contrast (Eq. \ref{AC_eq}) can be rewritten as:
\begin{align}
C_{ac}(t) = & [T_{def}(t) -T_0] - [T_s(t) - T_0] \\ 
= & \Delta T_{def}(t) - \sqrt{\frac{t_1}{t}}\cdot \Delta T(t_1)
\end{align}
DAC Applications on different type of material samples can be found in (\citet{pilla2002new}), which has proven this method has an effective improvement on the signal to noise ratio.

\subsection{Temperature Signal Reconstruction (TSR)}
The thermographic signal reconstruction (TSR) data processing technique is one of the most recent improvements which raise thermography to the level of the most established NDE techniques (\citet{shepard2003reconstruction, Balageas2015}).

Same as DAC method, Eq. \ref{PT_eq_2} can be rewritten  in logarithmic way as:
\begin{equation}
\log (\Delta T) = \log (\frac{Q}{e}) - \frac{1}{2}\log (\pi t)
\end{equation}
In a general way, as: %with degree $n$:
\begin{equation}
\log (\Delta T) = a_0 + a_1\log (t) + a_2[\log (t)]^2 +...+ [a_n\log(t)]^n
\end{equation}
This technique usually consists in the fitting of the experimental log-log plot by a polynomial of degree $n$, and also in the computation of the 1$^{st}$ and 2$^{nd}$ derivatives of the thermograms.

In practice, logarithmic data may vary from ideal one-dimensional behavior for a variety of reasons (e.g. material inhomogeneities, nonlinear camera response or background radiation contributions). Nevertheless, the logarithmic behavior exhibits remarkable consistency, in that pixels representing defect free areas are nearly linear, and pixels corresponding to subsurface defects depart from the near-linear signature at a particular time that is correlated to the depth of the defect (\citet{Shepard2002,Shepard2003}).


\subsection{Principal Component Thermography (PCT)}
Principal Component Analysis (PCA) is a statistical analysis tool used for identifying patterns in data and expressing the data in a way to highlight the similarities and differences in patterns. When data dimension is very high, the pattern matching of the data then becomes very difficult. Thus, PCA comes to extract features and reduce redundancy by projecting the data onto a system of orthogonal components, which is known as PCT (\citet{Rajic2002,Rajic2002a}). 

Singular Value Decomposition (SVD) is often used in PCA. Generally, a matrix $M$ of $m\times n$ can be decomposed in the form of:
\begin{equation}
A = U R V^T
\end{equation}
where $U$ is an $M\times N$ unitary matrix, $R$ is a diagonal $N \times N$ matrix with non-negative real numbers on the diagonal, $V$ is an $N\times N$ unitary matrix, and $V^T$ is the conjugate transpose of $V$.

The 3D thermogram matrix needs to be reorganized as a 2D $M\times N$ matrix $A$ with the dimension of time along the columns and space. The columns of $U$ represent a set of orthogonal statistical modes known as Empirical Orthogonal Functions (EOFs) which describe the data spatial variations, while the time variations are represented by the Principal Components (PCs) which are arranged row-wise in matrix $V_T$. Then applying the SVD to the 2D matrix, the resulting U matrix, providing the spatial information, can be rearranged as a 3D sequence (\citet{Ibarra-Castanedo2006}). The whole procedure is illustrated in Fig. \ref{PCT_SVD}.
\begin{figure}[htbp]
	\centering
	\subfloat[Thermographic data rearrangement from a 3D sequence to a 2D matrix]
	{
		\includegraphics[scale=0.7]{art/PCT_SVD_1}
	}
	\\
	\subfloat[Rearrangement of a 2D matrix into a 3D matrix containing the EOFs]
	{
		\includegraphics[scale=0.7]{art/PCT_SVD_2}
	}
	\caption{PCT algorithm procedure}
	\label{PCT_SVD}
\end{figure}

Beyond reducing redundancy, PCT is also proposed as a contrast enhancement approach.
The entire results obtained in this thesis are mostly post-processed by PCT method.

\subsection{Receiver Operating 	Characteristic (ROC)}
The Receiver operating characteristic (ROC) curve is a technique in statistics which helps visualize, organize and select classifiers based on their performance. ROC graphs have long been used in different fields as signal detection (\citet{swets2000better}), diagnostic systems(\citet{swets1988measuring}), medical decision(\citet{zou2002receiver}), and the earliest one in machine learning(\citet{spackman1989signal}). While being frequently chosen a standard method in several scientific fields, ROC are rarely applied in the thermographic field.(\citet{Bison2014a}).

\subsubsection{ROC concept}
Basically, a classification model (classifier or diagnosis) is a mapping of instances between certain classes/groups. For a two-class prediction problem (also known as binary classification), in which the results are marked either as positive ($p$) or negative ($n$). Then there are four results in a binary classification. Therefore, given an instance, it the instance is positive and it is classified as positive, it is called as a \textit{true positive}; if it is classified as negative, it is called as a \textit{false negative}. If the instance is negative and it is classified as negative, it is called as a \textit{true negative}; if it is classified as positive, it is called as a \textit{false positive}. When given a set of instances, a two-by-two confusion matrix (also called a contingency table) can be constructed representing the dispositions of the set of instances. 
\begin{figure}[ht]
	\centering
	\includegraphics[scale=1.2]{art/ROCintro_1}
	\caption{Confusion matrix}
	\label{ROCintro_1}
\end{figure}

The parameters' rates in Fig. \ref{ROCintro_1} are calculated as following:\\
The \textbf{true positive rate} (also called \textit{recall}):
\begin{equation}
tp\; rate = \frac{\Sigma \; \text{True positives}}{\Sigma \; \text{Total positives (or Condition positives)}}
\end{equation}
The \textbf{false positive rate} (also called \textit{false alarm rate}):
\begin{equation}
fp\; rate = \frac{\Sigma \; \text{False positives}}{\Sigma \; \text{Total negatives (or Condition negatives)}}
\end{equation}

Additional rates associated are defined as:\\
The \textbf{true negative rate} (also called \textit{specificity}): 
\begin{equation}
tn\; rate = \frac{\Sigma \; \text{True negatives}}{\Sigma \; \text{Total negatives (or Condition negatives)}}
\end{equation}
The \textbf{false negative rate}:
\begin{equation}
fn\; rate = \frac{\Sigma \; \text{False negatives}}{\Sigma \; \text{Total positives (or Condition positives)}}
\end{equation}
And:
\begin{equation*}
\text{sensitivity} = \text{recall}
\end{equation*}

\subsubsection{ROC space}
The corresponding ROC curve graph is created by plotting the true positive rate (TPR) against the false positive rate (FPR) at various threshold settings (\citet{Fawcett2006}).  Since TPR is equivalent to sensitivity and FPR is equal to 1 − specificity, the ROC graph is also called the sensitivity vs (1 − specificity) plot. In the ROC space, each prediction result or instance of a confusion matrix represents one point. Fig. \ref{ROC_space} shows one example of four prediction points.
\begin{figure}[ht]
	\centering
	\includegraphics[scale=0.56]{art/ROC_space-2}
	\caption{The ROC space}
	\label{ROC_space}
\end{figure}
There are several points in ROC space which are important to note. The lower left point $ (0,0) $ represents the strategy of never issuing a positive classification; such a classifier commits no false positive errors but also gains no true positives. The opposite strategy, of unconditionally issuing positive classifications, is represented by the upper right point $ (1,1) $. The point $ (0,1) $ represents perfect classification.  Informally, one point in ROC space is better than another if it is to the northwest (tp rate is higher, fp rate is lower, or both) of the first. In the figure, $C'$ has the best performance as shown.

The diagonal line (also called \textit{line of no-discrimination}) $y = x$ represents the strategy of randomly guessing a class. For example, if a classifier randomly guesses the positive class half the time, it can be expected to get half the positives and half the negatives correct; this yields the point $ (0.5, 0.5) $ in ROC space. The diagonal divides the ROC space. Points above the diagonal represent good classification results (better than random), points below the line represent poor results (worse than random). Note that the output of a consistently poor predictor could simply be inverted to obtain a good predictor. As $C$ and $C'$ shown in the figure.

In conclusion,  any classifier that produces a point in the lower right triangle can be negated to produce a point in the upper left triangle. Any classifier on the diagonal may be said to have no information about the class.

\subsubsection{Area under curve analysis}
In ROC analysis, a common method to compare different classifiers is to calculate the area under the ROC curve, known as AUC(\citet{Fawcett2006}). In this way, a single scalar value will represent the expected performance of the classifiers. As in the ROC curve profile, the AUC is a part of the unit square are, therefore, its value will be between 0 and 1. Thus, random guessing classifier produces a diagonal line between (0,0) and (1,1), which has an area of 0.5. It can be then indicated that a classifier AUC value less than 0.5 is even worse than a random guessing. 

An important statistical property of the AUC is that a classifier's AUC value is equivalent to the probability that the classifier will rank a randomly chosen positive instance higher than a randomly chosen negative instance. 

\section{Modeling and simulation}
In this thesis, all modeling and simulation work are undertaken in the platform COMSOL Multiphysics{\textregistered}, which might be helpful when comparing with the experimental results.
%\section{Introduction to COMSOL Multiphysics}

COMSOL Multiphysics is a general-purpose software platform, based on advanced numerical methods, for modeling and simulating physics-based problems. It is a finite element analysis, solver and Simulation software / Finite Element Analysis (FEA) Software package for various physics and engineering applications, especially coupled phenomena, or multiphysics.

The advantages of using COMSOL Multiphysics for modeling:
\begin{itemize}
	\item COMSOL Multiphysics has an integrated modeling environment.
	\item COMSOL Multiphysics takes a semi-analytic approach: once questions specified, COMSOL symbolically assembles finite-element method matrices and organizes the bookkeeping.
	\item COMSOL Multiphysics is fully compatible with MATLAB, so user could define programming for the modeling,organizing the computation, or the post-processing has full functionality. COMSOL Script is a MATLAB-like integrated programming environment that can also provide these facilities.
	\item COMSOL Multiphysics provides pre-built templates as Application Modes and in the Model Library for common modeling applications.
	\item COMSOL Multiphysics provides multiphysics modeling--linking well known ``application modes" transparently.
	\item COMSOL Multiphysics innovated extended multiphysics--coupling between logically distinct domains and models that permits simultaneous solution.
\end{itemize}


One example of application of COMSOL simulation in infrared thermography for NDT \& E can be found in (\citet{Cannas2012Modeling}). Where a small concrete wall with an inside cavity was heated by two halogen lamps to detect the defects position. A 3D model by finite element method under COMSOL platform has also been implemented to simulate the heating process. Experimental and numerical data have well matched, which allows parametric studies free from the experimental tests.

\bibliographystyle{unsrtnat}              % style de la bibliographie
\bibliography{U:/Desktop/Bibliography/Biblio_th} 
