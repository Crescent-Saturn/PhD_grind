\chapter*{Introduction}         % ne pas numéroter
\phantomsection\addcontentsline{toc}{chapter}{Introduction} % inclure dans TdM


%Une thèse ou un mémoire devrait normalement débuter par une
%introduction. Celle-ci est traitée comme un chapitre normal, sauf
%qu'elle n'est pas numérotée.
\section{Cold chain conception}
Nowadays, the increasing cost of energy has made energy saving a vital necessity in the current world.
According to a research \citep{coulomb2008refrigeration}, about 15\% of all electricity consumed worldwide is devoured by refrigeration technology. 

As one of the major applications of refrigeration technology, the cold chain concept has several main procedures as: production, processing, packaging, distribution and handling, retail and consumer's household \citep{Estrada-Flores2010}. It firstly starts at the procedure of production of raw materials, in which immediate cooling has been deployed to provide perfect quality. Then primary processing involves in with operations preparing the raw materials for a second processing stage.  In this stage, carcass chilling, precooling of fruits and vegetables or milk cooling and other cold chain operations are often represented. Transforming primary agricultural products into manufactured foods is performed in the secondary processing. After that, the cold chain continues in the form of freezing or chilling of foods, either in bulk or in packages. Food distribution and handling need to be performed always at controlled temperatures and planning of routes and schedules need to take into account the location and capacity of refrigerated distribution centres, the refrigerated modes (sea, air, land) and the volumes to be transported. Loading and unloading operations may be performed in refrigerated docks. Food at retail needs different levels of refrigeration: cold storage is required in supermarket distribution centres; preparation rooms  need to be air-conditioned; walk-ins and display cabinets need to be refrigerated. Finally, consumers store purchases of perishable products in domestic refrigerators.

Then, as one major procedure of the cold chain concept, the correct transport of perishable foodstuffs in refrigerated vehicles, especially for dairy products, meat and frozen foods, known as ``Maintaining the cold chain", is one example of energy saving involved in. It has also become the key part fo every distributor's food safety program. Therefore, a suitable thermal insulation implemented in refrigerated vehicles is essential for saving energy while maintaining an appropriate conservation of the foodstuffs. There are some agreements concerning thermal insulation tests which ensures the suitability for the transport of food in refrigerated conditions, for example ATP: ‘The Agreement on the Transport of Perishable foodstuffs’\citep{Geneva1970}, which establishes standards for the international transport of perishable food between the states that ratify the treaty since 1970.

The Construction Technologies Institute of the Italian National Research Council (ITC- CNR), our collaborator, hosts a wide testing facility for refrigerated vehicles or insulated roll containers and it is also authorized by the Italian Ministry of Transport to perform such ATP tests \citep{Tassou2009,dragano2009experimental}. The ATP standard test is a procedure that measures the insulating performance of truck panels with a global approach; however, if there are some local flaws or thermal bridges inside the panels, which could not be measured by ATP tests, then the insulation and the stability of the temperature can no longer be guaranteed. Infrared thermography can be particularly helpful regarding these issues. In fact, all of the defects mentioned above lead to a variation of the heat flux and temperature on the surface of the container \citep{grinzato2010r, grinzatoquality, grinzato1comparison}. Therefore, the local heat flux map obtained by infrared thermography could give a straightforward image of the structure, and also a local evaluation of the transmittance.

\section{Infrared thermography for NDT \& E}
In the industrial field, the Nondestructive Testing \& Evaluation (NDT \& E) techniques have been  widely developed and applied for many years, aiming to evaluate the properties of a material, component or system (such as automobiles, aircraft composite structures, etc.) without harming the products themselves. Small invisible under-surface defects are expected to be contained in these products. Using electromagnetic radiation, sound, and inherent properties of materials, are common NDT \& E methods to examine the defects. Among all of the different NDT \& E techniques, Infrared Thermography (IT), as the science of measuring and mapping surface temperatures, which combines an infrared image sensor, a data-interpreting computer processor and an interface providing feedback control machine, has emerged as an  outstanding approach \citep{maldague3introduction,cielo1987thermographie,Maldague2001theory,stanley1994non,pradere2009microscale,avdelidis2004applications}.

\begin{quote}
	\textit{	Infrared thermography, a nondestructive, remote sensing technique, has proved to be an effective, convenient, and economical method of testing concrete. It can detect internal voids, delaminations, and cracks in concrete structures such as bridge decks, highway pavements, garage floors, parking lot pavements, and building walls. As a testing technique, some of its most important qualities are that (1) it is accurate; (2) it is repeatable; (3) it need not inconvenience the public; and (4) it is economical."}\citep{malhotra2004handbook}
\end{quote} 

Due to the development of real-time thermography, \textit{Pulsed Thermography} (PT), thanks to its quickness and convenience of this inspection technique, is one of the most common methods in Active Thermography procedures, in which an energy source is required to produce a thermal contrast between the feature of interest and the background.  Therefore, numerous studies have been devoted to this technique \citep{Mayr2011active,Maldague1993Nondestructive,Maldague1994bInfra,Maldague2002intro,Maldague2004Double,2007-Ibarra-Castanedo,2011-ClementeIbarra-Castanedo,2007-ClementeIbarra-Castanedo,shoja2011inspection,duan2013quantitative,vahiddefect2014}. However, most research studies have applied the traditional PT: using hot thermal stimulation as sources (such as lamps, flashes, etc). Only a very limited number of studies using cold approach have been performed on industrial product inspection so far \citep{Maldague1993Nondestructive,Maldague1994bInfra,endohdynamical2012},(or mainly used in Electronic Systems) \citep{2012-LewisHom}. Thus, the advantage and convenience of using the cold stimulation in IT still remains to be investigated in detail and better understood.

Contrary to the traditional Pulsed Thermography, whose thermal response generated inside the specimen is often hot, the cold approach has its own advantages under certain circumstances: it would be more economical than the warming approach if the material specimen is already hot, or more beneficial since there's less thermal noises. These advantages have also been shown in the literature \citep{Maldague1993Nondestructive,Maldague1994bInfra,endohdynamical2012}. Therefore, the continuation of studies on cold stimulation resources in infrared thermography is encouraged by these preliminary researches.