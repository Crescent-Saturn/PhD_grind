\chapter*{Introduction}         % ne pas numéroter
\phantomsection\addcontentsline{toc}{chapter}{Introduction} % inclure dans TdM

%Une thèse ou un mémoire devrait normalement débuter par une
%introduction. Celle-ci est traitée comme un chapitre normal, sauf
%qu'elle n'est pas numérotée.
Nowadays, the increasing cost of energy has made energy saving a vital necessity in the current world.
According to a research \citep{coulomb2008refrigeration}, about 15\% of all electricity consumed worldwide is devoured by refrigeration technology. Therefore, the correct transport of perishable foodstuffs in refrigerated vehicles, especially for dairy products, meat and frozen foods, known as ``Maintaining the cold chain", is one of the examples involved in. It has also become the key part fo every distributor's food safety program.

As one of the major applications of refrigeration technology, the cold chain concept has several main procedures as: production, processing, packaging, distribution and handling, retail and consumer's household \citep{Estrada-Flores2010}. It firstly starts at the procedure of production of raw materials, in which immediate cooling has been deployed to provide perfect quality. Then primary processing involves in with operations preparing the raw materials for a second processing stage.  In this stage, carcass chilling, precooling of fruits and vegetables or milk cooling and other cold chain operations are often represented. Transforming primary agricultural products into manufactured foods is performed in the secondary processing. After that, the cold chain continues in the form of freezing or chilling of foods, either in bulk or in packages. Food distribution and handling need to be performed always at controlled temperatures and planning of routes and schedules need to take into account the location and capacity of refrigerated distribution centres, the refrigerated modes (sea, air, land) and the volumes to be transported. Loading and unloading operations may be performed in refrigerated docks. Food at retail needs different levels of refrigeration: cold storage is required in supermarket distribution centres; preparation rooms  need to be air-conditioned; walk-ins and display cabinets need to be refrigerated. Finally, consumers store purchases of perishable products in domestic refrigerators.